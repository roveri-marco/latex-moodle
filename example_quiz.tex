% !TeX encoding = UTF-8
% !TeX spellcheck = en_US
\documentclass[twocolumn]{article}
\usepackage[cm]{fullpage}
\usepackage[final]{moodle}
\usepackage{tikz}
\def\myequation{y=a\sqrt{x}+b}
\htmlregister{\myequation}

\begin{document}
\begin{quiz}[ % options that apply to all questions
%	points=1, % default 1
%	penalty=0, % penalty in case of wrong attempt (between 0 and 1). Default 0.1
%	fraction=0, % fraction of points for wrong or partially true answers
	%feedback={coucou}
	] {Example Quiz}

\begin{multi}[points=3,numbering=Alph]{Multiple Choice (single correct answer)}
	% only one correct answer
	What is the first derivative of $x^3$ w.r.t. $x$?
	\item[feedback={this is a very long feedback; it may even be displayed in 
		several lines. Here is a new sentence! Does that work? Yes. Now, let's 
		put an 
		equation: \[\myequation.\]}] $\frac{1}{4} x^4+C$
	\item[]* $3x^2$ % the star stands for fraction=1
	\item[feedback={$\myequation$,  $\sin(x)/x$}] $51$
\end{multi}

\begin{multi}[multiple,numbering=roman]{Multiple Choice (several correct 
		answers)} % 
	%when several answers are correct
	Select the following numbers that are prime.
	\item[feedback={it is only divided by 1 and itself!}]* 5
	\item[feedback={divided by 2 and 3!}] 6
	\item[]* 7 %feedback={it is only divided by 1 and itself!}
	\item[feedback={divided by 2 and 4! Normally this feedback would be short 	
		but I want to make it longer for testing purposes.}] 8
\end{multi}

\begin{numerical}[ % options that apply to all answers
%	points=1, % default 1
%	penalty=0, % penalty in case of wrong attempt (between 0 and 1). Default 0.1
%%	fraction=0, % fraction of points for wrong or partially true answers
%	feedback={coucou} % feedback is given regardless of the answer
	] {Numerical}
What is $8+3$?
\item[fraction=100,feedback={this is a very long feedback; it may even be 
displayed in several lines. Here is a new sentence! Does that work? Yes. Now, 
let's put an equation: \[\myequation.\]}] 11
\item[fraction=0] 12
\item[fraction=0,feedback={Pfff\dots}] 5
\end{numerical}

\begin{shortanswer}[case sensitive=true]{Short Answer (case sensitive)}
What was Newton's first name?
\item[feedback={this is a very long feedback; it may even be displayed in 
several lines. Here is a new sentence! Does that work? Yes. Now, let's put an 
equation: \[\myequation.\]}] Isaac
\item[fraction=50,feedback={forgot how to capitalize properly?}] isaac
\item[fraction=0] Fig Fag Fog
\item[fraction=0,feedback={how noble!}] Sir
\end{shortanswer}

\begin{shortanswer}{Short Answer (case insensitive)}
Newton's rival was Gottfried Wilhelm \blank.
\item[feedback={Correct! But why the hell did you put a dot?}] Leibniz.
\item Leibniz
\item[fraction=0,feedback={write it backwards!}] Zinbiel
\end{shortanswer}

\begin{matching}[dd]{Matching}
Answer-specific feedback is too complicated for matching questions (lots of 
possible combinations). Moodle does not support that\dots
	\item[feedback={this feedback is garbage: it is placed in the XML but won't 
	make it through the moodle import}] 
	Orange \answer Orange
	\item[feedback={Actually, moodle's matching question type does not seem to 
	support feedback}] Lemon \answer Yellow
	\item[feedback={sadly...}] Banana \answer Yellow
	\item[] Strawberry \answer Red
	\item[]  \answer Black
\end{matching}

\begin{essay}[response required,response format=text,template={put 
your answer here}]{Essay}
Is learning worth it?
\item if the answer is "yes" $\rightarrow$ full grade
\item if the answer is silly  $\rightarrow$ minimum grade
\end{essay}

\begin{cloze}{Cloze}
\begin{tikzpicture}
\draw[->] (-1.5,0)--(1.5,0);
\draw[->] (0,-1.5)--(0,1.5);
\draw[fill=red,fill opacity=.2,draw=green] (0,0)circle[radius=1];
\node[text width=3.5cm,anchor=west] at (2.5,0) {This is a picture test. 
$\myequation$};
\end{tikzpicture}
Within the cloze environment, when using multichoice, moodle allows to make 
only one choice. Therefore, multiple choice questions shall be made such that 
all points can be acquired with a single good answer only. In moodlen, there is 
a strong limitation to math display in answers or feedbacks: any curly braces 
($\{,\}$) make the moodle's internal parser fail.

\begin{multi}
First, an inline dropdown box. It is very compact on moodle. The drawback is 
that there is no LaTeX or HTML rendering in answers. The feedback renders HMTL 
and LaTeX. It is a bit hidden as it pops up only when hovering the 
answer with the mouse.
\item[feedback={yes $ax+b$}]* chip
\item[fraction=10] chop
\item[feedback={this is a quite long feedback with equation: $y=ax+b$.}] 
chap
\end{multi}

\begin{multi}[horizontal]
Second, an horizontal multichoice. May be quite compact as well but inadequate 
when possible answers are lengthy or numerous. Both answers and feedback can 
be rendered using LaTeX and HTML.
\item[feedback={$ax+b$}]* True
\item[] False
\item[feedback={silly!}] $Dunno$
\end{multi}

\begin{multi}[vertical]
Last, a vertical multichoice. Behaves like the standard single multichoice.
\item[feedback={yes! $ax+b$}]* yop
\item[fraction=20] yap
\item[feedback={no!}] yip
\item[feedback={nope...}] $yup$
\end{multi}
\end{cloze}

\end{quiz}

%\begin{quiz}{Second Quiz}	
%\begin{numerical}{Numerical}
%What is $1+1$?
%\item 2
%\end{numerical}	
%\end{quiz}

\end{document}