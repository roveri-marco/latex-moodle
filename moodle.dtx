% \iffalse meta-comment
%
% moodle.dtx
% Copyright 2016 by Anders O.F. Hendrickson (anders.hendrickson@snc.edu)
%
% This work may be distributed and/or modified under the
% conditions of the LaTeX Project Public License, either version 1.3
% of this license or (at your option) any later version.
% The latest version of this license is in
%   http://www.latex-project.org/lppl.txt
% and version 1.3 or later is part of all distributions of LaTeX%
% version 2005/12/01 or later.
%
% This work has the LPPL maintenance status `maintained'.
% 
% The Current Maintainer of this work is Anders O.F. Hendrickson.
%
% This work consists of the files moodle.dtx and moodle.ins
% and the derived files moodle.sty and getitems.sty.
%
% \fi
%
% \iffalse
%<*driver>
\ProvidesFile{moodle.dtx}
%</driver>
%<package>\NeedsTeXFormat{LaTeX2e}[1999/12/01]
%<package>\ProvidesPackage{moodle}
%<*package>
    [2020/07/09 v0.8 Moodle quiz XML generation]
%</package>
%
%<*driver>
\documentclass[a4paper]{ltxdoc}
\usepackage{iftex}
\ifxetex
  \usepackage{fontspec}
  \setmainfont[Mapping=tex-text]{Linux Libertine O}
\else
  \usepackage[utf8]{inputenc} % necessary
  \usepackage[T1]{fontenc} % necessary
  \usepackage{microtype}
  \usepackage[ttscale=.875]{libertine}
\fi
%\usepackage{moodle}[2020/07/09]
\usepackage{amssymb,metalogo,,multirow,threeparttable,booktabs,hyperref,tikz,minted,changelog}
\usetikzlibrary{arrows,positioning}
\usemintedstyle{tango}
\EnableCrossrefs         
\CodelineIndex
\OnlyDescription
\RecordChanges
\begin{document}
  \DocInput{moodle.dtx}
  %\PrintChanges
\end{document}
%</driver>
% \fi
%
% \CheckSum{6453} ^^A Comment \OnlyDescription above to adjust
%
% \CharacterTable
%  {Upper-case    \A\B\C\D\E\F\G\H\I\J\K\L\M\N\O\P\Q\R\S\T\U\V\W\X\Y\Z
%   Lower-case    \a\b\c\d\e\f\g\h\i\j\k\l\m\n\o\p\q\r\s\t\u\v\w\x\y\z
%   Digits        \0\1\2\3\4\5\6\7\8\9
%   Exclamation   \!     Double quote  \"     Hash (number) \#
%   Dollar        \$     Percent       \%     Ampersand     \&
%   Acute accent  \'     Left paren    \(     Right paren   \)
%   Asterisk      \*     Plus          \+     Comma         \,
%   Minus         \-     Point         \.     Solidus       \/
%   Colon         \:     Semicolon     \;     Less than     \<
%   Equals        \=     Greater than  \>     Question mark \?
%   Commercial at \@     Left bracket  \[     Backslash     \\
%   Right bracket \]     Circumflex    \^     Underscore    \_
%   Grave accent  \`     Left brace    \{     Vertical bar  \|
%   Right brace   \}     Tilde         \~}
%
%
% \changes{v0.5}{2016/01/05}{Initial version}
% \changes{v0.6}{2019/02/18}{Bux-fixing release}
% \changes{v0.7}{2020/07/09}{Feature extensions}
%
% \GetFileInfo{moodle.dtx}
%
% \DoNotIndex{\newcommand,\newenvironment,\def}
% 
%
% \title{The \textsf{moodle} package: \\
%        generating Moodle quizzes via \LaTeX%
%        \thanks{This document
%                corresponds to \textsf{moodle.sty}~\fileversion, dated \filedate.}}
% \author{Anders Hendrickson\footnote{original author of the package (\texttt{v0.5})}\\
% St.~Norbert College, De~Pere, WI, USA \\ \texttt{anders.hendrickson@snc.edu} \and
% Matthieu Guerquin-Kern\footnote{author of the updates (\texttt{v0.6} and \texttt{v0.7}),
% partially funded by \href{https://www.ensea.fr/en/}{ENSEA Graduate School}, France.}\\
% \href{mailto:guerquin-kern AT crans.org}{guerquin-kern AT crans.org}}
% \date{July 9, 2020}
% 
% \maketitle
%
% \providecommand\TikZ{Ti\emph{k}Z}
%
% \section{Motivation}
%
% The acronym Moodle stands for ``Modular Object-Oriented Dynamic Learning Environment.''
% It is an open source learning management system (LMS) employed by many universities,
% colleges, and high schools to provide digital access to course materials, such as
% notes, video lectures, forums, and the like; see
% \url{https://moodle.com/moodle-lms/} for more information. 
% One of the many useful
% features of Moodle is that mathematical and scientific notation can be entered in
% \LaTeX\ or \TeX\ code, which will be typeset either through a built-in \TeX\ filter
% or by invoking MathJax.
% 
% For instructors who want to give students frequent feedback,
% but lack the time to do so, a particularly valuable module in Moodle
% is the \emph{quiz}.  A Moodle quiz can consist of several different types of
% questions---not only multiple choice or true/false questions, but also 
% questions requiring a short phrase or numerical answer, and even essay 
% questions.  All but the essay questions are automatically graded by the 
% system, and the instructor has full control over how often the quiz may be 
% attempted, its duration, and so forth.  Feedback can be tailored to specific
% mistakes the student makes.
% 
% All these features make Moodle quizzes very useful tools for instructors 
% who have access to them.
% Unfortunately, the primary way to create or edit a Moodle quiz 
% is through a web-based interface that can be slow to operate.
% To users of \LaTeX, accustomed to the speed of typing source code on
% a keyboard alone, the agonizing slowness of switching between mouse and keyboard
% to navigate a web form with its myriad dropdown boxes, radio buttons,
% compounded with a perceptible time lag as one's Moodle server responds to requests,
% can produce a very frustrating experience.  Moreover, editing is entirely 
% impossible without network access.
% 
% Once the quiz is written, there is no easy way to view and proofread all the 
% information of which it is made.  Each question is edited on a separate webpage,
% which is so full of options that it cannot be viewed on a single screen.  
% An instructor has to spend much time checking over the newly created quiz in 
% order to be confident there are no errors.
% 
% Added to all this is the frustration of managing graphics.
% If a question requires an image---say, asking a calculus student to interpret
% the graph of a function---the image must first be produced as a standalone file
% (e.g., in JPG or PNG format), uploaded to Moodle, and then chosen in a web-based
% HTML editor.  Great is the vexation of the instructor who decided to alter a 
% question, as there are more and more possibilities of error whenever multiple 
% files must be kept synchronized.
% 
% Users of \LaTeX\ are also accustomed to the speed and flexibility that 
% comes from defining their own macros, which may be as brief as writing
% |\R| instead of |\mathbb{R}|
% or as complex as macros that generate entire paragraphs of text.
% The Moodle editor, by contrast, requires you to type |\mathbb{R}|
% every single time you want $\mathbb{R}$.
% 
% Finally, there is the question of archiving and reusing one's work.
% Much, much work goes into creating Moodle quizzes, which then reside
% on a Moodle server somewhere in the cloud in a format neither easily browsable
% nor easily modifiable.
% 
% \LaTeX\ itself has the power to solve all these difficulties: 
% it is swift to edit and swifter to compile a \LaTeX\ document,
% and the PDF may be previewed onscreen or printed out for ease of proofreading.
% Mathematical graphics can be integrated within the main file through \TikZ,
% and of course \LaTeX\ macros can be customized.
% Using the present \textsf{moodle} package,
% a quiz author can type a quiz using familiar \LaTeX\ syntax and document 
% structure.  Upon compilation, \LaTeX\ will generate both a well-organized
% PDF that is easy to proofread and an XML file that can be uploaded directly
% to Moodle.  The entire process is far faster than using Moodle's own
% web-based editor, makes it easier to catch one's mistakes, 
% and the ultimate source code of one's work is a human-readable |.tex| file 
% that can be archived, versioned, browsed, and edited offline.
% 
% Strictly speaking, the \textsf{moodle} package does not generate quizzes:
% it generates question banks that can be imported in the LMS. The teacher
% still needs to compose manually a quiz from the question banks. Hopefully,
% two Moodle features supported by the package make this task easier:
% categories and tags.
%
% In this documentation the LMS is referred to as Moodle (uppercase M and
% roman font) while the \LaTeX\ package that is documented here is referred
% to as \textsf{moodle} (all lower case and sans serif font).
% 
% \section{Workflow}
% The process of creating a quiz in Moodle using this package is depicted in
% Figure~\ref{fig:workflow}. It follows a few steps:
% \begin{enumerate}
%   \item Write a \LaTeX\ document using |\usepackage{moodle}| as described 
% below.
%   \item Compile the document to PDF using pdf\LaTeX\ or \XeLaTeX. This will 
% also produce the file \meta{jobname}|-moodle.xml|.
%   \item Open Moodle, navigate to the desired course,
%         and under ``Question bank'' select ``Import.''
%   \item Select ``Moodle XML format,'' choose the XML file to upload, and press ``Import.''
%   \item After Moodle verifies that the questions have been imported correctly,
%         you may add them to your quizzes.
% \end{enumerate}
% \begin{figure}[tbp]
% \centering
% \begin{tikzpicture}[node distance=1,auto,bend 
% angle=45,box/.style={rectangle,draw=blue!50,rounded corners=3,top 
% 	color=white,bottom color=black!20,thick,align=center,text 
% 	width=2.5cm},elmt/.style={font=\itshape,align=left},
% cmnt/.style={font=\footnotesize,align=center},
% bigcmnt/.style={font=\normalsize,align=center},pre/.style={<-,>=stealth',thick},
% post/.style={->,>=stealth',thick}, prepost/.style={<->,>=stealth',thick}]
% \draw[white,fill=orange!20,rounded corners=20]
% (-1.5,0)--++(.5,.8)--++(9.5,0)--++(0.5,-.8)--++(-5,-3)--cycle;
% \node[orange!80!black] (dev) at (5,-.3) {\textit{Developpement}};
% \fill[green,fill opacity=.2,rounded corners=20]
% (-1.5,1)--++(0,-5.8)--++(10.5,0)--++(0,1.5)--cycle;
% \node[green!80!black] (dev) at (6.2,-3) {\textit{Publishing}}; 
% \node[box,text width=2cm] (tex) {\texttt{.tex} source file};
% \node[rectangle,fill=white,draw,align=center,text 
% width=1.5cm,below=of tex] (compfinal) {\LaTeX{}\\ engine};
% \node[rectangle,fill=white,draw,align=center,text width=1.5cm,right=of tex,xshift=4.5cm]
% (compdraft) {\LaTeX{}\\ engine};
% \node[box,right=of compfinal,xshift=.5cm] (pdf) {\texttt{.pdf} file for proofreading};
% \node[rectangle,fill=white,draw,align=center,text width=2cm,below=of compfinal] (extern) 
% {Picture\\processing};
% \draw (pdf|-extern) node[box,anchor=center] (xml) {\texttt{.xml} file\\(pictures embedded)};
% \draw (compdraft|-xml)
% node[rectangle,fill=white,draw,align=center,text width=1.5cm,anchor=center] (moodle) 
% {Moodle\\Server};
% \draw (tex) edge [post,bend right=10] node[cmnt,pos=.5,left] {\texttt{final}} (compfinal);
% \draw (tex) edge [post,bend left=10] node[cmnt,pos=.5,above] {\texttt{draft}} (compdraft);
% \draw (compdraft) edge [post,bend left=10] (pdf);
% \draw (compfinal) edge [post] (pdf);
% \draw (compfinal) edge [post,bend left=10] (xml);
% \draw (compfinal) edge [post,bend right=30] node[cmnt,black!40,pos=.5,text width=1cm,left]
% {\texttt{tikz}, \texttt{.png}, \texttt{.jpg}} (extern);
% \draw (extern) edge [post,green,bend right=30] node[cmnt,black!40,pos=.5,below,right]
% {\texttt{base64}} (compfinal);
% \draw (pdf) edge [post,red,bend left=10] node[cmnt,pos=.5,above right] 
% {improvements} (tex);
% \draw (xml) edge [post,red] node[cmnt,pos=.5,above] {import} (moodle);
% \end{tikzpicture}
% \caption{Block diagram describing a typical workflow using the \textsf{moodle} package.}
% \label{fig:workflow}
% \end{figure}
%
% \section{Usage}\label{sect:usage}
% \subsection{Example Document}
%
% The following pages presume the reader already has some familiarity with creating
% and editing Moodle quizzes through the web interface.
% The |xkeyval| package is used to provide a key-value interface.
% Here is a simple example document:
% \begin{VerbatimOut}[gobble=1]{minted.doc.out}
%   \documentclass[12pt]{article}
%   \usepackage[section]{moodle}
%   \def\myreactiontoasillyanswer{What!?}
%   \htmlregister{\myreactiontoasillyanswer}
%   \begin{document}
%   \begin{quiz}{My first quiz}
%     \begin{numerical}[points=2]{Basic addition}
%       What is $8+3$?
%       \item 11
%     \end{numerical}
%     \begin{shortanswer}[usecase]{Newton's name}
%       What was Newton's first name?
%       \item Isaac
%       \item[fraction=0, feedback={No, silly!}] Fig
%       \item[fraction=0] Sir
%     \end{shortanswer}
%     \begin{multi}[points=3]{A first derivative}
%       What is the first derivative of $x^3$?
%       \item[feedback={no!}]  $\frac{1}{4} x^4+C$
%       \item* $3x^2$
%       \item[feedback={\myreactiontoasillyanswer}]  $51$
%     \end{multi}
%   \end{quiz}
%   \end{document}
% \end{VerbatimOut}
% \inputminted[gobble=2,frame=lines]{latex}{minted.doc.out}
% Key features to note in this first example are that a |quiz| environment
% contains several question environments.
% Each question takes a name as a mandatory argument, 
% and it may also take optional key-value arguments within brackets.
% The question environments resemble list environments
% such as |itemize| or |enumerate|, in that answers are set off by 
% |\item|'s, but the question itself is the text that occurs before
% the first |\item|.
% 
% \DescribeMacro{\htmlregister}
% Using |\htmlregister|\marg{macroname}, tells \LaTeX\ about the macros you 
% use. This way, the macros will be properly expanded in the XML file.
% This works only if the macros are defined \emph{without} optional argument.
%
%^^A \DescribeMacro{\moodleregisternewcommands} 
%^^A It can be cumbersome to record individually a list of macros for expansion.
%^^A Calling |\moodleregisternewcommands| triggers the automatic
%^^A expansion of macros defined subsequently using |\newcommand|.
%^^A Again, this works only if the macros are defined \emph{without} optional
%^^A argument.
% 
% \DescribeMacro{draft}
% \DescribeMacro{final}
% If the package option |draft| is invoked, by calling
% |\usepackage[draft]{moodle}| or |\documentclass[draft]{...}|, then no
% XML file will be generated. This is especially useful while editing a
% quiz containing graphics, so as to avoid the time spent converting
% image files. The package option |final| might be useful if one wants
% to avoid the option |draft| to be inherited from the |documentclass|.
%
% \DescribeMacro{nostamp}
% By default, the package will output a stamp as comment in the XML file.
% This stamp contains information gathered about the TeX engine, the
% operating system used and the package version. For instance:
% \begin{quote}\small
% |<!-- This file was generated on 2020-07-15 by pdfLaTeX -->|\\
% |<!-- running on Linux with the package moodle v0.7 -->|
% \end{quote}
% The package option
% |nostamp| prevents this stamp to be written in the XML file.
%
% \DescribeMacro{section}
% \DescribeMacro{section*}
% \DescribeMacro{subsection}
% \DescribeMacro{subsection*}
% If the package option |section| is invoked (|\usepackage[section]{moodle}|),
% then each quiz is represented by a different \LaTeX\ section. Starred
% variants correspond to unnumbered sections or subsections. To preserve
% compatibility with Version 0.5, the default is |subsection*|. Consequently,
% |\usepackage{moodle}| is equivalent to |\usepackage[subsection*]{moodle}|.
%
% \subsection{Quiz and Question Environments}
% 
% \DescribeMacro{quiz}
% A |.tex| document to generate Moodle quizzes contains one or more
% |quiz| environments, within which various question environments are nested.
% The required argument to the |quiz| environment names the ``question bank''
% to which the questions inside will belong after being uploaded to Moodle.
% \begin{center}
% |\begin{quiz}|\oarg{options}\marg{question bank name}%
% \end{center}
% There are no |quiz|-specific options,
% but any \meta{options} set with |\begin{quiz}|
% will be inherited by all questions contained within that |quiz| environment.
% 
% \DescribeMacro{\moodleset}
% Options may also be set at any time with |\moodleset|\marg{options};
% these changes are local to \TeX-groups.
% \bigskip
% 
% The syntax for each question environment is
% \begin{quote}
%   |\begin|\marg{question type}\oarg{question options}\marg{question name} \\
%   \rule{2em}{0pt}\meta{question text} \\
%   \rule{2em}{0pt}|\item| \meta{item} \\
%   \rule{2em}{0pt}\quad$\vdots$ \\
%   \rule{2em}{0pt}|\item| \meta{item} \\
%   |\end|\marg{question type}
% \end{quote}
% The meaning of the \meta{item}s varies depending on the question type,
% but they usually are answers to the question.
% Details will be given below.
% 
% The following key-value options may be set for all questions:
% 
% \DescribeMacro{points}\DescribeMacro{default grade}
% By default, each question is worth 1 point on the quiz.
% This may be changed with the |points| key or its synonym, |default grade|;
% for example, |points=2| makes that question worth two points.
% 
% \DescribeMacro{penalty}
% The |penalty| is the fraction of points that is taken off for each wrong attempt;
% it may be set to any value between 0 and 1.
% The default is |penalty=0.10|.
% 
% \DescribeMacro{fraction}\DescribeMacro{fractiontol}
% In most question types, it is possible to designate some answers as being
% worth partial credit---that is, some fraction of a completely correct answer.
% The |fraction| key may be set to any of the values given in Table~\ref{tab:fraction}, 
% from |0| (entirely wrong) to |100| (entirely correct).
% 
% \begin{table}[tbp]
% \centering
% \caption{\href{https://github.com/moodle/moodle/blob/MOODLE\_310\_STABLE/question/engine/bank.php\#L339}{Valid positive options} for the \texttt{fraction} key: $100\cdot(p/q)$.}
% \label{tab:fraction}
% \begin{tabular}{r*{10}{l}}
% \toprule
% & \multicolumn{10}{c}{Numerator integer $p$}\\
% \cmidrule{2-11}
% $q$ & 0 & 1 & 2 & 3 & 4 & 5 & 6 & 7 & 8 & 9\\
% \cmidrule(lr){1-1}\cmidrule(lr){2-2}\cmidrule(lr){3-3}\cmidrule(lr){4-4}
% \cmidrule(lr){5-5}\cmidrule(lr){6-6}\cmidrule(lr){7-7}\cmidrule(lr){8-8}
% \cmidrule(lr){9-9}\cmidrule(lr){10-10}\cmidrule(lr){11-11}
% 1 & 0&100&&&&&&&&\\
% 2 & 0&50&100&&&&&&&\\
% 3 & 0&33.33333&66.66667&100&&&&&&\\
% 4 & 0&25&&75&100&&&&&\\
% 5 & 0&20&&&80&100&&&&\\
% 6 & 0&16.66667&&&&83.33333&100&&&\\
% 7 & 0&14.28571&&&&&&100&&\\
% 8 & 0&12.5&&&&&&&100&\\
% 9 & 0&11.11111&&&&&&&&100\\
% 10 & 0&10&20&30&40&50&60&70&80&90\\
% 20 & 0&5&&&&&&&&\\
% \bottomrule
% \end{tabular}
% \end{table}
% In questions where several choices can be selected (see |multi| with the option
% |multiple|), positive fractions must sum up to exactly 100. It is also possible to set
% negative fractions (from -100 to 0) for wrong choices, in order to prevent the
% selection of all choices from leading to a good grade.
% In this case, the value ranging from -100 to 0 must be the opposite of one of the
% values listed in Table~\ref{tab:fraction}.
% 
% The package tries to match the |fraction| key to one of the admissible values.
% To this end, the tolerance is parameterized by the |fractiontol| key. It defaults
% to |0.01| but may be changed. When no admissible fraction value is matched, the
% package throws an error.
%
% \DescribeMacro{feedback}
% The |feedback| key sets text that will appear to the student after completing the quiz.
% For example, one might set 
% \begin{center}
%   |feedback={This question might show up in the final exam.}|
% \end{center}
% The desired feedback should be included in braces.
% 
% Two kinds of feedback can be given.  If the |feedback| key is set for a 
% question, then that feedback will appear to each student regardless of the student's answer.
% Answer-specific feedback (perhaps explaining a common mistake)
% may also be given by setting the |feedback| key \emph{at the individual answer}.
%
% \DescribeMacro{tags}
% The |tags| key sets a keyword for the question that will be taken into account by Moodle for
% filtering purposes or classification of questions inside the question bank. It is possible
% for instance to build a quiz with questions cherry-picked among the set of questions holding
% a particular tag.
% For example, one might set 
% \begin{center}
%   |tags={easy}|
% \end{center}
% The desired tag should be included in braces.
%
% Tags can be assigned at two levels.  If the |tags| key is set at the quiz level,
% then that tags will be assigned by default to each question of the quiz.
% Question-specific tags can be assigned by setting the |tags| key \emph{at the question level}.
% Since only single tag is supported, the tag a the question-level overrides eventual tags
% specified at the quiz-level.
% 
% \subsection{Question Types}
% 
% We next discuss the various question types supported by \textsf{moodle}
% and the options that may be set.
%
%\subsubsection{True/False}
% 
% \DescribeMacro{truefalse}
% The syntax for a True/False question is as follows:
% \begin{quote}
%   |\begin{truefalse}|\oarg{question options}\marg{question name} \\
%   \rule{2em}{0pt}\meta{question text} \\
%   \rule{2em}{0pt}|\item*| \meta{feedback when ``true" is chosen} \\
%   \rule{2em}{0pt}|\item| \meta{feedback when ``false" is chosen} \\
%   |\end{truefalse}|
% \end{quote}
% The correct answer is designated by the asterisk |*| after the |\item|;
% it need not appear first in the list.
%
% Answer-specific feedback can also be defined as an item option, similarly to
% other types.
% \begin{quote}
%   |\begin{truefalse}|\oarg{question options}\marg{question name} \\
%   \rule{2em}{0pt}\meta{question text} \\
%   \rule{2em}{0pt}|\item[feedback={|\meta{When ``true" is chosen}|}]*| \\
%   |\end{truefalse}|
% \end{quote}
% Note that, in this example, no feedback is defined for the incorrect answer
% ``False": the corresponding item can be omitted.
% 
% With the True/False question type, the |penalty| key has no effect.
%
% \subsubsection{Multiple Choice}
% 
% \DescribeMacro{multi}
% The syntax for a classic multiple choice question,
% with only one correct answer, is as follows:
% \begin{quote}
%   |\begin{multi}|\oarg{question options}\marg{question name} \\
%   \rule{2em}{0pt}\meta{question text} \\
%   \rule{2em}{0pt}|\item*| \meta{correct answer} \\
%   \rule{2em}{0pt}|\item|\oarg{options} \meta{wrong answer} \\
%   \rule{2em}{0pt}\quad$\vdots$ \\
%   \rule{2em}{0pt}|\item|\oarg{options} \meta{wrong answer} \\
%   |\end{multi}|
% \end{quote}
% The correct answer is designated by the asterisk |*| after the |\item|;
% it need not appear first in the list.
% 
% 
% \DescribeMacro{shuffle}
% The boolean key |shuffle| determines whether Moodle will
% rearrange the possible answers in a random order.
% Setting |shuffle=false| will guarantee that the answer appear
% in the order they were typed; the default is |shuffle=true|.
% 
% \DescribeMacro{numbering}
% Moodle offers different options for numbering the possible answers.
% You may set the |numbering| key to any of the following values,
% which mirror the usual \LaTeX\ syntax:
% |alph|, |Alph|, |arabic|, |roman|, |Roman|, and |none|.
% Calling |numbering=none| produces an unnumbered list of answers.
% The Moodle syntax of |abc|, |ABCD|, |123|, |iii|, and |IIII| is also 
% acceptable,
% but note that it requires \emph{four} capital letters to obtain
% upper-case Roman or alphabetic numerals this way.
% 
% \DescribeMacro{fraction}
% The |fraction| key can be used to designate some wrong answers
% as being worth partial credit.  For example, a question might read thus:
% \begin{VerbatimOut}[gobble=1]{minted.doc.out}
%   \begin{multi}{my question}
%     Compute $\int 4x^3\,dx$.
%     \item* $x^4+C$
%     \item[fraction=50] $x^4$
%     \item $12x^2$
%   \end{multi}
% \end{VerbatimOut}
% \inputminted[gobble=2,frame=lines]{latex}{minted.doc.out}
% Thus the asterisk |*| is shorthand for |fraction=100|, 
% whereas a bare |\item| sets |fraction=0|.
% 
% \DescribeMacro{single}
% By default, the |multi| environment produces
% a multiple choice question with only one correct answer;
% this is called |single| mode, and on Moodle it appears with radio buttons.
%
% \DescribeMacro{multiple}
% It is also possible to write questions with possibly more than one correct answer,
% asking the user to check all correct answers. To do this, use the key |multiple|
% or |single=false|.
% The worth of each correct answers in |multiple| mode may be set by |fraction|,
% but Moodle asks that all the fractions add up to \emph{exactly} 100.
% If you simply designate each correct answer with |\item*|, then \textsf{moodle}
% will divide equally among those answers the points lefts for a sum of 100\%.
% Items that are not given a |fraction| are considered incorrect and selecting them
% results in negative points such that the sum of all incorrect answers is -100\%.
% For example, the following two examples are equivalent:
% \begin{VerbatimOut}[gobble=1]{minted.doc.out}
%   \begin{multi}[multiple]{my question}
%     Which numbers are prime?
%     \item[fraction=20] 2
%     \item* 5
%     \item* 7
%     \item[fraction=-10] 1 
%     \item 6
%     \item 8
%   \end{multi}
%   
%   \begin{multi}[multiple]{my question}
%     Which numbers are prime?
%     \item[fraction=20] 2
%     \item[fraction=40] 5
%     \item[fraction=40] 7
%     \item[fraction=-10] 1
%     \item[fraction=-45] 6
%     \item[fraction=-45] 8
%   \end{multi}
% \end{VerbatimOut}
% \inputminted[gobble=2,frame=lines]{latex}{minted.doc.out}
% Note that, in this example, negative fractions are set for wrong choices. This
% prevents students selecting all options to obtain a good grade with no merit.
% 
% \subsubsection{Numerical}
%
% A numerical question in Moodle requires the student
% to input a real number in decimal form.
% Its typical format is
% \begin{quote}
%   |\begin{numerical}|\oarg{question options}\marg{question name} \\
%   \rule{2em}{0pt}\meta{question text} \\
%   \rule{2em}{0pt}|\item|\oarg{options} \meta{correct answer} \\
%   |\end{numerical}|
% \end{quote}
% If there is more than one correct answer, additional |\item|'s may be included.
% Because this is not a multiple choice question, there is no need to provide
% incorrect answers.  There may nevertheless be reasons to include incorrect answers.
% For example, partially correct answers may be specified by setting the |fraction| key.
% Feedback for a common mistake may be given by including the incorrect answer like this:
% \begin{quote}\footnotesize
%   |\item[fraction=0,feedback={You forgot to antidifferentiate!}]| \meta{incorrect answer}
% \end{quote}
%
% \DescribeMacro{tolerance} 
% The |tolerance| key can be used to specify the validity of answers within some margin.
% For example, with the question
% \begin{VerbatimOut}[gobble=1]{minted.doc.out}
%   \begin{numerical}[tolerance=0.01]{my question}
%     Approximate value of $\sqrt{2}$?
%     \item 1.4142
%     \item[fraction=0,feedback={twice this!}] 0.70711
%   \end{numerical}
% \end{VerbatimOut}
% \inputminted[gobble=2,frame=lines]{latex}{minted.doc.out}
% In this case, the input \textsf{1.41} will be validated and the input \textsf{0.71} will
% get the specific feedback.
% 
% Units, unit-handling and multipliers are currently unsupported.
%
% \subsubsection{Short Answer}
% A short answer question resembles a numerical question: the student is to fill
% in a text box with a missing word or phrase.
% \begin{quote}
%   |\begin{shortanswer}|\oarg{question options}\marg{question name} \\
%   \rule{2em}{0pt}\meta{question text} \\
%   \rule{2em}{0pt}|\item|\oarg{options} \meta{correct answer} \\
%   \rule{2em}{0pt}\quad$\vdots$ \\
%   \rule{2em}{0pt}|\item|\oarg{options} \meta{correct answer} \\
%   |\end{shortanswer}|
% \end{quote}
% You can make the text box appear as part of the question with the 
% control sequence |\blank|.  For example,
% your question might read 
% \begin{VerbatimOut}[gobble=1]{minted.doc.out}
%   \begin{shortanswer}{Leibniz}
%     Newton's rival was Gottfried Wilhelm \blank.
%     \item Leibniz
%     \item Leibniz.
%   \end{shortanswer}
% \end{VerbatimOut}
% \inputminted[gobble=2,frame=lines]{latex}{minted.doc.out}
% Note that as the blank occurred at the end of a sentence, 
% we included two answers, 
% lest students get the question wrong merely by
% including or omitting a period.
% 
% \DescribeMacro{case sensitive}\DescribeMacro{usecase}
% The default setting when creating a Short Answer question in Moodle
% is to ignore the distinction between upper- and lower-case letters
% when grading a short answer question. This default is preserved by
% \textsf{moodle}.
% You can make a question case-sensitive with the key |case sensitive| 
% or its shorter synonym |usecase|.
% 
% \subsubsection{Essay Questions}\label{subsubsect:essay}
% Instructors may ask essay questions on a Moodle quiz,
% although Moodle's software is not up to the task of grading them!
% Instead each essay question answer must be graded manually by the
% instructor or a teaching assistant.
% \begin{quote}
%   |\begin{essay}|\oarg{question options}\marg{question name} \\
%   \rule{2em}{0pt}\meta{question text} \\
%   \rule{2em}{0pt}|\item|\oarg{options} \meta{notes for grader} \\
%   \rule{2em}{0pt}\quad$\vdots$ \\
%   \rule{2em}{0pt}|\item|\oarg{options} \meta{notes for grader} \\
%   |\end{essay}|
% \end{quote}
% Instead of containing answers, the |\item| tags for the |essay| question 
% contain notes that will appear to whoever is grading the question manually.
% 
% \DescribeMacro{response required}
% Although Moodle cannot grade the content of an essay question,
% it can at least determine whether the question has been left blank.
% If the |response required| key is set, Moodle will insist that the student
% enter something in the blank before accepting the quiz as completed.
% 
% \DescribeMacro{response format}
% Moodle offers five different ways for students to enter and/or upload their
% answers to an essay question.  You may choose one of these five options:
% \begin{description}
%   \item[{\ttfamily html}] An editor with the ability to format HTML responses
%         including markup for italics, boldface, etc.  This is the default.
%   \item[{\ttfamily file}] A file picker allowing the student to upload a 
%         file, such as a PDF or DOC file, containing the essay.
%   \item[{\ttfamily html+file}] The same HTML editor as above, but with the 
%         ability to upload files as well.  This permits some students to type 
%         answers directly into the web form, and others to compose their 
%         essays in another program first.
%   \item[{\ttfamily text}] This editor allows only for entering plain text 
%         without any markup.
%   \item[{\ttfamily monospaced}] This yields a plain text editor, without any 
%         markup, and with a fixed-width font.  This could be useful for 
%         entering code snippets, for example.
% \end{description}
% 
% \DescribeMacro{response field lines}
% The key |response field lines| controls the height of the input box.
% The default is |response field lines=15|.
% 
% \DescribeMacro{attachments allowed}
% The |attachments allowed| key controls \emph{how many} attachments a student is
% allowed to upload.  Permissible values are |0|, |1|, |2|, |3|, or |unlimited|.
% 
% \DescribeMacro{attachments required}
% You may also require the student to upload a certain number of attachments
% by setting |attachments required| to |0|, |1|, |2|, or |3|.
% 
% \DescribeMacro{template}
% Finally, you may preload the essay question with a template that the student
% will edit and/or type over, with the key |template=|\marg{template}.
% The \meta{template} should be enclosed in braces.
% 
% \subsection{Matching Questions}
% 
% A matching question offers a series of subquestions
% and a set of possible answers from which to choose.
% If there are $m$ questions and $n\geq m$ possible answers,
% a matching question will look like this:
% 
% \begin{quote}
%   |\begin{matching}|\oarg{question options}\marg{question name} \\
%   \rule{2em}{0pt}\meta{question text} \\
%   \rule{2em}{0pt}|\item|\oarg{options} \meta{question 1} |\answer| \meta{answer 1}\\
%   \rule{2em}{0pt}|\item|\oarg{options} \meta{question 2} |\answer| \meta{answer 2}\\
%   \rule{2em}{0pt}\quad$\vdots$ \\
%   \rule{2em}{0pt}|\item|\oarg{options} \meta{question $m$} |\answer| \meta{answer $m$}\\
%   \rule{2em}{0pt}|\item|\oarg{options} |\answer| \meta{answer $m+1$}\\
%   \rule{2em}{0pt}\quad$\vdots$ \\
%   \rule{2em}{0pt}|\item|\oarg{options} |\answer| \meta{answer $n$}\\
%   |\end{matching}|
% \end{quote}
% Answers $1$ through $m$ correspond to questions $1$ through $m$;
% answers $m+1$ through $n$ are ``decoy'' answers.
% If multiple questions should have the same answer,
% be sure your typed answer match exactly, so that Moodle will not
% create duplicate copies of the same answer!
% 
% \DescribeMacro{shuffle}
% The |matching| question accepts the option of |shuffle| to randomly
% permute the questions and answers; by default |shuffle=true|.
% 
% \DescribeMacro{drag and drop}\DescribeMacro{dd}
% The standard matching question offered by Moodle corresponds to
% a dropdown box for choosing the answer to each question.
% There also exists a ``drag and drop matching'' plugin for Moodle that
% shows all questions in one column,
% all answers in a second column, and allows students to drag the correct
% answer to the question using a mouse.
% In this package, to enable drag-and-drop matching, use the key
% `|drag and drop|' or `|dd|' for short.  The default is |dd=false|.
% If you choose the standard format, then due to the limitations of
% dropdown boxes, no \LaTeX\ or HTML code can be used in the answers.
% 
% \subsection{Cloze Questions}
% 
% A ``cloze question'' has one or more subquestions embedded within a passage of text.  
% For example, you might ask students to fill in several missing words within
% a sentence, or calculate several coefficients of a polynomial.
% To encode cloze questions in \LaTeX\ using this package is easy:
% you simply nest one or more |multi|, |shortanswer|, or |numerical| environments
% within a |cloze| environment, as in the following example:
% \begin{VerbatimOut}[gobble=1]{minted.doc.out}
%   \begin{cloze}{my cloze question}
%     Thanks to calculus, invented by Isaac
%     \begin{shortanswer}[usecase]
%       \item Newton
%     \end{shortanswer},
%     we know that the derivative of $x^2$ is
%     \begin{multi}[horizontal]
%       \item $2x$
%       \item* $\frac{1}{3} x^3 + C$
%       \item $0$
%     \end{multi}
%     and that $\int_0^2 x^2\,dx$ equals
%     \begin{numerical}
%       \item[tolerance=0.0004] 2.667
%     \end{numerical}.
%     Thanks, Isaac!
%   \end{cloze}
% \end{VerbatimOut}
% \inputminted[gobble=2,frame=lines]{latex}{minted.doc.out}
% Note that when used as a subquestion within a cloze question,
% |\begin{multi}| is \emph{not} followed by name in braces;
% the same is true for the |shortanswer| and |numerical|
% environments.
% 
% \DescribeMacro{single=true}\DescribeMacro{single=false}\DescribeMacro{multiple}
% Before Moodle version 3.5, within a cloze question, a multiple choice question
% was necessarily of type |single|, i.e. with a single good answer. If you intend
% to export your quiz to Moodle 3.5+, the option |multiple| can be used, when
% multiple good answers are to be found.
%
% \DescribeMacro{vertical}\DescribeMacro{horizontal}\DescribeMacro{inline}
% Within a cloze question, by default, a multiple choice question is implemented
% as an |inline| dropdown box. This is visually compact, but it also prevents
% the use of mathematical or HTML formatting.
% Adding the option |vertical| displays the subquestion as a vertical column
% of radio buttons instead; likewise the option |horizontal| creates a horizontal 
% row of radio buttons.
% The option |vertical| is incompatible with |multiple| or |single=false|
% (dropdown boxes don't let you pick up several answers!).
%
% \DescribeMacro{shuffle}
% Starting from Moodle version 3.0, within a cloze question, the items of a
% multiple choice question can be shuffled. Setting |shuffle=false| will
% guarantee that the answer appear in the order they were typed; the
% default is |shuffle=true|.
%
% \DescribeMacro{case sensitive}\DescribeMacro{usecase}
% Within a cloze question, the short answer question can be made case sensitive.
% This option, disabled by default, is selected with |case sensitive| or |usecase|.
% 
% \subsection{Summary of the Key Options}
% 
% Table~\ref{tab:key-options} summarizes the key options available at the question
% and answer levels depending on the question type. For the essay questions,
% please refer to section~\ref{subsubsect:essay}.
% 
% \begin{table}[tbp]
% \centering
% \caption{Options offered at the question and answer levels for each question type.}
% \label{tab:key-options}
% \small
% \begin{tabular}{*{14}{l}}
% \toprule
% & \multicolumn{10}{l}{Question} & \multicolumn{3}{l}{Answer}\\
% \cmidrule(lr){2-11}\cmidrule(lr){12-14}
% Question type & \rotatebox{90}{points} & 
% \rotatebox{90}{penalty} & \rotatebox{90}{feedback} & \rotatebox{90}{tags} & 
% \rotatebox{90}{shuffle} & \rotatebox{90}{numbering} & \rotatebox{90}{multiple} &
% \rotatebox{90}{usecase} & \rotatebox{90}{tolerance} &\rotatebox{90}{dd} &
% \rotatebox{90}{fraction} & \rotatebox{90}{feedback} & 
% \rotatebox{90}{tolerance}\\\cmidrule(lr){1-1}\cmidrule(lr){2-2}\cmidrule(lr){3-3}
% \cmidrule(lr){4-4}\cmidrule(lr){5-5}\cmidrule(lr){6-6}\cmidrule(lr){7-7}
% \cmidrule(lr){8-8}\cmidrule(lr){9-9}\cmidrule(lr){10-10}\cmidrule(lr){11-11}
% \cmidrule(lr){12-12}\cmidrule(lr){13-13}\cmidrule(lr){14-14}
% \href{https://docs.moodle.org/35/en/Multiple_Choice_question_type} 
% {Multichoice} & $\bullet$ & $\bullet$ & $\bullet$ & $\bullet$ & 
% $\bullet$ & $\bullet$ & $\bullet$ & & & & $\bullet$ & $\bullet$ \\
% \href{https://docs.moodle.org/35/en/Numerical_question_type}{Numerical}
% & $\bullet$ & $\bullet$ & $\bullet$ & $\bullet$ & & & 
% & & $\bullet$ & & $\bullet$ & $\bullet$ & $\bullet$ \\
% \href{https://docs.moodle.org/35/en/Short-Answer_question_type}{Short
% Answer} & $\bullet$ & $\bullet$ & $\bullet$ & $\bullet$ & & & 
% & $\bullet$ & & & $\bullet$ & $\bullet$ \\
% \href{https://docs.moodle.org/35/en/Matching_question_type}{Matching}
% & $\bullet$ & $\bullet$ & $\bullet$ & $\bullet$ & $\bullet$ & & 
% & & & $\bullet$ & $\bullet$ & \\
% \href{https://docs.moodle.org/35/en/Multiple_True/False_Question_Type}
% {True/False} & $\bullet$ & & $\bullet$ & $\bullet$ & & & 
% & & & & & $\bullet$ \\
% ^^A\href{https://docs.moodle.org/35/en/Essay_question_type}{Essay} & \\\hline%
% \href{https://docs.moodle.org/35/en/Embedded_Answers_(Cloze)_question_type}{Cloze}
% & $\bullet$ & $\bullet$ & $\bullet$ & $\bullet$ & & & 
% & & & & & \\\cmidrule(lr){1-1}
% \hspace{1em}Numerical & $\bullet$ & & & & & & 
% & &$\bullet$ & & $\bullet$ & $\bullet$ & $\bullet$ \\
% \hspace{1em}Short Answer & $\bullet$ & & & & & & 
% & $\bullet$ & & & $\bullet$ & $\bullet$ \\
% \hspace{1em}Multi (regular) & $\bullet$ & & & & $\bullet$ & & $\bullet$ 
% & & & & $\bullet$ & $\bullet$ \\
% \hspace{1em}Multi (horizontal)& $\bullet$ & & & & $\bullet$ & & $\bullet$ 
% & & & & $\bullet$ & $\bullet$ \\
% \hspace{1em}Multi (vertical)& $\bullet$ & & & & $\bullet$ & & 
% & & & & $\bullet$ & $\bullet$ \\
% \bottomrule
% \end{tabular}
% \end{table}
% 
% \section{Conversion to HTML}
% 
% Questions should be typed as usual for \LaTeX,
% including |\$| to obtain dollar signs, |$|'s or |\(|...|\)| for math shifts,
% |$$|'s or |\[|...|\]| for display math, et cetera.
% The package \textsf{moodle.sty} automatically converts
% this \LaTeX\ code into HTML for web display.
% 
% Table~\ref{tab:html} lists \LaTeX\ macros, commands, and environments that 
% are specifically converted to HTML.
% \begin{table}[tbp]
% \centering
% \caption{Conversion of \LaTeX\ material to HTML.}
% \label{tab:html}
% \begin{tabular}{llll}
% \toprule
% \multicolumn{2}{l}{Macros} & Commands & Environnments \\
% \cmidrule(lr){1-2}\cmidrule(lr){3-3}\cmidrule(lr){4-4}
% |~| & |\#| &|\emph{}| & |\begin{center}|\\
% |\$| & |\&| &|\textbf{}| & |\begin{enumerate}|\\
% |\\| & |\par| &|\textit{}| & |\begin{itemize}|\\
% |\&| & |\S| & |\texttt{}| & |\begin{tikzpicture}|\\
% |\{| & |\}| & |\textsc{}| & \\
% |\|\textvisiblespace & |\relax| & |\underline{}| &\\
% |\,| & |\thinspace| & |\textsuperscript{}| & \\
% |\dots| & |\ldots| & |\textsubscript{}| & \\
% |\euro| & |\texteuro| & |\url{}| & \\
% |\TeX| & |\LaTeX|& |\href{}{}| & \\
% |\_| & |\textbackslash| & |\tikz[]{}| & \\
% & & |\includegraphics[]{}| & \\
% & & |\verbatiminput{}| & \\
% & & |\VerbatimInput[]{}| & \\
% & & |\LVerbatimInput[]{}| & \\
% & & |\BVerbatimInput[]{}| & \\
% & & |\inputminted[]{}{}| & \\
% \bottomrule
% \end{tabular}
% \end{table}
% Single and double quotation marks, french quotation marks and the diacritical 
% commands  |\^|, |\'|, |\`|, |\"|, |\~|, |\c|, |\u|, |\v| and |\H| are also handled,
% as are the characters |\aa|, |\ae|, |\oe|, |\o|, |\ss|, |\l|, and their 
% capitalizations. See Tables~\ref{tab:diacritical}, \ref{tab:ligatures}, and
% \ref{tab:other} for more details.
%   
% In addition, |<| and |>| will be converted to |&lt;| and |&gt;| \emph{within math mode only}.
% If they should be typed outside of math mode, they will be passed as typed to
% the HTML, and probably interpreted by students' browsers as HTML tags
% or other unpredicated results.
% 
% Be aware that \emph{\textsf{moodle} does not know how to convert any 
% other \TeX\ or \LaTeX\ commands to HTML.}
% If other sequences are used, they may be passed verbatim to the XML file
% or may cause unpredicted results. The |\htmlregister| command lets you specify
% the macros that must be expanded in the XML file. It works only when no
% optional argument is used.
% 
% \DescribeMacro{\moodleregisternewcommands} 
% It can be cumbersome to record individually a list of macros for expansion.
% Calling |\moodleregisternewcommands| triggers the automatic
% expansion of macros defined subsequently using |\newcommand|,
% |\renewcommand|, |\providecommand|, or their starred variants.
% Again, this works only if the macros are defined \emph{without} optional
% argument.
% 
% If you think of another \LaTeX\ command that should be changed to an HTML equivalent, 
% please contact the author at \url{anders.hendrickson@snc.edu}
% so that it may be added to a future revision of the package.
% 
% \section{Graphics}
% The \textsf{moodle} package can handle two kinds of graphics seamlessly.
% External graphics files may be included with the |\includegraphics| command
% from the |graphicx| package, and graphics may be generated internally using \TikZ. 
% In either case, the graphics will be embedded in base-64 encoding directly within
% the Moodle~XML produced.  This prevents the hassle of managing separate
% graphics files on the Moodle server, as Moodle will store the picture 
% within the question in the question bank.
% 
% \subsection{Default \texttt{includegraphics}}
% \DescribeMacro{\includegraphics}\DescribeMacro{height}\DescribeMacro{width}
% When using |\includegraphics|, the only options currently supported 
% are |height| and |width|.  Attempts to use other |\includegraphics| options,
% such as |scale| or |angle|, will affect the PDF but not the XML output.
% The dimensions set by |height| and |width| are \TeX\ dimensions such as \texttt{4\,in}
% or \texttt{2.3\,cm}.
% In order to prepare the image for web viewing, this package converts those 
% dimensions to pixels using a default of 
% 103 pixels per inch.\footnote{This
%   number was selected because an image with 
%   |<IMG HEIGHT=103 WIDTH=103 SRC="...">| showed up as almost exactly 1 inch 
%   tall and 1 inch wide on several of this author's devices and browsers 
%   as of January 2016.}
% \DescribeMacro{ppi}
% That value may be changed by setting the |ppi| key (e.g., |ppi=72|); 
% this is probably best done for the entire document with a |\moodleset| command,
% rather than image-by-image.
% \DescribeMacro{\graphicspath}
% You can use |\graphicspath{{|\emph{path}|}}| to specify a directory where the
% pictures to be included are located.
%
% \subsection{\TikZ\ Pictures}
% When \TikZ\ is loaded and used to define pictures, \textsf{moodle} invokes
% the |external| \TikZ\ library, so that each |tikzpicture| environment is compiled
% to a freestanding PDF file.
%
% \subsection{Package Option \texttt{tikz}}
% \DescribeMacro{tikz}
% The \textsf{moodle} package admits a \texttt{tikz} option which has the following effects:
% \begin{itemize}
% \item the package \texttt{tikz} is loaded.
% \item \texttt{includegraphics} is embedded in a \TikZ\ picture. Consequences are that
%   \begin{itemize}
%     \item the pictures encoded in the XML file are resampled. This prevents encoding
%           images at a higher resolution than rendered by Moodle.
%     \item the full set of \texttt{includegraphics} options is accessible,
% e.g.~|scale=.5|, |angle=90|, or |width=.2\textwidth|.
%   \end{itemize}
% \item \DescribeMacro{\embedaspict} a macro |\embedaspict{...}| is provided for the
% inclusion of inline \LaTeX\ material as images. This can serve as a workaround to
% overcome limitations of this package---like the conversion of tabulars to HTML---
% or limitations of Moodle itself.
% For the definition of this macro, the package \texttt{varwidth} is loaded.
% \item optimizations of the \TikZ-external library are disabled. Compilation might get
% sensibly slower.
% \end{itemize}
%
% \subsection{External Tools}
% The mechanisms used for handling graphics are somewhat fragile and rely upon
% three free external programs.
% \begin{enumerate}
%   \item GhostScript (\url{www.ghostscript.com}) is used to convert the PDF output
%         from \TikZ\ into a PNG raster graphics file.
%         The default command line is presumed to be |gswin64c.exe| 
%         (if |\ifwindows| from the |ifplatform| package returns true)
%         or |gs| (if |\ifwindows| returns false).
%         If your system requires a different command line to invoke Ghostscript,
%         \DescribeMacro{\ghostscriptcommand}
%         you may change it by invoking:
%            \begin{quote}
%             |\ghostscriptcommand|\marg{executable filename}
%            \end{quote}
%   \item When external graphics files such as JPG or GIF are included,
%         the open-source ImageMagick software (\url{www.imagemagick.org})
%         converts each file to PNG format.
%         The command line for ImageMagick is the nondescript word |convert|,
%         \DescribeMacro{\imagemagickcommand}
%         but may be changed by invoking |\imagemagickcommand|\marg{executable filename}.
%   \item OptiPNG (\url{http://optipng.sourceforge.net/}) is used to optimize the PNG images.
%         The command line is presumed to be |optipng|, but can be changed with
%         \DescribeMacro{\optipngcommand}
%         |\optipngcommand|\marg{executable filename}.
% \end{enumerate}
% 
% Please note the following vital points to make the graphics handling work:
% \begin{itemize}
%   \item As of now, graphics are only supported when compiling directly to a PDF
%         with |pdflatex|.  Including PS graphics or using \TikZ\ with the DVI$\to$PS workflow is not
%         yet supported.
%   \item Filenames should not contains spaces or, under windows, special characters like |_| or |\|.
%   \item You must have Ghostscript, ImageMagick, and OptiPNG installed on your system
%         to fully use the graphics-handling capabilities of \textsf{moodle}.
%   \item \LaTeX\ must be able to call system commands; that is, |\write18| must be enabled.
%         For Mik\TeX, this means adding |--enable-write18| to the command line of |pdflatex|;
%         for \TeX Live, this means adding |--shell-escape=true|.
%   \item Users of the |circuitikz| package must enclose their circuits in the |tikzpicture|
%         environment instead of |circuitikz|. That is required,
%         as of \TikZ\ 2.1, by the |external| library.
% \end{itemize}
%
% \section{Verbatim Code}
% Because, for HTML translation, \textsf{moodle} parses the body of questions, the use of 
% verbatim code results in compilation errors. This is why the use of |\verb|,
% |\begin{verbatim}| and other standard utilities is not supported.
%
% However, using the following three utilities, verbatim code can be imported from an external file:
% \begin{enumerate}
%   \item \DescribeMacro{\verbatiminput}|\verbatiminput|\marg{filename} from the \textsf{verbatim}
%         package inserts verbatim code in both the PDF and the XML for moodle, without fancy additions.
%   \item \DescribeMacro{\VerbatimInput}The macro |\VerbatimInput|\marg{options}\marg{filename} from
%         \textsf{fancyvrb} or \textsf{fvextra} does more, with several options and settings offered
%         (see below).
%         The variants \DescribeMacro{\BVerbatimInput}|\VBerbatimInput| and
%         \DescribeMacro{\LVerbatimInput}|\LBerbatimInput| are also supported, with identical
%         effect on the XML output.
%         The variants with a star are unsupported and result in errors when used.
%   \item \DescribeMacro{\inputminted}On top of that |\inputminted|\oarg{options}\marg{lang}
%         \marg{filename}from the \textsf{minted} package offers syntaxic highlighting tailored to the
%         specified language.
% \end{enumerate}
% The \textsf{moodle} package handles these three commands to pass the code in the output XML.
%
% With |\VerbatimInput| and |\inputminted|, the options that are taken care of for XML
% generation are listed in Table~\ref{tab:verbatim-options}. Using |\fvset|\marg{key=value,...},
% options can be set globally. Equivalently, with \textsf{minted},
% |\setminted|\oarg{lang}\marg{key=value,...} is available.
%
% \begin{table}[tbp]
% \centering
% \begin{threeparttable}[b]
% \caption{Options and corresponding values considered for XML generation of verbatim material
% with \texttt{VerbatimInput} and \texttt{inputminted}.}
% \label{tab:verbatim-options}
% \begin{tabular}{ll}
% \toprule
% Option keys & Possible values\\\cmidrule(lr){1-1}\cmidrule(lr){2-2}
% ^^A\texttt{commentchar} & \meta{character}\\
% \texttt{gobble} & \meta{integer}\\
% \texttt{autogobble}\tnote{1} & \texttt{true} or \texttt{false}\\
% \texttt{tabsize} & \meta{integer}\\
% \texttt{numbers} & \texttt{none}, \texttt{left}, \texttt{right}, or \texttt{both}\tnote{2}\\
% \texttt{firstnumber} & \texttt{auto}, \texttt{last}, or \meta{integer}\\
% \texttt{firstline} & \meta{integer}\\
% \texttt{lastline} & \meta{integer}\\
% \texttt{numberblanklines} & \texttt{true} or \texttt{false}\\
% \texttt{highlightlines}\tnote{2} & \marg{coma-separated list of integers or ranges}\\
% \texttt{style}\tnote{1} & \meta{string}\\
% \bottomrule
% \end{tabular}
%\begin{tablenotes}
%\item[1] \texttt{autogobble}, \texttt{numbers=both}, and \texttt{style} are offered only by \textsf{minted}.
%\item[2] line highlighting is offered only with \textsf{fvextra} or \textsf{minted} loaded.
%\end{tablenotes}
%\end{threeparttable}
%\end{table}
%
% In order to define the verbatim code from the \LaTeX\ document itself, it is still possible
% to use, outside the scope of the \textsf{moodle} questions, the environments |filecontents*|  (from the
% \textsf{filecontents} package or \LaTeX\ kernel itself since 2019) or |VerbatimOut|
% (from the \textsf{fancyvrb} and \textsf{fvextra} packages).
% 
% When code decorated with left-side line numbers is placed in question items, the output PDF could
% show a collision between numbers of the item and the first line. To avoid this, |\LVerbatimInput| or
% |\BVerbatimInput| can be used. Instead, when \textsf{minted} is used, the ``left-right'' mode can be
% enforced with the \LaTeX\ command:
% \begin{quote}
%   |\RecustomVerbatimEnvironment{Verbatim}{LVerbatim}{}|
%  \end{quote}
%
% When using utilies from \textsf{fancyvrb}, \textsf{fvextra}, or \textsf{minted}, \textsf{moodle} sets
% framing options for the display of code in the output PDF:
% \begin{quote}
%   |\fvset{frame=lines,label={[Beginning of code]End of code},|\\
%   |       framesep=3mm,numbersep=9pt}|
% \end{quote}
% These settings can be overidden using |\fvset| after the preamble.
%
% \section{Known Limitations and Call for Bug Reports}
% Table~\ref{tab:support-limitations} lists some different features supported, limitations, and
% bugs.
%
%\begin{table}[tbp]
%\centering
%\begin{threeparttable}[b]
%\caption{Content enrichment (pictures, equations) support after XML import in Moodle v3.5.7,
% depending on the question type.}
%\label{tab:support-limitations}
%\begin{tabular}{lccc}
% \toprule
% & \multicolumn{3}{l}{XML rendering in\dots}\\\cmidrule(lr){2-4}
%Question type & Question & Answer & Feedback\\\cmidrule(lr){1-1}\cmidrule(lr){2-2}
%\cmidrule(lr){3-3}\cmidrule(lr){4-4}
% \href{https://docs.moodle.org/31/en/Multiple_Choice_question_type}{Multichoice}
%& yes & yes & yes \\
% \href{https://docs.moodle.org/31/en/Numerical_question_type}{Numerical}
%& yes & no\tnote{1} & yes \\
% \href{https://docs.moodle.org/31/en/Short-Answer_question_type}{Short Answer}
% & yes & no\tnote{1} & yes \\
% Matching (\href{https://docs.moodle.org/31/en/Matching_question_type}{std})
%& yes & no\tnote{2} & no\tnote{3} \\
% Matching (\href{https://docs.moodle.org/31/en/Drag_and_drop_matching_question_type}{dd})
% & yes & yes\tnote{4} & no\tnote{3} \\
% \href{https://docs.moodle.org/31/en/Essay_question_type}{Essay}
%& yes & yes\tnote{5,6} & yes\tnote{5} \\\cmidrule(lr){1-1}
%\href{https://docs.moodle.org/31/en/Embedded_Answers_(Cloze)_question_type}{Cloze} & yes & & \\
%\hspace{1em}Numerical & yes & no\tnote{1} & yes \\
%\hspace{1em}Short Answer & yes & no\tnote{1} & yes\tnote{7} \\
%\hspace{1em}Multi (regular) & yes & no\tnote{2} & yes\tnote{7} \\
%\hspace{1em}Multi (horizontal) & yes & yes & yes \\
%\hspace{1em}Multi (vertical) & yes & yes & yes \\\bottomrule
%\end{tabular}
%\begin{tablenotes}
%\item[1] Moodle prompts the student for an answer and then compares it to the 
%solutions provided. This is text-only.
%\item[2] Moodle uses a dropdown list to let one choose among the possible 
%answers. This forbids either picture inclusion and \LaTeX\ rendering.
%\item[3] Not supported by Moodle (in this context, answer-specific feedback 
%represents lots of possible combinations).
%\item[4] The drag-and-drop-matching plugin seems broken before version 1.6 
%20190409. Moodle's XML import fails with a \textsf{dmlwriteexception} when 
%the field content exceeds few hundreds characters. This prevents the inclusion 
%of most base64 images and maybe some complicated equations.
%\item[5] For this question type and in the context of XML generation, the 
%Answer column represents the ``template" while the Feedback column represents 
%the ``notes for the grader". Obviously, the grading process is not automatic 
%and there is no answer-specific feedback.
%\item[6] Picture and \LaTeX\ rendering could be done, but only after 
%submission and only if the keyval ``response format" is set to ``html".
%\item[7] Moodle only reveals the feedback when hovering the checkmark or X 
%mark with the mouse.
%\end{tablenotes}
%\end{threeparttable}
%\end{table}
%
% Tables~\ref{tab:diacritical}, \ref{tab:ligatures}, and \ref{tab:other} describe the current
% support for some special characters, accents and other diacritical marks.
%
%\begin{table}[tbp]
% \centering
% \caption{Support for diacritical marks 
% in a UTF8-coded \TeX~document compiled with (pdf)\LaTeX\ (packages 
% \texttt{inputenc} with option \texttt{utf8} and \texttt{fontenc} with option \texttt{T1})
% or \XeLaTeX\ (package \texttt{fontspec}).}
% \label{tab:diacritical}
% \begin{threeparttable}[t]
% \begin{tabular}{*{6}{l}}
% \toprule
% \multicolumn{4}{l}{Input type} & \multicolumn{2}{l}{Engine support}\\
% \cmidrule(lr){1-4}\cmidrule(lr){5-6}
% \multicolumn{2}{l}{Unicode} & \multicolumn{2}{l}{\LaTeX} & \XeLaTeX & (pdf)\LaTeX \\
% \cmidrule(lr){1-2}\cmidrule(lr){3-4}\cmidrule(lr){5-5}\cmidrule(lr){6-6}
% \aa & \AA & \verb|\aa| & \verb|\AA| & Unicode and \LaTeX & Unicode and \LaTeX \\
% \`a & \`A & \verb|\`a| & \verb|\`A| & Unicode and \LaTeX & Unicode and \LaTeX \\
% \^a & \^A & \verb|\^a| & \verb|\^A| & Unicode and \LaTeX & Unicode and \LaTeX \\
% \~a & \~A & \verb|\~a| & \verb|\~A| & Unicode and \LaTeX & Unicode and \LaTeX \\
% \'e & \'E & \verb|\'e| & \verb|\'E| & Unicode and \LaTeX & Unicode and \LaTeX \\
% \`e & \`E & \verb|\`e| & \verb|\`E| & Unicode and \LaTeX & Unicode and \LaTeX \\
% \"e & \"E & \verb|\"e| & \verb|\"E| & Unicode and \LaTeX & Unicode and \LaTeX \\
% \^e & \^E & \verb|\^e| & \verb|\^E| & Unicode and \LaTeX & Unicode and \LaTeX \\
% \^i & \^I & \verb|\^i| & \verb|\^I| & Unicode and \LaTeX & Unicode and \LaTeX \\
% \"i & \"I & \verb|\"i| & \verb|\"I| & Unicode and \LaTeX & Unicode and \LaTeX \\
% \~n & \~N & \verb|\~n| & \verb|\~N| & Unicode and \LaTeX & Unicode and \LaTeX \\
% \~o & \~O & \verb|\~o| & \verb|\~O| & Unicode and \LaTeX & Unicode and \LaTeX \\
% \"o & \"O & \verb|\"o| & \verb|\"O| & Unicode and \LaTeX & Unicode and \LaTeX \\
% \^o & \^O & \verb|\^o| & \verb|\^O| & Unicode and \LaTeX & Unicode and \LaTeX \\
% \`u & \`U & \verb|\`u| & \verb|\`U| & Unicode and \LaTeX & Unicode and \LaTeX \\
% \"u & \"U & \verb|\"u| & \verb|\"U| & Unicode and \LaTeX & Unicode and \LaTeX \\
% \^u & \^U & \verb|\^u| & \verb|\^U| & Unicode and \LaTeX & Unicode and \LaTeX \\
% \"y & \"Y & \verb|\"y| & \verb|\"Y| & Unicode and \LaTeX & Unicode and \LaTeX \\
% \c{c} & \c{C} & \verb|\c{c}| & \verb|\c{C}| & Unicode and \LaTeX & Unicode and \LaTeX \\
% \c{s} & \c{S} & \verb|\c{s}| & \verb|\c{S}| & Unicode and \LaTeX & \LaTeX\ only\\
% \c{t} & \c{T} & \verb|\c{t}| & \verb|\c{T}| & Unicode and \LaTeX & \LaTeX\ only\\
% \H{o} & \H{O} & \verb|\H{o}| & \verb|\H{O}| & Unicode and \LaTeX & \LaTeX\ only\\
% \H{u} & \H{U} & \verb|\H{u}| & \verb|\H{U}| & Unicode and \LaTeX & \LaTeX\ only\\
% \u{a} & \u{A} & \verb|\u{a}| & \verb|\u{A}| & Unicode and \LaTeX & \LaTeX\ only\\
% \u{e} & \u{E} & \verb|\u{e}| & \verb|\u{E}| & Unicode and \LaTeX & \LaTeX\ only\\
% \u{g} & \u{G} & \verb|\u{g}| & \verb|\u{G}| & Unicode and \LaTeX & \LaTeX\ only\\
% \u{\i} & \u{I} & \verb|\u{\i}| & \verb|\u{I}| & Unicode and \LaTeX & \LaTeX\ only\tnote{1}\\
% \u{o} & \u{O} & \verb|\u{o}| & \verb|\u{O}| & Unicode and \LaTeX & \LaTeX\ only\\
% \v{c} & \v{C} & \verb|\v{c}| & \verb|\v{C}| & Unicode and \LaTeX & \LaTeX\ only\\
% \v{d} & \v{D} & \verb|\v{d}| & \verb|\v{D}| & Unicode and \LaTeX & \LaTeX\ only\\
% \v{e} & \v{E} & \verb|\v{e}| & \verb|\v{E}| & Unicode and \LaTeX & \LaTeX\ only\\
% \v{l} & \v{L} & \verb|\v{l}| & \verb|\v{L}| & Unicode and \LaTeX & \LaTeX\ only\\
% \v{n} & \v{N} & \verb|\v{n}| & \verb|\v{N}| & Unicode and \LaTeX & \LaTeX\ only\\
% \v{r} & \v{R} & \verb|\v{r}| & \verb|\v{R}| & Unicode and \LaTeX & \LaTeX\ only\\
% \v{s} & \v{S} & \verb|\v{s}| & \verb|\v{S}| & Unicode and \LaTeX & \LaTeX\ only\\
% \v{t} & \v{T} & \verb|\v{t}| & \verb|\v{T}| & Unicode and \LaTeX & \LaTeX\ only\\
% \v{z} & \v{Z} & \verb|\v{z}| & \verb|\v{Z}| & Unicode and \LaTeX & \LaTeX\ only\\
% \bottomrule
% \end{tabular}
% \begin{tablenotes}
%   \item[1] \XeLaTeX\ renders correctly |\u{i}|, that is, without a superscript dot.
%            Instead, with (pdf)\LaTeX\ the rendering of |\u{i}| is flawed by the
%            superposition of the superscript dot and the breve diacritical mark.
%            Both engines render |\u{\i}| as expected.
% \end{tablenotes}
% \end{threeparttable}
% \end{table}
%
%\begin{table}[tbp]
% \centering
% \begin{threeparttable}[t]
% \caption{Support for ligatures in a UTF8-coded \TeX~document compiled with (pdf)\LaTeX\ (packages 
% \texttt{inputenc} with option \texttt{utf8}, \texttt{fontenc} with option \texttt{T1})
% or \XeLaTeX\ (package \texttt{fontspec}).}
% \label{tab:ligatures}
% \begin{tabular}{llllll}
% \toprule
% \multicolumn{4}{l}{Input type} & \multicolumn{2}{l}{Engine support}\\
% \cmidrule(lr){1-4}\cmidrule(lr){5-6}
% \multicolumn{2}{l}{Unicode} & \multicolumn{2}{l}{\LaTeX} & \XeLaTeX & (pdf)\LaTeX \\
% \cmidrule(lr){1-2}\cmidrule(lr){3-4}\cmidrule(lr){5-6}\cmidrule(lr){6-6}
% \ae & \AE & \verb|\ae| & \verb|\AE| & Unicode and \LaTeX & \LaTeX\ only\\
% \oe & \OE & \verb|\oe| & \verb|\OE| & Unicode and \LaTeX & \LaTeX\ only\\
% \ss & \SS & \verb|\ss| & \verb|\SS| & Unicode and \LaTeX\tnote{1} & \LaTeX\ only\tnote{2}\\
% \bottomrule
% \end{tabular}
% \begin{tablenotes}[b]
%   \item[1] the Libertine font, used in this documentation and available for instance via the
%            package |libertine|, defines the glyph \SS. Most fonts do not define this glyph.
%   \item[2] \LaTeX\ defines the \verb|\SS| macro but (pdf)\LaTeX\ renders it as a doubled capital S.
% \end{tablenotes}
% \end{threeparttable}
% \end{table}
%
%\begin{table}[tbp]
% \centering
% \begin{threeparttable}[b]
% \caption{Support for other glyphs and punctuation marks in a UTF8-coded \TeX~document compiled
% with (pdf)\LaTeX\ (packages \texttt{inputenc} with option \texttt{utf8}, \texttt{fontenc} with
% option \texttt{T1}) or \XeLaTeX\ (package \texttt{fontspec}).}
% \label{tab:other}
% \begin{tabular}{llllll}
% \toprule
% \multicolumn{4}{l}{Input type} & \multicolumn{2}{l}{Engine support}\\
% \cmidrule(lr){1-4}\cmidrule(lr){5-6}
% \multicolumn{2}{l}{Unicode} & \multicolumn{2}{l}{\LaTeX} & \XeLaTeX & (pdf)\LaTeX \\
% \cmidrule(lr){1-2}\cmidrule(lr){3-4}\cmidrule(lr){5-5}\cmidrule(lr){6-6}
% \l & \L & \verb|\l| & \verb|\L| & Unicode and \LaTeX & \LaTeX\ only\\
% \o & \O & \verb|\o| & \verb|\O| & Unicode and \LaTeX & \LaTeX\ only\\
% « & » & \verb|\guillemotleft| & \verb|\guillemotright|\tnote{1} & Unicode and \LaTeX & \LaTeX\ only\\
% \bottomrule
% \end{tabular}
% \begin{tablenotes}
%   \item[1] the package |babel| loaded with option |french| defines |\og| and |\fg| for
%            the same symbols. These are also supported by \textsf{moodle}.
% \end{tablenotes}
% \end{threeparttable}
% \end{table}
% Some features of Moodle quizzes have not yet been implemented in \textsf{moodle}.
% Here is a non-exhaustive list.
% \begin{itemize}
%   \item Moodle's feature of designating feedback for correct, 
%         partially correct, and incorrect answers.
%   \item Calculated questions; that is, automatically generated numerical questions 
%         using randomly chosen numbers.
%   \item Hints
%   \item Multiple keywords (tags) for questions
%   \item So-called ``description'' questions.
% \end{itemize}
% I have used Version 0.5 for one semester's teaching,
% but if other users adopt this package, I fully expect them to find bugs.
% \emph{Please} send all bugs you find to \url{anders.hendrickson@snc.edu},
% so that I can fix them for subsequent versions.
% 
% \section{Compatibility}
% This package has been written for and tested with the implementation of 
% Moodle 2.9 run by Moodlerooms for St.~Norbert College in January 2016.
% Future versions of this package will probably include some support for
% specifying your version of Moodle in the |.tex| file to help ensure compatibility.
% 
% As the ultimate purpose of this package is the generation of XML files,
% future versions of \textsf{moodle} will attempt to maintain backwards 
% compatability with earlier versions of regarding the XML output, apart from
% bug fixes.  
% Backwards compatibility of the PDF output is not yet guaranteed, however,
% in case the author or users discover better ways for the PDF to display 
% the underlying XML data to be proofread.
% 
% In other words, compiling your current |.tex| file with a future version 
% of \textsf{moodle} should produce the same XML file it does now 
% (apart from bug fixes),
% but it might produce a more informative, and hence different,
% PDF output.
% 
% 
% \section{Unrelated Tip: Quality of Moodle \TeX\ Images}
% This has nothing to do with \textsf{moodle}, but is a Frequently Asked Question
% in is own right.
% On some servers, at least, Moodle's default ``\TeX\ Filter'' for 
% displaying mathematical notation is of abysmally poor quality, rending mathematics
% as low-resolution PNG's.  One solution that has worked for me is to go 
% to ``Course Administration $\to$ Filters,'' turn ``\TeX\ Notation'' \emph{off},
% but turn ``MathJax'' \emph{on}.  This forces \TeX\ code to be rendered by MathJax
% instead of Moodle, producing much higher-quality results.
% 
% \changes{v0.5}{2016/01/05}{Initial version}
% \changes{v0.6}{2019/02/18}{Bux-fixing release}
% \changes{v0.7}{2020/07/09}{Feature extensions}
% \begin{changelog}[title={Version History},author={Matthieu Guerquin-Kern}]
%   \begin{version} ^^A[version=0.8,date=\today]
%     \added
%       \item Command to trigger the automatic recording of new commands.
%       \item Mechanism to match |fraction| key to values accepted by Moodle.
%       \item A |fractiontol| key to control the tolerance in this mechanism.
%       \item Support for |\_| and |\textbackslash|.
%     \changed
%       \item The package \texttt{iftex} is now required.
%       \item An error is thrown when |fraction| is set to an invalid value.
%     \fixed
%       \item Closing braces escaped in cloze subquestions outside math environment.
%   \end{version}
%   \begin{version}[version=0.7,date=2020-09-06]
%     \added
%       \item Support for inclusion of verbatim code.
%       \item Package option \texttt{tikz}.
%       \item Support for |\"Y| and |\"y|.
%       \item New commands converted to XML.
%       \item Adding a stamp comment in XML, package option offered to disable
%             this behavior.
%       \item Support for the |\tikz| command.
%       \item A different directory can be specified for picture inclusion.
%       \item Warn user of the \texttt{babel} package set for french that autospacing
%             must be deactivated.
%       \item Square bracket math delimiters are recognized and converted properly.
%       \item Support of breve and caron diacritical marks.
%     \changed
%       \item In multi with multiple answers allowed, choosing all options no longer
%             results in a good grade. An automatic penalty mechanism is introduced.
%             Can be overridden by manually setting fractions.
%     \removed
%       \item Irrelevant \texttt{penalty} tag in cloze subquestions.
%     \fixed
%       \item Non-integer fractions can now be specified in cloze subquestions.
%       \item Signifiantly squeeze PNG images size by skipping ancillary data.
%       \item Enumerate or itemize environment can now be nested in question items.
%       \item Several pictures can be included in a question without being mixed
%             in the XML file.
%       \item management and rendering of fraction in questions.
%       \item Correctly handling a \LaTeX\ starting the last item of a question.
%       \item Closing braces escaped in cloze subquestions. This allows \LaTeX\
%             equations or images to be included.
%       \item Image inclusion with MacOS.
%   \end{version}
%   \begin{version}[version=0.6b,date=2019-11-27]
%     \added
%       \item New package options to set section or subsection at the quiz level.
%       \item True/False question type is now supported.
%       \item Moodle tags can now be specified for questions (and rendered in PDF
%             as well).
%       \item In cloze questions, the \texttt{multiresponse} subquestion type is
%             now supported.
%     \removed
%       \item External dependency on \texttt{OpenSSL}.
%       \item Irrelevant tags were written in XML for matching questions.
%     \fixed
%       \item \TikZ\ externalization now works when using Xe\LaTeX.
%       \item It is now possible to set points manually among several correct
%             answers in multichoice questions.
%       \item General feedbacks can now contain backslashes.
%       \item Several quizzes can now be defined in a single source file, each
%             specifying a category for Moodle's question bank.
%       \item Correct encoding information in now written in XML depending on
%             the \LaTeX\ compiler used.
%   \end{version}
%   \begin{version}[version=0.6a,date=2019-06-21]
%     \added
%       \item Xe\LaTeX\ is now recommended when using UT8-encoded sources (support
%             of accents).
%       \item Feedbacks are now displayed in the PDF file produced.
%       \item The \texttt{optipng} utility is used (and required) to reduce the size
%             of images embedded in the XML file.
%       \item Question options and settings are now displayed in the PDF file
%       \item Supporting more \LaTeX\ macros for symbols and accents (mostly
%             diacritical marks and ligatures).
%       \item Introduce shuffle options in cloze-multi subquestions.
%       \item Package option \texttt{final}.
%     \changed
%       \item In draft mode, \TikZ\ externalization in no longer triggered.
%     \fixed
%       \item In the different question types, the feedback fields are now converted
%             for HTML allowing \LaTeX\ equation and images.
%       \item Documentation improvements (limitations and previously undocumented
%             features).
%   \end{version}
%   \shortversion{version=0.5,date=2016-01-05,simple,changes=Initial version,author={Anders O.F. Hendrickson}}
% \end{changelog}
%
% \StopEventually{}
%
% \section{Implementation}
% \subsection{Packages, Options, and Utilities}
%    \begin{macrocode}
\newif\ifmoodle@draftmode
\newif\ifmoodle@stampmode
\newif\ifmoodle@tikz
\newif\ifmoodle@tikzloaded
\newif\ifmoodle@section
\newif\ifmoodle@subsection
\newif\ifmoodle@numbered

%%DECLARATION OF OPTIONS
\DeclareOption{draft}{\moodle@draftmodetrue}
\DeclareOption{final}{\moodle@draftmodefalse}
\DeclareOption{nostamp}{\moodle@stampmodefalse}
\DeclareOption{tikz}{\moodle@tikztrue}
\DeclareOption{section}{\moodle@sectiontrue\moodle@numberedtrue}
\DeclareOption{section*}{\moodle@sectiontrue\moodle@numberedfalse}
\DeclareOption{subsection}{\moodle@sectionfalse\moodle@numberedtrue}
\DeclareOption{subsection*}{\moodle@sectionfalse\moodle@numberedfalse}

\moodle@draftmodefalse
\moodle@stampmodetrue
\moodle@tikzfalse
\moodle@tikzloadedfalse
\moodle@subsectiontrue
\moodle@numberedfalse

\ProcessOptions

\RequirePackage{environ} %To be able to take environment body as a macro argument
\RequirePackage{xkeyval} %For key-handling
\RequirePackage{amssymb} %For \checkmark symbol
\RequirePackage{trimspaces} %To remove extra spaces from strings
\RequirePackage{etex}    %Expansion control, detokenization, etc.
\RequirePackage{etoolbox}%List management
\RequirePackage{xpatch}  %To patch commands easily in HTML mode
\RequirePackage{array}   %For formatting tables in the LaTeX mode of Clozes
\RequirePackage{ifplatform} % To choose Ghostscript commands
\@ifpackageloaded{iftex}{}{\RequirePackage{iftex}}
% iftex already required by recent versions of ifplatform. Needed to know:
%     1) whether we can convert output from PDF to PNG (ifpdf), 
%     2) when output pdf is latin1-encoded (ifpdf)
%     3) when output xml is utf8-encoded (if?tex)
\RequirePackage{fancybox} % Needed for fancy LaTeX tags
\RequirePackage{getitems} %To gather the header and items

\let\xa=\expandafter
\def\@star{*}%
\def\@hundred{100}%
\def\@fifty{50}%
\def\@moodle@empty{}%
\def\@relax{\relax}%
\def\@moodle@par{\par}%
%    \end{macrocode}
% As the package involves a fair bit of file processing,
% we automate the naming of auxiliary files.
%    \begin{macrocode}
\def\jobnamewithsuffixtomacro#1#2{%
  \filenamewithsuffixtomacro{#1}{\jobname}{#2}%
}
\def\@jn@quote{"}%
\def\filenamewithsuffixtomacro#1#2#3{%
  % #1 = macro to create
  % #2 = filename to add suffix to
  % #3 = suffix to add
  \edef\jn@suffix{#3}%
  \def\jn@macro{#1}%
  \xa\testforquote#2\@jn@rdelim
}
\def\testforquote#1#2\@jn@rdelim{%
  \def\jn@test@i{#1}%
  \ifx\jn@test@i\@jn@quote
    % Involves quotes
    \edef\jn@next{"\jn@stripquotes#1#2\jn@suffix"}%
  \else
    \edef\jn@next{#1#2\jn@suffix}%
  \fi
  \xa\xdef\jn@macro{\jn@next}%
}
\def\jn@stripquotes"#1"{#1}%


\jobnamewithsuffixtomacro{\outputfilename}{-moodle.xml}
%    \end{macrocode}
% Next, we create macros to open and close the Moodle XML file
% we will be writing.
%    \begin{macrocode}
\newwrite\moodle@outfile
\def\openmoodleout{%
  \immediate\openout\moodle@outfile=\outputfilename\relax
  \ifpdftex % pdflatex or latex
    \writetomoodle{<?xml version="1.0" encoding="iso-8859-1"?>}%
  \else
    \ifxetex % xetex
      \writetomoodle{<?xml version="1.0" encoding="UTF-8"?>}%
    \else % what shall we do?
      \writetomoodle{<?xml version="1.0" encoding="UTF-8"?>}%
      %\stop
    \fi
  \fi
  \ifmoodle@stampmode
    \def\moodle@stamp{This file was generated on \the\year-\two@digits\month-\two@digits\day}
    \ifpdftex % pdflatex or latex
      \ifpdf % pdflatex
        \g@addto@macro{\moodle@stamp}{ by pdfLaTeX }%
      \else % latex
        \g@addto@macro{\moodle@stamp}{ by LaTeX }%
      \fi
    \else
      \ifxetex % xetex
        \g@addto@macro{\moodle@stamp}{ by XeLaTeX }%
      \else
        \ifluatex % 
          \g@addto@macro{\moodle@stamp}{ LuaLaTeX }%
        \else
          \g@addto@macro{\moodle@stamp}{ a TeX engine }%
          \fi
      \fi
    \fi
    \writetomoodle{<!-- \moodle@stamp -->}%
    \def\moodle@stamp{running on \platformname}%
    \g@addto@macro{\moodle@stamp}{ with the package moodle v0.7 }%
    \writetomoodle{<!-- \moodle@stamp -->}%
  \fi
  \immediate\write\moodle@outfile{}%
  \writetomoodle{<quiz>}%
}%
\def\closemoodleout{%
  \writetomoodle{ }%
  \writetomoodle{</quiz>}%
  \immediate\closeout\moodle@outfile
}%
%    \end{macrocode}
% 
% To both make this |.sty| file and the XML output more readable,
% we create a mechanism for writing to the output file with indents.
% The macro |\calculateindent|\marg{$n$} globally defines 
% |\moodle@indent| to be a string of \meta{$n$} |\otherspace|'s.
%    \begin{macrocode}
\def\calculateindent#1{%
  \bgroup
    \count0=\number#1\relax
    \gdef\moodle@indent{}%
    \calculateindent@int
  \egroup
}%
\def\calculateindent@int{%
  \ifnum\count0>0\relax
    \g@addto@macro{\moodle@indent}{\otherspace}%
    \advance\count0 by -1\relax
    \expandafter
    \calculateindent@int
  \fi
}%
%    \end{macrocode}
% Now the command |\writetomoodle|\oarg{n}\marg{stuff} adds the line 
% ``\meta{stuff}'' to the XML file
% preceded by an indent of \meta{n} spaces.
% Note that this command |\edef|'s its second argument.
%    \begin{macrocode}
\newcommand\writetomoodle[2][0]{%
  \edef\test@ii{#2}%
  \ifnum#1=0\relax
    \immediate\write\moodle@outfile{\test@ii}%
  \else
    \calculateindent{#1}%
    \immediate\write\moodle@outfile{\moodle@indent\trim@pre@space{\test@ii}}%
  \fi
}%
%    \end{macrocode}
% We now create the main |quiz| environment that will contain
% the questions we write.  It outputs to XML as a |<question type="category">| tag.
%    \begin{macrocode}
\newenvironment{quiz}[2][]%
{
  \setkeys{moodle}{#1}%
  \@moodle@ifgeneratexml{%
    %\openmoodleout%
    \setcategory{#2}%
  }{}%
   \ifmoodle@section
     \ifmoodle@numbered
       \section{#2}%
     \else
       \section*{#2}%
     \fi
   \else
     \ifmoodle@numbered
       \subsection{#2}%
     \else
       \subsection*{#2}%
     \fi
   \fi
   \begin{enumerate}%
}{
  \end{enumerate}%
  %\@moodle@ifgeneratexml{\closemoodleout}{}
}

{\catcode`\$=12\catcode`\ =12%
\gdef\setcategory#1{%
\writetomoodle{ }%
\writetomoodle{<question type="category">}%
\writetomoodle{  <category>}%
\writetomoodle{    <text>$course$/top/#1</text>}%
\writetomoodle{  </category>}%
\writetomoodle{</question>}%
\writetomoodle{ }%
}}%
%    \end{macrocode}                       
% The next utility takes a single macro control sequence |#1|,
% and allows that macro's current value to persist
% after the next |\egroup|, |}|, or |\endgroup|.
%    \begin{macrocode}
\def\passvalueaftergroup#1{%
  \xa\xa\xa\gdef\xa\xa\csname moodle@remember@\string#1\endcsname\xa{\xa\def\xa#1\xa{#1}}%
  \xa\aftergroup\csname moodle@remember@\string#1\endcsname
}
%    \end{macrocode}                       
%
% \subsubsection{Main Switch: to create XML or not}
% 
%    \begin{macrocode}
\long\def\@moodle@ifgeneratexml#1#2{%
  % If we are generating XML, do #1; otherwise do #2.
  \tikzifexternalizing{%
    % This run of LaTeX is currently ONLY generating a Tikz image
    % to be saved in an external file.  We do NOT want to waste time
    % generating XML, and moreover trying to do so would cause errors
    % because of file dependencies.
    #2%
  }{%
    \ifmoodle@draftmode
      #2%
    \else
      #1%
    \fi
  }%
}
%    \end{macrocode}
% Now the macros |openmoodleout| and |closemoodleout| are triggered at Begin and End Document, respectively
%    \begin{macrocode}
\AfterEndPreamble{
  \@moodle@ifgeneratexml{%
    \openmoodleout%
  }{}%
}
\AtEndDocument{
  \@moodle@ifgeneratexml{%
    \closemoodleout%
  }{}%
}
%    \end{macrocode}
%
% \subsection{Key-Value Pairs for Quiz Questions}
% 
% The various options are set using key-value syntax of |xkeyval|.
%    \begin{macrocode}
\def\moodleset#1{\setkeys{moodle}{#1}}%
%    \end{macrocode}
% We first define some macros that will help us write other macros.
% Calling |\generate@moodle@write@code|\marg{name}|<|\meta{HTML tag}|>|\marg{text to write}
% creates a macro |\moodle@write|\meta{name}, taking no parameters,
% which writes the code |<|\meta{HTML tag}|>...</|\meta{HTML tag}|>|
% to the output XML file.
% 
% The ordinary version |\generate@moodle@write@code| passes its output text |#3|
% through the HTMLizer, producing HTML code, while the starred variant
% |\generate@moodle@write@code*| passes |#3| verbatim as text.
% 
% For example, 
% |\generate@moodle@write@code{excuse}<EXC>{\theexcuse}|
% would expand to
% \begin{Verbatim}[gobble=4,frame=single]
%   \gdef\moodle@writeexcuse{%
%     \xa\def\xa\test@iii\xa{\theexcuse}%
%     \ifx\test@iii\@moodle@empty
%       \writetomoodle[2]{  <EXC format="html"><text/></EXC>}%
%     \else
%       \xa\converttohtmlmacro\xa\moodle@htmltowrite\xa{\theexcuse}%
%       \writetomoodle[2]{  <EXC format="html">}%
%       \writetomoodle[4]{    <text><![CDATA[<p>\moodle@htmltowrite</p>]]></text>}%
%       \writetomoodle[2]{  </EXC>}%
%     \fi
%   }%
% \end{Verbatim}
% but 
% |\generate@moodle@write@code*{excuse}<EXC>{\theexcuse}|
% would expand only to
% \begin{Verbatim}[gobble=4,frame=single]
%   \gdef\moodle@writeexcuse{%
%      \writetomoodle[2]{  <EXC>\theexcuse</EXC>}%
%   }
% \end{Verbatim}
%    \begin{macrocode}
\def\generate@moodle@write@code{%
  \@ifnextchar*\generate@moodle@write@data\generate@moodle@write@html
}%

\def\generate@moodle@write@html#1<#2>#3{%
  % #1 = NAME for \moodle@writeNAME
  % #2 = HTML tag
  % #3 = what, exactly, to write
  \xa\gdef\csname moodle@write#1\endcsname{%
    \xa\def\xa\test@iii\xa{#3}%
    \ifx\test@iii\@moodle@empty
      \writetomoodle[2]{  <#2 format="html"><text/></#2>}%
    \else
      \xa\converttohtmlmacro\xa\moodle@htmltowrite\xa{#3}%
      \writetomoodle[2]{  <#2 format="html">}%
      \writetomoodle[4]{    <text><![CDATA[<p>\moodle@htmltowrite</p>]]></text>}%
      \writetomoodle[2]{  </#2>}%
    \fi
  }%
}%

\def\generate@moodle@write@html@noptag#1<#2>#3{%
	% No <P>..</P> introduced
	% #1 = NAME for \moodle@writeNAME
	% #2 = HTML tag
	% #3 = what, exactly, to write
	\xa\gdef\csname moodle@write#1\endcsname{%
		\xa\def\xa\test@iii\xa{#3}%
		\ifx\test@iii\@moodle@empty
		\writetomoodle[2]{  <#2 format="html"><text/></#2>}%
		\else
		\xa\converttohtmlmacro\xa\moodle@htmltowrite\xa{#3}%
		\writetomoodle[2]{  <#2 format="html">}%
		\writetomoodle[4]{    <text><![CDATA[\moodle@htmltowrite]]></text>}%
		\writetomoodle[2]{  </#2>}%
		\fi
	}%
}%

\def\generate@moodle@write@data*#1<#2>#3{%
  % #1 = NAME for \moodle@writeNAME
  % #2 = HTML tag
  % #3 = what, exactly, to write
  \xa\gdef\csname moodle@write#1\endcsname{%
    \writetomoodle[2]{  <#2>#3</#2>}%
  }%
}%

\def\generate@moodle@write@tags#1{%
  % #1 = NAME for \moodle@writeNAME
  % #3 = what, exactly, to write
  \xa\gdef\csname moodle@writetags\endcsname{%
    %\xa\xa\xdef\xa\xa\ds\xa\xa{Encountered '\string #1'}\show\ds
    \xa\xdef\xa\test@iii\xa{\moodle@tags}%
    \ifx\test@iii\@moodle@empty\relax\else
      \xa\converttohtmlmacro\xa\moodle@htmltowrite\xa{\moodle@tags}%
      \writetomoodle[2]{  <tags>}%
      \writetomoodle[4]{    <tag><text><![CDATA[\moodle@htmltowrite]]></text></tag>}%
      \writetomoodle[2]{  </tags>}%
    \fi
  }%
}%

%    \end{macrocode}
% \subsubsection{Keys for all question types}
%    \begin{macrocode}
%% QUESTIONNAME
      \define@cmdkey{moodle}[moodle@]{questionname}{}%
%      \gdef\moodle@writequestionname{%
%        \writetomoodle[2]{<name>}%
%        \writetomoodle[4]{  <text>\moodle@questionname</text>}%
%        \writetomoodle[2]{</name>}%
%      }%
%\generate@moodle@write@code{questionname}<name>{\moodle@questionname}%
\generate@moodle@write@html@noptag{questionname}<name>{\moodle@questionname}%

%% QUESTIONTEXT
      %I tried to use questiontext as a key, but it doesn't seem to work.
      %The trouble is that xkeyval has trouble parsing a key with a \par token followed by a comma within brackets,
      %like \setkeys{moodle}{questiontext={ABC\par [D,E]}}
      %It's not worth trying to fix.

      \long\def\questiontext#1{%
        %\converttohtmlmacro\myoutput{#1}%
        %\let\moodle@questiontext=\myoutput%
        \def\moodle@questiontext{#1}%
      }%
      \generate@moodle@write@code{questiontext}<questiontext>{\moodle@questiontext}%{%

%% PENALTY FOR WRONG ATTEMPT
      \define@cmdkey{moodle}[moodle@]{penalty}[0.10]{}%
      \generate@moodle@write@code*{penalty}<penalty>{\moodle@penalty}%
      \moodleset{penalty=0.10}%

%% FEEDBACK
      % Moodle allows for feedback tailored to each question,
      % and feedback tailored to each right or wrong answer.
      % We shall use the key 'feedback' to record both kinds of feedback,
      % relying on TeX's grouping mechanism to keep them apart.
      % When it comes time to write them to XML,
      % \moodle@writegeneralfeedback uses the HTML tag <generalfeedback>
      % whereas \moodle@writefeedback uses the tag <feedback>.
      % Note that the general feedback is NOT inherited by each answer!
      \define@cmdkey{moodle}[moodle@]{feedback}[]{}%
      \generate@moodle@write@code{generalfeedback}<generalfeedback>{\moodle@feedback}%
      \generate@moodle@write@code{feedback}<feedback>{\moodle@feedback}%
      \moodleset{feedback={}}%

%% DEFAULT GRADE
      %The next line creates \moodle@defaultgrade,
      %which is how many points the quiz question is worth.
      %Key calls like [default grade=7] set \moodle@defaultgrade.
      \define@cmdkey{moodle}[moodle@]{default grade}[1.0]{}%
      %Next, makes 'points' a synonym for 'default grade'
      \define@key{moodle}{points}[1.0]{\xa\def\csname moodle@default grade\endcsname{#1}} 
      \generate@moodle@write@code*{defaultgrade}<defaultgrade>{\csname moodle@default grade\endcsname}%
      \moodleset{default grade=1.0} %This sets the default.

%% HIDDEN
      \define@boolkey{moodle}[moodle@]{hidden}[true]{}%
      \generate@moodle@write@code*{hidden}<hidden>{\ifmoodle@hidden 1\else 0\fi}%
      \moodleset{hidden=false}%

\def\moodle@writecommondata{%
  \moodle@writequestionname%
  \moodle@writequestiontext%
  \moodle@writedefaultgrade%
  \moodle@writegeneralfeedback%
  \moodle@writepenalty%
  \moodle@writehidden%
}%

%% TAGS
      %The next line creates \moodle@tags,
      %which defines a "tag" (i.e., keyword) for the question.
      %Key calls like [tags={random}] set \moodle@tags.
      \define@cmdkey{moodle}[moodle@]{tags}[]{}%
      \generate@moodle@write@tags{\csname moodle@tags\endcsname}%
      \moodleset{tags}%
% TODO: handle multiple 'tags' for one question

%    \end{macrocode}
% \subsubsection{Keys for all answers}
%    \begin{macrocode}
%% FRACTION -- how much this answer is worth out of 100 percent
      \define@cmdkey{moodle}[moodle@]{fraction}[100]{}%
      %We do not create \moodle@writefraction, because the fraction occurs in
      %the XML within the answer tag, like <answer fraction="75">.
      \moodleset{fraction=100} %This sets the default.
%    \end{macrocode}
%    \begin{macrocode}
%% FRACTIONTOL -- the tolerance for fractions with respect to valid values
      \define@cmdkey{moodle}[moodle@]{fractiontol}[0.1]{}%
      \moodleset{fractiontol=0.1} %This sets the default.
%    \end{macrocode}
% \subsubsection{Keys for multiple choice questions}
%    \begin{macrocode}

%% SINGLE and MULTIPLE -- for multichoice, is there 1 right answer or more than 1?
      \define@boolkey{moodle}[moodle@]{single}[true]{}%
      \generate@moodle@write@code*{single}<single>{\ifmoodle@single true\else false\fi}%
      \moodleset{single=true}%
      %The key 'multiple' is an antonym to 'single'.
      \define@boolkey{moodle}[moodle@]{multiple}[true]{\ifmoodle@multiple\moodle@singlefalse\else\moodle@singletrue\fi}%

%% SHUFFLE ANSWERS
      \define@boolkey{moodle}[moodle@]{shuffle}[true]{}%
      \generate@moodle@write@code*{shuffle}<shuffleanswers>{\ifmoodle@shuffle 1\else 0\fi}%
      \moodleset{shuffle=true}%

%% TODO: CORRECTFEEDBACK
%% TODO: PARTIALLYCORRECTFEEDBACK
%% TODO: INCORRECTFEEDBACK
%% TODO: NUMCORRECT key

%% NUMBERING -- for numbering of multichoice questions
      \define@choicekey{moodle}{numbering}%
                       {alpha,alph,Alpha,Alph,arabic,roman,Roman,%
                        abc,ABCD,123,iii,IIII,none}[abc]{%
                        \def\moodle@numbering{#1}%
                        \def\test@@i{#1}%
                        \ifx\test@@i\@moodle@alpha
                          \def\moodle@numbering{abc}\fi
                        \ifx\test@@i\@moodle@alph
                          \def\moodle@numbering{abc}\fi
                        \ifx\test@@i\@moodle@Alpha
                          \def\moodle@numbering{ABCD}\fi
                        \ifx\test@@i\@moodle@Alph
                          \def\moodle@numbering{ABCD}\fi
                        \ifx\test@@i\@moodle@arabic
                          \def\moodle@numbering{123}\fi
                        \ifx\test@@i\@moodle@roman
                          \def\moodle@numbering{iii}\fi
                        \ifx\test@@i\@moodle@Roman
                          \def\moodle@numbering{IIII}\fi
                        }%
      %'answer numbering' will be a synonym to 'numbering'
      \define@key{moodle}{answer numbering}[abc]{\setkeys{moodle}{numbering={#1}}}% 
      \generate@moodle@write@code*{answernumbering}<answernumbering>{\moodle@numbering}%
      %N.B. if we did not set the default here, then \moodle@numbering would be undefined, causing problems.
      \moodleset{answer numbering=abc}% 
      
      \def\@moodle@alpha{alpha}%
      \def\@moodle@Alpha{Alpha}%
      \def\@moodle@alph{alph}%
      \def\@moodle@Alph{Alph}%
      \def\@moodle@arabic{arabic}%
      \def\@moodle@roman{roman}%
      \def\@moodle@Roman{Roman}%      
      \def\@moodle@abc{abc}%
      \def\@moodle@ABCD{ABCD}%
      \def\@moodle@arabicnumbers{123}%
      \def\@moodle@iii{iii}%
      \def\@moodle@IIII{IIII}%
      \def\@moodle@none{none}%
      \def\moodle@obeynumberingstyle{%
        \ifx\moodle@numbering\@moodle@abc
          \renewcommand\theenumii{\alph{enumii}}%
        \fi
        \ifx\moodle@numbering\@moodle@ABCD
          \renewcommand\theenumii{\Alph{enumii}}%
        \fi
        \ifx\moodle@numbering\@moodle@arabicnumbers
          \renewcommand\theenumii{\arabic{enumii}}%
        \fi
        \ifx\moodle@numbering\@moodle@iii
          \renewcommand\theenumii{\roman{enumii}}%
        \fi
        \ifx\moodle@numbering\@moodle@IIII
          \renewcommand\theenumii{\Roman{enumii}}%
        \fi
        \ifx\moodle@numbering\@moodle@none
          \renewcommand\labelenumii{$\bullet$~}%
        \fi
      }
      %TODO: * In the PDF, how should 'none' in a multi look different from 
      %         short answer or numerical options?
      %       * Instead of \theenumi and \labelenumi,
      %         use \@enumdepth to automatically set the correct depth.
      
%% DISPLAY MODE -- affects Cloze multiple choice questions only.
      % 0 = inline, 1 = vertical, 2 = horizontal
      \def\moodle@multi@mode{0}%
      \define@key{moodle}{inline}[]{\def\moodle@multi@mode{0}}%
      \define@key{moodle}{vertical}[]{\def\moodle@multi@mode{1}}%
      \define@key{moodle}{horizontal}[]{\def\moodle@multi@mode{2}}%
%    \end{macrocode}
% \subsubsection{Keys for numerical questions}
%    \begin{macrocode}
%% TOLERANCE
      \define@cmdkey{moodle}[moodle@]{tolerance}[0]{}%
      \moodleset{tolerance=0}%
      %There is no \moodle@writetolerance, because in the XML the
      %tolerance is given within the answer tag,
      %like <answer fraction=100 tolerance=0.03>.

% TODO: implement unit-handling for numerical questions!
%    \end{macrocode}
% \subsubsection{Keys for short answer questions}
%    \begin{macrocode}
%% CASE SENSITIVE
      \define@boolkey{moodle}[moodle@]{case sensitive}[true]{}%
      \generate@moodle@write@code*{usecase}<usecase>{\csname ifmoodle@case sensitive\endcsname 1\else 0\fi}%
      % We make 'usecase' a synonym for 'case sensitive'.
      \define@boolkey{moodle}[moodle@]{usecase}[true]{\ifmoodle@usecase\csname moodle@case sensitivetrue\endcsname\else\csname moodle@case sensitivefalse\endcsname\fi}%
      \moodleset{case sensitive=false}%
%    \end{macrocode}
% \subsubsection{Keys for matching questions}
%    \begin{macrocode}
%% DRAG-AND-DROP FORMAT
      \define@boolkey{moodle}[moodle@]{draganddrop}[true]{}%
      % We make 'dd' and 'dragdrop' and 'drag and drop' synonyms for 'draganddrop'.
      \define@boolkey{moodle}[moodle@]{dd}[true]{\ifmoodle@dd\moodle@draganddroptrue\else\moodle@draganddropfalse\fi}%
      \define@boolkey{moodle}[moodle@]{drag and drop}[true]{\moodle@ddsynonym}%
      \def\moodle@ddsynonym{%
        \csname ifmoodle@drag and drop\endcsname
          \moodle@draganddroptrue
        \else
          \moodle@draganddropfalse
        \fi
      }
      \moodleset{draganddrop=false}%
%    \end{macrocode}
% \subsubsection{Keys for essay questions}
%    \begin{macrocode}
%% EDITOR
      \def\@moodle@html{html}%
      \def\@moodle@htmlfile{html+file}%
      \def\@moodle@text{text}%
      \def\@moodle@plain{plain}%
      \def\@moodle@monospaced{monospaced}%
      \def\@moodle@file{file}%
      \def\@moodle@noinline{noinline}%
      \define@choicekey{moodle}{response format}%
                       {html,html+file,text,monospaced,file}[html]%
                       {\def\test@i{#1}%
                        \ifx\test@i\@moodle@html
                          % HTML Editor
                          \def\moodle@responseformat{editor}% 
                        \fi
                        \ifx\test@i\@moodle@htmlfile
                          % HTML Editor with File Picker
                          \def\moodle@responseformat{editorfilepicker}%
                        \fi
                        \ifx\test@i\@moodle@text
                          % Plain text
                          \def\moodle@responseformat{plain}%
                        \fi
                        \ifx\test@i\@moodle@plain
                          % Plain text
                          \def\moodle@responseformat{plain}%
                        \fi
                        \ifx\test@i\@moodle@monospaced
                          % Plain text, monospaced font
                          \def\moodle@responseformat{monospaced}%
                        \fi
                        \ifx\test@i\@moodle@file
                          % No inline text (i.e., attachments only)
                          \def\moodle@responseformat{noinline}%
                        \fi
                        \ifx\test@i\@moodle@noinline
                          % No inline text (i.e., attachments only)
                          \def\moodle@responseformat{noinline}%
                        \fi
                       }%
      \generate@moodle@write@code*{responseformat}<responseformat>{\moodle@responseformat}%
      \moodleset{response format=html}% 
      %N.B. if we did not set a default, then \moodle@responseformat would be undefined, causing problems.

%% RESPONSE REQUIRED
      \define@boolkey{moodle}[moodle@]{response required}[true]{}%
      % TODO: Make synonym 'required'
      \generate@moodle@write@code*{responserequired}<responserequired>{\csname ifmoodle@response required\endcsname 1\else 0\fi}%
      \moodleset{response required=false}%

%% RESPONSEFIELDLINES
      \define@cmdkey{moodle}[moodle@]{response field lines}[15]{}%
      \generate@moodle@write@code*{responsefieldlines}<responsefieldlines>{\csname moodle@response field lines\endcsname}%
      %Make synonyms 'input box size' or 'height' or 'lines'?
      \moodleset{response field lines=15}% N.B. if we do not set a default, then \moodle@responseformat will be undefined, causing problems.

%% ATTACHMENTS ALLOWED
      \def\@moodle@unlimited{unlimited}%
      \define@choicekey{moodle}{attachments allowed}{0,1,2,3,unlimited}[1]{%
        \def\test@i{#1}%
        \ifx\test@i\@moodle@unlimited
          \def\moodle@attachmentsallowed{-1}%
        \else
          \def\moodle@attachmentsallowed{#1}%
        \fi
      }
      \generate@moodle@write@code*{attachmentsallowed}<attachments>{\moodle@attachmentsallowed}
      \moodleset{attachments allowed=0}%

%% ATTACHMENTS REQUIRED
      \define@choicekey{moodle}{attachments required}{0,1,2,3}[1]{\def\moodle@attachmentsrequired{#1}}%
      \generate@moodle@write@code*{attachmentsrequired}<attachmentsrequired>{\moodle@attachmentsrequired}
      \moodleset{attachments required=0}%

%% RESPONSE TEMPLATE
      \define@key{moodle}{template}{\long\def\moodle@responsetemplate{#1}}%
      \generate@moodle@write@html@noptag{responsetemplate}<responsetemplate>{\moodle@responsetemplate}
      \moodleset{template={}}%
%    \end{macrocode}
% \subsubsection{Hint tags}
% The following are not yet fully implemented.
%    \begin{macrocode}
%% SHOWNUMCORRECT
      \define@boolkey{moodle}[moodle@]{shownumcorrect}[true]{}%
      \gdef\moodle@writeshownumcorrect{%
        \if\moodle@shownumcorrect
          \writetomoodle[4]{    <shownumcorrect/>}%
        \fi
      }%
      \moodleset{shownumcorrect=false}%

%% CLEARWRONG
      \define@boolkey{moodle}[moodle@]{clearwrong}[true]{}%
      \gdef\moodle@writeclearwrong{%
        \if\moodle@clearwrong
          \writetomoodle[4]{    <clearwrong/>}%
        \fi
      }%
      \moodleset{clearwrong=false}%

% TODO: Implement hints
%    \end{macrocode}
%
% \subsection{Answer handling}
% 
%    \begin{macrocode}
%The Answers XML depends heavily on the question type.
%Each type of question defines how it obtains answers from the LaTeX input,
%how it typesets those in a PDF or DVI, and how it writes them as XML code.
%It will write that XML to the macro \moodle@answers@xml,
%which them gets written to the file when \moodle@writeanswers
%is invoked.

\def\moodle@answers@xml{}%
\gdef\moodle@writeanswers{%
  \writetomoodle{\moodle@answers@xml}%
}%

\newcommand\addto@xml[3][0]{%
  % #1 = spaces to indent (default=0)
  % #2 = macro containing XML code (possibly empty)
  % #3 = XML text to be appended to that macro (will be \edef'd)
  \calculateindent{#1}%
  \edef\xml@to@add{\moodle@indent\trim@pre@space{#3}}%
  \ifx#2\@moodle@empty
    \edef\newxml{\noexpand#2\xml@to@add}%
  \else
    \edef\newxml{\noexpand#2^^J\xml@to@add}%
  \fi
  \xa\xa\xa\def\xa\xa\xa#2\xa\xa\xa{\newxml}%
}%
%    \end{macrocode}
%
% \subsubsection{Not yet implemented}
% 
%    \begin{macrocode}

%%%%%%%%%%%%%%%%%%%%%%%%%%%%%%%%%%%%%%%%
%% DESCRIPTION 'QUESTIONS' %%%%%%%%%%%%%

% TODO: implement the \writedescription and a suitable front-end.
%       Should this be \begin{description}...\end{description},
%       or should \begin{quiz}...\end{quiz} just scoop up all
%       text outside question environments and package it in descriptions?

%%%%%%%%%%%%%%%%%%%%%%%%%%%%%%%%%%%%%%%%
%% CALCULATED %%%%%%%%%%%%%%%%%%%%%%%%%%

% TODO: I don't think I really want to handle this.  Not now.

%    \end{macrocode}

% \subsection{Front Ends}
% This section creates the user interface for the various question types.
% First, we define a generic command to create 
% a front-end environment for a Moodle question type.
% In order to function, the following macros must be hard-coded:
% \begin{itemize}
%   \item |\moodle@|\meta{type}|@latexprocessing|: 
%     Loops through the saved |\item|'s to typeset them in LaTeX,
%     usually inside an itemize or enumerate environment.
%   \item |\save|\meta{type}|answer#1|: 
%     Processes the text of a single |\item| to save the information to memory,
%     usually inside |\moodle@answers@xml|.
%   \item |\write|\meta{type}|question|:
%     Writes the information, hitherto saved only in macros,
%     into the XML file.
% \end{itemize}
% For example, to create the `shortanswer' question type,
% we shall call 
% \begin{Verbatim}[gobble=5,frame=single]
%    \moodle@makefrontend{shortanswer}
%    \def\moodle@shortanswer@latexprocessing{...}
%    \def\saveshortansweranswer#1{...}
%    \def\writeshortanswerquestion{...}
% \end{Verbatim}
% 
%    \begin{macrocode}

\def\moodle@makelatextagbox#1{%
%  \ifmoodle@tikzloaded
%    \tikzset{external/export next=false}
%    \tikz[baseline]{\node[draw,minimum height=1.2em,rounded corners,fill=black!20] {\tiny #1};}
%  % Fancy but interferes with the tikzexternalize counter
%  \else
    \Ovalbox{\tiny #1}
    %\ovalbox{\tiny #1}
    %\shadowbox{\tiny #1}
%  \fi
}%

\def\moodle@makelatextag@qtype#1{%
  \doublebox{\tiny \textsc{#1}}
}%

\def\moodle@makelatextag@value#1#2{%
  \moodle@makelatextagbox{\csname moodle@#1\endcsname~#2}
}%

\def\moodle@makelatextag@key#1{%
  \moodle@makelatextagbox{\csname moodle@#1\endcsname}
}%

\def\moodle@marks#1{point\ifdim#1pt=1pt \else s\fi}

\def\moodle@makefrontend#1#2{%
  \NewEnviron{#1}[2][]{%
    \bgroup
      \setkeys{moodle}{##1,questionname={##2}}%
      \expandafter\gatheritems\xa{\BODY}%
      \let\moodle@questionheader=\gatheredheader
      %First, the LaTeX processing
      \item \textbf{\moodle@questionname}
      \xa\xdef\xa\test@iii\xa{\moodle@tags}%
      \ifx\test@iii\@moodle@empty\relax\else
        \hfill tags: \texttt{\moodle@tags}
      \fi
      \par
      \noindent
      \moodle@makelatextag@qtype{#1}
      \moodle@makelatextag@value{default grade}{\moodle@marks{\csname moodle@default grade\endcsname}}
      \moodle@makelatextag@value{penalty}{penalty}
      #2\par
      \noindent
      \moodle@questionheader
      \edef\moodle@generalfeedback{\expandonce\moodle@feedback}
      \csname moodle@#1@latexprocessing\endcsname
      %Now, writing information to XML
      \@moodle@ifgeneratexml{%
        \xa\questiontext\xa{\moodle@questionheader}% Save the question text.
        \csname write#1question\endcsname
        \bgroup
          \gdef\moodle@answers@xml{}%
          \setkeys{moodle}{feedback={}}%
          \xa\loopthroughitemswithcommand\xa{\csname save#1answer\endcsname}%
          \passvalueaftergroup{\moodle@answers@xml}%
        \egroup
        \moodle@writeanswers%
        \moodle@writetags%
        \writetomoodle{</question>}%
      }{}%
    \egroup
  }%
}
%    \end{macrocode}
%
% \subsubsection{Essay Question Front-End}
% The essay question is the only question type whose front end
% is not yet created by |\moodle@makefrontend|.
% This is because of what it must do with its |\item|'s.
% 
%    \begin{macrocode}
\def\moodle@essay@latexprocessing{%
  % Moodle cannot automatically grade an essay, 
  % but if the user puts \item's in, we can list them in an itemize as notes.
  \ifnum\c@numgathereditems>0\relax
    \par\noindent \emph{Notes for grader:}
    \begin{itemize} \setlength\itemsep{0pt}\setlength\parskip{0pt}%
      \loopthroughitemswithcommand{\moodle@print@essay@answer}%
    \end{itemize}%
  \fi
  \ifx\moodle@generalfeedback\@empty\relax\else%
    \fbox{\parbox{\linewidth}{\emph{\moodle@generalfeedback}}}%
  \fi
}

\NewEnviron{essay}[2][]{%
  \bgroup
    \setkeys{moodle}{#1,questionname={#2}}%
    \expandafter\gatheritems\expandafter{\BODY}%
    \let\moodle@questionheader=\gatheredheader
    %First, the LaTeX processing.
      \item \textbf{\moodle@questionname}
      \xa\xdef\xa\test@iii\xa{\moodle@tags}%
      \ifx\test@iii\@moodle@empty\relax\else
        \hfill tags: \texttt{\moodle@tags}
      \fi
      \par
      \noindent
      \moodle@makelatextag@qtype{essay}
      \moodle@makelatextag@value{default grade}{\moodle@marks{\csname moodle@default grade\endcsname}}
      \moodle@makelatextag@value{penalty}{penalty}
      \moodle@makelatextag@key{responseformat}\par
      \noindent
      \moodle@questionheader
      \edef\moodle@generalfeedback{\expandonce\moodle@feedback}
      \csname moodle@essay@latexprocessing\endcsname
    %Now, writing information to memory.
    \@moodle@ifgeneratexml{%
      \xa\questiontext\xa{\moodle@questionheader}% Save the question text.
      \writeessayquestion
      \bgroup
        \gdef\moodle@answers@xml{}%
        %
        \ifnum\c@numgathereditems=0\relax
          \addto@xml[2]\moodle@answers@xml{<graderinfo format="html"><text/></graderinfo>}%
        \else
          \addto@xml[2]\moodle@answers@xml{<graderinfo format="html"><text><![CDATA[}%
          \ifnum\c@numgathereditems>1\relax
            \addto@xml[4]\moodle@answers@xml{<ul>}%
          \fi
          \loopthroughitemswithcommand{\moodle@savegraderinfo}%
          \ifnum\c@numgathereditems>1\relax
            \addto@xml[4]\moodle@answers@xml{</ul>}%
          \fi
          \addto@xml[2]\moodle@answers@xml{]]></text></graderinfo>}%
        \fi
        %
        \passvalueaftergroup{\moodle@answers@xml}%
      \egroup
      \moodle@writeanswers% The 'answers' XML really contains the grader info.
      \moodle@writeresponsetemplate%
      \moodle@writetags%
      \writetomoodle{</question>}%
    }{}%
  \egroup
}%

%%%% TODO
%%%% To make essay work will be tough.
%%%% Every line from \ifnum\c@numgathereditems=0\relax through its \else and \fi,
%%%% with the exception of
%%%%          \xa\loopthroughitemswithcommand\xa{\csname save#1answer\endcsname}%
%%%% , does not exist in our current \moodle@makefrontend code.
%%%% How can we cope?
%%%%
%%%% Idea: change \moodle@makefrontend so that
%%%%       1. if \c@numgathereditems=0, we don't do anything.
%%%%       2. it calls a preamble and postamble around the \loopthroughitemswithcommand.
%%%%          Like this:
%%%%            
%%%%      \@moodle@ifgeneratexml{%
%%%%        \xa\questiontext\xa{\moodle@questionheader}% Save the question text.
%%%%        \bgroup
%%%%          \gdef\moodle@answers@xml{}%
%%%%          \setkeys{moodle}{feedback={}}%
%%%%          \@ifundefined{moodle@#1@answers@preamble}{}{}%
%%%%          \csname moodle@#1@answers@preamble\endcsname
%%%%          \ifnum\c@numgathereditems=0\relax
%%%%            \relax
%%%%          \else
%%%%            \xa\loopthroughitemswithcommand\xa{\csname save#1answer\endcsname}%
%%%%          \fi
%%%%          \@ifundefined{moodle@#1@answers@postamble}{}{}%
%%%%          \csname moodle@#1@answers@postamble\endcsname
%%%%          \passvalueaftergroup{\moodle@answers@xml}%
%%%%        \egroup
%%%%        \csname write#1question\endcsname
%%%%      }{}%
%%%% The \@ifundefined lines should automatically define the 
%%%% \...@preamble \...@postamble macros to be \relax if they don't exist already.

\gdef\writeessayquestion{%
  \writetomoodle{<question type="essay">}%
    \moodle@writecommondata%
    \moodle@writeresponserequired%
    \moodle@writeresponseformat%
    \moodle@writeresponsefieldlines%
    \moodle@writeattachmentsallowed%
    \moodle@writeattachmentsrequired%
}%

\def\moodle@print@essay@answer#1{%
    \item #1%
}%


\def\moodle@savegraderinfo#1{%
  %\def\ds{#1}\show\ds
  \bgroup
    \moodle@savegraderinfo@int#1\moodle@answer@rdelim
    \passvalueaftergroup{\moodle@answers@xml}%
  \egroup
}%
\newcommand\moodle@savegraderinfo@int[1][]{%
  \setkeys{moodle}{fraction=0,#1}%
  \moodle@savegraderinfo@int@int%
}%
\def\moodle@savegraderinfo@int@int#1\moodle@answer@rdelim{%
  \def\moodle@answertext{#1}
  \xa\converttohtmlmacro\xa\moodle@answertext@html\xa{\moodle@answertext}%
  %\trim@spaces@in\moodle@answertext
  \ifnum\c@numgathereditems>1\relax
    \addto@xml[6]{\moodle@answers@xml}{<li>\moodle@answertext@html</li>}%
  \else
    \addto@xml[4]{\moodle@answers@xml}{\moodle@answertext@html}%
  \fi
}%
%    \end{macrocode}
%
% \subsubsection{Short Answer Question Front-End}
% 
%    \begin{macrocode}
\def\blank{\rule{1in}{0.5pt}}%
% TODO: Make an optional argument for width?  This wouldn't affect Moodle,
%        only the appearance in the PDF.  It doesn't seem worth it.

%\NewEnviron{shortanswer}[2][]{%
%   \bgroup
%     \setkeys{moodle}{#1,questionname={#2}}%
%     \expandafter\gatheritems\xa{\BODY}%
%     \let\moodle@questionheader=\gatheredheader
%     %First, the LaTeX processing.
%       \item \textbf{\moodle@questionname}
%       \csname ifmoodle@case sensitive\endcsname
%         \framebox{\tiny Case-Sensitive}\relax
%       \fi
%       \framebox{\tiny\csname moodle@default grade\endcsname~points}
%       \framebox{\tiny\csname moodle@penalty\endcsname~penalty}\par
%       \noindent
%       \moodle@questionheader
%       \csname moodle@shortanswer@latexprocessing\endcsname
%     %Now, writing information to memory.
%     \@moodle@ifgeneratexml{%
%       \xa\questiontext\xa{\moodle@questionheader}% Save the question text.
%       \bgroup
%         \gdef\moodle@answers@xml{}%
%         \setkeys{moodle}{feedback={}}%
%         \xa\loopthroughitemswithcommand\xa{\csname 
%         saveshortansweranswer\endcsname}%
%         \passvalueaftergroup{\moodle@answers@xml}%
%       \egroup
%       \csname writeshortanswerquestion\endcsname
%     }{}%
%   \egroup
% }%

\moodle@makefrontend{shortanswer}{\moodle@makelatextag@shortanswer}%

% LATEX PROCESSING

\def\moodle@makelatextag@shortanswer{%
  \csname ifmoodle@case sensitive\endcsname
    \moodle@makelatextagbox{Case-Sensitive}\relax
  \else
    \moodle@makelatextagbox{Case-Insensitive}\relax
  \fi
}

\def\moodle@shortanswer@latexprocessing{%
  \begin{itemize} \setlength\itemsep{0pt}\setlength\parskip{0pt}%
    \loopthroughitemswithcommand{\moodle@print@shortanswer@answer}%
  \end{itemize}%
  \ifx\moodle@generalfeedback\@empty\relax\else%
    \fbox{\parbox{\linewidth}{\emph{\moodle@generalfeedback}}}%
  \fi
}

   \def\moodle@print@shortanswer@answer#1{%
       \let\moodle@feedback=\@empty
       \moodle@print@shortanswer@answer@int#1\@rdelim
   }%
   \newcommand\moodle@print@shortanswer@answer@int[1][]{%
     \setkeys{moodle}{#1}%
     \moodle@print@shortanswer@answer@int@int%
   }%
   \def\moodle@print@shortanswer@answer@int@int#1\@rdelim{%
     \ifx\moodle@fraction\@hundred
       \item #1$~\checkmark$%
     \else
       \moodle@checkfraction
       \item #1$~(\moodle@fraction\%)$%
     \fi
     \ifx\moodle@feedback\@empty\relax\else
       \hfill \emph{$\rightarrow$ \moodle@feedback}
     \fi
   }%

% SAVING ANSWERS TO MEMORY
\def\saveshortansweranswer#1{%
  \bgroup
    \saveshortansweranswer@int#1\moodle@answer@rdelim
    \passvalueaftergroup{\moodle@answers@xml}%
  \egroup
}%
   \newcommand\saveshortansweranswer@int[1][]{%
     \setkeys{moodle}{fraction=100,#1}%                  %%%%%% DEFAULT VALUE IS 100%
     \saveshortansweranswer@int@int%
   }%
   \def\saveshortansweranswer@int@int#1\moodle@answer@rdelim{%
     \def\moodle@answertext{#1}%
     \trim@spaces@in\moodle@answertext
     \moodle@checkfraction
     \addto@xml[2]{\moodle@answers@xml}{<answer fraction="\moodle@fraction" format="plain_text">}%
     \addto@xml[4]{\moodle@answers@xml}{  <text>\moodle@answertext</text>}%
     \ifx\moodle@feedback\@empty\relax\else
       \trim@spaces@in\moodle@feedback
       \xa\converttohtmlmacro\xa\moodle@feedback@html\xa{\moodle@feedback}%
       \addto@xml[4]{\moodle@answers@xml}{  <feedback format="html"><text><![CDATA[<p>\moodle@feedback@html</p>]]></text></feedback>}%
     \fi
     \addto@xml[2]{\moodle@answers@xml}{</answer>}%
   }%

% WRITING QUESTION TO XML FILE
\gdef\writeshortanswerquestion{%
  \writetomoodle{<question type="shortanswer">}%
    \moodle@writecommondata%
    \moodle@writeusecase%
}%
%    \end{macrocode}
%
% \subsubsection{Numerical Question Front-End}
% 
%    \begin{macrocode}
\moodle@makefrontend{numerical}{\moodle@makelatextag@numerical}%

% LATEX PROCESSING

\def\moodle@makelatextag@numerical{}

\def\moodle@numerical@latexprocessing{%
      \begin{itemize} \setlength\itemsep{0pt}\setlength\parskip{0pt}%
        \loopthroughitemswithcommand{\moodle@print@numerical@answer}%
      \end{itemize}%
      \ifx\moodle@generalfeedback\@empty\relax\else%
        \fbox{\parbox{\linewidth}{\emph{\moodle@generalfeedback}}}%
      \fi
}

   \def\moodle@print@numerical@answer#1{%
       \let\moodle@feedback=\@empty
       \moodle@print@numerical@answer@int#1\@rdelim
   }%
   \newcommand\moodle@print@numerical@answer@int[1][]{%
     \setkeys{moodle}{#1}%
     \moodle@print@numerical@answer@int@int%
   }%
   \def\moodle@print@numerical@answer@int@int#1\@rdelim{%
     \ifdim0pt=\moodle@tolerance pt\relax
       \def\moodle@numericalprint@tolerance{}%
     \else
       \edef\moodle@numericalprint@tolerance{\noexpand\pm\moodle@tolerance}%
     \fi
     \ifx\moodle@fraction\@hundred
       \item $#1\moodle@numericalprint@tolerance~\checkmark$%
     \else
       \moodle@checkfraction
       \item $#1\moodle@numericalprint@tolerance~(\moodle@fraction\%)$%
     \fi
     \ifx\moodle@feedback\@empty\relax\else
       \hfill \emph{$\rightarrow$ \moodle@feedback}%
     \fi
   }%

% SAVING ANSWERS TO MEMORY
\def\savenumericalanswer#1{%
  \bgroup
    \savenumericalanswer@int#1\moodle@answer@rdelim
    \passvalueaftergroup{\moodle@answers@xml}%
  \egroup
}%
   \newcommand\savenumericalanswer@int[1][]{%
     \setkeys{moodle}{fraction=100,#1}%                  %%%%%% DEFAULT VALUE IS 100%
     \savenumericalanswer@int@int%
   }%
   \def\savenumericalanswer@int@int#1\moodle@answer@rdelim{%
     \def\moodle@answertext{#1}%
     \trim@spaces@in\moodle@answertext
     \moodle@checkfraction
     \addto@xml[2]{\moodle@answers@xml}{<answer fraction="\moodle@fraction" format="plain_text">}%
     \addto@xml[4]{\moodle@answers@xml}{  <text>\moodle@answertext</text>}%
     \addto@xml[4]{\moodle@answers@xml}{  <tolerance>\moodle@tolerance</tolerance>}%
     \ifx\moodle@feedback\@empty\relax\else
       \trim@spaces@in\moodle@feedback
       \xa\converttohtmlmacro\xa\moodle@feedback@html\xa{\moodle@feedback}%
       \addto@xml[4]{\moodle@answers@xml}{  <feedback format="html"><text><![CDATA[<p>\moodle@feedback@html</p>]]></text></feedback>}%
     \fi
     \addto@xml[2]{\moodle@answers@xml}{</answer>}%
   }%


% WRITING QUESTION TO XML FILE
\gdef\writenumericalquestion{%
  \writetomoodle{<question type="numerical">}%
    \moodle@writecommondata%
}%
%    \end{macrocode}
%
% \subsubsection{Multiple Choice Question Front-End}
% 
%    \begin{macrocode}
%Multiple choice has the structure
% \begin{multi}[options]{name}%
%   What is 5+7?
%   \item 13
%   \item* 12
%   \item 11
% \end{multi}%

\moodle@makefrontend{multi}{\moodle@makelatextag@multi}%

% LATEX PROCESSING

\def\moodle@makelatextag@multi{%
  \csname ifmoodle@multiple\endcsname%
    \moodle@makelatextagbox{Multiple}\relax%
  \else%
    \moodle@makelatextagbox{Single}\relax%
  \fi%
  \csname ifmoodle@shuffle\endcsname%
    \moodle@makelatextagbox{Shuffle}\relax%
  \fi%
}

\def\moodle@multi@latexprocessing{%
  \moodle@countcorrectanswers%
  \begin{enumerate}\moodle@obeynumberingstyle%
    %\renewcommand{\theenumi}{\alph{enumi}}%
    \setlength\itemsep{0pt}\setlength\parskip{0pt}%
    \loopthroughitemswithcommand{\moodle@print@multichoice@answer}%
  \end{enumerate}%
  \ifx\moodle@generalfeedback\@empty\relax\else%
    \fbox{\parbox{\linewidth}{\emph{\moodle@generalfeedback}}}%
  \fi%
}
  \def\moodle@print@multichoice@answer#1{%
    \let\moodle@feedback=\@empty%
    \moodle@print@multichoice@answer@int#1 \@rdelim%
  }%
  \newcommand\moodle@print@multichoice@answer@int[1][]{%
    \let\moodle@fraction\@empty%
    \setkeys{moodle}{#1}%
    \moodle@print@multichoice@answer@int@int%
  }%
  \def\moodle@print@multichoice@answer@int@int#1#2\@rdelim{%
    \def\test@i{#1}%
    \def\test@ii{#2}%
    \def\moodle@answertext{\item }%
    \ifx\test@i\@star%
      \g@addto@macro\moodle@answertext{#2}%
      \ifmoodle@single%
        \setkeys{moodle}{fraction=100}%
      \else
        \setkeys{moodle}{fraction=\moodle@autopoints}%
      \fi
    \else
      \g@addto@macro\moodle@answertext{#1#2}%
    \fi
    \trim@spaces@in\moodle@answertext%
    \trim@spaces@in\moodle@answertext%
    \ifmoodle@single%
      \ifx\moodle@fraction\@empty\relax%
        \setkeys{moodle}{fraction=0}%
      \fi
      \ifx\moodle@fraction\@hundred%
        \trim@spaces@in\moodle@answertext%
        \g@addto@macro\moodle@answertext{$~\checkmark$}%
      \else
        \moodle@checkfraction
        \ifdim0pt=\moodle@fraction pt\relax\else%
          \g@addto@macro\moodle@answertext{$~(\moodle@fraction\%)$}%
        \fi
      \fi
    \else% multiple
      \ifx\moodle@fraction\@empty\relax%
        \setkeys{moodle}{fraction=\moodle@autosanctions}%
      \fi
      \moodle@checkfraction
      \g@addto@macro\moodle@answertext{$~(\moodle@fraction\%)$}%
    \fi
    \moodle@answertext
    \ifx\moodle@feedback\@empty\relax\else%
      \hfill \emph{$\rightarrow$ \moodle@feedback}%
    \fi%
  }%

% COMMON UTILITY: COUNTING CORRECT ANSWERS (AND A BIT MORE...)
   \newcounter{moodle@numcorrectanswers}% count the stars
   \newcounter{moodle@numincorrectanswers}% count the items without fraction key indicated
   \newlength{\moodle@pointspercorrect}%
   \newlength{\moodle@pointsperincorrect}%
   \newlength{\moodle@sumofpositivefractions}% sums user-set positive fractions
   \newlength{\moodle@sumofnegativefractions}% sums user-set negative fractions
   \def\moodle@countcorrectanswers{%
     \setcounter{moodle@numcorrectanswers}{0}%
     \setcounter{moodle@numincorrectanswers}{0}%
     \global\setlength{\moodle@pointspercorrect}{100pt}%
     \global\setlength{\moodle@pointsperincorrect}{-100pt}%
     \global\setlength{\moodle@sumofpositivefractions}{0pt}%
     \global\setlength{\moodle@sumofnegativefractions}{0pt}%
     \loopthroughitemswithcommand{\moodle@countcorrectanswers@a}%
     \global\advance\moodle@pointspercorrect by-\moodle@sumofpositivefractions\relax%
     \def\ds{\strip@pt\moodle@sumofpositivefractions}%
     \ifnum0=\c@moodle@numcorrectanswers\relax%
       % autopoints will never be used but we check if the sum of positive fractions is 100%
       \ifdim\moodle@pointspercorrect<-\moodle@fractiontol pt\relax%
         \PackageWarning{moodle}{Positive fractions sum up to more than 100 (here: \ds)}%
       \else
         \ifdim\moodle@pointspercorrect>\moodle@fractiontol pt\relax%
           \PackageError{moodle}{Positive fractions sum up to less than 100 (here: \ds)}%
         \fi
       \fi
     \else
       \ifdim0pt<\moodle@pointspercorrect\relax\else%
         % we have starred items so the sum of user-set positive fractions must be less than 100%
         % otherwise, starred items would lead to penalties
         \PackageError{moodle}{Positive fractions sum up to 100 or more (here: \ds):
                                there is no positive points left to be given to starred items.}%
       \fi
       \global\divide\moodle@pointspercorrect by \c@moodle@numcorrectanswers\relax%
     \fi
     \gdef\moodle@autopoints{\strip@pt\moodle@pointspercorrect}%
     \global\advance\moodle@pointsperincorrect by-\moodle@sumofnegativefractions\relax%
     \def\ds{\strip@pt\moodle@sumofnegativefractions}%
     \ifnum0=\c@moodle@numincorrectanswers\relax%
       % autosanctions will never be used and
       % we do not care about the sum of negative fractions (might be less than -100)
     \else
       \ifdim0pt<\moodle@pointsperincorrect\relax%
         % we have items without fractions set: to prevent auto sanctions from becoming bonuses,
         % such items are neutralized.
         \PackageWarning{moodle}{Negative fractions sum up to -100 or less (here: \ds):
                                  items with no fraction key set will be considered as neutral.}%
         \global\setlength{\moodle@pointsperincorrect}{0pt}%
       \fi
       \global\divide\moodle@pointsperincorrect by \c@moodle@numincorrectanswers\relax%
     \fi
     \gdef\moodle@autosanctions{\strip@pt\moodle@pointsperincorrect}%
   }
   \def\moodle@countcorrectanswers@a#1{%
     %The grouping is to keep key answer-specific key changes local.
     \bgroup
       \moodle@countcorrectanswers@b#1\moodle@answer@rdelim
     \egroup
   }%
   \newcommand\moodle@countcorrectanswers@b[1][]{%
     %\ifx&#1&%
       \let\moodle@fraction\@empty%
       \setkeys{moodle}{#1}%
       \moodle@countcorrectanswers@c%
     %\fi
   }%
   \def\moodle@countcorrectanswers@c#1#2\moodle@answer@rdelim{%
     \def\test@i{#1}%
     \ifx\test@i\@star
       \stepcounter{moodle@numcorrectanswers}%
     \else
       \ifx\moodle@fraction\@empty\relax%
         \stepcounter{moodle@numincorrectanswers}%
       \else
         \ifdim0pt<\moodle@fraction pt\relax%
           \global\addtolength{\moodle@sumofpositivefractions}{\moodle@fraction pt}%
         \else
           \global\addtolength{\moodle@sumofnegativefractions}{\moodle@fraction pt}%
         \fi
       \fi
     \fi
   }%
   \newlength{\test@fraction}%
   \newlength{\test@lower}%
   \newlength{\test@upper}%
   \def\moodle@fractionerror{%
     \def\ds{\moodle@fraction}%
     \PackageError{moodle}{the current fraction is not a valid value (here: \ds)}%
   }
   {\catcode`|=3\relax
   \gdef\moodle@validfractionlist{0|5|10|11.11111|12.5|14.28571|16.66667|20|25|30|33.33333|40|50|60|66.66667|70|75|80|83.33333|90|100}}
   %\forcsvlist{\listadd\moodle@validfractionlist}{0pt,5pt,10pt,11.11111pt,12.5pt,14.28571pt,16.66667pt,20pt,25pt,30pt,33.33333pt,40pt,50pt,60pt,66.66667pt,70pt,75pt,80pt,83.33333pt,90pt,100pt}%
   \def\moodle@isfractionnear#1{%
     \setlength{\test@lower}{#1 pt}%
     \addtolength{\test@lower}{-\moodle@fractiontol pt}%
     \setlength{\test@upper}{#1 pt}%
     \addtolength{\test@upper}{\moodle@fractiontol pt}%
     \ifdim\test@upper>\test@fraction\relax
       \ifdim\test@lower<\test@fraction\relax
         \gdef\test@fractionmatched{#1}%
       \fi
     \fi
   }
   \def\moodle@checkfraction{%
     %\def\test@i{#1}%
     \setlength{\test@fraction}{\moodle@fraction pt}%
     % take the absolute value
     \ifdim0pt>\test@fraction\relax%
       \setlength{\test@fraction}{-\moodle@fraction pt}%
     \fi
     % test if the fraction is an admissible value
     \let\test@fractionmatched\@empty
     \renewcommand*{\do}[1]{\moodle@isfractionnear{##1}}%
     \dolistloop{\moodle@validfractionlist}%
     \ifx\test@fractionmatched\@empty\relax
       \moodle@fractionerror%
     \fi
     \ifdim\moodle@fraction pt<-\moodle@fractiontol pt\relax%
       \setkeys{moodle}{fraction=-\test@fractionmatched}%
     \else
       \setkeys{moodle}{fraction=\test@fractionmatched}%
     \fi
   }
% TODO: Put these macros in same order as other sections'.

% SAVING ANSWERS TO MEMORY
\def\savemultianswer#1{%
  \bgroup
    \savemultianswer@int#1 \moodle@answer@rdelim
    \passvalueaftergroup{\moodle@answers@xml}%
  \egroup
}%
  \newcommand\savemultianswer@int[1][]{%
    \let\moodle@fraction\@empty%
    \setkeys{moodle}{#1}%
    \savemultianswer@int@int%
  }%
  \def\savemultianswer@int@int#1#2\moodle@answer@rdelim{%
    \def\test@i{#1}%
    \ifx\test@i\@star
      \ifmoodle@single
        \setkeys{moodle}{fraction=100}%
      \else
        \setkeys{moodle}{fraction=\moodle@autopoints}%
      \fi
      \def\moodle@answertext{#2}%
    \else
      \def\moodle@answertext{#1#2}%
    \fi
    \ifx\moodle@fraction\@empty\relax%
      \ifmoodle@single\relax
        \setkeys{moodle}{fraction=0}%
      \else% multiple
        \setkeys{moodle}{fraction=\moodle@autosanctions}%
      \fi
    \fi
    \trim@spaces@in\moodle@answertext
    \trim@spaces@in\moodle@answertext
    \moodle@checkfraction
    \addto@xml[2]{\moodle@answers@xml}{<answer fraction="\moodle@fraction" format="html">}%
    \xa\converttohtmlmacro\xa\moodle@answertext@html\xa{\moodle@answertext}%
    \addto@xml[4]{\moodle@answers@xml}{  <text><![CDATA[<p>\moodle@answertext@html</p>]]></text>}%
    \ifx\moodle@feedback\@empty\relax\else
      \trim@spaces@in\moodle@feedback
      \xa\converttohtmlmacro\xa\moodle@feedback@html\xa{\moodle@feedback}%
      \addto@xml[4]{\moodle@answers@xml}{  <feedback format="html"><text><![CDATA[<p>\moodle@feedback@html</p>]]></text></feedback>}%
    \fi
    \addto@xml[2]{\moodle@answers@xml}{</answer>}%
  }%

% WRITING QUESTION TO XML FILE
\gdef\writemultiquestion{%
  \writetomoodle{<question type="multichoice">}%
    \moodle@writecommondata%
    \moodle@writesingle%
    \moodle@writeshuffle%
    \moodle@writeanswernumbering%
}%
%    \end{macrocode}
%
% \subsubsection{True/False Question Front-End}
% 
%    \begin{macrocode}
% True/False has structure
% \begin{truefalse}[options]{name}%
%   This is a matching question.
%   \item[feedback={feedback for student answering incorrectly "true"}] % first item is for true
%   \item* this is an other way of specifying answer-specific feedback
% \end{truefalse}%

%\moodle@makefrontend{truefalse}{}% We dont use the generic frontend because truefalse has no tunable penalty

\NewEnviron{truefalse}[2][]{%
    \bgroup
      \setkeys{moodle}{#1,questionname={#2}}%
      \expandafter\gatheritems\xa{\BODY}%
      \let\moodle@questionheader=\gatheredheader
      %First, the LaTeX processing
      \item \textbf{\moodle@questionname}
      \xa\xdef\xa\test@iii\xa{\moodle@tags}%
      \ifx\test@iii\@moodle@empty\relax\else
        \hfill tags: \texttt{\moodle@tags}
      \fi
      \par
      \noindent
      \moodle@makelatextag@qtype{truefalse}
      \moodle@makelatextag@value{default grade}{\moodle@marks{\csname moodle@default grade\endcsname}}
      \par
      \noindent
      \moodle@questionheader
      \edef\moodle@generalfeedback{\expandonce\moodle@feedback}
      \moodle@truefalse@latexprocessing
      %Now, writing information to XML
      \@moodle@ifgeneratexml{%
        \setkeys{moodle}{penalty=1}%
        \xa\questiontext\xa{\moodle@questionheader}% Save the question text.
        \csname writetruefalsequestion\endcsname
        \bgroup
          \gdef\moodle@answers@xml{}%
          \setkeys{moodle}{feedback={}}%
          \xa\loopthroughitemswithcommand\xa{\xa\savetruefalseanswer}%
          \ifnum\c@numgathereditems=1\relax%
            \setcounter{currentitemnumber}{2}%
            \savetruefalseanswer{}
          \fi
          \passvalueaftergroup{\moodle@answers@xml}%
        \egroup
        \moodle@writeanswers%
        \moodle@writetags%
        \writetomoodle{</question>}%
      }{}%
    \egroup
  }%

% LATEX PROCESSING

\def\moodle@truefalse@latexprocessing{%
%  \ifnum\c@numgathereditems>2\relax%
%    \PackageError{moodle}{Expecting at max two answers with truefalse type}
%  \fi
  \setcounter{moodle@numcorrectanswers}{0}%
  \begin{itemize} \setlength\itemsep{0pt}\setlength\parskip{0pt}%
    \loopthroughitemswithcommand{\moodle@print@truefalse@answer}%
  \end{itemize}
  \ifx\moodle@generalfeedback\@empty\relax\else%
    \fbox{\parbox{\linewidth}{\emph{\moodle@generalfeedback}}}%
  \fi
  \ifnum\c@moodle@numcorrectanswers=0\relax%
    \PackageError{moodle}{No answer is explicitly marked as correct (*). Be sure one answer leads to points.}%
  \fi
  \ifnum\c@moodle@numcorrectanswers>1\relax%
    \PackageError{moodle}{Two answers are explicitly marked as correct (*). Be sure only one answer leads to points.}%
  \fi
}

   \def\moodle@print@truefalse@answer#1{% here # is all what comes after "\item", that is "[options]* text"
       \let\moodle@feedback=\@empty
       \moodle@print@truefalse@answer@int#1\@rdelim % add an end delimiter:
   }%
   \newcommand\moodle@print@truefalse@answer@int[1][]{% with the optional argument, catch options and set them as keys
     \setkeys{moodle}{#1}%
     \moodle@print@truefalse@answer@int@int% applies to the rest: "* text\@rdelim"
   }%
   \def\moodle@print@truefalse@answer@int@int#1\@rdelim{% this is just to treat appart the case where nothing follows
     \def\test@i{#1}
     \trim@spaces@in\test@i
     \ifx\test@i\@empty\relax
       \moodle@print@truefalse@answer@int@int@empty
     \else
       \moodle@print@truefalse@answer@int@int@int#1\@rdelim
     \fi
   }%
   \def\moodle@print@truefalse@answer@int@int@empty{%
     \ifnum\c@currentitemnumber=1%
	\def\moodle@answertext{True}%
     \fi
     \ifnum\c@currentitemnumber=2%
	\def\moodle@answertext{False}%
     \fi
     \ifx\moodle@feedback\@empty\relax\else
       \item[\bf \moodle@answertext] ~\hfill \emph{$\rightarrow$ \moodle@feedback}%
     \fi
   }%
   \def\moodle@print@truefalse@answer@int@int@int#1#2\@rdelim{%
     \ifnum\c@currentitemnumber=1%
	\def\moodle@answertext{True}%
     \fi
     \ifnum\c@currentitemnumber=2%
	\def\moodle@answertext{False}%
     \fi
     \ifnum\c@currentitemnumber<3%
       \def\test@i{#1}%
       %\trim@spaces@in\test@i
       \ifx\test@i\@star
         \item[\bf \moodle@answertext] $\checkmark$%
         \stepcounter{moodle@numcorrectanswers}
       \else
         \item[\bf \moodle@answertext] ~%
       \fi
       \ifx\moodle@feedback\@empty\relax
         \def\test@ii{#2}
         \trim@spaces@in\test@ii
         \ifx\test@ii\@empty\relax\else
           \ifx\test@i\@star%
             \hfill \emph{$\rightarrow$ #2}%
           \else%
             \hfill \emph{$\rightarrow$ #1#2}%
           \fi
         \fi
       \else
         \hfill \emph{$\rightarrow$ \moodle@feedback}%
       \fi
     \fi
   }%

% SAVING ANSWERS TO MEMORY
\def\savetruefalseanswer#1{%
  \bgroup
    \savetruefalseanswer@int#1\moodle@answer@rdelim
    \passvalueaftergroup{\moodle@answers@xml}%
  \egroup
}%
   \newcommand\savetruefalseanswer@int[1][]{%
     \setkeys{moodle}{#1}%
     \savetruefalseanswer@int@int%
   }%
   \def\savetruefalseanswer@int@int#1\moodle@answer@rdelim{%
     \def\test@i{#1}
     \trim@spaces@in\test@i
     \ifx\test@i\@empty\relax
       \savetruefalseanswer@int@int@empty
     \else
       \savetruefalseanswer@int@int@int#1\moodle@answer@rdelim
     \fi
   }%
   \def\savetruefalseanswer@int@int@empty{%
     \setkeys{moodle}{fraction=0}%
     \ifnum\c@currentitemnumber=1%
	\def\moodle@answertext{true}%
     \fi
     \ifnum\c@currentitemnumber=2%
	\def\moodle@answertext{false}%
     \fi
     \ifnum\c@currentitemnumber<3%
       \addto@xml[2]{\moodle@answers@xml}{<answer fraction="\moodle@fraction" format="plain_text">}%
       \addto@xml[4]{\moodle@answers@xml}{  <text>\moodle@answertext</text>}%
       \ifx\moodle@feedback\@empty\relax\else
         \trim@spaces@in\moodle@feedback
         \xa\converttohtmlmacro\xa\moodle@feedback@html\xa{\moodle@feedback}%
         \addto@xml[4]{\moodle@answers@xml}{  <feedback format="html"><text><![CDATA[<p>\moodle@feedback@html</p>]]></text></feedback>}%
       \fi
       \addto@xml[2]{\moodle@answers@xml}{</answer>}%
     \fi
   }%
   \def\savetruefalseanswer@int@int@int#1#2\moodle@answer@rdelim{%
     \def\test@i{#1}%
     \ifx\test@i\@star
       \setkeys{moodle}{fraction=100}%
     \else
       \setkeys{moodle}{fraction=0}%
     \fi
     \ifnum\c@currentitemnumber=1%
	\def\moodle@answertext{true}%
     \fi
     \ifnum\c@currentitemnumber=2%
	\def\moodle@answertext{false}%
     \fi
     \ifnum\c@currentitemnumber<3%
       \addto@xml[2]{\moodle@answers@xml}{<answer fraction="\moodle@fraction" format="plain_text">}%
       \addto@xml[4]{\moodle@answers@xml}{  <text>\moodle@answertext</text>}%
       \ifx\moodle@feedback\@empty\relax
         \def\test@ii{#2}
         \ifx\test@ii\@empty\relax\else
           \ifx\test@i\@star
             \xa\converttohtmlmacro\xa\moodle@feedback@html\xa{#2}%
           \else%
             \xa\converttohtmlmacro\xa\moodle@feedback@html\xa{#1#2}%
           \fi%
           \addto@xml[4]{\moodle@answers@xml}{  <feedback format="html"><text><![CDATA[<p>\moodle@feedback@html</p>]]></text></feedback>}%
         \fi
       \else
         \trim@spaces@in\moodle@feedback
         \xa\converttohtmlmacro\xa\moodle@feedback@html\xa{\moodle@feedback}%
         \addto@xml[4]{\moodle@answers@xml}{  <feedback format="html"><text><![CDATA[<p>\moodle@feedback@html</p>]]></text></feedback>}%
       \fi
       \addto@xml[2]{\moodle@answers@xml}{</answer>}%
     \fi
   }%

% WRITING QUESTION TO XML FILE
\gdef\writetruefalsequestion{%
  \writetomoodle{<question type="truefalse">}%
    \moodle@writecommondata%
}%
%    \end{macrocode}
%
% \subsubsection{Matching Question Front-End}
% 
%    \begin{macrocode}
\let\answer=\hfill

\moodle@makefrontend{matching}{\moodle@makelatextag@matching}

% LATEX PROCESSING

\def\moodle@makelatextag@matching{%
  \csname ifmoodle@draganddrop\endcsname
    \moodle@makelatextagbox{Drag and drop}\relax
  \fi
  \csname ifmoodle@shuffle\endcsname
    \moodle@makelatextagbox{Shuffle}\relax
  \fi
}

\def\moodle@matching@latexprocessing{%
  \bgroup
    \let\answer=\hfill
    \begin{enumerate}\renewcommand{\theenumi}{\alph{enumi}}\setlength\itemsep{0pt}\setlength\parskip{0pt}%
      \loopthroughitemswithcommand{\moodle@print@matching@answer}%
    \end{enumerate}%
    \ifx\moodle@generalfeedback\@empty\relax\else%
      \fbox{\parbox{\linewidth}{\emph{\moodle@generalfeedback}}}%
    \fi
  \egroup
}

\long\def\moodle@print@matching@answer#1{%
  \moodle@print@matching@answer@int#1\@rdelim
}%
\newcommand\moodle@print@matching@answer@int[1][]{%
  \moodle@print@matching@answer@int@int\relax
}%
\long\def\moodle@print@matching@answer@int@int#1\answer#2\@rdelim{%
  \item #1\hfill #2%
}%


% SAVING ANSWERS TO MEMORY
\long\def\savematchinganswer#1{%
  \bgroup
    \savematchinganswer@int#1\moodle@answer@rdelim%
    \passvalueaftergroup{\moodle@answers@xml}%
  \egroup
}%
   \newcommand\savematchinganswer@int[1][]{%
     \setkeys{moodle}{#1}%
     \xa\savematchinganswer@int@int\space%
   }%
   \long\def\savematchinganswer@int@int#1\answer#2\moodle@answer@rdelim{%
     % Note that #1 may simply be \relax.
     \def\moodle@subquestiontext{#1}%
     \def\moodle@subanswertext{#2}%
     \trim@spaces@in\moodle@subquestiontext
     \xa\converttohtmlmacro\xa\moodle@subquestiontext@htmlized\xa{\moodle@subquestiontext}%
     \trim@spaces@in\moodle@subanswertext
     \ifmoodle@draganddrop
       \xa\converttohtmlmacro\xa\moodle@subanswertext@htmlized\xa{\moodle@subanswertext}%
     \fi
     \addto@xml[2]{\moodle@answers@xml}{<subquestion format="html">}%
     \ifx\moodle@subquestiontext\@empty
       \addto@xml[4]{\moodle@answers@xml}{  <text></text>}%
     \else
       \addto@xml[4]{\moodle@answers@xml}{  <text><![CDATA[<p>\moodle@subquestiontext@htmlized</p>]]></text>}%
     \fi
     \ifmoodle@draganddrop
%       \show\moodle@subanswertext@htmlized
       \addto@xml[4]{\moodle@answers@xml}{  <answer format="html"><text><![CDATA[<p>\moodle@subanswertext@htmlized</p>]]></text></answer>}%
     \else
%       \show\moodle@subanswertext
       \addto@xml[4]{\moodle@answers@xml}{  <answer><text>\moodle@subanswertext</text></answer>}%
     \fi
%     \ifx\moodle@feedback\@empty\relax\else
%       \trim@spaces@in\moodle@feedback
%       \xa\converttohtmlmacro\xa\moodle@feedback@html\xa{\moodle@feedback}%
%       \addto@xml[4]{\moodle@answers@xml}{  <feedback 
%format="html"><text><![CDATA[<p>\moodle@feedback@html</p>]]></text></feedback>}%
%     \fi
     \addto@xml[2]{\moodle@answers@xml}{</subquestion>}%
   }%

% WRITING QUESTION TO XML FILE
\gdef\writematchingquestion{%
  \ifmoodle@draganddrop
    \writetomoodle{<question type="ddmatch">}%
  \else
    \writetomoodle{<question type="matching">}%
  \fi
    \moodle@writecommondata%
%    \moodle@writesingle% %unappropriate for the matching type
    \moodle@writeshuffle%
%    \moodle@writeanswernumbering% %unappropriate for the matching type
}%
%    \end{macrocode}
%
% \subsection{Cloze Questions}
% Because cloze questions are so complicated, they get their own section of code.
% The cloze strategy is as follows.
% 
% All subquestions show up as part of the question body text.
% For each type of subquestion, we have a cloze-version environment
% that actually has 2 versions, depending on whether we are doing LaTeX or XML processing.
% So the main environment is quite typical: 
% \begin{enumerate}
%   \item Process the body as LaTeX.
%         During this run, a |\begin{multi}| etc.~will be processed for display onscreen.
%   \item Then save the body as the questiontext for XML.
%         During this run, a |\begin{multi}| etc.~will be parsed and turned into
%         cloze code as part of the XML questiontext.
% \end{enumerate}
%    \begin{macrocode}
% LATEX PROCESSING
% SAVING ANSWERS TO MEMORY

\newif\ifmoodle@clozemode
\moodle@clozemodefalse
\NewEnviron{cloze}[2][]{%
  \bgroup
    \setkeys{moodle}{default grade=1}%
    \setkeys{moodle}{#1,questionname={#2}}%
    % A cloze question won't have any \item's in it, so we just use \BODY.
    \moodle@enableclozeenvironments
    %First, the LaTeX processing.
      \item \textbf{\moodle@questionname}
      \xa\xdef\xa\test@iii\xa{\moodle@tags}%
      \ifx\test@iii\@moodle@empty\relax\else
        \hfill tags: \texttt{\moodle@tags}
      \fi
      \par
      \noindent
      \moodle@makelatextag@qtype{cloze}
      \moodle@makelatextag@value{default grade}{\moodle@marks{\csname moodle@default grade\endcsname}}
      \moodle@makelatextag@value{penalty}{penalty}\par
      \noindent
      \BODY
      \edef\moodle@generalfeedback{\expandonce\moodle@feedback}
      %\csname moodle@cloze@latexprocessing\endcsname
      \ifx\moodle@generalfeedback\@empty\relax\else%
        \fbox{\parbox{\linewidth}{\emph{\moodle@generalfeedback}}}%
      \fi
    %Now, writing information to memory.
    \@moodle@ifgeneratexml{%
      \xa\questiontext\xa{\BODY}% Save the question text as HTML.
      \writeclozequestion
    }{}%
  \egroup%
}

\def\moodle@enableclozeenvironments{%
  \let\multi=\clozemulti
  \let\endmulti=\endclozemulti
  \let\numerical=\clozenumerical
  \let\endnumerical=\endclozenumerical
  \let\shortanswer=\clozeshortanswer
  \let\endshortanswer=\endclozeshortanswer
}

% WRITING QUESTION TO XML FILE
\gdef\writeclozequestion{%
  \writetomoodle{<question type="cloze">}%
    \moodle@writecommondata%
    \moodle@writetags%
  \writetomoodle{</question>}%
}%
%    \end{macrocode}
%
% \subsubsection{Cloze Multiple Choice Questions}
% 
%    \begin{macrocode}

\NewEnviron{clozemulti}[1][]{%
  \bgroup
    \setkeys{moodle}{default grade=1}%
    \setkeys{moodle}{#1}%
    \expandafter\gatheritems\xa{\BODY}%
    \let\moodle@questionheader=\gatheredheader
    \ifhtmlizer@active
      %HTML version
      \def\moodle@clozemulti@output{}%
      \xa\g@addto@macro\xa\moodle@clozemulti@output\xa{\moodle@questionheader}%
      \def\clozemulti@coding{}%
      \edef\clozemulti@coding{\csname moodle@default grade\endcsname:}%
      \ifmoodle@multiple
        \PackageWarning{moodle}{Cloze Multiresponse only supported by Moodle 3.5+}
        \g@addto@macro{\clozemulti@coding}{MULTIRESPONSE}%
      \else
        \g@addto@macro{\clozemulti@coding}{MULTICHOICE}%
      \fi
      \ifcase\moodle@multi@mode\relax
         % Case 0: dropdown box style
         \ifmoodle@shuffle
           \g@addto@macro{\clozemulti@coding}{_}%
         \fi
      \or
        % Case 1: vertical style
        \ifmoodle@multiple
          \PackageError{moodle}{Vertical mode (dropdown box) incompatible with multiresponse.}
        \else
          \g@addto@macro{\clozemulti@coding}{_V}%
        \fi
      \else
        % Case 2: horizontal radio buttons
        \g@addto@macro{\clozemulti@coding}{_H}%
      \fi
      \ifmoodle@shuffle
        \PackageWarning{moodle}{Cloze Multi Shuffling only supported by Moodle 3.0+}
        \g@addto@macro{\clozemulti@coding}{S:}%
      \else
        \g@addto@macro{\clozemulti@coding}{:}%
      \fi
      \bgroup
        \setkeys{moodle}{feedback={}}%
        \loopthroughitemswithcommand{\saveclozemultichoiceanswer}%
      \egroup
      %\xa\g@addto@macro\xa\clozemulti@coding\xa{\clozerbrace}%
      \xa\g@addto@macro\xa\moodle@clozemulti@output\xa{\xa\clozelbrace\clozemulti@coding\clozerbrace}%
      %\show\moodle@clozemulti@output
      \xa\gdef\xa\htmlize@afteraction@hook\xa{\moodle@clozemulti@output}%
      \def\endclozemulti@code{\htmlize@patchendenvironment}%
    \else
      %LaTeX version
      \moodle@questionheader% %Any introductory text just continues to be typeset.
      \par
      \noindent
      \moodle@makelatextag@qtype{multi}
      \moodle@makelatextag@value{default grade}{\moodle@marks{\csname moodle@default grade\endcsname}}
      \moodle@makelatextag@multi
      \def\cloze@multichoice@table@text{}%
      %\let\moodle@feedback=\@empty
      \loopthroughitemswithcommand{\moodle@print@clozemultichoice@answer}%
      \ifcase\moodle@multi@mode\relax
        %Case 0: dropdown box style
        \par\noindent
        \begin{tabular}[t]{|p{.45\linewidth}|p{.45\linewidth}|}
          \firsthline% (\firsthline is from the array package.)
%          answer & feedback \\\hline\hline
          \cloze@multichoice@table@text%
        \end{tabular}%
        \par%
      \or
        %Case 1: vertical style
        \par\noindent
        \begin{itemize}\setlength\itemsep{0pt}\setlength\parskip{0pt}%
          \cloze@multichoice@table@text%
        \end{itemize}%
        \par%
      \else
        %Case 2: horizontal radio buttons
        \par{\cloze@multichoice@table@text}\par%
      \fi
      \def\endclozemulti@code{\relax}%
    \fi
    \passvalueaftergroup\endclozemulti@code%
    \passvalueaftergroup\htmlize@afteraction@hook%
  \egroup%
}[\endclozemulti@code]%


\def\moodle@print@clozemultichoice@answer#1{%
  \let\moodle@feedback=\@empty
  \moodle@print@clozemultichoice@answer@int#1 \@rdelim%
}%
\newcommand\moodle@print@clozemultichoice@answer@int[1][]{%
  \setkeys{moodle}{fraction=0,#1}%
  \moodle@print@clozemultichoice@answer@int@int%
}%
\def\moodle@print@clozemultichoice@answer@int@int#1#2\@rdelim{%
  % Case 0: "(answer) \\ \hline"
  % Case 1: "\item (answer)"
  % Case 2: "$\bullet~$(answer)\hfill"
  \ifcase\moodle@multi@mode\relax
    \relax% Case 0
  \or
    \g@addto@macro\cloze@multichoice@table@text{\item}% Case 1
  \else
    \g@addto@macro\cloze@multichoice@table@text{$\bullet~$}% Case 2
  \fi
  \def\test@i{#1}%
  \ifx\test@i\@star
    \setkeys{moodle}{fraction=100}%
    \g@addto@macro\cloze@multichoice@table@text{#2}%
  \else
    \g@addto@macro\cloze@multichoice@table@text{#1#2}%
  \fi
  \trim@spaces@in\cloze@multichoice@table@text
  \trim@spaces@in\cloze@multichoice@table@text
  \ifx\moodle@fraction\@hundred
    \g@addto@macro\cloze@multichoice@table@text{$~\checkmark$}%
  \else
    \moodle@checkfraction
    \ifdim0pt=\moodle@fraction pt\relax\else
      \xdef\cloze@multichoice@table@text{\expandonce\cloze@multichoice@table@text$~(\moodle@fraction\%)$}%
    \fi
  \fi
  \ifcase\moodle@multi@mode\relax
    % Case 0
      \xdef\cloze@multichoice@table@text{\expandonce\cloze@multichoice@table@text &\expandonce\emph{\expandonce\moodle@feedback}}%
      \g@addto@macro{\cloze@multichoice@table@text}{\\\hline}
  \or % Case 1
    \ifx\moodle@feedback\@empty\relax\else
      \xdef\cloze@multichoice@table@text{\expandonce\cloze@multichoice@table@text \hfill \expandonce\emph{$\rightarrow$ \expandonce\moodle@feedback}}%
    \fi
  \else % otherwise
    \ifx\moodle@feedback\@empty\relax\else
      \xdef\cloze@multichoice@table@text{\expandonce\cloze@multichoice@table@text\,\expandonce\emph{$\rightarrow$ \expandonce\moodle@feedback}}%
    \fi
    \g@addto@macro{\cloze@multichoice@table@text}{\hfill}% Case 2
  \fi
}%

\def\saveclozemultichoiceanswer#1{%
  \bgroup
    \saveclozemultichoiceanswer@int#1 \moodle@answer@rdelim
  \egroup
}%
\newcommand\saveclozemultichoiceanswer@int[1][]{%
  \setkeys{moodle}{fraction=0,#1}%
  \saveclozemultichoiceanswer@int@int%
}%
\def\saveclozemultichoiceanswer@int@int#1#2\moodle@answer@rdelim{%
  \def\test@i{#1}%
  \ifgatherbeginningofloop\else
    \xa\gdef\xa\clozemulti@coding\xa{\clozemulti@coding\clozetilde}% separator between answers
  \fi
  \ifx\test@i\@star
    \setkeys{moodle}{fraction=100}%
    \def\moodle@answertext{#2}%
  \else
    \def\moodle@answertext{#1#2}%
  \fi
  \trim@spaces@in\moodle@answertext
  \trim@spaces@in\moodle@answertext
  \ifx\moodle@fraction\@hundred
    \g@addto@macro\clozemulti@coding{=}%
  \else
    \moodle@checkfraction
    \ifdim0pt=\moodle@fraction pt\relax\else
      \xdef\clozemulti@coding{\expandonce\clozemulti@coding\otherpercent\moodle@fraction\otherpercent}%
    \fi
  \fi
  \xdef\clozemulti@coding{\expandonce\clozemulti@coding\expandonce\moodle@answertext}%
  \ifx\moodle@feedback\@empty\else
    \xdef\clozemulti@coding{\expandonce\clozemulti@coding\otherbackslash\otherhash\expandonce\moodle@feedback}%
  \fi
}%
%    \end{macrocode}
%
% \subsubsection{Cloze Numerical Questions}
% 
%    \begin{macrocode}
\NewEnviron{clozenumerical}[1][]{%
  \bgroup
    \expandafter\gatheritems\expandafter{\BODY}%
    \let\moodle@questionheader=\gatheredheader
    \setkeys{moodle}{default grade=1}%
    \setkeys{moodle}{#1}%
    \ifhtmlizer@active
      %HTML version
      \def\moodle@clozenumerical@output{}%
      \xa\g@addto@macro\xa\moodle@clozenumerical@output\xa{\moodle@questionheader}%
      \def\clozenumerical@coding{}%
      \edef\clozenumerical@coding{\csname moodle@default grade\endcsname:NUMERICAL:}%
      \bgroup
        \setkeys{moodle}{feedback={}}%
        \loopthroughitemswithcommand{\saveclozenumericalanswer}%
      \egroup
      %\xa\g@addto@macro\xa\clozenumerical@coding\xa{\otherrbrace}%
      \xa\g@addto@macro\xa\moodle@clozenumerical@output\xa{\xa\clozelbrace\clozenumerical@coding\clozerbrace}%
      \xa\gdef\xa\htmlize@afteraction@hook\xa{\moodle@clozenumerical@output}%
      \def\endclozenumerical@code{\htmlize@patchendenvironment}%
    \else
      %LaTeX version
      \moodle@questionheader% %Any introductory text just continues to be typeset.
      \par
      \noindent
      \moodle@makelatextag@qtype{numerical}
      \moodle@makelatextag@value{default grade}{\moodle@marks{\csname moodle@default grade\endcsname}}
      \moodle@makelatextag@numerical
      \par
      \noindent
      \def\cloze@numerical@table@text{}%
      \loopthroughitemswithcommand{\moodle@print@clozenumerical@answer}%
      \begin{tabular}[t]{|p{.45\linewidth}|p{.45\linewidth}|}
        \firsthline% (\firsthline is from the array package.)
%       answer & feedback \\\hline\hline
        \cloze@numerical@table@text%
      \end{tabular}%
      \par%
      \def\endclozenumerical@code{\relax}%
    \fi
    \passvalueaftergroup\endclozenumerical@code%
    \passvalueaftergroup\htmlize@afteraction@hook%
  \egroup
}[\endclozenumerical@code]%

\def\moodle@print@clozenumerical@answer#1{%
  \let\moodle@feedback=\@empty
  \bgroup
    \moodle@print@clozenumerical@answer@int#1\@rdelim
  \egroup
}%
\newcommand\moodle@print@clozenumerical@answer@int[1][]{%
  \setkeys{moodle}{#1}%
  \moodle@print@clozenumerical@answer@int@int%
}%
\def\moodle@print@clozenumerical@answer@int@int#1\@rdelim{%
  \ifx\moodle@fraction\@hundred
    \def\moodle@clozenumericalprint@fraction{$~\checkmark$}%
  \else
    \moodle@checkfraction
    \edef\moodle@clozenumericalprint@fraction{$(~\moodle@fraction\%)$}%
  \fi
  \ifdim0pt=\moodle@tolerance pt\relax
    \def\moodle@clozenumericalprint@tolerance{}%
  \else
    \edef\moodle@clozenumericalprint@tolerance{\noexpand\pm\moodle@tolerance}%
  \fi
  \xdef\moodle@clozenumericalprint@line{$#1\moodle@clozenumericalprint@tolerance$~\moodle@clozenumericalprint@fraction & \expandonce\emph{\expandonce\moodle@feedback}}%
  \xa\g@addto@macro\xa\cloze@numerical@table@text\xa{\moodle@clozenumericalprint@line \\\hline}%
}%



\def\saveclozenumericalanswer#1{%
  \bgroup
    \saveclozenumericalanswer@int#1\moodle@answer@rdelim
  \egroup
}%
\newcommand\saveclozenumericalanswer@int[1][]{%
  \setkeys{moodle}{fraction=100,#1}%                  %%%%%% DEFAULT VALUE IS 100%
  \saveclozenumericalanswer@int@int%
}%
\def\saveclozenumericalanswer@int@int#1\moodle@answer@rdelim{%
  \ifgatherbeginningofloop\else
    \xa\gdef\xa\clozenumerical@coding\xa{\clozenumerical@coding\clozetilde}% separator between answers
  \fi
  \def\moodle@answertext{#1}%
  \trim@spaces@in\moodle@answertext
  \ifx\moodle@fraction\@hundred
    \g@addto@macro\clozenumerical@coding{=}%
  \else
    \moodle@checkfraction
    \ifdim0pt=\moodle@fraction pt\relax\else
      \xdef\clozenumerical@coding{\expandonce\clozenumerical@coding\otherpercent\moodle@fraction\otherpercent}%
    \fi
  \fi
  \xdef\clozenumerical@coding{\expandonce\clozenumerical@coding\moodle@answertext:\moodle@tolerance}%
  \ifx\moodle@feedback\@empty\else
    %\trim@spaces@in\moodle@feedback
    \xdef\clozenumerical@coding{\expandonce\clozenumerical@coding\otherbackslash\otherhash\expandonce\moodle@feedback}%
  \fi
}%
%    \end{macrocode}
%
% \subsubsection{Cloze Short Answer Questions}
% 
%    \begin{macrocode}
\NewEnviron{clozeshortanswer}[1][]{%
  \bgroup
    \expandafter\gatheritems\expandafter{\BODY}%
    \let\moodle@questionheader=\gatheredheader
    \setkeys{moodle}{default grade=1}%
    \setkeys{moodle}{#1}%
    %Because nesting conditionals built by \csname doesn't work well,
    %we'll test '\ifmoodle@case sensitive' now and save the result in \count0.
    \csname ifmoodle@case sensitive\endcsname
      \count0=1\relax
    \else
      \count0=0\relax
    \fi
    \ifhtmlizer@active
      %HTML version
      \def\moodle@clozeshortanswer@output{}%
      \xa\g@addto@macro\xa\moodle@clozeshortanswer@output\xa{\moodle@questionheader}%
      \def\clozeshortanswer@coding{}%
      \ifnum\count0=1\relax
        \edef\clozeshortanswer@coding{\csname moodle@default grade\endcsname:SHORTANSWER_C:}%
      \else
        \edef\clozeshortanswer@coding{\csname moodle@default grade\endcsname:SHORTANSWER:}%
      \fi
      \bgroup
        \setkeys{moodle}{feedback={}}%
        \loopthroughitemswithcommand{\saveclozeshortansweranswer}%
      \egroup
      %\xa\g@addto@macro\xa\clozeshortanswer@coding\xa{\otherrbrace}%
      \xa\g@addto@macro\xa\moodle@clozeshortanswer@output\xa{\xa\clozelbrace\clozeshortanswer@coding\clozerbrace}%
      \xa\gdef\xa\htmlize@afteraction@hook\xa{\moodle@clozeshortanswer@output}%
      \def\endclozeshortanswer@code{\htmlize@patchendenvironment}%
    \else
      %LaTeX version
      \moodle@questionheader% %Any introductory text just continues to be typeset.
      \par
      \noindent
      \moodle@makelatextag@qtype{shortanswer}
      \moodle@makelatextag@value{default grade}{\moodle@marks{\csname moodle@default grade\endcsname}}
      \moodle@makelatextag@shortanswer
      \par
      \noindent
      \def\cloze@shortanswer@table@text{}%
      \loopthroughitemswithcommand{\moodle@print@clozeshortanswer@answer}%
      \begin{tabular}[t]{|p{.45\linewidth}|p{.45\linewidth}|}
        \firsthline% (\firsthline is from the array package.)
%       answer & feedback \\\hline\hline
        \cloze@shortanswer@table@text%
      \end{tabular}%
      \par%
      \def\endclozeshortanswer@code{\relax}%
    \fi
    \passvalueaftergroup\endclozeshortanswer@code%
    \passvalueaftergroup\htmlize@afteraction@hook%
  \egroup
}[\endclozeshortanswer@code]%

\def\moodle@print@clozeshortanswer@answer#1{%
  \let\moodle@feedback=\@empty
  \bgroup
    \moodle@print@clozeshortanswer@answer@int#1\@rdelim
  \egroup
}%
\newcommand\moodle@print@clozeshortanswer@answer@int[1][]{%
  \setkeys{moodle}{#1}%
  \moodle@print@clozeshortanswer@answer@int@int%
}%
\def\moodle@print@clozeshortanswer@answer@int@int#1\@rdelim{%
  \ifx\moodle@fraction\@hundred
    \def\moodle@clozeshortanswerprint@fraction{$~\checkmark$}%
  \else
    \moodle@checkfraction
    \edef\moodle@clozeshortanswerprint@fraction{$~(\moodle@fraction\%)$}%
  \fi
  \xdef\moodle@clozeshortanswerprint@line{#1~\moodle@clozeshortanswerprint@fraction & \expandonce\emph{\expandonce\moodle@feedback}}%
  \xa\g@addto@macro\xa\cloze@shortanswer@table@text\xa{\moodle@clozeshortanswerprint@line \\\hline}%
}%



\def\saveclozeshortansweranswer#1{%
  \bgroup
    \saveclozeshortansweranswer@int#1\moodle@answer@rdelim
  \egroup
}%
\newcommand\saveclozeshortansweranswer@int[1][]{%
  \setkeys{moodle}{fraction=100,#1}%                  %%%%%% DEFAULT VALUE IS 100%
  \saveclozeshortansweranswer@int@int%
}%
\def\saveclozeshortansweranswer@int@int#1\moodle@answer@rdelim{%
  \ifgatherbeginningofloop\else
    \xa\gdef\xa\clozeshortanswer@coding\xa{\clozeshortanswer@coding\clozetilde}% separator between answers
  \fi
  \def\moodle@answertext{#1}%
  \trim@spaces@in\moodle@answertext
  \ifx\moodle@fraction\@hundred
    \g@addto@macro\clozeshortanswer@coding{=}%
  \else
    \moodle@checkfraction
    \ifdim0pt=\moodle@fraction pt\relax\else
      \xdef\clozeshortanswer@coding{\expandonce\clozeshortanswer@coding\otherpercent\moodle@fraction\otherpercent}%
    \fi
  \fi
  \xdef\clozeshortanswer@coding{\expandonce\clozeshortanswer@coding\moodle@answertext}%
  \ifx\moodle@feedback\@empty\else
    \xdef\clozeshortanswer@coding{\expandonce\clozeshortanswer@coding\otherbackslash\otherhash\expandonce\moodle@feedback}%
  \fi
}%
%    \end{macrocode}
%
% \section{Converting \LaTeX\ to HTML}
% A lot of work must now be done to convert the \LaTeX\ code 
% of a question or answer into HTML code with embedded \TeX\ for math.
% 
% \subsection{Catcode Setup}
% First, we create versions of the special characters with catcode 12, ``other.''
%    \begin{macrocode}
{\catcode`\#=12\gdef\otherhash{#}%
 \catcode`\~=12\gdef\othertilde{~}%
 \catcode`\&=12\gdef\otherampersand{&}%
 \catcode`\^=12\gdef\othercaret{^}%
 \catcode`\$=12\gdef\otherdollar{$}%
 \catcode`\%=12\gdef\otherpercent{%}
 \catcode`\%=12\gdef\otherlbracket{[}
 \catcode`\%=12\gdef\otherrbracket{]}}%
{\catcode`\ =12\gdef\otherspace{ }}%
{\ttfamily\catcode`\|=0\catcode`\\=12\relax|gdef|otherbackslash{\}}%
{\catcode`\[=1\catcode`\]=2\catcode`\{=12\catcode`\}=12%
 \gdef\otherlbrace[{]\gdef\otherrbrace[}]\gdef\clozelbrace[{]\gdef\clozerbrace[}]]

\edef\@otherlbrace{\otherlbrace}%
\edef\@otherrbrace{\otherrbrace}%
\edef\@otherlbracket{\otherlbracket}%
\edef\@otherrbracket{\otherrbracket}%
\edef\@clozelbrace{\clozelbrace}%
\edef\@clozerbrace{\clozerbrace}%
\edef\@otherdollar{\otherdollar}%
\edef\@otherbackslash{\otherbackslash}%
\edef\@othertilde{\othertilde}%
%    \end{macrocode}
%
% Next, we define commands to change catcodes to a suitable verbatim mode
% for transcription.
% 
%    \begin{macrocode}
{ \catcode`\[=1\relax
  \catcode`\]=2\relax
  \catcode`\|=0\relax
  |gdef|verbcatcodesweirdest[
    |catcode`\{=12|relax
    |catcode`\}=12|relax
    |catcode`\\=12|relax
  ]%
}%
\def\verbcatcodes{%
  \catcode`\$=12\relax
  \catcode`\&=12\relax
  \catcode`\#=12\relax
  \catcode`\^=12\relax
  \catcode`\_=12\relax
  \catcode`\~=12\relax
  \makeatletter
  \catcode`\%=12\relax
  \catcode`\ =12\relax\catcode\newlinechar=12\verbcatcodesweirdest}%

\def\normalcatcodes{%
  \catcode`\\=0\relax
  \catcode`\{=1\relax
  \catcode`\}=2\relax
  \catcode`\$=3\relax
  \catcode`\&=4\relax
  \catcode\endlinechar=5\relax
  \catcode`\#=6\relax
  \catcode`\^=7\relax
  \catcode`\_=8\relax
  \catcode`\ =10\relax
  \makeatletter% We will be detokenizing and retokenizing internal control sequences, so we need this.
  \catcode`\~=13\relax
  \catcode`\%=14\relax}%

\def\retokenizingcatcodes{%
  %For rescanning previously scanned text, all true comments will already be gone,
  %but % signs may have been inserted by Cloze questions, so we want to treat them as 'other.'
  %
  % TODO: #'s are more worrisome.
  \normalcatcodes
  \catcode`\%=12\relax
}
%    \end{macrocode}
%
% \subsection{Detokenization and Retokenization}
% 
% We will be processing a \TeX\ token list.
% Based on its content, sometimes we will want it to be detokenized to 
% individual characters, but other times we want it retokenized so that
% \TeX's own parsing mechanism can gather up the parameters of macros.
% We use the e\TeX\ primitive command |\scantokens| to do this.
% 
% The following code (catcodes, groupings and all) defines a |\scantokens@to@macro| macro.
% That will assemble and disassemble strings of tokens
% using any changing schemes of catcodes we desire.
% 
% We define |^^A| to be |\gdef\stm@saved|, while |^^B| and |^^C| are substitutes for |{| and |}|, respectively.
% This permits us to define |\scantokens@to@macro| in a peculiar catcode regime.
%    \begin{macrocode}
\begingroup
  \catcode`\^^A=13\gdef^^A{\gdef\stm@saved}%
  \catcode`\^^B=1\catcode`\^^C=2\relax
  \long\gdef\scantokens@to@macro#1#2#3{%
    % #1 = control sequence to be defined
    % #2 = command to change catcodes, e.g. \verbcatcodes,
    %      and define any command sequences
    % #3 = text to be retokenized and saved into #1.
    \bgroup
      \def\texttorescan{#3}%
      \catcode`\^^A=13\catcode`\^^B=1\catcode`\^^C=2\relax
      \xa\def\xa\arg\xa{\xa^^A\xa^^B\texttorescan^^C}%
      #2%
      \catcode\endlinechar=9\relax% 
      %\scantokens always sees an end-of-line character at its end and converts it to a space.
      %The catcode change sets \scantokens to ignore end-of-line chars.
      %In practice, we're always calling \scantokens on previously scanned text anyway,
      %so we won't miss any real end-of-line chars, since they were already converted to spaces.
      \xa\scantokens\xa{\arg}%
    \egroup
    \xa\def\xa#1\xa{\stm@saved}%
  }%
\endgroup%

\long\def\ultradetokenize@to@macro#1#2{%
  \scantokens@to@macro#1{\verbcatcodes}{#2}%
}%
\def\retokenizenormal@to@macro#1#2{%
  \scantokens@to@macro#1{\retokenizingcatcodes}{#2}%
}%
%    \end{macrocode}
%
% \subsection{Level-Tracking}
%
% \subsubsection{TeX groups}
% While parsing, we'll need to keep track of how deeply nested in \TeX\ groups we are.
%    \begin{macrocode}
\newcount\grouplevel
%    \end{macrocode}
%
% \subsubsection{Math mode}
% While parsing, we'll need to keep track of whether
% we are in math mode (and how many levels deep the math mode might be nested).
%    \begin{macrocode}
\newcount\moodle@mathmodedepth
\moodle@mathmodedepth=0\relax
\def\moodle@ifmathmode#1#2{%
  \ifnum\moodle@mathmodedepth>0\relax
    #1%
  \else
    #2%
  \fi
}%
%    \end{macrocode}
%
% \subsubsection{Nested Lists}
% While parsing, we'll need to keep track levels of nested list.
%    \begin{macrocode}
\newcount\moodle@listdepth
\moodle@listdepth=0\relax

%    \end{macrocode}
%
% \subsection{Separation}
% 
% This code separates a string of tokens into two parts.
% Its parameters, |#1#2|, consist of tokenized text, 
% plus one terminal |\@htmlize@stop|.
% We are trying to break up the text into its first group and the remainder.
% This |\@htmlize@stop| is needed in case |#2| has the form ``|{...}|'',
% since we don't want \TeX\ to strip the braces off.
% Thus |\@htmlize@remainder| will definitely end in ``|\@htmlize@stop|''.
%    \begin{macrocode}
\long\def\htmlize@grabblock#1#2\htmlize@rdelim@ii{%
  \long\def\htmlize@blockinbraces{#1}%
  \long\def\htmlize@remainder{#2}%
}%
%    \end{macrocode}
%
% The next line defines the macro |\@htmlize@stop@detokenized| to contain
% the string of tokens |\@htmlize@stop|, all of category code 12 (other) or 10 (letter).
% We'll need this for comparison purposes later.
%    \begin{macrocode}
\ultradetokenize@to@macro\@htmlize@stop@detokenized{\@htmlize@stop}%
%    \end{macrocode}
%
% The next line creates the macro |\htmlize@remove@stopcode|,
% which removes the characters ``|\@htmlize@stop |'' from the end of a 
% detokenized sequence.
% Its syntax when called is simply |\htmlize@remove@stopcode |\meta{material},
% with no delimiters, since the ``|\@htmlize@stop |'' is itself the delimiter.
%    \begin{macrocode}
\xa\def\xa\htmlize@remove@stopcode\xa#\xa1\@htmlize@stop@detokenized{#1}%
%    \end{macrocode}
% 
% \subsection{Main Code: the HTMLizer}
% 
%    \begin{macrocode}
\newif\ifhtmlizer@active
\htmlizer@activefalse
\newif\ifhtmlize@actioncs
\newif\ifhtmlize@expandcs
\newif\ifhtmlize@passcs

\long\def\@@begin@cs{\begin}%
\def\@@htmlize@stop{\@htmlize@stop}%

\long\def\converttohtmlmacro#1#2{%
  \grouplevel=0\relax
  \def\htmlize@output{}%
  \htmlizer@activetrue%
  \converttohtml@int{#2}%
  \htmlizer@activefalse%
  \let#1=\htmlize@output\relax
}

\long\def\converttohtml@int#1{%
  \advance\grouplevel by 1\relax
  \bgroup
    \ultradetokenize@to@macro\htmlize@texttoscan{#1}%
    \xa\htmlize@recursive@i\htmlize@texttoscan\@htmlize@stop\@htmlize@stop\@htmlize@stop\htmlize@rdelim@i%
  \egroup
  \advance\grouplevel by -1\relax
}%

\def\@lt{<}%
\def\@gt{>}%

\long\def\htmlize@recursive@i#1#2#3\htmlize@rdelim@i{%
  % #1#2#3 is a sequence of tokens.  All should be categories 11 (letter) or 12 (other).
  % It terminates with the control sequences \@htmlize@stop\@htmlize@stop\@htmlize@stop.
  %\long\def\ds{(#1|#2|#3)}\show\ds
  \def\test@i{#1}%
  \def\test@ii{#2}%
  \ifx\test@i\@@htmlize@stop
    \let\htmlize@next@i=\relax
  \else
    \ifx\test@i\@otherlbrace%
      \xa\g@addto@macro\xa\htmlize@output\xa{\otherlbrace}%
      \bgroup
        \normalcatcodes
        %We need to rescan the input as TeX code,
        % so TeX can automatically pull off the first group in braces.
        % First, let's get rid of the terminal \@htmlize@stop codes.
        {\def\@htmlize@stop{}\xdef\htmlize@scrap{#1#2#3}}%
        \let\htmlize@text@to@rescan=\htmlize@scrap%
        % Next, we retokenize the code.
        \xa\retokenizenormal@to@macro\xa\htmlize@rescanned\xa{\htmlize@text@to@rescan}%
        % Now break it up into two pieces.
        \xa\htmlize@grabblock\htmlize@rescanned\@htmlize@stop\htmlize@rdelim@ii%
        % The first piece, \htmlize@blockinbraces, will be passed as a unit to \converttohtml@int.
        % The second part, \htmlize@remainder, will continue at this depth of grouping.
        % Therefore we'll detokenize \htmlize@remainder here.
        \xa\ultradetokenize@to@macro\xa\htmlize@remainder@detokenized\xa{\htmlize@remainder}%
        \edef\htmlize@remainder@detokenized{\xa\htmlize@remove@stopcode\htmlize@remainder@detokenized}%
        % 
        % Now build \htmlize@next@i.
        % When done, should look like
        %   \converttohtml@int{\htmlize@blockinbraces}%
        %   \g@addto@macro\htmlize@output{\otherrbrace}%
        %   \htmlize@recursive@i\htmlize@remainder@detokenized\@htmlize@stop\@htmlize@stop\@htmlize@stop\htmlize@rdelim@i%
        % but with all three arguments expanded.
        % Note that we are running
        \gdef\htmlize@scrap{\converttohtml@int}%
        \xa\g@addto@macro\xa\htmlize@scrap\xa{\xa{\htmlize@blockinbraces}}%
        \g@addto@macro\htmlize@scrap{\g@addto@macro\htmlize@output}%
        \ifmoodle@clozemode
          \xa\g@addto@macro\xa\htmlize@scrap\xa{\xa{\otherbackslash\otherrbrace}}
%          \moodle@ifmathmode{\xa\g@addto@macro\xa\htmlize@scrap\xa{\xa{\otherbackslash}}}%
%                            {}%
        \else
          \xa\g@addto@macro\xa\htmlize@scrap\xa{\xa{\otherrbrace}}%
        \fi
        \g@addto@macro\htmlize@scrap{\htmlize@recursive@i}%
        \xa\g@addto@macro\xa\htmlize@scrap\xa{\htmlize@remainder@detokenized\@htmlize@stop\@htmlize@stop\@htmlize@stop\htmlize@rdelim@i}%
        % Okay, that's done.  It's stored in a global macro.
        % Now we get it out of this group.
      \egroup
      \let\htmlize@next@i=\htmlize@scrap
    \else
      \ifx\test@i\@otherdollar%
        % Math shift character.
        \ifx\test@ii\@otherdollar
          % Double dollar sign, so we're entering display math mode.
          % We grab everything between $$...$$, sanitize it, and add it verbatim to
          % our output.
          \htmlize@displaymathshift@replace#1#2#3\htmlize@rdelim@iii%
        \else
          % Single dollar sign, so we're entering inline math mode.
          % We grab everything between $$...$$, sanitize it, and add it verbatim to
          % our output.
          \htmlize@inlinemathshift@replace#1#2#3\htmlize@rdelim@iii%
        \fi% \ifx\test@ii\@otherdollar
        % Now we resume work.
        % The \htmlize@xxxxxxmathshift@replace macro stored the remaining text in \htmlize@remaining@text.
        % Note that since we never detokenized and retokenized #1#2#3, 
        % \htmlize@remaining@text still includes the terminating \@htmlize@stop\@htmlize@stop\@htmlize@stop.
        \def\htmlize@next@i{\xa\htmlize@recursive@i\htmlize@remaining@text\htmlize@rdelim@i}%
      \else
        \ifx\test@i\@otherbackslash%
          % Control sequence.  Oh boy.
          % There are three possible things to do:
          % 1. Retokenize everything, so we get a token list.
          %    Expand this control sequence, the first one in the list,
          %    to obtain a new token list.  Then resume processing that list.
          %    Examples: \def\emph#1{<EM>#1</EM>}, \def\rec#1{\frac{1}{#1}}, \def\inv{^{-1}}
          %              \& --> &amp; \# --> #; etc.
          %    Environments: \begin{center}...\end{center} --> <CENTER>...</CENTER>
          % 2. Retokenize everything, so we get a token list.
          %    Let this first command (with its parameters) ACT.
          %    This may involve work in TeX's stomach (e.g., with counters)
          %    or with external files (e.g., image processing).
          %    The command may directly add material to \htmlize@output,
          %    but it should not typeset anything and should vanish from the 
          %    input stream when it is done.
          %       When it's done, we somehow need to detokenize and resume 
          %    processing the remainder of the input stream.
          %       Only commands explicitly crafted (or modified) to work
          %    with moodle.sty can possibly do all this!
          %    Examples: (modified) \includegraphics
          %    Environments: \begin{clozemulti}, \begin{enumerate}
          % 3. Ignore that it's a command.  Pass it right on as a character
          %    sequence to \htmlize@output.
          %    Examples: \alpha, \frac
          %    Environments: \begin{array}
          %
          % #2 is only for items on a specific list.
          % #1 is anything that runs in TeX's mouth.  
          %    We could keep a list and give users a way to add to it.
          %    I could also try expanding macros, using \ifcsmacro from etoolbox.sty
          % 
          % The first step is to figure out what control sequence we're dealing with.
          % First, let's get rid of the terminal \@htmlize@stop codes.
          {\def\@htmlize@stop{}\xdef\htmlize@scrap{#1#2#3}}%
          \let\htmlize@text@to@rescan=\htmlize@scrap%
          % Next, we retokenize the code.
          \xa\retokenizenormal@to@macro\xa\htmlize@rescanned\xa{\htmlize@text@to@rescan}%
          % Now break it up into two pieces.
          \xa\htmlize@grabblock\htmlize@rescanned\@htmlize@stop\htmlize@rdelim@ii%
          \let\@htmlize@cs\htmlize@blockinbraces%
          \edef\htmlize@cs@string{\xa\string\@htmlize@cs}%
          % The first piece, \htmlize@blockinbraces, will contain the single token in \@htmlize@cs.
          % We'll need to keep the second part, \htmlize@remainder, since it probably
          % contains arguments to the cs in \@htmlize@cs.
          %\edef\ds{Encountered '\xa\string\@htmlize@cs'}\show\ds
          %
          % N.B. that \@htmlize@cs is a macro *containing* a single control sequence.
          % This is good for testing with \ifx.
          % \htmlize@cs@string contains the cs as a string, e.g., the characters "\emph".
          %
          \ifx\@htmlize@cs\@@begin@cs
            %This is a \begin.  Begin environment-handling routine.
            %
            % Grab the first {...} from \htmlize@remainder, which is the argument
            % to \begin.
            \xa\htmlize@grabblock\htmlize@remainder\@htmlize@stop\htmlize@rdelim@ii%
            \let\htmlize@envname=\htmlize@blockinbraces%
            %We do not need the rest, so we won't pay any attention to the new
            %content of \htmlize@remainder.
            %
            %Now environments are non-expandable,
            %so there are only two possibilities: action or pass.
            \xa\ifinlist\xa{\htmlize@envname}{\htmlize@env@actionlist}%
              {% Action environment!
                %\bgroup
                  %\def\ds{Encountered active environment \string\begin\{{\htmlize@envname}\}}\show\ds
                  \def\htmlize@next@i{\xa\htmlize@do@actionenv\htmlize@rescanned\@htmlize@stop\htmlize@actionsequence@rdelim}%
                %The \bgroup is to active the environments.
                %The matching \egroup is found in \htmlize@do@actionenv.
              }{%An environment to pass to the HTML
                %We just pass the backslash from "\begin" and move on. 
                \g@addto@macro\htmlize@output{#1}%
                \def\htmlize@next@i{\htmlize@recursive@i#2#3\htmlize@rdelim@i}%
              }%
          \else%
            %This is not an environment.  Begin macro-handling routine.
            \htmlize@actioncsfalse
            \htmlize@expandcsfalse
            \htmlize@passcsfalse
            \xa\ifinlist\xa{\htmlize@cs@string}{\htmlize@cs@actionlist}%
              {%Action sequence!
               \htmlize@actioncstrue}%
              {% Not action sequence!
               \xa\ifinlist\xa{\htmlize@cs@string}{\htmlize@cs@expandlist}%
                 {%CS to be expanded!
                  \htmlize@expandcstrue%
                 }%
                 {%CS to be transcribed to XML
                  \htmlize@passcstrue%
                 }%
              }%
            %Now exactly one of \ifhtmlize@actioncs, \ifhtmlize@expandcs, and \ifhtmlize@passcs is true.
            \ifhtmlize@actioncs
              % It's an action-sequence.
              %\edef\ds{Must let \xa\string\@htmlize@cs\ act!}\show\ds
              %\show\htmlize@rescanned
              \def\htmlize@next@i{\xa\htmlize@do@actioncs\htmlize@rescanned\@htmlize@stop\htmlize@actionsequence@rdelim}%
              %\show\htmlize@rescanned
              % Note that \htmlize@do@actioncs should patch the command to have it
              % restart the scanning in time.
            \else
              \ifhtmlize@expandcs
                % control sequence to be expanded
                %\edef\ds{Must expand \xa\string\@htmlize@cs}\show\ds
                \bgroup
                  \htmlize@redefine@expansionmacros
                  %The \expandafters first expand \htmlize@rescanned,
                  %and then expand its first token just once.
                  \xa\xa\xa\gdef\xa\xa\xa\htmlize@scrap\xa\xa\xa{\htmlize@rescanned}%
                \egroup
                \xa\ultradetokenize@to@macro\xa\htmlize@remaining@text\xa{\htmlize@scrap}%
                \def\htmlize@next@i{\xa\htmlize@recursive@i\htmlize@remaining@text\@htmlize@stop\@htmlize@stop\@htmlize@stop\htmlize@rdelim@i}%
              \else
                % control sequence to be transcribed to XML.
                %\edef\ds{Must pass on \xa\string\@htmlize@cs}\show\ds
                \g@addto@macro\htmlize@output{#1}%
                \def\htmlize@next@i{\htmlize@recursive@i#2#3\htmlize@rdelim@i}%
              \fi% \ifhtmlize@expandcs
            \fi% \ifhtmlize@actioncs
          \fi% \ifx\@htmlize@cs\@@begin@cs
        \else%
          \ifx\test@i\@othertilde%
            % The ~ becomes non-breaking space &nbsp;
            \g@addto@macro\htmlize@output{\otherampersand nbsp;}%
            \def\htmlize@next@i{\htmlize@recursive@i#2#3\htmlize@rdelim@i}%
          \else
            \ifx\test@i\@lsinglequote%
              \ifx\test@ii\@lsinglequote%
                % Double left quote
                \g@addto@macro\htmlize@output{\otherampersand ldquo;}%
                \def\htmlize@next@i{\htmlize@recursive@i#3\htmlize@rdelim@i}%
              \else
                \g@addto@macro\htmlize@output{\otherampersand lsquo;}%
                \def\htmlize@next@i{\htmlize@recursive@i#2#3\htmlize@rdelim@i}%
              \fi% \ifx\test@ii\@lsinglequote%
            \else
              \ifx\test@i\@rsinglequote%
                \ifx\test@ii\@rsinglequote%
                  %Double right quote
                  \g@addto@macro\htmlize@output{\otherampersand rdquo;}%
                  \def\htmlize@next@i{\htmlize@recursive@i#3\htmlize@rdelim@i}%
                \else
                  \g@addto@macro\htmlize@output{\otherampersand rsquo;}%
                  \def\htmlize@next@i{\htmlize@recursive@i#2#3\htmlize@rdelim@i}%
                \fi% \ifx\test@ii\@rsinglequote%
              \else
                \ifx\test@i\@doublequote
                  \g@addto@macro\htmlize@output{\otherampersand rdquo;}%
                  \def\htmlize@next@i{\htmlize@recursive@i#2#3\htmlize@rdelim@i}%
                \else
                  \ifx\test@i\@lt
                    \moodle@ifmathmode{\g@addto@macro\htmlize@output{\otherampersand lt;}}%
                                      {\g@addto@macro\htmlize@output{<}}%
                    \def\htmlize@next@i{\htmlize@recursive@i#2#3\htmlize@rdelim@i}%
                  \else
                    \ifx\test@i\@gt
                      \moodle@ifmathmode{\g@addto@macro\htmlize@output{\otherampersand gt;}}%
                                        {\g@addto@macro\htmlize@output{>}}%
                      \def\htmlize@next@i{\htmlize@recursive@i#2#3\htmlize@rdelim@i}%
                    \else
                      % Default case: write first token to output, call self on remaining tokens.
                      \g@addto@macro\htmlize@output{#1}%
                      \def\htmlize@next@i{\htmlize@recursive@i#2#3\htmlize@rdelim@i}%
                    \fi% \ifx\test@i\@gt
                  \fi% \ifx\test@i\@lt
                \fi% \ifx\test@i\@doublequote
              \fi% \ifx\test@i\@rsinglequote%
            \fi% \ifx\test@i\@lsinglequote%
          \fi% \ifx\test@i\@othertilde%
        \fi% \ifx\test@i\@otherbackslash%
      \fi% \ifx\test@i\@otherdollar%
    \fi% \ifx\test@i\@otherlbrace%
  \fi% \ifx\test@i\@@htmlize@stop
  \htmlize@next@i
}%

\def\@lsinglequote{`}%
\def\@rsinglequote{'}%
\def\@doublequote{"}%
%    \end{macrocode}
% 
% \subsection{Math Mode handling}
% 
% In the following, note that the |\|\meta{*}|mathrightdelim|'s gobble an argument.
% This is so ``|$a$ is...|" can turn into
%       ``\ldots |a\|\meta{*}|mathrightdelim{} is...|"
% and preserve a trailing space.
%    \begin{macrocode}
\edef\inlinemathleftdelim{\otherbackslash(}%
\def\inlinemathrightdelim#1{\advancemathmodecounter{-1}%
                            \g@addto@macro\htmlize@output{\otherbackslash)}}%
\edef\displaymathleftdelim{<CENTER>\otherbackslash[}%
\def\displaymathrightdelim#1{\advancemathmodecounter{-1}%
                             \g@addto@macro\htmlize@output{\otherbackslash]</CENTER>}}%
{\catcode`\$=12\relax%
  \gdef\htmlize@inlinemathshift@replace$#1$#2\htmlize@rdelim@iii{%
    %\def\ds{inline math shift has '#1' and '#2'}\show\ds
    \xa\g@addto@macro\xa\htmlize@output\xa{\inlinemathleftdelim}%
    \advancemathmodecounter{1}%
    \def\mathtext{#1}%
    \def\aftertext{#2}%
    \xdef\htmlize@remaining@text{\expandonce\mathtext%
                                 \otherbackslash inlinemathrightdelim{}%
                                 \expandonce\aftertext}%
    %\show\htmlize@remaining@text
  }%
  \gdef\htmlize@displaymathshift@replace$$#1$$#2\htmlize@rdelim@iii{%
    \xa\g@addto@macro\xa\htmlize@output\xa{\displaymathleftdelim}%
    \advancemathmodecounter{1}%
    \def\mathtext{#1}%
    \def\aftertext{#2}%
    \xdef\htmlize@remaining@text{\expandonce\mathtext%
                                 \otherbackslash displaymathrightdelim{}%
                                 \expandonce\aftertext}%
  }%
}

%    \end{macrocode}
% 
% \subsection{Engines for Control Sequences}
% 
% There are three kinds of control sequences that need special handling:
% \begin{enumerate}
%   \item Action environments
%   \item Action command sequences
%   \item Expansion macros
% \end{enumerate}
% 
% \subsubsection{Engine for running action environments}
%    \begin{macrocode}
\long\def\htmlize@do@actionenv#1#2\@htmlize@stop\htmlize@actionsequence@rdelim{%
  \bgroup %The corresponding \egroup is given in \htmlize@proceedwiththerest, 
          %to localize the changes to the environment definitions.
    \htmlize@activate@environments
    \gdef\htmlize@afteraction@hook{}%
    #1#2\@htmlize@stop\htmlize@actionsequence@rdelim%
}

\def\htmlize@patchendenvironment{\swaptotrueendenvironment{\xa\htmlize@proceedwiththerest\htmlize@afteraction@hook}}%

\def\swaptotrueendenvironment#1#2\if@ignore\@ignorefalse\ignorespaces\fi{#2\if@ignore\@ignorefalse\ignorespaces\fi#1}%


\long\def\htmlize@record@environment#1{%
  \listadd{\htmlize@env@actionlist}{#1}%
}
\long\def\html@newenvironment#1#2{%
  \listadd{\htmlize@env@actionlist}{#1}%
  \g@addto@macro\htmlize@activate@environments{%
    \xa\let\csname #1\endcsname\relax%
    \xa\let\csname end#1\endcsname\relax%
    \NewEnviron{#1}{%
      #2%
    }[\htmlize@patchendenvironment]%
  }%
}


\def\htmlize@activate@environments{}%
%    \end{macrocode}
% 
% \subsubsection{Engine for running action command sequences}
% 
% The following automatically adds the ``engine'' to do the command
% and then resume processing the \LaTeX\ into HTML.
% It uses the |xpatch| package, which says it works with anything
% defined using |\newcommand| etc. and |\newenvironment| etc.
%    \begin{macrocode}
\gdef\htmlize@afteraction@hook{}%

\long\def\htmlize@do@actioncs#1#2\htmlize@actionsequence@rdelim{%
  % #1#2 contains the current string to be rendered into HTML;
  %      N.B. it has been tokenized at this point, 
  %      so TeX can process it directly.
  % #1 = the command sequence we need to execute
  % #2 = the rest of the string
  %
  % First, we patch the desired command so that, when it is over,
  % it calls \htmlize@proceedwiththerest.
  % We do this within the group, so as not to permanently change the command.
  \bgroup
    % The matching \egroup is issued in \htmlize@proceedwiththerest,
    % so that the changes made by \htmlize@activate@css are localized to just the command itself.
    \gdef\htmlize@afteraction@hook{}%
    \htmlize@activate@css%
    \def\test@i{#1}%
    \ifx\test@i\@relax
      \def#1{\xa\htmlize@proceedwiththerest\htmlize@afteraction@hook}%
    \else
      \xapptocmd#1{\xa\htmlize@proceedwiththerest\htmlize@afteraction@hook}{}{\PackageError{Could not patch the command \string#1!}}%
    \fi
    % Now we call that patched command.
    #1#2\htmlize@actionsequence@rdelim%
  %The matching \egroup now is built into the command #1.
}
\long\def\htmlize@proceedwiththerest#1\htmlize@actionsequence@rdelim{%
    % The action cs has done its work.
    % Now we gather up the remaining tokens, detokenize them,
    % remove the \@htmlize@stop, and get back to work transcribing it.
  \egroup %This \egroup matches the \bgroup that was issued either in \htmlize@do@actioncs or in \htmlize@do@actionenv
  \ultradetokenize@to@macro\htmlize@remainder@detokenized{#1}%
  %This will contain an extra \@htmlize@stop, so we remove it.
  \xa\xa\xa\def\xa\xa\xa\htmlize@remainder@detokenized\xa\xa\xa{\xa\htmlize@remove@stopcode\htmlize@remainder@detokenized}%
  %Now we get back to work transcribing the remainder.
  \xa\htmlize@recursive@i\htmlize@remainder@detokenized\@htmlize@stop\@htmlize@stop\@htmlize@stop\htmlize@rdelim@i%  
}

\long\def\htmlize@record@action#1{%
  \xa\listadd\xa\htmlize@cs@actionlist\xa{\string#1}%
}

\def\htmlize@activate@css{}%
\def\html@action@def#1{%
  \htmlize@record@action{#1}%
  \xa\def\xa\htmlize@scrap\xa{\xa\let\xa#1\csname html@\string#1\endcsname}%
  \xa\g@addto@macro\xa\htmlize@activate@css\xa{\htmlize@scrap}%
  \xa\def\csname html@\string#1\endcsname% %And this \def\html@\oldcsname is follows by the remainder of the definition.
}
\def\html@action@newcommand#1[#2][#3]#4{%
  %\message{>>> Defining #1[#2][#3]{...} ^^J}
  \ifmoodle@draftmode
  \else
    \xa\html@action@def\csname #1\endcsname{\csname moodle@#1@int\endcsname}%
  \fi
  % Note that \htmlize@do@actioncs will 'patch' this by putting 
  % '\xa\htmlize@proceedwiththerest\htmlize@afteraction@hook'
  % at the end.  We want those 3 tokens to occur instead after
  % the graphics filename.
  \xa\csdef{moodle@#1@int}##1##2##3{\csname moodle@#1@int@int\endcsname}%
  % This gobbles up those three spurious tokens,
  % which we will re-insert after our work is done.
  \xa\newcommand\csname moodle@#1@int@int\endcsname[#2][#3]{%
    #4%
    % Now we re-insert the code to get the HTMLizing going again.
    \xa\htmlize@proceedwiththerest\htmlize@afteraction@hook
  }%
}
%    \end{macrocode}
% 
% \subsubsection{Engine for expansion control sequences}
% 
% Calling |\htmlize@redefine@expansionmacros| will redefine
% the macros for us.  It starts out empty.
%    \begin{macrocode}
\def\htmlize@redefine@expansionmacros{}%
%    \end{macrocode}
% If |\mymacro| needs no changes to be suited for expansion,
% you can simply call |\htmlize@record@expand{\mymacro}|
% or |\htmlregister{\mymacro}|
% to record that it should be expanded on its way to the HTML.
% Examples would be user-built macros such as |\inv|$\to$|^{-1}|
% or |\N|$\to$|\mathbb{N}|.
%    \begin{macrocode}
\long\def\htmlize@record@expand#1{%
  \xa\listadd\xa\htmlize@cs@expandlist\xa{\string#1}%
}
\let\htmlregister=\htmlize@record@expand
%    \end{macrocode}
% Often users define a list of macros at the end of the preamble.
% It can be cumbersome to record individually these macros for expansion.
% By calling |\moodleregisternewcommands| they trigger the automatic
% expansion of macros defined subsequently using |\newcommand|,
% |\renewcommand|, |\providecommand| or their starred variants.
%    \begin{macrocode}
\def\moodleregisternewcommands{%
  %% INSPIRED FROM 
  %https://tex.stackexchange.com/questions/73271/how-to-redefine-or-patch-the-newcommand-command
  \newcommand*{\saved@ifdefinable}{}
  \let\saved@ifdefinable\@ifdefinable
  \renewcommand{\@ifdefinable}[2]{%
    \saved@ifdefinable{##1}{##2}%
    \htmlregister{##1}
  }%
  \let\@@ifdefinable\@ifdefinable
}%
%    \end{macrocode}
% On the other hand, if an alternate version of the macro is 
% needed for HTML purposes, you can define its HTML version with
% |\html@def\mymacro...|
% Parameters are okay.
% An example would be
% |\html@def\emph#1{<EM>#1</EM>}|.
%    \begin{macrocode}
\long\def\html@def#1{%
  \htmlize@record@expand{#1}%
  \xa\def\xa\htmlize@scrap\xa{\xa\let\xa#1\csname html@\string#1\endcsname}%
  \xa\g@addto@macro\xa\htmlize@redefine@expansionmacros\xa{\htmlize@scrap}%
  \xa\def\csname html@\string#1\endcsname%
}
%    \end{macrocode}
% Note that when |\html@def| expands out, it ends with |\def\html@\oldcsname| 
% which abuts directly on the remainder of the definition.
% 
% \subsection{Specific Control Sequences for Action and Expansion}
% 
% Now that we have that machinery in place,
% we define specific environments, action control sequences, and macros to 
% expand to accomplish our purposes.
% 
% \subsubsection{Action Environments}
%    \begin{macrocode}
\htmlize@record@environment{clozemulti}
\htmlize@record@environment{multi}
\htmlize@record@environment{clozenumerical}
\htmlize@record@environment{numerical}
\htmlize@record@environment{clozeshortanswer}
\htmlize@record@environment{shortanswer}

\html@newenvironment{center}{\xdef\htmlize@afteraction@hook{<CENTER>\expandonce\BODY</CENTER>}}%

\def\moodle@save@getitems@state{%
  \global\xa\xdef\csname moodle@currentitemnumber@level@\the\moodle@listdepth\xa\endcsname\xa{\thecurrentitemnumber}%
  \global\xa\xdef\csname moodle@numgathereditems@level@\the\moodle@listdepth\xa\endcsname\xa{\thenumgathereditems}%
  \moodle@saveitems{\thenumgathereditems}%
}%
\def\moodle@restore@getitems@state{%
  \setcounter{numgathereditems}{\csname moodle@numgathereditems@level@\the\moodle@listdepth\endcsname}%
  \setcounter{currentitemnumber}{\csname moodle@currentitemnumber@level@\the\moodle@listdepth\endcsname}%
  \moodle@restoreitems{\thenumgathereditems}%
}%
\def\moodle@saveitems#1{%
  \ifnum#1>0%
    \global\csletcs{moodle@level@\the\moodle@listdepth @item@#1}{getitems@item@#1}%
    \xa\moodle@saveitems\xa{\number\numexpr#1-1\expandafter}%
  \fi
}%
\def\moodle@restoreitems#1{%
  \ifnum#1>0%
    \global\csletcs{getitems@item@#1}{moodle@level@\the\moodle@listdepth @item@#1}%
    \global\xa\let\csname moodle@level@\the\moodle@listdepth @item@#1\endcsname=\@undefined
    \xa\moodle@restoreitems\xa{\number\numexpr#1-1\expandafter}%
  \fi
}%
\def\moodle@makelistenv#1#2{%
  \html@newenvironment{#1}{%
    \advance\moodle@listdepth by 1\relax
    \moodle@save@getitems@state%
      \xa\gatheritems\xa{\BODY}%
      \gdef\htmlize@afteraction@hook{<#2>}%
      \loopthroughitemswithcommand{\moodle@itemtoLI}%
      \g@addto@macro\htmlize@afteraction@hook{</#2>}%
    \moodle@restore@getitems@state%
    \advance\moodle@listdepth by -1\relax
  }%
}%

\moodle@makelistenv{enumerate}{OL}%
\moodle@makelistenv{itemize}{UL}%

\def\moodle@itemtoLI#1{%
  \g@addto@macro\htmlize@afteraction@hook{<LI>#1}%
  \trim@spaces@in\htmlize@afteraction@hook%
  \g@addto@macro\htmlize@afteraction@hook{</LI>}%
}%

%    \end{macrocode}
% 
% \subsubsection{Action Control Sequences}
% 
%    \begin{macrocode}
\def\advancemathmodecounter#1{%
  \global\advance\moodle@mathmodedepth by #1\relax
}
\def\openclozemode{%
  \global\moodle@clozemodetrue\relax
}
\def\endclozemode{%
  \global\moodle@clozemodefalse\relax
}
\htmlize@record@action{\advancemathmodecounter}%
\htmlize@record@action{\openclozemode}%
\htmlize@record@action{\endclozemode}%
\htmlize@record@action{\relax}%

\html@action@def\#{\g@addto@macro\htmlize@output{\otherhash}}%
\html@action@def\&{\g@addto@macro\htmlize@output{\otherampersand}}%
\html@action@def\\{\moodle@ifmathmode{\g@addto@macro\htmlize@output{\otherbackslash\otherbackslash}}%
                                     {\g@addto@macro\htmlize@output{<BR/>}}}%
\html@action@def\{{%
    \moodle@ifmathmode{\g@addto@macro\htmlize@output{\otherbackslash\otherlbrace}}%
                      {\g@addto@macro\htmlize@output{\otherlbrace}}%
  }%
\html@action@def\}{%
    \moodle@ifmathmode{\g@addto@macro\htmlize@output{\otherbackslash\otherrbrace}}%
                      {\ifmoodle@clozemode\g@addto@macro\htmlize@output{\otherbackslash}\fi
                       \g@addto@macro\htmlize@output{\otherrbrace}}%
  }%
\html@action@def\[{%
    \advancemathmodecounter{1}
    \g@addto@macro\htmlize@output{<CENTER>\otherbackslash\otherlbracket}%
  }%
\html@action@def\]{%
    \g@addto@macro\htmlize@output{\otherbackslash\otherrbracket</CENTER>}%
    \advancemathmodecounter{-1}
  }%
\html@action@def\({%
    \advancemathmodecounter{1}
    \g@addto@macro\htmlize@output{\otherbackslash(}%
  }%
\html@action@def\){%
    \g@addto@macro\htmlize@output{\otherbackslash)}%
    \advancemathmodecounter{-1}
  }%
\html@action@def\ldots{%
    \moodle@ifmathmode{\g@addto@macro\htmlize@output{\string\ldots}}%
                      {\g@addto@macro\htmlize@output{\otherampersand hellip\othersemicol}}%
  }%
\html@action@def\dots{%
    \moodle@ifmathmode{\g@addto@macro\htmlize@output{\string\dots}}%
                      {\g@addto@macro\htmlize@output{\otherampersand hellip\othersemicol}}%
  }%
\html@action@def\ {%
    \moodle@ifmathmode{\g@addto@macro\htmlize@output{\otherbackslash\otherspace}}%
                      {\g@addto@macro\htmlize@output{\otherspace}}%
  }%
\html@action@def\${%
     \g@addto@macro\htmlize@output{\otherdollar}%
  }%
\html@action@def\clozetilde{%
    \xa\g@addto@macro\xa\htmlize@output\xa{\othertilde}%
  }%
\html@action@def\clozelbrace{%
    \openclozemode
    \xa\g@addto@macro\xa\htmlize@output\xa{\otherlbrace}%
  }%
\html@action@def\clozerbrace{%
    \xa\g@addto@macro\xa\htmlize@output\xa{\otherrbrace}%
    \endclozemode
  }%
\html@action@def\TeX{%
    \g@addto@macro\htmlize@output{\otherbackslash(\otherbackslash rm\otherbackslash TeX\otherbackslash)}
  }%
\html@action@def\LaTeX{%
    \g@addto@macro\htmlize@output{\otherbackslash(\otherbackslash rm\otherbackslash LaTeX\otherbackslash)}
  }%

{\catcode`;=12\relax\gdef\othersemicol{;}}

%Diacritical marks over vowels
{\catcode`|=3\relax
 \gdef\htmlize@vowels{a|e|i|o|u|A|E|I|O|U|}}
\def\htmlize@define@diacritic#1#2{%
  \htmlize@record@action{#1}%
  \g@addto@macro\htmlize@activate@css{%
    \def#1##1{%
      \ifinlist{##1}{\htmlize@vowels}%
        {\g@addto@macro\htmlize@output{\otherampersand##1#2\othersemicol}}%
        {\xa\g@addto@macro\htmlize@output\xa{\string#1##1}}%
    }%
  }%
}
\htmlize@define@diacritic{\^}{circ}%
\htmlize@define@diacritic{\'}{acute}%
\htmlize@define@diacritic{\`}{grave}%

%Diaeresis/Tréma/Umlaut
{\catcode`|=3\relax
 \gdef\htmlize@diaeresis{a|e|i|o|u|y|A|E|I|O|U|Y|}}
\html@action@def\"#1{%
    \ifinlist{#1}{\htmlize@diaeresis}%
      {\g@addto@macro\htmlize@output{\otherampersand#1uml\othersemicol}}%
      {\xa\g@addto@macro\htmlize@output\xa{\string\"#1}}%
}%

%Hungarian long-umlaut diacritics
\def\@o{o}\def\@O{O}\def\@u{u}\def\@U{U}%
\html@action@def\H#1{%
  \bgroup
    \def\test@i{#1}%
    \ifx\test@i\@O
      \def\toadd{\otherampersand\otherhash336\othersemicol}%
    \else
      \ifx\test@i\@o
        \def\toadd{\otherampersand\otherhash337\othersemicol}%
      \else
        \ifx\test@i\@U
          \def\toadd{\otherampersand\otherhash368\othersemicol}%
        \else
          \ifx\test@i\@u
            \def\toadd{\otherampersand\otherhash369\othersemicol}%
          \else
            \def\toadd{\otherbackslash\otherlbrace#1\otherrbrace}%
          \fi
        \fi
      \fi
    \fi
    \xa\g@addto@macro\xa\htmlize@output\xa{\toadd}%
  \egroup
}%

%Cedilla
\def\@c{c}\def\@C{C}\def\@s{s}\def\@S{S}\def\@t{t}\def\@T{T}%
\html@action@def\c#1{%
	\bgroup
	  \def\test@i{#1}%
	  \ifx\test@i\@c
	    \def\toadd{\otherampersand ccedil\othersemicol}%
	  \else
	    \ifx\test@i\@C
	      \def\toadd{\otherampersand Ccedil\othersemicol}%
	    \else
	      \ifx\test@i\@s
	        \def\toadd{\otherampersand\otherhash351\othersemicol}%
	      \else
	        \ifx\test@i\@S
	          \def\toadd{\otherampersand\otherhash350\othersemicol}%
	        \else
	          \ifx\test@i\@t
	            \def\toadd{\otherampersand\otherhash355\othersemicol}%
	          \else
	            \ifx\test@i\@T
	              \def\toadd{\otherampersand\otherhash354\othersemicol}%
	            \else
	              \def\toadd{\otherbackslash\otherlbrace#1\otherrbrace}%
	            \fi
	          \fi
	        \fi
	      \fi
	    \fi
	  \fi
	  \xa\g@addto@macro\xa\htmlize@output\xa{\toadd}%
	\egroup
}%

%Tilde over a, n, o
{\catcode`|=3\relax
 \gdef\htmlize@tilde{A|N|O|a|n|o|}}
\html@action@def\~#1{%
    \ifinlist{#1}{\htmlize@tilde}%
      {\g@addto@macro\htmlize@output{\otherampersand#1tilde\othersemicol}}%
      {\xa\g@addto@macro\htmlize@output\xa{\string\~#1}}%
}%

% breve diacritics
{\catcode`|=3\relax
 \gdef\htmlize@breve{A|G|U|a|g|u|}}
\def\@e{e}\def\@E{E}\def\@i{i}\def\@ii{\i}\def\@I{I}\def\@o{o}\def\@O{O}%
\html@action@def\u#1{%
  \ifinlist{#1}{\htmlize@breve}%
     {\g@addto@macro\htmlize@output{\otherampersand#1breve\othersemicol}}%
     {
      \bgroup
        \def\test@i{#1}%
        \ifx\test@i\@E
          \def\toadd{\otherampersand\otherhash276\othersemicol}%
        \else
          \ifx\test@i\@e
            \def\toadd{\otherampersand\otherhash277\othersemicol}%
          \else
            \ifx\test@i\@I
              \def\toadd{\otherampersand\otherhash300\othersemicol}%
            \else
              \ifx\test@i\@i
                \def\toadd{\otherampersand\otherhash301\othersemicol}%
              \else
                \ifx\test@i\@ii
                  \def\toadd{\otherampersand\otherhash301\othersemicol}%
                \else
                  \ifx\test@i\@O
                    \def\toadd{\otherampersand\otherhash334\othersemicol}%
                  \else
                    \ifx\test@i\@o
                      \def\toadd{\otherampersand\otherhash335\othersemicol}%
                    \else
                      \def\toadd{\otherbackslash\otherlbrace#1\otherrbrace}%
                    \fi
                  \fi
                \fi
              \fi
            \fi
          \fi
        \fi
        \xa\g@addto@macro\xa\htmlize@output\xa{\toadd}%
      \egroup
    }%
}%

% caron diacritics
{\catcode`|=3\relax
 \gdef\htmlize@caron{C|D|E|L|N|R|S|T|Z|c|d|e|l|n|r|s|t|z|}}
\html@action@def\v#1{%
    \ifinlist{#1}{\htmlize@caron}%
      {\g@addto@macro\htmlize@output{\otherampersand#1caron\othersemicol}}%
      {\xa\g@addto@macro\htmlize@output\xa{\string\v#1}}%
}%
%    \end{macrocode}
% 
% \subsubsection{Command sequences for Expansion}
% 
%    \begin{macrocode}
\html@def\underline#1{<SPAN STYLE=\&\#34;text-decoration: underline;\&\#34;>#1</SPAN>}
\html@def\emph#1{<EM>#1</EM>}%
\html@def\textit#1{<I>#1</I>}%
\html@def\textbf#1{<B>#1</B>}%
\html@def\texttt#1{<CODE>#1</CODE>}%
\html@def\textsc#1{<SPAN STYLE=\&\#34;font-variant: small-caps;\&\#34;>#1</SPAN>}
\html@def\url#1{<A href=\&\#34;#1\&\#34;>#1</A>}%
\html@def\href#1#2{<A href=\&\#34;#1\&\#34;>#2</A>}%
\html@def\textsuperscript#1{<SUP>#1</SUP>}%
\html@def\up#1{<SUP>#1</SUP>}%
\html@def\fup#1{<SUP>#1</SUP>}%
\html@def\textsubscript#1{<SUB>#1</SUB>}%
\html@def\blank{____________}%
\html@def\par{</P><P>}%
\html@def\aa{\&aring;}%
\html@def\AA{\&Aring;}%
\html@def\ae{\&aelig;}%
\html@def\AE{\&AElig;}%
\html@def\oe{\&oelig;}%
\html@def\OE{\&OElig;}%
\html@def\S{\&sect;}%
\html@def\euro{\&euro;}%
\html@def\texteuro{\&euro;}%
\html@def\o{\&oslash;}%
\html@def\O{\&Oslash;}%
\html@def\ss{\&szlig;}%
\html@def\SS{\&\#7838;}%
\html@def\l{\&lstrok;}%
\html@def\L{\&Lstrok;}%
\html@def\og{\&laquo;\&\#8239;}%
\html@def\guillemotleft{\&laquo;\&\#8239;}%
\html@def\fg{\&\#8239;\&raquo;}%
\html@def\guillemotright{\&\#8239;\&raquo;}%
\html@def\,{\&\#8239;}%
\html@def\thinspace{\&\#8239;}%
\html@def\textbackslash{\&\#92;}%
\html@def\_{\&\#95;}%

% AH CUSTOM MACROS TO EXPAND --- remove these before publishing!
%\htmlize@record@expand{\inv}%
%\htmlize@record@expand{\rec}%

\htmlize@record@action\inlinemathrightdelim
\htmlize@record@action\displaymathrightdelim

%    \end{macrocode}
% 
% \subsection{Graphics via {\ttfamily\string\includegraphics}}
% 
% \subsubsection{External program command lines}
% We first set up commands for the external programs.
%    \begin{macrocode}
\def\htmlize@setexecutable#1{%
  % Defines macro #1 to be #2 in a verbatim mode suitable for filenames
  \def\htmlize@executable@macro{#1}%
  \bgroup\catcode`\|=0\catcode`\\=12\relax%
  \htmlize@setexecutable@int
}
\def\htmlize@setexecutable@int#1{%
  \egroup 
  \expandafter\def\htmlize@executable@macro{#1}%
}

\def\ghostscriptcommand{\htmlize@setexecutable\gs}
\def\baselxivcommand{\htmlize@setexecutable\baselxiv}
\def\imagemagickcommand{\htmlize@setexecutable\htmlize@imagemagick@convert}
\def\optipngcommand{\htmlize@setexecutable\optipng}

\ifwindows%
  \ghostscriptcommand{gswin64c.exe}%
  \baselxivcommand{certutil}%
\else%
  \ghostscriptcommand{gs}%
  \baselxivcommand{base64}%
\fi%

\imagemagickcommand{convert}%
\optipngcommand{optipng}
%    \end{macrocode}
% 
% \subsubsection{Graphics key-handling}
% Next, we get ready to handle keys like |height=4cm| or |width=3cm| or |ppi=72|.
%    \begin{macrocode}
\define@cmdkeys{moodle@includegraphics}[moodle@graphics@]{ppi}
\define@cmdkey{moodle}[moodle@graphics@]{ppi}{}% This is so the ppi key can be set at the document, quiz, or question level.
\define@cmdkeys{Gin}{ppi}% This is so the original \includegraphics will not object to a key of ppi.
\setkeys{moodle@includegraphics}{ppi=103}

\newdimen\moodle@graphics@temp@dimen
\newcount\moodle@graphics@height@pixels
\newcount\moodle@graphics@width@pixels
\def\moodle@graphics@dimentopixels#1#2{%
  \moodle@graphics@temp@dimen=#2\relax
  \moodle@graphics@temp@dimen=0.013837\moodle@graphics@temp@dimen
  \xa\moodle@graphics@temp@dimen\xa=\moodle@graphics@ppi\moodle@graphics@temp@dimen
  #1=\moodle@graphics@temp@dimen
  \divide #1 by 65536\relax
}
\define@key{moodle@includegraphics}{height}[]{%
  \moodle@graphics@dimentopixels{\moodle@graphics@height@pixels}{#1}%
}
\define@key{moodle@includegraphics}{width}[]{%
  \moodle@graphics@dimentopixels{\moodle@graphics@width@pixels}{#1}%
}
\setkeys{moodle@includegraphics}{height=0pt,width=0pt}
%    \end{macrocode}
% 
% \subsubsection{Graphics conversion to HTML}
% If the |tikz| option is loaded, we define the |embedaspict| command.
% Furthermore, |includegraphics| is packed into a TikZ node.
% This allows externalization with regular options for |includegraphics|.
% Otherwise, |includegraphics| is redefined with a limited set of options supported.
%    \begin{macrocode}
\ifmoodle@tikz
\AfterEndPreamble{%
  %\htmlize@record@expand{\embedaspict}%
  \let\oldincludegraphics=\includegraphics
  % patching includegraphics to trigger externalization
  \renewcommand{\includegraphics}[2][]{%
    %\message{moodle.sty: Processing \string\includegraphics[#1]{#2} for HTML^^J}%
    \tikz{\node[inner sep=0pt]{\oldincludegraphics[#1]{#2}};}%
  }%
  % externalized images must be included with the regular command
  \pgfkeys{/pgf/images/include external/.code={\oldincludegraphics{#1}}}%
  \html@action@newcommand{includegraphics}[2][]{%
    \message{moodle.sty: Processing \string\includegraphics[#1]{#2} ^^J}
    \global\advance\numpicturesread by 1\relax
    \edef\htmlize@imagetag{<IMG SRC="data:image/png;base64,\csname picbaselxiv@\the\numpicturesread\endcsname">}%
    \xa\g@addto@macro\xa\htmlize@output\xa{\htmlize@imagetag}%
  }%
}%
\else
\html@action@newcommand{includegraphics}[2][]{%
  \bgroup% The grouping is to localize the changes caused by \setkeys.
    \message{moodle.sty: Processing \string\includegraphics[#1]{#2} for HTML...^^J}
    \setkeys*{moodle@includegraphics}{#1}%
    % Height or width should be given in TeX dimensions like cm or pt or in,
    % and are converted to pixels for web use using the ppi key.  
    % TODO: Can we modify \includegraphics to accept height or width in 
    %        pixels?
    % TODO: What about \includegraphics[scale=0.7] ?
    %        Other keys: keepaspectratio=true|false, angle (rotation), clip & trim
    %           -> the package option 'tikz' offers a workaround for this
    \ifnum\moodle@graphics@height@pixels=0\relax
      \ifnum\moodle@graphics@width@pixels=0\relax
        % No size specified.  Default to height of 200 pixels.
        \def\moodle@graphics@geometry{x200}%
        \def\moodle@graphics@htmlgeometry{}%
      \else
        % Width only specified.
        \edef\moodle@graphics@geometry{\number\moodle@graphics@width@pixels}%
        \edef\moodle@graphics@htmlgeometry{width=\number\moodle@graphics@width@pixels}%
      \fi
    \else
      \ifnum\moodle@graphics@width@pixels=0\relax
        % Height only specified.  The `x' is part of the syntax.
        \edef\moodle@graphics@geometry{x\number\moodle@graphics@height@pixels}%
        \edef\moodle@graphics@htmlgeometry{height=\number\moodle@graphics@height@pixels}%
      \else
        % Height and width both specified.  The `!' is part of the syntax.
        \edef\moodle@graphics@geometry{\number\moodle@graphics@width@pixels x\number\moodle@graphics@height@pixels!}%
        \edef\moodle@graphics@htmlgeometry{width=\number\moodle@graphics@width@pixels\otherspace height=\number\moodle@graphics@height@pixels}%
      \fi
    \fi
    %First, convert it to PNG
    \def\moodle@graphicspath{\@ifundefined{Ginput@path}{}{\xa\@firstofone\Ginput@path}}
    \edef\cmdline{\htmlize@imagemagick@convert\otherspace \moodle@graphicspath#2 -resize \moodle@graphics@geometry\otherspace #2.png}%
    \message{moodle.sty:   Converting '#2' to PNG...^^J}%
    \expandafter\immediate\expandafter\write18\expandafter{\cmdline}%
    %Next, optimize
    \edef\cmdline{\optipng\otherspace -clobber -strip all -quiet #2.png}
    \message{moodle.sty:   Optimizing '#2.png'...^^J}%
    \expandafter\immediate\expandafter\write18\expandafter{\cmdline}%
    %Next, convert the PNG to base64 encoding
    \ifwindows
      \def\cmdline{\baselxiv\otherspace -encode #2.png tmp.b64 && findstr /vbc:"---" tmp.b64 > #2.enc && del tmp.b64"}%
      % Starting from Windows 7, CertUtil is included by default. There should be no windows XP still running
    \else
      \ifmacosx
        \def\cmdline{\baselxiv\otherspace -b 64 -i #2.png -o #2.enc}
      \else % Linux, Cygwin
        \def\cmdline{\baselxiv\otherspace #2.png > #2.enc}
      \fi
      % base64 is part of coreutils, add "-w 64" to get exactly the previous behavior  %
    \fi
    \message{moodle.sty:   Converting '#2.png' to base64...^^J}%
    \expandafter\immediate\expandafter\write18\expandafter{\cmdline}%
    %Now, save that base64 encoding in a TeX macro
    \def\moodle@newpic@baselxiv{}%
    \message{moodle.sty:   Reading base64 file '#2.enc'...^^J}%
    \openin\baseLXIVdatafile=#2.enc\relax
      \savebaselxivdata@recursive
    \closein\baseLXIVdatafile
    \ifwindows
      \immediate\write18{del #2.enc #2.png}%
    \else
      \immediate\write18{rm -f #2.enc #2.png}%
    \fi
    \xa\global\xa\let\csname picbaselxiv@graphics@#2\endcsname=\moodle@newpic@baselxiv%
    \edef\htmlize@imagetag{<IMG \moodle@graphics@htmlgeometry\otherspace SRC="data:image/png;base64,\csname picbaselxiv@graphics@#2\endcsname">}%
    \xa\g@addto@macro\xa\htmlize@output\xa{\htmlize@imagetag}%
    \message{moodle.sty:   <IMG> tag inserted.^^J}%
  \egroup
}%
\fi
%    \end{macrocode}
% The following code accomplishes the reading of an |.enc| file into memory.
% It is also used by the \TikZ\ code below.
%    \begin{macrocode}
\newread\baseLXIVdatafile
\def\savebaselxivdata@recursive{%
  \ifeof\baseLXIVdatafile
    \let\baselxiv@next=\relax
  \else
    \read\baseLXIVdatafile to \datalinein
    %\message{<<\datalinein>>^^J}
    \ifx\datalinein\@moodle@par
      \let\baselxiv@next=\relax
    \else
      %We add tokens manually, rather than with \g@addto@macro, to save time.
      \xa\xa\xa\gdef\xa\xa\xa\moodle@newpic@baselxiv\xa\xa\xa{\xa\moodle@newpic@baselxiv\datalinein^^J}%
      \let\baselxiv@next=\savebaselxivdata@recursive
    \fi
  \fi
  \baselxiv@next
}
%    \end{macrocode}
% 
% \subsection{\TikZ\ Picture Handling}
% If the user is not using the \TikZ\ package, there is no need to waste time 
% loading it.  Without \TikZ\ loaded, however, many of the following commands
% are undefined.
% Our solution is to wait until |\AtBeginDocument| and then test whether 
% \TikZ\ is loaded.  If so, we make the appropriate definitions.
% \begin{macro}{TikZ}
% \changes{v0.7}{2020/07/14}{Support \emph{tikz}\ command}
%    \begin{macrocode}
\newif\ifmoodle@tikzloaded
\moodle@tikzloadedfalse
\AtBeginDocument{
  \ifx\tikzpicture\@undefined
    \moodle@tikzloadedfalse
  \else
    \moodle@tikzloadedtrue
  \fi
  \ifmoodle@draftmode
    \long\def\tikzifexternalizing#1#2{#2}%
  \else
  \ifmoodle@tikzloaded
    \usetikzlibrary{external}%
    \tikzexternalize%
    \tikzset{external/force remake}%

    \ifpdftex % pdflatex or latex
      \ifpdf % pdflatex
      \def\pdftopng{\edef\gscmdline{\gs\otherspace -dBATCH -dNOPAUSE -sDEVICE=pngalpha -sOutputFile=\tikzexternalrealjob-tikztemp-\the\numconvertedpictures.png -r150 \tikzexternalrealjob-tikztemp-\the\numconvertedpictures.pdf}%
                    \expandafter\immediate\expandafter\write18\expandafter{\gscmdline}}%
      \else % latex
        \def\pstopng{\edef\gscmdline{\gs\otherspace -dBATCH -dNOPAUSE -sDEVICE=pngalpha -sOutputFile=\tikzexternalrealjob-tikztemp-\the\numconvertedpictures.png -r150 \tikzexternalrealjob-tikztemp-\the\numconvertedpictures.ps}%
                    \expandafter\immediate\expandafter\write18\expandafter{\gscmdline}}%
      \fi
    \else
      \ifxetex % xetex
        \def\pdftopng{\edef\gscmdline{\gs\otherspace -dBATCH -dNOPAUSE -sDEVICE=pngalpha -sOutputFile=\tikzexternalrealjob-tikztemp-\the\numconvertedpictures.png -r150 \tikzexternalrealjob-tikztemp-\the\numconvertedpictures.pdf}%
                    \expandafter\immediate\expandafter\write18\expandafter{\gscmdline}}%
      \else % What to do?
        \def\pdftopng{\edef\gscmdline{\gs\otherspace -dBATCH -dNOPAUSE -sDEVICE=pngalpha -sOutputFile=\tikzexternalrealjob-tikztemp-\the\numconvertedpictures.png -r150 \tikzexternalrealjob-tikztemp-\the\numconvertedpictures.pdf}%
                    \expandafter\immediate\expandafter\write18\expandafter{\gscmdline}}%
      \fi
    \fi
    \def\pngoptim{\edef\cmdline{\optipng\otherspace -clobber -strip all -quiet \tikzexternalrealjob-tikztemp-\the\numconvertedpictures.png}%
      \expandafter\immediate\expandafter\write18\expandafter{\cmdline}}%
    \ifwindows
      \def\pngtobaselxiv{\edef\baselxivcmdline{\baselxiv\otherspace -encode \tikzexternalrealjob-tikztemp-\the\numconvertedpictures.png tmp.b64 && findstr /vbc:"---" tmp.b64 > \tikzexternalrealjob-tikztemp-\the\numconvertedpictures.enc && del tmp.b64}%
                       \expandafter\immediate\expandafter\write18\expandafter{\baselxivcmdline}}
      % Starting from Windows 7, CertUtils is included by default. There should be no windows XP still running in 2020
    \else
      \ifmacosx
        \def\pngtobaselxiv{\edef\baselxivcmdline{\baselxiv\otherspace -b 64 -i \tikzexternalrealjob-tikztemp-\the\numconvertedpictures.png -o \tikzexternalrealjob-tikztemp-\the\numconvertedpictures.enc}%
                       \expandafter\immediate\expandafter\write18\expandafter{\baselxivcmdline}}
      \else
        \def\pngtobaselxiv{\edef\baselxivcmdline{\baselxiv\otherspace \tikzexternalrealjob-tikztemp-\the\numconvertedpictures.png > \tikzexternalrealjob-tikztemp-\the\numconvertedpictures.enc}%
                       \expandafter\immediate\expandafter\write18\expandafter{\baselxivcmdline}}
      % base64 is part of coreutils, add "-w 64" to get exactly the previous behavior %
      \fi
    \fi
    
    \let\moodle@oldtikzpicture=\tikzpicture

    %The following code lets us run things *before* the normal \begin{tikzpicture}.
    \renewenvironment{tikzpicture}{%
      \global\advance\numconvertedpictures by 1\relax
      %\jobnamewithsuffixtomacro{\htmlize@picbasename}{-tikztemp-\the\numconvertedpictures}%
      %\xa\tikzsetnextfilename\xa{\htmlize@picbasename}%
      \tikzsetnextfilename{\tikzexternalrealjob-tikztemp-\the\numconvertedpictures}%
      \moodle@oldtikzpicture%
    }{}%
    % However, the tikz externalize library does *not* run \end{tikzpicture}.
    % In order to run commands after the tikz picture is done compiling, we need to 
    % use a hook into \tikzexternal@closeenvironments.

    \g@addto@macro{\tikzexternal@closeenvironments}{%
      \moodle@endtikzpicture@hook
    }
    \def\moodle@endtikzpicture@hook{%
      \@moodle@ifgeneratexml{%
        \ifpdftex % pdflatex or latex
          \ifpdf % pdflatex
            \message{moodle.sty: Converting picture '\tikzexternalrealjob-tikztemp-\the\numconvertedpictures.pdf' to PNG...^^J}%
            \pdftopng
          \else % latex
            \message{moodle.sty: Converting picture '\tikzexternalrealjob-tikztemp-\the\numconvertedpictures.ps' to PNG...^^J}%
            \pstopng
          \fi
        \else
          \ifxetex % xetex
            \message{moodle.sty: Converting picture '\tikzexternalrealjob-tikztemp-\the\numconvertedpictures.pdf' to PNG...^^J}%
            \pdftopng
          \else % What to do?
            \message{moodle.sty: Converting picture '\tikzexternalrealjob-tikztemp-\the\numconvertedpictures.pdf' to PNG...^^J}%
            \pdftopng
          \fi
        \fi
        \message{moodle.sty:   Optimizing '\tikzexternalrealjob-tikztemp-\the\numconvertedpictures.png'...^^J}%
        \pngoptim
        \message{moodle.sty:   Converting '\tikzexternalrealjob-tikztemp-\the\numconvertedpictures.png' to base64...^^J}%
        \pngtobaselxiv
        \message{moodle.sty:   Reading base64 file '\tikzexternalrealjob-tikztemp-\the\numconvertedpictures.enc'...^^J}%
        \savebaselxivdata
        \message{moodle.sty:   base64 data saved.^^J}%
      }{}%
    }
    \ifmoodle@tikz
      \tikzset{external/optimize=false} % due to redefinition, includegraphics must not be optimized away
    \else
      \tikzset{external/optimize=true}
      \tikzset{external/optimize command away={\VerbatimInput}{1}}
    \fi
    %
    % The HTMLizer version of the tikzpicture environment,
    % which writes an <IMG> tag to the XML file.
    \htmlize@record@environment{tikzpicture}
    \g@addto@macro\htmlize@activate@environments{%
      \let\tikzpicture\relax\let\endtikzpicture\relax
      \NewEnviron{tikzpicture}[1][]{%
        \global\advance\numpicturesread by 1\relax
        \edef\htmlize@imagetag{<IMG SRC="data:image/png;base64,\csname picbaselxiv@\the\numpicturesread\endcsname">}%
        \xa\g@addto@macro\xa\htmlize@output\xa{\htmlize@imagetag}%
      }[\htmlize@patchendenvironment]%
    }%
    \html@action@newcommand{tikz}[2][]{%
%      \message{>>> Processing \string\tikz[#1]{...} ^^J}
      \global\advance\numpicturesread by 1\relax
      \edef\htmlize@imagetag{<IMG SRC="data:image/png;base64,\csname picbaselxiv@\the\numpicturesread\endcsname">}%
      \xa\g@addto@macro\xa\htmlize@output\xa{\htmlize@imagetag}%
    }%
  \else
    %TikZ not loaded.  Provide dummy definitions for commands.
    \long\def\tikzifexternalizing#1#2{#2}%
  \fi
  \fi
  \ifmoodle@tikz
    \tikzstyle{moodlepict}=[minimum height=1em,inner sep=0pt,execute at begin node={\begin{varwidth}{\linewidth}},execute at end node={\end{varwidth}}]
    \newcommand\embedaspict[1]{\tikz[baseline=-\the\dimexpr\fontdimen22\textfont2\relax]{\node[moodlepict]{\mbox{#1}};}}
    \htmlize@record@expand{\embedaspict}
  \fi
}

\newcount\numconvertedpictures
\numconvertedpictures=0\relax
\newcount\numpicturesread
\numpicturesread=0\relax

\def\savebaselxivdata{%
  \def\moodle@newpic@baselxiv{}%
  \openin\baseLXIVdatafile=\tikzexternalrealjob-tikztemp-\the\numconvertedpictures.enc\relax
    \savebaselxivdata@recursive
  \closein\baseLXIVdatafile
  \xa\global\xa\let\csname picbaselxiv@\the\numconvertedpictures\endcsname=\moodle@newpic@baselxiv%
}

\ifmoodle@tikz
  \ifmoodle@tikzloaded
    \PackageWarning{moodle}{With package option 'tikz', you should not load TikZ manually.}%
  \fi
  \RequirePackage{tikz}%
  \RequirePackage{varwidth}% for the command |embedaspict|
\fi
%    \end{macrocode}
% Finally, we clean up our mess by deleting the temporary PDF, PNG, and ENC 
% files we created.
%    \begin{macrocode}
\AtEndDocument{%
  \ifmoodle@tikzloaded
    \@moodle@ifgeneratexml{%
      \ifxetex
        % we must leave picture pdf's for later linking
        \ifwindows
          \immediate\write18{powershell.exe -noexit "del * -include \tikzexternalrealjob-tikztemp-*.* -exclude *.pdf}%
        \else
          \immediate\write18{find . -type f -name "\tikzexternalrealjob-tikztemp-*.*" -not -name "*.pdf" -delete}%
        \fi
      \else
        \ifwindows
          \immediate\write18{del \tikzexternalrealjob-tikztemp-*.*}%
        \else
          \immediate\write18{rm -f \tikzexternalrealjob-tikztemp-*.*}%
        \fi
      \fi
    }{}%
  \fi
}
%    \end{macrocode}
% TODO:
% * sizing options for TikZ pictures?
% * In cloze multi, how to handle HTML or TeX in answers?  In particular, what about ~?
%\end{macro}
%
% \subsection{Verbatim Code}
% 
% We start by defining a macro to parameter a style for code box display in Moodle
%    \begin{macrocode}
\def\xmlDisplayVerbatimBox{border-top: thin solid; border-bottom: thin solid}%
%    \end{macrocode}
% First, let us handle |\verbatiminput| from the `verbatim' package
%    \begin{macrocode}
\html@action@def\verbatiminput#1{%
  \message{moodle.sty: Processing \string\verbatiminput{#1} for HTML ^^J}%
  \g@addto@macro\htmlize@output{<PRE style="\xmlDisplayVerbatimBox"><CODE>}%
  %%%%%%%%%%%%%% from verbatim %%%%%%%%%%%%%%%%%
  \@bsphack
  \let\do\@makeother\dospecials
  \catcode`\^^M\active
  \def\verbatim@processline{\xa\g@addto@macro\xa\htmlize@output\xa{\the\verbatim@line<BR/>}}
  \verbatim@readfile{#1}%
  \@esphack
   %%%%%%%%%%%%%%%%%%%%%%%%%%%%%%%%%%%%%%%%%%%%%%
  \g@addto@macro\htmlize@output{</CODE></PRE>}%
}%
%    \end{macrocode}
% Second, we deal with |\VerbatimInput| from `fancyvrb' or `fvextra'
%    \begin{macrocode}
\@ifpackageloaded{minted}{\PackageError{moodle}{'moodle' should be loaded before 'minted'.}%
{'moodle' loads 'fancybox' which, unfortunately, redefines verbatim commands.}}%
\@ifpackageloaded{fvextra}{\PackageError{moodle}{'moodle' should be loaded before 'fvextra'.}%
{'moodle' loads 'fancybox' which, unfortunately, redefines verbatim commands.}}%
\@ifpackageloaded{fancyvrb}{\PackageError{moodle}{'moodle' should be loaded before 'fancyvrb'.}%
{'moodle' loads 'fancybox' which, unfortunately, redefines verbatim commands.}}%

\html@action@newcommand{VerbatimInput}[2][]{%
  \message{moodle.sty: Processing \string\VerbatimInput[#1]{#2} for HTML ^^J}%
  \def\FV@KeyValues{#1}%
  \FV@UseKeyValues% import options defined in #1
  \moodle@VerbatimInput{#2}%
}%
\html@action@newcommand{LVerbatimInput}[2][]{%
  \message{moodle.sty: Processing \string\LVerbatimInput[#1]{#2} for HTML ^^J}%
  \def\FV@KeyValues{#1}%
  \FV@UseKeyValues% import options defined in #1
  \moodle@VerbatimInput{#2}%
}%
\html@action@newcommand{BVerbatimInput}[2][]{%
  \message{moodle.sty: Processing \string\BVerbatimInput[#1]{#2} for HTML ^^J}%
  \def\FV@KeyValues{#1}%
  \FV@UseKeyValues% import options defined in #1
  \moodle@VerbatimInput{#2}%
}%
\def\moodle@VerbatimInput#1{%
  \g@addto@macro\htmlize@output{<PRE style="\xmlDisplayVerbatimBox"><CODE>}%
  %%%%%%%% using material from fancyvrb and fvextra  %%%%%%%%
  %\begingroup
  \def\moodle@verbatim@addlinenumber##1{%
    \g@addto@macro\htmlize@output{<span style="font-size: 80\otherpercent; 
         background-color: \otherhash f0f0f0; padding: 0 5px 0 5px; display:
         inline-block; width: 16pt; ##1">}%
    \if@FV@NumberBlankLines
      \xa\g@addto@macro\xa\htmlize@output\xa{\the\c@FancyVerbLine</span>}%
    \else
      \ifx\FV@Line\empty
        \xa\g@addto@macro\xa\htmlize@output\xa{\otherampersand nbsp;</span>}%
      \else
        \xa\g@addto@macro\xa\htmlize@output\xa{\the\c@FancyVerbLine</span>}%
      \fi 
    \fi
  }
  % redefine the ProcessLine routine ('fancyvrb' and 'fvextra') for XML output
  \def\FV@ProcessLine##1{%
    \ifcsname FV@HighlightLine:\number\c@FancyVerbLine\endcsname
      \xdef\moodle@FV@tagB{<mark>}% fvextra triggered highlighting for this line
      \xdef\moodle@FV@tagE{</mark>}%
    \else
      \xdef\moodle@FV@tagB{}% no highlighting
      \xdef\moodle@FV@tagE{}%
    \fi
    \catcode`\`=12%
    \def\FV@Line{##1}%
    \ifx\FV@LeftListNumber\relax
    
    \else% line numbers displayed on the left side
      \moodle@verbatim@addlinenumber{text-align: right}%
    \fi
    \xa\g@addto@macro\xa\htmlize@output\xa{\moodle@FV@tagB}%
    \xa\g@addto@macro\xa\htmlize@output\xa{\FV@Line}%
    \xa\g@addto@macro\xa\htmlize@output\xa{\moodle@FV@tagE}%
    \ifx\FV@RightListNumber\relax\else% line numbers on the right side
      \moodle@verbatim@addlinenumber{text-align: left; float: right}%
    \fi
    \g@addto@macro\htmlize@output{<BR/>}%linebreak
  }
  \global\FV@CodeLineNo\z@% reset codeline counter
  \frenchspacing% Cancels special punctuation spacing.
  \FV@DefineWhiteSpace
  \def\FV@Space{\space}
  \FV@DefineTabOut% replace tabs with a series a whitespaces
  \ifdefined\FV@HighlightLinesPrep
    \FV@HighlightLinesPrep% prepare highlighting if 'fvextra' is loaded
  \fi
  \FV@Input{#1}%
  %\endgroup
  %%%%%%%%%%%%%%%%%%%%%%%%%%%%%%%%%%%%%%%%%%%%%%%%%%%%%%%%
  \g@addto@macro\htmlize@output{</CODE></PRE>}%
}%
\AtEndPreamble{%
  \@ifpackageloaded{fancyvrb}{%
    % custom settings for display
    \fvset{frame=lines,label={[Beginning of code]End of code},framesep=3mm,numbersep=9pt}%
  }{}%
}
%    \end{macrocode}
% Third, we patch `minted' so that it also calls pygmentize to generate HTML code.
%    \begin{macrocode}
\AtEndPreamble{% this definition should prevail because `minted' gets loaded after `moodle'
\@ifpackageloaded{minted}{%
  \renewcommand{\minted@pygmentize}[2][\minted@outputdir\minted@jobname.pyg]{%
    \minted@checkstyle{\minted@get@opt{style}{default}}%
    \stepcounter{minted@pygmentizecounter}%
    \ifthenelse{\equal{\minted@get@opt{autogobble}{false}}{true}}%
      {\def\minted@codefile{\minted@outputdir\minted@jobname.pyg}}%
      {\def\minted@codefile{#1}}%
    \ifthenelse{\boolean{minted@isinline}}%
      {\def\minted@optlistcl@inlines{%
        \minted@optlistcl@g@i
        \csname minted@optlistcl@lang\minted@lang @i\endcsname}}%
      {\let\minted@optlistcl@inlines\@empty}%
    \def\minted@cmdtemplate##1##2{%
      \ifminted@kpsewhich
        \ifwindows
          \detokenize{for /f "usebackq tokens=*"}\space\@percentchar\detokenize{a 
          in (`kpsewhich}\space\minted@codefile\detokenize{`) do}\space
        \fi
      \fi
      \MintedPygmentize\space -l #2 -f ##1 -F tokenmerge
      \minted@optlistcl@g \csname minted@optlistcl@lang\minted@lang\endcsname
      \minted@optlistcl@inlines
      \minted@optlistcl@cmd -o \minted@outputdir##2\space
      \ifminted@kpsewhich
        \ifwindows
          \@percentchar\detokenize{a}%
        \else
          \detokenize{`}kpsewhich \minted@codefile\space
            \detokenize{||} \minted@codefile\detokenize{`}%
        \fi
      \else
        \minted@codefile
      \fi}%
    \def\minted@cmd{\minted@cmdtemplate{latex -P commandprefix=PYG}{\minted@infile}}
    % For debugging, uncomment: %%%%
    \immediate\typeout{\minted@cmd}%
    % %%%%
    \def\minted@cmdHTML{\minted@cmdtemplate{html -P noclasses -P 
      nowrap -P hl_lines="\FV@HighlightLinesList" -P
      style="\minted@get@opt{style}{default}"}{\csname minted@infileHTML\the\c@minted@pygmentizecounter\endcsname}}%
    \def\minted@cmdPNG{\minted@cmdtemplate{png -P    
      line_numbers=True}{\minted@infilePNG}}%
    \def\minted@cmdSVG{\minted@cmdtemplate{svg -P 
      linenos=True}{\minted@infileSVG}}%
    \ifthenelse{\boolean{minted@cache}}%
      {%
        \ifminted@frozencache
        \else
          \ifx\XeTeXinterchartoks\minted@undefined
            \ifthenelse{\equal{\minted@get@opt{autogobble}{false}}{true}}%
              {\edef\minted@hash{\pdf@filemdfivesum{#1}%
                \pdf@mdfivesum{\minted@cmd autogobble(\ifx\FancyVerbStartNum\z@ 
                0\else\FancyVerbStartNum\fi-\ifx\FancyVerbStopNum\z@ 
                0\else\FancyVerbStopNum\fi)}}}%
              {\edef\minted@hash{\pdf@filemdfivesum{#1}%
                \pdf@mdfivesum{\minted@cmd}}}%
          \else
            \ifx\mdfivesum\minted@undefined
              \immediate\openout\minted@code\minted@jobname.mintedcmd\relax
              \immediate\write\minted@code{\minted@cmd}%
              \ifthenelse{\equal{\minted@get@opt{autogobble}{false}}{true}}%
                {\immediate\write\minted@code{autogobble(\ifx\FancyVerbStartNum\z@
                 0\else\FancyVerbStartNum\fi-\ifx\FancyVerbStopNum\z@ 
                0\else\FancyVerbStopNum\fi)}}{}%
              \immediate\closeout\minted@code
              \edef\minted@argone@esc{#1}%
              \StrSubstitute{\minted@argone@esc}{\@backslashchar}{\@backslashchar\@backslashchar}[\minted@argone@esc]%
              \StrSubstitute{\minted@argone@esc}{"}{\@backslashchar"}[\minted@argone@esc]%
              \edef\minted@tmpfname@esc{\minted@outputdir\minted@jobname}%
              \StrSubstitute{\minted@tmpfname@esc}{\@backslashchar}{\@backslashchar\@backslashchar}[\minted@tmpfname@esc]%
              \StrSubstitute{\minted@tmpfname@esc}{"}{\@backslashchar"}[\minted@tmpfname@esc]%
              %Cheating a little here by using ASCII codes to write `{` and `}`
              %in the Python code
              \def\minted@hashcmd{%
                \detokenize{python -c "import hashlib; import os;
                  hasher = hashlib.sha1();
                  f = 
                  open(os.path.expanduser(os.path.expandvars(\"}\minted@tmpfname@esc.mintedcmd\detokenize{\")),
                   \"rb\");
                  hasher.update(f.read());
                  f.close();
                  f = 
                  open(os.path.expanduser(os.path.expandvars(\"}\minted@argone@esc\detokenize{\")),
                   \"rb\");
                  hasher.update(f.read());
                  f.close();
                  f = 
                  open(os.path.expanduser(os.path.expandvars(\"}\minted@tmpfname@esc.mintedmd5\detokenize{\")),
                   \"w\");
                  macro = \"\\edef\\minted@hash\" + chr(123) + hasher.hexdigest() 
                  + chr(125) + \"\";
                  f.write(\"\\makeatletter\" + macro + 
                  \"\\makeatother\\endinput\n\");
                  f.close();"}}%
              \ShellEscape{\minted@hashcmd}%
              \minted@input{\minted@outputdir\minted@jobname.mintedmd5}%
            \else
              \ifthenelse{\equal{\minted@get@opt{autogobble}{false}}{true}}%
               {\edef\minted@hash{\mdfivesum file {#1}%
                  \mdfivesum{\minted@cmd autogobble(\ifx\FancyVerbStartNum\z@ 
                  0\else\FancyVerbStartNum\fi-\ifx\FancyVerbStopNum\z@ 
                  0\else\FancyVerbStopNum\fi)}}}%
               {\edef\minted@hash{\mdfivesum file {#1}%
                  \mdfivesum{\minted@cmd}}}%
            \fi
          \fi
          \edef\minted@infile{\minted@cachedir/\minted@hash.pygtex}%
          \edef\minted@temp@infileHTML{\minted@cachedir/\minted@hash.html}%
          \global\cslet{minted@infileHTML\the\c@minted@pygmentizecounter}{\minted@temp@infileHTML}%
          %\global\edef\minted@infilePNG{\minted@cachedir/\minted@hash.png}%
          %\global\edef\minted@infileSVG{\minted@cachedir/\minted@hash.svg}%
          \IfFileExists{\minted@infile}{}{%
            \ifthenelse{\equal{\minted@get@opt{autogobble}{false}}{true}}{%
              \minted@autogobble{#1}}{}%
            \ShellEscape{\minted@cmd}%
            \ShellEscape{\minted@cmdHTML}%
            %\ShellEscape{\minted@cmdPNG}%
            %\ShellEscape{\minted@cmdSVG}%
            }%
        \fi
        \ifthenelse{\boolean{minted@finalizecache}}%
         {%
            \edef\minted@cachefilename{listing\arabic{minted@pygmentizecounter}.pygtex}%
            \edef\minted@actualinfile{\minted@cachedir/\minted@cachefilename}%
            \ifwindows
              \StrSubstitute{\minted@infile}{/}{\@backslashchar}[\minted@infile@windows]
              \StrSubstitute{\minted@actualinfile}{/}{\@backslashchar}[\minted@actualinfile@windows]
              \ShellEscape{move /y 
              \minted@outputdir\minted@infile@windows\space\minted@outputdir\minted@actualinfile@windows}%
            \else
              \ShellEscape{mv -f 
              \minted@outputdir\minted@infile\space\minted@outputdir\minted@actualinfile}%
            \fi
            \let\minted@infile\minted@actualinfile
            \expandafter\minted@addcachefile\expandafter{\minted@cachefilename}%
         }%
         {\ifthenelse{\boolean{minted@frozencache}}%
           {%
              \edef\minted@cachefilename{listing\arabic{minted@pygmentizecounter}.pygtex}%
              \edef\minted@infile{\minted@cachedir/\minted@cachefilename}%
              \expandafter\minted@addcachefile\expandafter{\minted@cachefilename}}%
           {\expandafter\minted@addcachefile\expandafter{\minted@hash.pygtex}}%
         }%
        \minted@inputpyg}%
      {%
        \ifthenelse{\equal{\minted@get@opt{autogobble}{false}}{true}}{%
          \minted@autogobble{#1}}{}%
        \ShellEscape{\minted@cmd}%
        \ShellEscape{\minted@cmdHTML}%
        %\ShellEscape{\minted@cmdPNG}%
        %\ShellEscape{\minted@cmdSVG}%
        \minted@inputpyg}%
  }%
}{}%
}%
\newcounter{moodle@pygmentizecounter}
\html@action@newcommand{inputminted}[3][]{%
  \message{moodle.sty: Processing \string\inputminted[#1]{#2}{#3} for HTML ^^J}%
  % arguments #2 and #3 are thrown away: the job is done previously by minted when
  % calling pygmentize. The file |\minted@infileHTML| generated with our hack will be used.
  % Since minted is based upon `fvextra' the macro |\moodle@VerbatimInput| works here.
  \minted@configlang{#2}% grab options set for this specific language
  \setkeys{minted@opt@cmd}{#1}% grab options in #1
  \minted@fvset% import options 
  \stepcounter{moodle@pygmentizecounter}
  \xa\moodle@VerbatimInput\xa{\csname minted@infileHTML\the\c@moodle@pygmentizecounter\endcsname}%
}%
%    \end{macrocode}
%
% \subsection{Warning users of \texttt{babel}}
% Users of the |babel| package loaded with the 'french' option may experience problems
% related to autospacing when using |pdfLaTeX|. We shall warn them.
%    \begin{macrocode}
\AtBeginDocument{%
  \ifpdftex % pdflatex or latex
    \@ifpackagewith{babel}{french}{%
      \PackageWarning{moodle}{Be careful when using LaTeX, moodle, and the babel package with option 'french'. Autospacing produces in undesired symbols in the XML. You can either 1) compile with xelatex or 2) add '\NoAutoSpacing' after '\begin{quiz}'.}%
    }{\relax}%
  \fi
}%
%    \end{macrocode}
% 
%
% \Finale
\endinput
