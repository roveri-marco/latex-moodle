% \iffalse meta-comment
%
% moodle.dtx
% Copyright 2016 by Anders O.F. Hendrickson (anders.hendrickson@snc.edu)
%
% This work may be distributed and/or modified under the
% conditions of the LaTeX Project Public License, either version 1.3
% of this license or (at your option) any later version.
% The latest version of this license is in
%   http://www.latex-project.org/lppl.txt
% and version 1.3 or later is part of all distributions of LaTeX
% version 2005/12/01 or later.
%
% This work has the LPPL maintenance status `maintained'.
% 
% The Current Maintainer of this work is Anders O.F. Hendrickson.
%
% This work consists of the files moodle.dtx and moodle.ins
% and the derived files moodle.sty and getitems.sty.
%
% \fi
%
% \iffalse
%<*driver>
\ProvidesFile{moodle.dtx}
%</driver>
%<package>\NeedsTeXFormat{LaTeX2e}[1999/12/01]
%<package>\ProvidesPackage{moodle}
%<*package>
    [2019/02/18 v0.6 Moodle quiz XML generation]
%</package>
%
%<*driver>
\documentclass{ltxdoc}
\usepackage{moodle}
\usepackage{amssymb}
\usepackage{fancyvrb}
\usepackage{multirow,hhline,colortbl,threeparttable,xcolor,graphicx,hyperref}
\EnableCrossrefs         
\CodelineIndex
\OnlyDescription
\RecordChanges
\begin{document}
  \DocInput{moodle.dtx}
\end{document}
%</driver>
% \fi
%
% \CheckSum{3985}
%
% \CharacterTable
%  {Upper-case    \A\B\C\D\E\F\G\H\I\J\K\L\M\N\O\P\Q\R\S\T\U\V\W\X\Y\Z
%   Lower-case    \a\b\c\d\e\f\g\h\i\j\k\l\m\n\o\p\q\r\s\t\u\v\w\x\y\z
%   Digits        \0\1\2\3\4\5\6\7\8\9
%   Exclamation   \!     Double quote  \"     Hash (number) \#
%   Dollar        \$     Percent       \%     Ampersand     \&
%   Acute accent  \'     Left paren    \(     Right paren   \)
%   Asterisk      \*     Plus          \+     Comma         \,
%   Minus         \-     Point         \.     Solidus       \/
%   Colon         \:     Semicolon     \;     Less than     \<
%   Equals        \=     Greater than  \>     Question mark \?
%   Commercial at \@     Left bracket  \[     Backslash     \\
%   Right bracket \]     Circumflex    \^     Underscore    \_
%   Grave accent  \`     Left brace    \{     Vertical bar  \|
%   Right brace   \}     Tilde         \~}
%
%
% \changes{v0.5}{2016/01/05}{Initial version}
% \changes{v0.6}{2019/02/18}{Bux-fixing release}
%
% \GetFileInfo{moodle.sty}
%
% \DoNotIndex{\newcommand,\newenvironment,\def}
% 
%
% \title{The \textsf{moodle} package: \\
%        generating Moodle quizzes via \LaTeX%
%        \thanks{This document
%                corresponds to \textsf{moodle.sty}~\fileversion, dated \filedate.}}
% \author{Anders Hendrickson\\ St.~Norbert College, De~Pere, WI, USA \\ \url{anders.hendrickson@snc.edu}}
% \date{February 18, 2019}
% 
% \maketitle
%
% \providecommand\TikZ{Ti\emph{k}Z}
%
% \section{Motivation}
%
% The acronym Moodle stands for ``Modular Object-Oriented Dynamic Learning Environment.''
% It is an open source learning management system employed by many universities, colleges, 
% and high schools to provide digital access to course materials, such as
% notes, video lectures, forums, and the like; see
% \url{https://moodle.com/moodle-lms/} for more information. 
% One of the many useful
% features of Moodle is that mathematical and scientific notation can be entered in
% \LaTeX\ or \TeX\ code, which will be typeset either through a built-in \TeX\ filter
% or by invoking MathJax.
% 
% For instructors who want to give students frequent feedback,
% but lack the time to do so, a particularly valuable module in Moodle
% is the \emph{quiz}.  A Moodle quiz can consist of several different types of
% questions---not only multiple choice or true/false questions, but also 
% questions requiring a short phrase or numerical answer, and even essay 
% questions.  All but the essay questions are automatically graded by the 
% system, and the instructor has full control over how often the quiz may be 
% attempted, its duration, and so forth.  Feedback can be tailored to specific
% mistakes the student makes.
% 
% All these features make Moodle quizzes very useful tools for instructors 
% who have access to them.
% Unfortunately, the primary way to create or edit a Moodle quiz 
% is through a web-based interface that can be slow to operate.
% To users of \LaTeX, accustomed to the speed of typing source code on
% a keyboard alone, the agonizing slowness of switching between mouse and keyboard
% to navigate a web form with its myriad dropdown boxes, radio buttons,
% compounded with a perceptible time lag as one's Moodle server responds to requests,
% can produce a very frustrating experience.  Moreover, editing is entirely 
% impossible without network access.
% 
% Once the quiz is written, there is no easy way to view and proofread all the 
% information of which it is made.  Each question is edited on a separate webpage,
% which is so full of options that it cannot be viewed on a single screen.  
% An instructor has to spend much time checking over the newly created quiz in 
% order to be confident there are no errors.
% 
% Added to all this is the frustration of managing graphics.
% If a question requires an image---say, asking a calculus student to interpret
% the graph of a function---the image must first be produced as a standalone file
% (e.g., in JPG or PNG format), uploaded to Moodle, and then chosen in a web-based
% HTML editor.  Great is the vexation of the instructor who decided to alter a 
% question, as there are more and more possibilities of error whenever multiple 
% files must be kept synchronized.
% 
% Users of \LaTeX\ are also accustomed to the speed and flexibility that 
% comes from defining their own macros, which may be as brief as writing
% |\R| instead of |\mathbb{R}|
% or as complex as macros that generate entire paragraphs of text.
% The Moodle editor, by contrast, requires you to type |\mathbb{R}|
% every single time you want $\mathbb{R}$.
% 
% Finally, there is the question of archiving and reusing one's work.
% Much, much work goes into creating Moodle quizzes, which then reside
% on a Moodle server somewhere in the cloud in a format neither easily browsable
% nor easily modifiable.
% 
% \LaTeX\ itself has the power to solve all these difficulties: 
% it is swift to edit and swifter to compile a \LaTeX\ document,
% and the PDF may be previewed onscreen or printed out for ease of proofreading.
% Mathematical graphics can be integrated within the main file through \TikZ,
% and of course \LaTeX\ macros can be customized.
% Using the present \textsf{moodle.sty} package,
% a quiz author can type a quiz using familiar \LaTeX\ syntax and document 
% structure.  Upon compilation, \LaTeX\ will generate both a well-organized
% PDF that is easy to proofread and an XML file that can be uploaded directly
% to Moodle.  The entire process is far faster than using Moodle's own
% web-based editor, makes it easier to catch one's mistakes, 
% and the ultimate source code of one's work is a human-readable |.tex| file 
% that can be archived, browsed, and edited offline.
% 
% \section{Workflow}
% The process of creating a quiz in Moodle using this package is as follows:
% \begin{enumerate}
%   \item Write a \LaTeX\ document using |\usepackage{moodle}| as described below.
%   \item Compile the document to PDF using pdf\LaTeX.  This will also produce the file
%         \meta{jobname}|-moodle.xml|.
%   \item Open Moodle, navigate to the desired course,
%         and under ``Question bank'' select ``Import.''
%   \item Select ``Moodle XML format,'' choose the XML file to upload, and press ``Import.''
%   \item After Moodle verifies that the questions have been imported correctly,
%         you may add them to your quizzes.
% \end{enumerate}
% 
% \section{Usage}\label{sect:usage}
% \subsection{Example Document}
%
% The following pages presume the reader already has some familiarity with creating
% and editing Moodle quizzes through the web interface.
% The |xkeyval| package is used to provide a key-value interface.
% Here is a simple example document:
% \begin{verbatim}
%   \documentclass[12pt]{article}
%   \usepackage{moodle}
%   \def\myreactiontoasillyanswer{What!?}
%   \htmlregister{\myreactiontoasillyanswer}
%   \begin{document}
%   
%   \begin{quiz}{My first quiz}
%     \begin{numerical}[points=2]{Basic addition}
%       What is $8+3$?
%       \item 11
%     \end{numerical}
%     
%     \begin{shortanswer}[case sensitive=true]{Newton's name}
%       What was Newton's first name?
%       \item Isaac
%       \item[fraction=0, feedback={No, silly!}] Fig
%       \item[fraction=0] Sir
%     \end{shortanswer}
%     
%     \begin{multi}[points=3]{A first derivative}
%       What is the first derivative of $x^3$?
%       \item[feedback={\myreactiontoasillyanswer}]  $\frac{1}{4} x^4+C$
%       \item* $3x^2$
%       \item[feedback={\myreactiontoasillyanswer}]  $51$
%     \end{multi}
%   \end{quiz}
%   
%   \end{document}
% \end{verbatim}
% Key features to note in this first example are that a |quiz| environment
% contains several question environments.
% Each question takes a name as a mandatory argument, 
% and it may also take optional key-value arguments within brackets.
% The question environments resemble list environments
% such as |itemize| or |enumerate|, in that answers are set off by 
% |\item|'s, but the question itself is the text that occurs before
% the first |\item|.
% 
% \DescribeMacro{\htmlregister}
% Using |\htmlregister|\marg{macroname}, tells \LaTeX\ about the macros you 
% use. This way, the macros will be properly expanded in the XML file.
%
% \newpage
% 
% \DescribeMacro{draft}
% \DescribeMacro{final}
% If the package option |draft| is invoked, by calling
% |\usepackage[draft]{moodle}| or |\documentclass[draft]{...}|, then no
% XML file will be generated. This is especially useful while editing a
% quiz containing graphics, so as to avoid the time spent converting
% image files. The package option |final| might be useful if one wants
% to avoid the option |draft| to be inherited from the |documentclass|.
%
% \subsection{Quiz and Question Environments}
% 
% \DescribeMacro{quiz}
% A |.tex| document to generate Moodle quizzes contains one or more
% |quiz| environments, within which various question environments are nested.
% The required argument to the |quiz| environment names the ``question bank''
% to which the questions inside will belong after being uploaded to moodle.
% \begin{center}
% |\begin{quiz}|\oarg{options}\marg{question bank name}%
% \end{center}
% There are no |quiz|-specific options,
% but any \meta{options} set with |\begin{quiz}|
% will be inherited by all questions contained within that |quiz| environment.
% 
% \DescribeMacro{\moodleset}
% Options may also be set at any time with |\moodleset|\marg{options};
% these changes are local to \TeX-groups.
% \bigskip
% 
% The syntax for each question environment is
% \begin{quote}
%   |\begin|\marg{question type}\oarg{question options}\marg{question name} \\
%   \rule{2em}{0pt}\meta{question text} \\
%   \rule{2em}{0pt}|\item| \meta{item} \\
%   \rule{2em}{0pt}\quad$\vdots$ \\
%   \rule{2em}{0pt}|\item| \meta{item} \\
%   |\end|\marg{question type}
% \end{quote}
% The meaning of the \meta{item}s varies depending on the question type,
% but they usually are answers to the question.
% Details will be given below.
% 
% The following key-value options may be set for all questions:
% 
% \DescribeMacro{points}\DescribeMacro{default grade}
% By default, each question is worth 1 point on the quiz.
% This may be changed with the |points| key or its synonym, |default grade|;
% for example, |points=2| makes that question worth two points.
% 
% \DescribeMacro{penalty}
% The |penalty| is the fraction of points that is taken off for each wrong attempt;
% it may be set to any value between 0 and 1.
% The default is |penalty=0.10|.
% 
% \DescribeMacro{fraction}
% In most question types, it is possible to designate some answers as being
% worth partial credit---that is, some fraction of a completely correct answer.
% The |fraction| key may be set to any integer from |0| (entirely wrong) to |100| (entirely correct).
% 
% \DescribeMacro{feedback}
% The |feedback| key sets text that will appear to the student after completing the quiz.
% For example, one might set 
% \begin{center}
%   |feedback={A question like this will appear on the exam.}|
% \end{center}
% The desired feedback should be included in braces.
% 
% Two kinds of feedback can be given.  If the |feedback| key is set for a 
% question, then that feedback will appear to each student regardless of the student's answer.
% Answer-specific feedback (perhaps explaining a common mistake)
% may also be given by setting the |feedback| key \emph{at the individual answer}.
% 
% \subsection{Question Types}
% 
% We next discuss the various question types supported by \textsf{moodle.sty}
% and the options that may be set.
% 
% \subsubsection{Multiple Choice}
% 
% \DescribeMacro{multi}
% The syntax for a classic multiple choice question,
% with only one correct answer, is as follows:
% \begin{quote}
%   |\begin{multi}|\oarg{question options}\marg{question name} \\
%   \rule{2em}{0pt}\meta{question text} \\
%   \rule{2em}{0pt}|\item*| \meta{correct answer} \\
%   \rule{2em}{0pt}|\item|\oarg{options} \meta{wrong answer} \\
%   \rule{2em}{0pt}\quad$\vdots$ \\
%   \rule{2em}{0pt}|\item|\oarg{options} \meta{wrong answer} \\
%   |\end{multi}|
% \end{quote}
% The correct answer is designated by the asterisk |*| after the |\item|;
% it need not appear first in the list.
% 
% 
% \DescribeMacro{shuffle}
% The boolean key |shuffle| determines whether Moodle will
% rearrange the possible answers in a random order.
% Setting |shuffle=false| will guarantee that the answer appear
% in the order they were typed; the default is |shuffle=true|.
% 
% \DescribeMacro{numbering}
% Moodle offers different options for numbering the possible answers.
% You may set the |numbering| key to any of the following values,
% which mirror the usual \LaTeX\ syntax:
% |alph|, |Alph|, |arabic|, |roman|, |Roman|, and |none|.
% Calling |numbering=none| produces an unnumbered list of answers.
% The Moodle syntax of |abc|, |ABCD|, |123|, |iii|, and |IIII| is also 
%acceptable,
% but note that it requires \emph{four} |I|'s to obtain upper-case Roman numerals this way.
% 
% \DescribeMacro{fraction}
% The |fraction| key can be used to designate some wrong answers
% as being worth partial credit.  For example, a question might read thus:
% \begin{verbatim}
%   \begin{multi}{my question}
%     Compute $\int 4x^3\,dx$.
%     \item* $x^4+C$
%     \item[fraction=50] $x^4$
%     \item $12x^2$
%   \end{multi}
% \end{verbatim}
% Thus the asterisk |*| is shorthand for |fraction=100|, 
% whereas a bare |\item| sets |fraction=0|.
% 
% \DescribeMacro{single}\DescribeMacro{multiple}
% By default, the |multi| environment produces
% a multiple choice question operates with only one correct answer;
% this is called |single| mode, and on Moodle it appears with radio buttons.
% It is also possible to write questions with more than one correct answer,
% asking the user to check all correct answers.
% To do this, use the key |multiple| or |single=false|.
% 
% The worth of each correct answers in |multiple| mode may be set by |fraction|,
% but Moodle will require that all the fractions add up to at least 100.
% If you simply designate each correct answer with |\item*|, then \textsf{moodle.sty}
% will divide the 100\% equally among those answers.
% For example, the following two examples produce equivalent results:
% \begin{verbatim}
%   \begin{multi}[multiple]{my question}
%     Which numbers are prime?
%     \item* 5
%     \item 6
%     \item* 7
%     \item 8
%   \end{multi}
% \end{verbatim}
% \begin{verbatim}
%   \begin{multi}[multiple]{my question}
%     Which numbers are prime?
%     \item[fraction=50] 5
%     \item 6
%     \item[fraction=50] 7
%     \item 8
%   \end{multi}
% \end{verbatim}
% 
% \subsubsection{Numerical}
%
% A numerical question in Moodle requires the student
% to input a real number in decimal form.
% Its typical format is
% \begin{quote}
%   |\begin{numerical}|\oarg{question options}\marg{question name} \\
%   \rule{2em}{0pt}\meta{question text} \\
%   \rule{2em}{0pt}|\item|\oarg{options} \meta{correct answer} \\
%   |\end{numerical}|
% \end{quote}
% If there is more than one correct answer, additional |\item|'s may be included.
% Because this is not a multiple choice question, there is no need to provide
% incorrect answers.  There may nevertheless be reasons to include incorrect answers.
% For example, partially correct answers may be specified by setting the |fraction| key.
% Feedback for a common mistake may be given by including the incorrect answer like this:
% \begin{quote}\footnotesize
%   |\item[fraction=0,feedback={You forgot to antidifferentiate!}]| \meta{incorrect answer}
% \end{quote}
%
% \DescribeMacro{tolerance} 
% The |tolerance| key can be used to specify the validity of answers within some margin.
% For example, with the question
% \begin{verbatim}
%   \begin{numerical}[tolerance=0.01]{my question}
%     Approximate value of $\sqrt{2}$?
%     \item 1.4142
%     \item[fraction=0,feedback={twice this!}] 0.70711
%   \end{numerical}
% \end{verbatim}
% In this case, the input \textsf{1.41} will be validated and the input \textsf{0.71} will
% get the specific feedback.
% 
% Units, unit-handling and multipliers are currently unsupported.
%
% \subsubsection{Short Answer}
% A short answer question resembles a numerical question: the student is to fill
% in a text box with a missing word or phrase.
% \begin{quote}
%   |\begin{shortanswer}|\oarg{question options}\marg{question name} \\
%   \rule{2em}{0pt}\meta{question text} \\
%   \rule{2em}{0pt}|\item|\oarg{options} \meta{correct answer} \\
%   \rule{2em}{0pt}\quad$\vdots$ \\
%   \rule{2em}{0pt}|\item|\oarg{options} \meta{correct answer} \\
%   |\end{shortanswer}|
% \end{quote}
% You can make the text box appear as part of the question with the 
% control sequence |\blank|.  For example,
% your question might read 
% \begin{verbatim}
%   \begin{shortanswer}{Leibniz}
%     Newton's rival was Gottfried Wilhelm \blank.
%     \item Leibniz
%     \item Leibniz.
%   \end{shortanswer}
% \end{verbatim}
% Note that as the blank occurred at the end of a sentence, 
% we included two answers, 
% lest students get the question wrong merely by
% including or omitting a period.
% 
% \DescribeMacro{case sensitive}\DescribeMacro{usecase}
% By default, Moodle ignores the distinction between upper- and lower-case letters
% when grading a short answer question.
% You can make a question case-sensitive with the key |case sensitive| 
% or its shorter synonym |usecase|.
% 
% \subsubsection{Essay Questions}
% Instructors may ask essay questions on a Moodle quiz,
% although Moodle's software is not up to the task of grading them!
% Instead each essay question answer must be graded manually by the
% instructor or a teaching assistant.
% \begin{quote}
%   |\begin{essay}|\oarg{question options}\marg{question name} \\
%   \rule{2em}{0pt}\meta{question text} \\
%   \rule{2em}{0pt}|\item|\oarg{options} \meta{notes for grader} \\
%   \rule{2em}{0pt}\quad$\vdots$ \\
%   \rule{2em}{0pt}|\item|\oarg{options} \meta{notes for grader} \\
%   |\end{essay}|
% \end{quote}
% Instead of containing answers, the |\item| tags for the |essay| question 
% contain notes that will appear to whoever is grading the question manually.
% 
% \DescribeMacro{response required}
% Although Moodle cannot grade the content of an essay question,
% it can at least determine whether the question has been left blank.
% If the |response required| key is set, Moodle will insist that the student
% enter something in the blank before accepting the quiz as completed.
% 
% \DescribeMacro{response format}
% Moodle offers five different ways for students to enter and/or upload their
% answers to an essay question.  You may choose one of these five options:
% \begin{description}
%   \item[{\tt html}] An editor with the ability to format HTML responses
%         including markup for italics, boldface, etc.  This is the default.
%   \item[{\tt file}] A file picker allowing the student to upload a file, such as
%         a PDF or DOC file, containing the essay.
%   \item[{\tt html+file}] The same HTML editor as above, but with the ability
%         to upload files as well.  This permits some students to type answers
%         directly into the web form, and others to compose their essays in
%         another program first.
%   \item[{\tt text}] This editor allows only for entering plain text without any markup.
%   \item[{\tt monospaced}] This yields a plain text editor, without any markup,
%         and with a fixed-width font.  This could be useful for entering code snippets, for example.
% \end{description}
% 
% \DescribeMacro{response field lines}
% The key |response field lines| controls the height of the input box.
% The default is |response field lines=15|.
% 
% \DescribeMacro{attachments allowed}
% The |attachments allowed| key controls \emph{how many} attachments a student is
% allowed to upload.  Permissible values are |0|, |1|, |2|, |3|, or |unlimited|.
% 
% \DescribeMacro{attachments required}
% You may also require the student to upload a certain number of attachments
% by setting |attachments required| to |0|, |1|, |2|, or |3|.
% 
% \DescribeMacro{template}
% Finally, you may preload the essay question with a template that the student
% will edit and/or type over, with the key |template=|\marg{template}.
% The \meta{template} should be enclosed in braces.
% 
% \subsection{Matching Questions}
% 
% A matching question offers a series of subquestions
% and a set of possible answers from which to choose.
% If there are $m$ questions and $n$ possible answers,
% a matching question will look like this:
% 
% \begin{quote}
%   |\begin{matching}|\oarg{question options}\marg{question name} \\
%   \rule{2em}{0pt}\meta{question text} \\
%   \rule{2em}{0pt}|\item|\oarg{options} \meta{question 1} |\answer| \meta{answer 1}\\
%   \rule{2em}{0pt}|\item|\oarg{options} \meta{question 2} |\answer| \meta{answer 2}\\
%   \rule{2em}{0pt}\quad$\vdots$ \\
%   \rule{2em}{0pt}|\item|\oarg{options} \meta{question $m$} |\answer| \meta{answer $m$}\\
%   \rule{2em}{0pt}|\item|\oarg{options} |\answer| \meta{answer $m+1$}\\
%   \rule{2em}{0pt}\quad$\vdots$ \\
%   \rule{2em}{0pt}|\item|\oarg{options} |\answer| \meta{answer $n$}\\
%   |\end{matching}|
% \end{quote}
% Answers $1$ through $m$ correspond to questions $1$ through $m$;
% answers $m+1$ through $n$ are ``decoy'' answers.
% If multiple questions should have the same answer,
% be sure your typed answer match exactly, so that Moodle will not
% create duplicate copies of the same answer!
% 
% \DescribeMacro{shuffle}
% The |matching| question accepts the option of |shuffle| to randomly
% permute the questions and answers; by default |shuffle=true|.
% 
% \DescribeMacro{drag and drop}\DescribeMacro{dd}
% Moodle has two formats for matching questions.
% The older matching question offered a dropdown box
% for choosing the answer to each question;
% the newer ``drag and drop matching'' format
% shows all questions in one column,
% all answers in a second column, and allows students to drag the correct
% answer to the question using a mouse.
% To enable drag-and-drop matching, use the key `|drag and drop|' or `|dd|' for 
% short.  The default is |dd=false|.  If you choose the older format,
% then due to the limitations of dropdown boxes, no \LaTeX\ or HTML code can be used
% in the answers.
% 
% \subsection{Cloze Questions}
% 
% A ``cloze question'' has one or more subquestions embedded within a passage of text.  
% For example, you might ask students to fill in several missing words within
% a sentence, or calculate several coefficients of a polynomial.
% To encode cloze questions in \LaTeX\ using this package is easy:
% you simply nest one or more |multi|, |shortanswer|, or |numerical| environments
% within a |cloze| environment, as in the following example:
% \begin{verbatim}
%   \begin{cloze}{my cloze question} \\
%     Thanks to calculus, invented by Isaac
%     \begin{shortanswer}
%       \item Newton
%     \end{shortanswer},
%     we know that the derivative of $x^2$ is
%     \begin{multi}[horizontal]
%       \item $2x$
%       \item* $\frac{1}{3} x^3 + C$
%       \item $0$
%     \end{multi}
%     and that $\int_0^2 x^2\,dx$ equals
%     \begin{numerical}
%       \item[tolerance=0.0004] 2.667
%     \end{numerical}.
%     Thanks, Isaac!
%   \end{cloze}
% \end{verbatim}
% Note that when used as a subquestion within a cloze question,
% |\begin{multi}| is \emph{not} followed by name in braces;
% the same is true for the |shortanswer| and |numerical|
% environments.
% 
% \DescribeMacro{single=true}
% Within a cloze question, a multiple choice question is necessarily of type
% |single|, i.e. with a single good answer.
%
% \DescribeMacro{vertical}\DescribeMacro{horizontal}\DescribeMacro{inline}
% Within a cloze question, by default a multiple choice question is implemented
% as an |inline| dropdown box. This is visually compact, but it also prevents
% the use of mathematical or HTML formatting.
% Adding the option |vertical| displays the subquestion as a vertical column
% of radio buttons instead; likewise the option |horizontal| creates a horizontal 
% row of radio buttons.
% 
% \section{Conversion to HTML}
% 
% Questions should be typed as usual for \LaTeX,
% including |\$| to obtain dollar signs, |$|'s for math shifts,
% |$$|'s for display math, et cetera.
% The package \textsf{moodle.sty} automatically converts
% this \LaTeX\ code into HTML for web display.
% 
% The following table lists \LaTeX\ commands and environments that 
% will be converted to HTML: 
% \begin{center}
%   \begin{tabular}{lll}
%     |~| & |\#| & |\$| \\
%     |\&| & |\\| & |\S| \\
%     |\{| & |\}| & {\tt\catcode`\ =12\relax\catcode`\\=12\relax\ }\\
%     |\emph| & |\textbf| & |\par| \\
%     |\ldots| & |\dots| & |\relax| \\
%     |\euro| & |\texteuro| \\
%     |\begin{center}| \\
%     |\begin{enumerate}| \\
%     |\begin{itemize}| \\
%     |\begin{tikzpicture}| \\
%   \end{tabular}
% \end{center}
% Single and double quotation marks and the diacritical commands  |\^|, |\'|, |\`|, |\"|, |\~|, and |\H| are also handled,
% as are the characters |\aa|, |\ae|, |\o|, |\ss|, |\l|, and their capitalizations.
%   
% In addition, |<| and |>| will be converted to |&lt;| and |&gt;| \emph{within math mode only}.
% If they should be typed outside of math mode, they will be passed as typed to
% the HTML, and probably interpreted by students' browsers as HTML tags
% or other unpredicated results.
% 
% Be aware that \emph{\textsf{moodle.sty} does not know how to convert any 
% other \TeX\ or \LaTeX\ commands to HTML.}
% If other sequences are used, they may be passed verbatim to the XML file
% or may cause unpredicted results.
% 
% If you think of another \LaTeX\ command that should be changed to an HTML equivalent, 
% please contact the author at \url{anders.hendrickson@snc.edu}
% so that it may be added to a future revision of the package.
% 
% \section{Graphics}
% The \textsf{moodle.sty} package can handle two kinds of graphics seamlessly.
% External graphics files may be included with the |\includegraphics| command
% from the |graphicx| package,
% and graphics may be generated internally using \TikZ.  In either case, the 
% graphics will be embedded in base-64 encoding directly within
% the Moodle~XML produced.  This prevents the hassle of managing separate
% graphics files on the Moodle server, as Moodle will store the picture 
% within the question in the question bank.
% 
% \DescribeMacro{\includegraphics}\DescribeMacro{height}\DescribeMacro{width}
% When using |\includegraphics|, the only options currently supported 
% are |height| and |width|.  Attempts to use other |\includegraphics| options,
% such as |scale| or |angle|, will affect the PDF but not the XML output.
% The dimensions set by |height| and |width| are \TeX\ dimensions such as |4in| or |2.3cm|.
% In order to prepare the image for web viewing, this package converts those 
% dimensions to pixels using a default of 
% 103 pixels per inch.\footnote{This
%   number was selected because an image with 
%   |<IMG HEIGHT=103 WIDTH=103 SRC="...">| showed up as almost exactly 1 inch 
%   tall and 1 inch wide on several of this author's devices and browsers 
%   as of January 2016.}
% \DescribeMacro{ppi}
% That value may be changed by setting the |ppi| key (e.g., |ppi=72|); 
% this is probably best done for the entire document with a |\moodleset| command,
% rather than image-by-image.
% 
% The mechanism used is somewhat fragile and relies upon one \TikZ\ library and three free external programs.
% \begin{enumerate}
%   \item \textsf{moodle.sty} invokes |\usetikzlibrary{external}|,
%         so that each |tikzpicture| environment is compiled to a freestanding PDF file.
%   \item GhostScript (\url{www.ghostscript.com}) is used to convert the PDF output
%         from \TikZ\ into a PNG raster graphics file.
%         The default command line is presumed to be |gswin64c.exe| 
%         (if |\ifwindows| from the |ifplatform| package returns true)
%         or |gs| (if |\ifwindows| returns false).
%         If your system requires a different command line to invoke Ghostscript,
%         \DescribeMacro{\ghostscriptcommand}
%         you may change it by invoking |\ghostscriptcommand|\marg{executable filename}.
%   \item When external graphics files such as JPG or GIF are included,
%         the open-source ImageMagick software (\url{www.imagemagick.org})
%         converts each file to PNG format.
%         The command line for ImageMagick is the nondescript word |convert|,
%         \DescribeMacro{\imagemagickcommand}
%         but may be changed by invoking |\imagemagickcommand|\marg{executable filename}.
%   \item OptiPNG (\url{http://optipng.sourceforge.net/}) is used to optimize the PNG images.
%         The command line is presumed to be |optipng|, but can be changed with
%         \DescribeMacro{\optipngcommand}
%         |\optipngcommand|\marg{executable filename}.
%   \item Finally, the OpenSSL toolkit (\url{www.openssl.org}) is used to base64-encode 
%         the PNG file so that it can be included directly in the text of the XML file.
%         The command line is presumed to be |openssl|, but can be changed with
%         \DescribeMacro{\opensslcommand}
%         |\opensslcommand|\marg{executable filename}.
% \end{enumerate}
% 
% Please note the following vital points to make the graphics handling work:
% \begin{itemize}
%   \item As of now, graphics are only supported when compiling directly to a PDF
%         with |pdflatex|.  Including PS graphics or using \TikZ\ with the DVI$\to$PS workflow is not
%         yet supported.
%   \item Filenames should not contains spaces or special characters like |_| or |\|.
%   \item You must have Ghostscript, ImageMagick, OptiPNG, and OpenSSL installed on your system
%         to fully use the graphics-handling capabilities of \textsf{moodle.sty}.
%   \item \LaTeX\ must be able to call system commands; that is, |\write18| must be enabled.
%         For Mik\TeX, this means adding |--enable-write18| to the command line of |pdflatex|;
%         for \TeX Live, this means adding |--shell-escape=true|.
% \end{itemize}
% 
% \section{Known Limitations and Call for Bug Reports}
% Table~\ref{tab:2} lists some different features supported, limitations, and
% bugs.
%
%\begin{table*}[tbp]
%\centering
%\small
%\def\OKcell{\cellcolor{green}}
%\def\KOcell{\cellcolor{red}}
%\def\DNAcell{\cellcolor{black!20}}
%\def\Warncell{\cellcolor{orange}}
%\def\MyLine{\hhline{~-|>{\arrayrulecolor{green}}->{\arrayrulecolor{black}}--|>{\arrayrulecolor{green}}->{\arrayrulecolor{black}}--|}
%}
%\begin{threeparttable}[b]
%\caption{Content enrichment support after XML import in Moodle 
%\texttt{v3.1}, depending on the question type. Color convention: green for 
%\colorbox{green}{full support}, orange for \colorbox{orange}{problems 
%limiting the support}, red for \colorbox{red}{show-stopping problems}, and 
%gray for \colorbox{black!20}{support limitations intrinsic to the question 
%type} defined by Moodle.}
%\label{tab:2}
%\begin{tabular}{rl|ccc|ccc}
%\multicolumn{2}{c|}{}& \multicolumn{3}{c|}{\LaTeX{} equation rendering} & 
%\multicolumn{3}{c}{Picture inclusion}\\
%\multicolumn{2}{c|}{Question type}& \rotatebox{55}{Question} & 
%\rotatebox{55}{Answer} & \rotatebox{55}{Feedback} & \rotatebox{55}{Question} 
%& \rotatebox{55}{Answer} & \rotatebox{55}{Feedback}\\\hhline{*{8}{:=}:}
%
%&\href{https://docs.moodle.org/31/en/Multiple_Choice_question_type} 
%{Multichoice} & \OKcell & \OKcell & \OKcell & \Warncell bug\tnote{9} & 
%\Warncell bug\tnote{9} & \Warncell bug\tnote{9}\\\hhline{*{8}{-}}
%
%& \href{https://docs.moodle.org/31/en/Numerical_question_type}{Numerical}
%& \OKcell & \DNAcell Text only\tnote{3} & \OKcell & \Warncell bug\tnote{9} & 
%\DNAcell Text only\tnote{3} & \Warncell bug\tnote{9}\\\hhline{*{8}{-}}
%
%& \href{https://docs.moodle.org/31/en/Short-Answer_question_type}{Short
%Answer} & \OKcell & \DNAcell Text only\tnote{3} & \OKcell & \Warncell 
%bug\tnote{9} & \DNAcell Text only\tnote{3} & \Warncell 
%bug\tnote{9}\\\hhline{*{8}{-}}
%
%&Matching 
%(\href{https://docs.moodle.org/31/en/Matching_question_type}{std})
%& \OKcell & \DNAcell Text only\tnote{4} & \DNAcell Undef.\tnote{5} & 
%\OKcell & \DNAcell Text only\tnote{4} & \DNAcell Undef.\tnote{5} 
%\\\hhline{*{8}{-}}
%
%&Matching 
%(\href{https://docs.moodle.org/31/en/Drag_and_drop_matching_question_type}
%{dd}) & \OKcell & \DNAcell bug\tnote{2} & \DNAcell 
%Undef.\tnote{5} & \Warncell bug\tnote{9} & \DNAcell bug\tnote{9}, 
%bug\tnote{2} & \DNAcell Undef.\tnote{5} \\\hhline{*{8}{-}}
%
%&\href{https://docs.moodle.org/31/en/Essay_question_type}{Essay}  & \OKcell & 
%\OKcell note\tnote{6}, note\tnote{7} & \OKcell note\tnote{6} & 
%\Warncell bug\tnote{9} & \OKcell note\tnote{6}, note\tnote{7} & \Warncell 
%bug\tnote{9}\\\hhline{*{8}{-}}
%
%\parbox[t]{2mm}{\multirow{5}{*}{\rotatebox[origin=c]{90}{\href{https://docs.moodle.org/31/en/Embedded_Answers_(Cloze)_question_type}
%{Cloze}}}}
%&Numerical & \OKcell & \DNAcell Text only\tnote{3} & \KOcell bug\tnote{8} & 
%\OKcell & \DNAcell Text only\tnote{3} & \OKcell \\\MyLine
%
%&Short Answer & \OKcell & \DNAcell Text only\tnote{3} & \KOcell note\tnote{1}, 
%bug\tnote{8} & \OKcell & \DNAcell Text only\tnote{3} & \OKcell 
%note\tnote{1}\\\MyLine
%
%&Multi (dropdown) & \OKcell & \DNAcell Text only\tnote{4} &\KOcell 
%note\tnote{1}, bug\tnote{8} & \OKcell & \DNAcell Text only\tnote{4} & \OKcell 
%note\tnote{1} \\\MyLine
%
%&Multi (horizontal)& \OKcell & \Warncell bug\tnote{8} & \KOcell 
%bug\tnote{8} & \OKcell & \OKcell & \OKcell\\\MyLine
%
%&Multi (vertical)& \OKcell & \Warncell bug\tnote{8} & \KOcell bug\tnote{8} & 
%\OKcell & \OKcell & \OKcell \\\hline
%\end{tabular}
%\begin{tablenotes}
%\item[1] Moodle only reveals the feedback when hovering the checkmark or X 
%mark with the mouse.
%\item[2] The drag-and-drop-matching plugin seems broken. Moodle's XML import 
%fails with a \textsf{dmlwriteexception} when the field content exceeds few 
%hundreds characters. This prevents the inclusion of most base64 images and 
%maybe some complicated equations. See this reported
%\href{https://github.com/jmvedrine/moodle-qtype_ddmatch/issues/6}{issue}.
%\item[3] Moodle prompts the student for an answer and then compares it to the 
%solutions provided. This is text-only.
%\item[4] Moodle uses a dropdown list to let one choose among the possible 
%answers. This forbids either picture inclusion and \LaTeX{} rendering.
%\item[5] Not supported by Moodle (in this context, answer-specific feedback 
%represents lots of possible combinations).
%\item[6] For this question type and in the context of XML generation, the 
%Answer column represents the ``template" while the Feedback column represents 
%the ``notes for the grader". Obviously, the grading process is not automatic 
%and there is no answer-specific feedback.
%\item[7] Picture and \LaTeX{} rendering could be done, but only after 
%submission and only if the keyval ``response format" is set to ``html".
%\item[8] In the generated XML, the symbol $\}$ is not escaped with a 
%backslash as required by the Moodle cloze parser. The XML import will fail 
%when the field contains \LaTeX{} code like \verb|\sqrt{...}| and 
%\verb|\frac{...}{...}|.
%\item[9] In the XML, the pictures included in the question field and answer or 
%feedback fields are mixed.
%\end{tablenotes}
%\end{threeparttable}
%\end{table*}
%
% Some features of Moodle quizzes have not yet been implemented in \textsf{moodle.sty}.
% Here is a non-exhaustive list.
% \begin{itemize}
%   \item Moodle's feature of designating feedback for correct, 
%         partially correct, and incorrect answers.
%   \item Calculated questions; that is, automatically generated numerical questions 
%         using randomly chosen numbers.
%   \item Hints
%   \item Keywords
%   \item So-called ``description'' questions.
% \end{itemize}
% I have used Version 0.5 for one semester's teaching,
% but if other users adopt this package, I fully expect them to find bugs.
% \emph{Please} send all bugs you find to \url{anders.hendrickson@snc.edu},
% so that I can fix them for subsequent versions.
% 
% \section{Compatibility}
% This package has been written for and tested with the implementation of 
% Moodle 2.9 run by Moodlerooms for St.~Norbert College in January 2016.
% Future versions of this package will probably include some support for
% specifying your version of Moodle in the |.tex| file to help ensure compatibility.
% 
% As the ultimate purpose of this package is the generation of XML files,
% future versions of \textsf{moodle.sty} will attempt to maintain backwards 
% compatability with earlier versions of regarding the XML output, apart from
% bug fixes.  
% Backwards compatibility of the PDF output is not yet guaranteed, however,
% in case the author or users discover better ways for the PDF to display 
% the underlying XML data to be proofread.
% 
% In other words, compiling your current |.tex| file with a future version 
% of \textsf{moodle.sty} should produce the same XML file it does now 
% (apart from bug fixes),
% but it might produce a more informative, and hence different,
% PDF output.
% 
% 
% \section{Unrelated Tip: Quality of Moodle \TeX\ Images}
% This has nothing to do with |moodle.sty|, but is a Frequently Asked Question
% in is own right.
% On some servers, at least, Moodle's default ``\TeX\ Filter'' for 
% displaying mathematical notation is of abysmally poor quality, rending mathematics
% as low-resolution PNG's.  One solution that has worked for me is to go 
% to ``Course Administration $\to$ Filters,'' turn ``\TeX\ Notation'' \emph{off},
% but turn ``MathJax'' \emph{on}.  This forces \TeX\ code to be rendered by MathJax
% instead of Moodle, producing much higher-quality results.
% 
% \PrintChanges
%
% \StopEventually{}
%
% \section{Implementation}
% \subsection{Packages, Options, and Utilities}
%    \begin{macrocode}
\newif\ifmoodle@draftmode
\newif\ifmoodle@tikzloaded

%%DECLARATION OF OPTIONS
\DeclareOption{draft}{\moodle@draftmodetrue}
\DeclareOption{final}{\moodle@draftmodefalse}

\moodle@draftmodefalse
\moodle@tikzloadedfalse

\ProcessOptions

\RequirePackage{environ} %To be able to take environment body as a macro argument
\RequirePackage{xkeyval} %For key-handling
\RequirePackage{amssymb} %For \checkmark symbol
\RequirePackage{trimspaces} %To remove extra spaces from strings
\RequirePackage{etex}    %Expansion control, detokenization, etc.
\RequirePackage{etoolbox}%List management
\RequirePackage{xpatch}  %To patch commands easily in HTML mode
\RequirePackage{array}   %For formatting tables in the LaTeX mode of Clozes
\RequirePackage{ifplatform} % To choose Ghostscript commands
\RequirePackage{ifpdf} % Needed to know whether we can convert output from PDF to PNG

\RequirePackage{getitems} %To gather the header and items

\let\xa=\expandafter
\def\@star{*}%
\def\@hundred{100}%
\def\@moodle@empty{}%
\def\@relax{\relax}%
%    \end{macrocode}
% As the package involves a fair bit of file processing,
% we automate the naming of auxiliary files.
%    \begin{macrocode}
\def\jobnamewithsuffixtomacro#1#2{%
  \filenamewithsuffixtomacro{#1}{\jobname}{#2}%
}
\def\@jn@quote{"}%
\def\filenamewithsuffixtomacro#1#2#3{%
  % #1 = macro to create
  % #2 = filename to add suffix to
  % #3 = suffix to add
  \edef\jn@suffix{#3}%
  \def\jn@macro{#1}%
  \xa\testforquote#2\@jn@rdelim
}
\def\testforquote#1#2\@jn@rdelim{%
  \def\jn@test@i{#1}%
  \ifx\jn@test@i\@jn@quote
    % Involves quotes
    \edef\jn@next{"\jn@stripquotes#1#2\jn@suffix"}%
  \else
    \edef\jn@next{#1#2\jn@suffix}%
  \fi
  \xa\xdef\jn@macro{\jn@next}%
}
\def\jn@stripquotes"#1"{#1}%


\jobnamewithsuffixtomacro{\outputfilename}{-moodle.xml}
%    \end{macrocode}
% Next, we create macros to open and close the Moodle XML file
% we will be writing.
%    \begin{macrocode}
\newwrite\moodle@outfile
\def\openmoodleout{%
  \immediate\openout\moodle@outfile=\outputfilename\relax
  \writetomoodle{<?xml version="1.0" encoding="UTF-8"?>}%
  \writetomoodle{<quiz>}%
  \writetomoodle{ }%
}%
\def\closemoodleout{%
  \writetomoodle{ }%
  \writetomoodle{</quiz>}%
  \immediate\closeout\moodle@outfile
}%
%    \end{macrocode}
% To both make this |.sty| file and the XML output more readable,
% we create a mechanism for writing to the output file with indents.
% The macro |\calculateindent|\marg{$n$} globally defines 
% |\moodle@indent| to be a string of \meta{$n$} |\otherspace|'s.
%    \begin{macrocode}
\def\calculateindent#1{%
  \bgroup
    \count0=\number#1\relax
    \gdef\moodle@indent{}%
    \calculateindent@int
  \egroup
}%
\def\calculateindent@int{%
  \ifnum\count0>0\relax
    \g@addto@macro{\moodle@indent}{\otherspace}%
    \advance\count0 by -1\relax
    \expandafter
    \calculateindent@int
  \fi
}%
%    \end{macrocode}
% Now the command |\writetomoodle|\oarg{n}\marg{stuff} adds the line 
% ``\meta{stuff}'' to the XML file
% preceded by an indent of \meta{n} spaces.
% Note that this command |\edef|'s its second argument.
%    \begin{macrocode}
\newcommand\writetomoodle[2][0]{%
  \edef\test@ii{#2}%
  \ifnum#1=0\relax
    \immediate\write\moodle@outfile{\test@ii}%
  \else
    \calculateindent{#1}%
    \immediate\write\moodle@outfile{\moodle@indent\trim@pre@space{\test@ii}}%
  \fi
}%
%    \end{macrocode}
% We now create the main |quiz| environment that will contain
% the questions we write.  It outputs to XML as a |<question type="category">| tag.
%    \begin{macrocode}
\newenvironment{quiz}[2][]%
  {\setkeys{moodle}{#1}%
   \@moodle@ifgeneratexml{%
     \openmoodleout%
     \setcategory{#2}%
    }{}%
    \subsection*{#2}%
    \begin{enumerate}%
   }
  {\end{enumerate}%
   \@moodle@ifgeneratexml{\closemoodleout}{}}

{\catcode`\$=12\catcode`\ =12%
\gdef\setcategory#1{%
\writetomoodle{<question type="category">}%
\writetomoodle{  <category>}%
\writetomoodle{    <text>$module$/#1</text>}%
\writetomoodle{  </category>}%
\writetomoodle{</question>}%
\writetomoodle{ }%
}}%
%    \end{macrocode}                       
% The next utility takes a single macro control sequence |#1|,
% and allows that macro's current value to persist
% after the next |\egroup|, |}|, or |\endgroup|.
%    \begin{macrocode}
\def\passvalueaftergroup#1{%
  \xa\xa\xa\gdef\xa\xa\csname moodle@remember@\string#1\endcsname\xa{\xa\def\xa#1\xa{#1}}%
  \xa\aftergroup\csname moodle@remember@\string#1\endcsname
}
%    \end{macrocode}                       
%
% \subsubsection{Main Switch: to create XML or not}
% 
%    \begin{macrocode}
\long\def\@moodle@ifgeneratexml#1#2{%
  % If we are generating XML, do #1; otherwise do #2.
  \tikzifexternalizing{%
    % This run of LaTeX is currently ONLY generating a Tikz image
    % to be saved in an external file.  We do NOT want to waste time
    % generating XML, and moreover trying to do so would cause errors
    % because of file dependencies.
    #2%
  }{%
    \ifmoodle@draftmode
      #2%
    \else
      #1%
    \fi
  }%
}
%    \end{macrocode}
%
% \subsection{Key-Value Pairs for Quiz Questions}
% 
% The various options are set using key-value syntax of |xkeyval|.
%    \begin{macrocode}
\def\moodleset#1{\setkeys{moodle}{#1}}%
%    \end{macrocode}
% We first define some macros that will help us write other macros.
% Calling |\generate@moodle@write@code|\marg{name}|<|\meta{HTML tag}|>|\marg{text to write}
% creates a macro |\moodle@write|\meta{name}, taking no parameters,
% which writes the code |<|\meta{HTML tag}|>...</|\meta{HTML tag}|>|
% to the output XML file.
% 
% The ordinary version |\generate@moodle@write@code| passes its output text |#3|
% through the HTMLizer, producing HTML code, while the starred version 
% |\generate@moodle@write@code*| passes |#3| verbatim as text.
% 
% For example, 
% |\generate@moodle@write@code{excuse}<EXC>{\theexcuse}|
% would expand to
% \begin{Verbatim}[gobble=4,frame=single]
%   \gdef\moodle@writeexcuse{%
%     \xa\def\xa\test@iii\xa{\theexcuse}%
%     \ifx\test@iii\@moodle@empty
%       \writetomoodle[2]{  <EXC format="html"><text/></EXC>}%
%     \else
%       \xa\converttohtmlmacro\xa\moodle@htmltowrite\xa{\theexcuse}%
%       \writetomoodle[2]{  <EXC format="html">}%
%       \writetomoodle[4]{    <text><![CDATA[<p>\moodle@htmltowrite</p>]]></text>}%
%       \writetomoodle[2]{  </EXC>}%
%     \fi
%   }%
% \end{Verbatim}
% but 
% |\generate@moodle@write@code*{excuse}<EXC>{\theexcuse}|
% would expand only to
% \begin{Verbatim}[gobble=4,frame=single]
%   \gdef\moodle@writeexcuse{%
%      \writetomoodle[2]{  <EXC>\theexcuse</EXC>}%
%   }
% \end{Verbatim}
%    \begin{macrocode}
\def\generate@moodle@write@code{%
  \@ifnextchar*\generate@moodle@write@data\generate@moodle@write@html
}%

\def\generate@moodle@write@html#1<#2>#3{%
  % #1 = NAME for \moodle@writeNAME
  % #2 = HTML tag
  % #3 = what, exactly, to write
  \xa\gdef\csname moodle@write#1\endcsname{%
    \xa\def\xa\test@iii\xa{#3}%
    \ifx\test@iii\@moodle@empty
      \writetomoodle[2]{  <#2 format="html"><text/></#2>}%
    \else
      \xa\converttohtmlmacro\xa\moodle@htmltowrite\xa{#3}%
      \writetomoodle[2]{  <#2 format="html">}%
      \writetomoodle[4]{    <text><![CDATA[<p>\moodle@htmltowrite</p>]]></text>}%
      \writetomoodle[2]{  </#2>}%
    \fi
  }%
}%

\def\generate@moodle@write@html@noptag#1<#2>#3{%
	% No <P>..</P> introduced
	% #1 = NAME for \moodle@writeNAME
	% #2 = HTML tag
	% #3 = what, exactly, to write
	\xa\gdef\csname moodle@write#1\endcsname{%
		\xa\def\xa\test@iii\xa{#3}%
		\ifx\test@iii\@moodle@empty
		\writetomoodle[2]{  <#2 format="html"><text/></#2>}%
		\else
		\xa\converttohtmlmacro\xa\moodle@htmltowrite\xa{#3}%
		\writetomoodle[2]{  <#2 format="html">}%
		\writetomoodle[4]{    <text><![CDATA[\moodle@htmltowrite]]></text>}%
		\writetomoodle[2]{  </#2>}%
		\fi
	}%
}%

\def\generate@moodle@write@data*#1<#2>#3{%
  % #1 = NAME for \moodle@writeNAME
  % #2 = HTML tag
  % #3 = what, exactly, to write
  \xa\gdef\csname moodle@write#1\endcsname{%
    \writetomoodle[2]{  <#2>#3</#2>}%
  }%
}%

%    \end{macrocode}
% \subsubsection{Keys for all question types}
%    \begin{macrocode}
%% QUESTIONNAME
      \define@cmdkey{moodle}[moodle@]{questionname}{}%
      \gdef\moodle@writequestionname{%
        \writetomoodle[2]{<name>}%
        \writetomoodle[4]{  <text>\moodle@questionname</text>}%
        \writetomoodle[2]{</name>}%
      }%

%% QUESTIONTEXT
      %I tried to use questiontext as a key, but it doesn't seem to work.
      %The trouble is that xkeyval has trouble parsing a key with a \par token followed by a comma within brackets,
      %like \setkeys{moodle}{questiontext={ABC\par [D,E]}}
      %It's not worth trying to fix.

      \long\def\questiontext#1{%
        %\converttohtmlmacro\myoutput{#1}%
        %\let\moodle@questiontext=\myoutput%
        \def\moodle@questiontext{#1}%
      }%
      \generate@moodle@write@code{questiontext}<questiontext>{\moodle@questiontext}%{%

%% PENALTY FOR WRONG ATTEMPT
      \define@cmdkey{moodle}[moodle@]{penalty}[0.10]{}%
      \generate@moodle@write@code*{penalty}<penalty>{\moodle@penalty}%
      \moodleset{penalty}%

%% FEEDBACK
      % Moodle allows for feedback tailored to each question,
      % and feedback tailored to each right or wrong answer.
      % We shall use the key 'feedback' to record both kinds of feedback,
      % relying on TeX's grouping mechanism to keep them apart.
      % When it comes time to write them to XML,
      % \moodle@writegeneralfeedback uses the HTML tag <generalfeedback>
      % whereas \moodle@writefeedback uses the tag <feedback>.
      % Note that the general feedback is NOT inherited by each answer!
      \define@cmdkey{moodle}[moodle@]{feedback}[]{}%
      \generate@moodle@write@code{generalfeedback}<generalfeedback>{\moodle@feedback}%
      \generate@moodle@write@code{feedback}<feedback>{\moodle@feedback}%
      \moodleset{feedback}%

%% DEFAULT GRADE
      %The next line creates \moodle@defaultgrade,
      %which is how many points the quiz question is worth.
      %Key calls like [default grade=7] set \moodle@defaultgrade.
      \define@cmdkey{moodle}[moodle@]{default grade}[1.0]{}%
      %Next, makes 'points' a synonym for 'default grade'
      \define@key{moodle}{points}[1.0]{\xa\def\csname moodle@default grade\endcsname{#1}} 
      \generate@moodle@write@code*{defaultgrade}<defaultgrade>{\csname moodle@default grade\endcsname}%
      \moodleset{default grade=1.0} %This sets the default.

%% HIDDEN
      \define@boolkey{moodle}[moodle@]{hidden}[true]{}%
      \generate@moodle@write@code*{hidden}<hidden>{\ifmoodle@hidden 1\else 0\fi}%
      \moodleset{hidden=false}%

\def\moodle@writecommondata{%
  \moodle@writequestionname%
  \moodle@writequestiontext%
  \moodle@writedefaultgrade%
  \moodle@writegeneralfeedback%
  \moodle@writepenalty%
  \moodle@writehidden%
}%

% TO DO: 'tags' (i.e., keywords)

%    \end{macrocode}
% \subsubsection{Keys for all answers}
%    \begin{macrocode}
%% FRACTION -- how much this answer is worth out of 100 percent
      \define@cmdkey{moodle}[moodle@]{fraction}[100]{}%
      %We do not create \moodle@writefraction, because the fraction occurs in
      %the XML within the answer tag, like <answer fraction="75">.
      \moodleset{fraction=100} %This sets the default.
%    \end{macrocode}
% \subsubsection{Keys for multiple choice questions}
%    \begin{macrocode}

%% SINGLE and MULTIPLE -- for multichoice, is there 1 right answer or more than 1?
      \define@boolkey{moodle}[moodle@]{single}[true]{}%
      \generate@moodle@write@code*{single}<single>{\ifmoodle@single true\else false\fi}%
      \moodleset{single}%
      %The key 'multiple' is an antonym to 'single'.
      \define@boolkey{moodle}[moodle@]{multiple}[true]{\ifmoodle@multiple\moodle@singlefalse\else\moodle@singletrue\fi}%

%% SHUFFLE ANSWERS
      \define@boolkey{moodle}[moodle@]{shuffle}[true]{}%
      \generate@moodle@write@code*{shuffle}<shuffleanswers>{\ifmoodle@shuffle 1\else 0\fi}%
      \moodleset{shuffle=true}%

%% TO DO: CORRECTFEEDBACK
%% TO DO: PARTIALLYCORRECTFEEDBACK
%% TO DO: INCORRECTFEEDBACK
%% TO DO: NUMCORRECT key

%% NUMBERING -- for numbering of multichoice questions
      \define@choicekey{moodle}{numbering}%
                       {alpha,alph,Alpha,Alph,arabic,roman,Roman,%
                        abc,ABCD,123,iii,IIII,none}[abc]{%
                        \def\moodle@numbering{#1}%
                        \def\test@@i{#1}%
                        \ifx\test@@i\@moodle@alpha
                          \def\moodle@numbering{abc}\fi
                        \ifx\test@@i\@moodle@alph
                          \def\moodle@numbering{abc}\fi
                        \ifx\test@@i\@moodle@Alpha
                          \def\moodle@numbering{ABCD}\fi
                        \ifx\test@@i\@moodle@Alph
                          \def\moodle@numbering{ABCD}\fi
                        \ifx\test@@i\@moodle@arabic
                          \def\moodle@numbering{123}\fi
                        \ifx\test@@i\@moodle@roman
                          \def\moodle@numbering{iii}\fi
                        \ifx\test@@i\@moodle@Roman
                          \def\moodle@numbering{IIII}\fi
                        }%
      %'answer numbering' will be a synonym to 'numbering'
      \define@key{moodle}{answer numbering}[abc]{\setkeys{moodle}{numbering={#1}}}% 
      \generate@moodle@write@code*{answernumbering}<answernumbering>{\moodle@numbering}%
      %N.B. if we did not set the default here, then \moodle@numbering would be undefined, causing problems.
      \moodleset{answer numbering=abc}% 
      
      \def\@moodle@alpha{alpha}%
      \def\@moodle@Alpha{Alpha}%
      \def\@moodle@alph{alph}%
      \def\@moodle@Alph{Alph}%
      \def\@moodle@arabic{arabic}%
      \def\@moodle@roman{roman}%
      \def\@moodle@Roman{Roman}%      
      \def\@moodle@abc{abc}%
      \def\@moodle@ABCD{ABCD}%
      \def\@moodle@arabicnumbers{123}%
      \def\@moodle@iii{iii}%
      \def\@moodle@IIII{IIII}%
      \def\@moodle@none{none}%
      \def\moodle@obeynumberingstyle{%
        \ifx\moodle@numbering\@moodle@abc
          \renewcommand\theenumii{\alph{enumii}}%
        \fi
        \ifx\moodle@numbering\@moodle@ABCD
          \renewcommand\theenumii{\Alph{enumii}}%
        \fi
        \ifx\moodle@numbering\@moodle@arabicnumbers
          \renewcommand\theenumii{\arabic{enumii}}%
        \fi
        \ifx\moodle@numbering\@moodle@iii
          \renewcommand\theenumii{\roman{enumii}}%
        \fi
        \ifx\moodle@numbering\@moodle@IIII
          \renewcommand\theenumii{\Roman{enumii}}%
        \fi
        \ifx\moodle@numbering\@moodle@none
          \renewcommand\labelenumii{$\bullet$~}%
        \fi
      }
      %TO DO: * In the PDF, how should 'none' in a multi look different from 
      %         short answer or numerical options?
      %       * Instead of \theenumi and \labelenumi,
      %         use \@enumdepth to automatically set the correct depth.
      
%% DISPLAY MODE -- affects Cloze multiple choice questions only.
      % 0 = inline, 1 = vertical, 2 = horizontal
      \def\moodle@multi@mode{0}%
      \define@key{moodle}{inline}[]{\def\moodle@multi@mode{0}}%
      \define@key{moodle}{vertical}[]{\def\moodle@multi@mode{1}}%
      \define@key{moodle}{horizontal}[]{\def\moodle@multi@mode{2}}%
%    \end{macrocode}
% \subsubsection{Keys for numerical questions}
%    \begin{macrocode}
%% TOLERANCE
      \define@cmdkey{moodle}[moodle@]{tolerance}[0]{}%
      \moodleset{tolerance=0}%
      %There is no \moodle@writetolerance, because in the XML the
      %tolerance is given within the answer tag,
      %like <answer fraction=100 tolerance=0.03>.

% TO DO: implement unit-handling for numerical questions!
%    \end{macrocode}
% \subsubsection{Keys for short answer questions}
%    \begin{macrocode}
%% CASE SENSITIVE
      \define@boolkey{moodle}[moodle@]{case sensitive}[true]{}%
      \generate@moodle@write@code*{usecase}<usecase>{\csname ifmoodle@case sensitive\endcsname 1\else 0\fi}%
      % We make 'usecase' a synonym for 'case sensitive'.
      \define@boolkey{moodle}[moodle@]{usecase}[true]{\ifmoodle@usecase\csname moodle@case sensitivetrue\endcsname\else\csname moodle@case sensitivefalse\endcsname\fi}%
      \moodleset{case sensitive=false}%
%    \end{macrocode}
% \subsubsection{Keys for matching questions}
%    \begin{macrocode}
%% DRAG-AND-DROP FORMAT
      \define@boolkey{moodle}[moodle@]{draganddrop}[true]{}%
      % We make 'dd' and 'dragdrop' and 'drag and drop' synonyms for 'draganddrop'.
      \define@boolkey{moodle}[moodle@]{dd}[true]{\ifmoodle@dd\moodle@draganddroptrue\else\moodle@draganddropfalse\fi}%
      \define@boolkey{moodle}[moodle@]{drag and drop}[true]{\moodle@ddsynonym}%
      \def\moodle@ddsynonym{%
        \csname ifmoodle@drag and drop\endcsname
          \moodle@draganddroptrue
        \else
          \moodle@draganddropfalse
        \fi
      }
      \moodleset{draganddrop=false}%
%    \end{macrocode}
% \subsubsection{Keys for essay questions}
%    \begin{macrocode}
%% EDITOR
      \def\@moodle@html{html}%
      \def\@moodle@htmlfile{html+file}%
      \def\@moodle@text{text}%
      \def\@moodle@plain{plain}%
      \def\@moodle@monospaced{monospaced}%
      \def\@moodle@file{file}%
      \def\@moodle@noinline{noinline}%
      \define@choicekey{moodle}{response format}%
                       {html,html+file,text,monospaced,file}[html]%
                       {\def\test@i{#1}%
                        \ifx\test@i\@moodle@html
                          % HTML Editor
                          \def\moodle@responseformat{editor}% 
                        \fi
                        \ifx\test@i\@moodle@htmlfile
                          % HTML Editor with File Picker
                          \def\moodle@responseformat{editorfilepicker}%
                        \fi
                        \ifx\test@i\@moodle@text
                          % Plain text
                          \def\moodle@responseformat{plain}%
                        \fi
                        \ifx\test@i\@moodle@plain
                          % Plain text
                          \def\moodle@responseformat{plain}%
                        \fi
                        \ifx\test@i\@moodle@monospaced
                          % Plain text, monospaced font
                          \def\moodle@responseformat{monospaced}%
                        \fi
                        \ifx\test@i\@moodle@file
                          % No inline text (i.e., attachments only)
                          \def\moodle@responseformat{noinline}%
                        \fi
                        \ifx\test@i\@moodle@noinline
                          % No inline text (i.e., attachments only)
                          \def\moodle@responseformat{noinline}%
                        \fi
                       }%
      \generate@moodle@write@code*{responseformat}<responseformat>{\moodle@responseformat}%
      \moodleset{response format=html}% 
      %N.B. if we did not set a default, then \moodle@responseformat would be undefined, causing problems.

%% RESPONSE REQUIRED
      \define@boolkey{moodle}[moodle@]{response required}[true]{}%
      % TO DO: Make synonym 'required'
      \generate@moodle@write@code*{responserequired}<responserequired>{\csname ifmoodle@response required\endcsname 1\else 0\fi}%
      \moodleset{response required=false}%

%% RESPONSEFIELDLINES
      \define@cmdkey{moodle}[moodle@]{response field lines}[15]{}%
      \generate@moodle@write@code*{responsefieldlines}<responsefieldlines>{\csname moodle@response field lines\endcsname}%
      %Make synonyms 'input box size' or 'height' or 'lines'?
      \moodleset{response field lines=15}% N.B. if we do not set a default, then \moodle@responseformat will be undefined, causing problems.

%% ATTACHMENTS ALLOWED
      \def\@moodle@unlimited{unlimited}%
      \define@choicekey{moodle}{attachments allowed}{0,1,2,3,unlimited}[1]{%
        \def\test@i{#1}%
        \ifx\test@i\@moodle@unlimited
          \def\moodle@attachmentsallowed{-1}%
        \else
          \def\moodle@attachmentsallowed{#1}%
        \fi
      }
      \generate@moodle@write@code*{attachmentsallowed}<attachments>{\moodle@attachmentsallowed}
      \moodleset{attachments allowed=0}%

%% ATTACHMENTS REQUIRED
      \define@choicekey{moodle}{attachments required}{0,1,2,3}[1]{\def\moodle@attachmentsrequired{#1}}%
      \generate@moodle@write@code*{attachmentsrequired}<attachmentsrequired>{\moodle@attachmentsrequired}
      \moodleset{attachments required=0}%

%% RESPONSE TEMPLATE
      \define@key{moodle}{template}{\long\def\moodle@responsetemplate{#1}}%
      \generate@moodle@write@html@noptag{responsetemplate}<responsetemplate>{\moodle@responsetemplate}
      \moodleset{template={}}%
%    \end{macrocode}
% \subsubsection{Hint tags}
% The following are not yet fully implemented.
%    \begin{macrocode}
%% SHOWNUMCORRECT
      \define@boolkey{moodle}[moodle@]{shownumcorrect}[true]{}%
      \gdef\moodle@writeshownumcorrect{%
        \if\moodle@shownumcorrect
          \writetomoodle[4]{    <shownumcorrect/>}%
        \fi
      }%
      \moodleset{shownumcorrect=false}%

%% CLEARWRONG
      \define@boolkey{moodle}[moodle@]{clearwrong}[true]{}%
      \gdef\moodle@writeclearwrong{%
        \if\moodle@clearwrong
          \writetomoodle[4]{    <clearwrong/>}%
        \fi
      }%
      \moodleset{clearwrong=false}%

% TO DO: Implement hints
%    \end{macrocode}
%
% \subsection{Answer handling}
% 
%    \begin{macrocode}
%The Answers XML depends heavily on the question type.
%Each type of question defines how it obtains answers from the LaTeX input,
%how it typesets those in a PDF or DVI, and how it writes them as XML code.
%It will write that XML to the macro \moodle@answers@xml,
%which them gets written to the file when \moodle@writeanswers
%is invoked.

\def\moodle@answers@xml{}%
\gdef\moodle@writeanswers{%
  \writetomoodle{\moodle@answers@xml}%
}%

\newcommand\addto@xml[3][0]{%
  % #1 = spaces to indent (default=0)
  % #2 = macro containing XML code (possibly empty)
  % #3 = XML text to be appended to that macro (will be \edef'd)
  \calculateindent{#1}%
  \edef\xml@to@add{\moodle@indent\trim@pre@space{#3}}%
  \ifx#2\@moodle@empty
    \edef\newxml{\noexpand#2\xml@to@add}%
  \else
    \edef\newxml{\noexpand#2^^J\xml@to@add}%
  \fi
  \xa\xa\xa\def\xa\xa\xa#2\xa\xa\xa{\newxml}%
}%
%    \end{macrocode}
%
% \subsubsection{Not yet implemented}
% 
%    \begin{macrocode}
%%%%%%%%%%%%%%%%%%%%%%%%%%%%%%%%%%%%%%%%
%% TRUE/FALSE QUESTIONS %%%%%%%%%%%%%%%%%

% TODO: implement the \writetruefalsequestion
% and a front-end.

%%%%%%%%%%%%%%%%%%%%%%%%%%%%%%%%%%%%%%%%
%% DESCRIPTION 'QUESTIONS' %%%%%%%%%%%%%

% TODO: implement the \writedescription and a suitable front-end.
%       Should this be \begin{description}...\end{description},
%       or should \begin{quiz}...\end{quiz} just scoop up all
%       text outside question environments and package it in descriptions?



%%%%%%%%%%%%%%%%%%%%%%%%%%%%%%%%%%%%%%%%
%% CALCULATED %%%%%%%%%%%%%%%%%%%%%%%%%%

% TODO: I don't think I really want to handle this.  Not now.

%    \end{macrocode}

% \subsection{Front Ends}
% This section creates the user interface for the various question types.
% First, we define a generic command to create 
% a front-end environment for a Moodle question type.
% In order to function, the following macros must be hard-coded:
% \begin{itemize}
%   \item |\moodle@|\meta{type}|@latexprocessing|: 
%     Loops through the saved |\item|'s to typeset them in LaTeX,
%     usually inside an itemize or enumerate environment.
%   \item |\save|\meta{type}|answer#1|: 
%     Processes the text of a single |\item| to save the information to memory,
%     usually inside |\moodle@answers@xml|.
%   \item |\write|\meta{type}|question|:
%     Writes the information, hitherto saved only in macros,
%     into the XML file.
% \end{itemize}
% For example, to create the `shortanswer' question type,
% we shall call 
% \begin{Verbatim}[gobble=5,frame=single]
%    \moodle@makefrontend{shortanswer}
%    \def\moodle@shortanswer@latexprocessing{...}
%    \def\saveshortansweranswer#1{...}
%    \def\writeshortanswerquestion{...}
% \end{Verbatim}
% 
%    \begin{macrocode}

\def\moodle@makelatextagbox#1{%
  \framebox{\tiny #1}
}%

\def\moodle@makelatextag@qtype#1{%
  \moodle@makelatextagbox{\textsc{#1}}
}%

\def\moodle@makelatextag@value#1#2{%
  \moodle@makelatextagbox{\csname moodle@#1\endcsname~#2}
}%

\def\moodle@makelatextag@key#1{%
  \moodle@makelatextagbox{\csname moodle@#1\endcsname}
}%

\def\moodle@makefrontend#1#2{%
  \NewEnviron{#1}[2][]{%
    \bgroup
      \setkeys{moodle}{##1,questionname={##2}}%
      \expandafter\gatheritems\xa{\BODY}%
      \let\moodle@questionheader=\gatheredheader
      %First, the LaTeX processing
      \item \textbf{\moodle@questionname}\par
      \noindent
      \moodle@makelatextag@qtype{#1}
      \moodle@makelatextag@value{default grade}{points}
      \moodle@makelatextag@value{penalty}{penalty}
      #2\par
      \noindent
      \moodle@questionheader
      \csname moodle@#1@latexprocessing\endcsname
      %Now, writing information to XML
      \@moodle@ifgeneratexml{%
        \xa\questiontext\xa{\moodle@questionheader}% Save the question text.
        \bgroup
          \gdef\moodle@answers@xml{}%
          \setkeys{moodle}{feedback={}}%
          \xa\loopthroughitemswithcommand\xa{\csname save#1answer\endcsname}%
          \passvalueaftergroup{\moodle@answers@xml}%
        \egroup
        \csname write#1question\endcsname
      }{}%
    \egroup
  }%
}
%    \end{macrocode}
%
% \subsubsection{Essay Question Front-End}
% The essay question is the only question type whose front end
% is not yet created by |\moodle@makefrontend|.
% This is because of what it must do with its |\item|'s.
% 
%    \begin{macrocode}
\def\moodle@essay@latexprocessing{%
  % Moodle cannot automatically grade an essay, 
  % but if the user puts \item's in, we can list them in an itemize as notes.
  \ifnum\c@numgathereditems>0\relax
    \par\noindent \emph{Notes for grader:}
    \begin{itemize} \setlength\itemsep{0pt}\setlength\parskip{0pt}%
      \loopthroughitemswithcommand{\moodle@print@essay@answer}%
    \end{itemize}%
  \fi
}

\NewEnviron{essay}[2][]{%
  \bgroup
    \setkeys{moodle}{#1,questionname={#2}}%
    \expandafter\gatheritems\expandafter{\BODY}%
    \let\moodle@questionheader=\gatheredheader
    %First, the LaTeX processing.
      \item \textbf{\moodle@questionname}\par
      \noindent
      \moodle@makelatextag@qtype{essay}
      \moodle@makelatextag@value{default grade}{points}
      \moodle@makelatextag@value{penalty}{penalty}
      \moodle@makelatextag@key{responseformat}\par
      \noindent
      \moodle@questionheader
      \csname moodle@essay@latexprocessing\endcsname
    %Now, writing information to memory.
    \@moodle@ifgeneratexml{%
      \xa\questiontext\xa{\moodle@questionheader}% Save the question text.
      \bgroup
        \gdef\moodle@answers@xml{}%
        %
        \ifnum\c@numgathereditems=0\relax
          \addto@xml[2]\moodle@answers@xml{<graderinfo format="html"><text/></graderinfo>}%
        \else
          \addto@xml[2]\moodle@answers@xml{<graderinfo format="html"><text><![CDATA[}%
          \ifnum\c@numgathereditems>1\relax
            \addto@xml[4]\moodle@answers@xml{<ul>}%
          \fi
          \loopthroughitemswithcommand{\moodle@savegraderinfo}%
          \ifnum\c@numgathereditems>1\relax
            \addto@xml[4]\moodle@answers@xml{</ul>}%
          \fi
          \addto@xml[2]\moodle@answers@xml{]]></text></graderinfo>}%
        \fi
        %
        \passvalueaftergroup{\moodle@answers@xml}%
      \egroup
      \writeessayquestion
    }{}%
  \egroup
}%

%%%% TO DO
%%%% To make essay work will be tough.
%%%% Every line from \ifnum\c@numgathereditems=0\relax through its \else and \fi,
%%%% with the exception of
%%%%          \xa\loopthroughitemswithcommand\xa{\csname save#1answer\endcsname}%
%%%% , does not exist in our current \moodle@makefrontend code.
%%%% How can we cope?
%%%%
%%%% Idea: change \moodle@makefrontend so that
%%%%       1. if \c@numgathereditems=0, we don't do anything.
%%%%       2. it calls a preamble and postamble around the \loopthroughitemswithcommand.
%%%%          Like this:
%%%%            
%%%%      \@moodle@ifgeneratexml{%
%%%%        \xa\questiontext\xa{\moodle@questionheader}% Save the question text.
%%%%        \bgroup
%%%%          \gdef\moodle@answers@xml{}%
%%%%          \setkeys{moodle}{feedback={}}%
%%%%          \@ifundefined{moodle@#1@answers@preamble}{}{}%
%%%%          \csname moodle@#1@answers@preamble\endcsname
%%%%          \ifnum\c@numgathereditems=0\relax
%%%%            \relax
%%%%          \else
%%%%            \xa\loopthroughitemswithcommand\xa{\csname save#1answer\endcsname}%
%%%%          \fi
%%%%          \@ifundefined{moodle@#1@answers@postamble}{}{}%
%%%%          \csname moodle@#1@answers@postamble\endcsname
%%%%          \passvalueaftergroup{\moodle@answers@xml}%
%%%%        \egroup
%%%%        \csname write#1question\endcsname
%%%%      }{}%
%%%% The \@ifundefined lines should automatically define the 
%%%% \...@preamble \...@postamble macros to be \relax if they don't exist already.

\gdef\writeessayquestion{%
  \writetomoodle{<question type="essay">}%
    \moodle@writecommondata%
    \moodle@writeresponserequired%
    \moodle@writeresponseformat%
    \moodle@writeresponsefieldlines%
    \moodle@writeattachmentsallowed%
    \moodle@writeattachmentsrequired%
    \moodle@writeanswers% The 'answers' XML really contains the grader info.
    \moodle@writeresponsetemplate%
  \writetomoodle{</question>}%
}%

\def\moodle@print@essay@answer#1{%
    \item #1%
}%


\def\moodle@savegraderinfo#1{%
  \bgroup
    \moodle@savegraderinfo@int#1\moodle@answer@rdelim
    \passvalueaftergroup{\moodle@answers@xml}%
  \egroup
}%
\newcommand\moodle@savegraderinfo@int[1][]{%
  \setkeys{moodle}{fraction=0,#1}%
  \moodle@savegraderinfo@int@int%
}%
\def\moodle@savegraderinfo@int@int#1\moodle@answer@rdelim{%
  \xa\converttohtmlmacro\xa\moodle@answertext@html\xa{#1}%
  %\trim@spaces@in\moodle@answertext
  \ifnum\c@numgathereditems>1\relax
    \addto@xml[6]{\moodle@answers@xml}{<li>\moodle@answertext@html</li>}%
  \else
    \addto@xml[4]{\moodle@answers@xml}{\moodle@answertext@html}%
  \fi
}%
%    \end{macrocode}
%
% \subsubsection{Short Answer Question Front-End}
% 
%    \begin{macrocode}
\def\blank{\rule{1in}{0.5pt}}%
% TO DO: Make an optional argument for width?  This wouldn't affect Moodle,
%        only the appearance in the PDF.  It doesn't seem worth it.

%\NewEnviron{shortanswer}[2][]{%
%   \bgroup
%     \setkeys{moodle}{#1,questionname={#2}}%
%     \expandafter\gatheritems\xa{\BODY}%
%     \let\moodle@questionheader=\gatheredheader
%     %First, the LaTeX processing.
%       \item \textbf{\moodle@questionname}
%       \csname ifmoodle@case sensitive\endcsname
%         \framebox{\tiny Case-Sensitive}\relax
%       \fi
%       \framebox{\tiny\csname moodle@default grade\endcsname~points}
%       \framebox{\tiny\csname moodle@penalty\endcsname~penalty}\par
%       \noindent
%       \moodle@questionheader
%       \csname moodle@shortanswer@latexprocessing\endcsname
%     %Now, writing information to memory.
%     \@moodle@ifgeneratexml{%
%       \xa\questiontext\xa{\moodle@questionheader}% Save the question text.
%       \bgroup
%         \gdef\moodle@answers@xml{}%
%         \setkeys{moodle}{feedback={}}%
%         \xa\loopthroughitemswithcommand\xa{\csname 
%         saveshortansweranswer\endcsname}%
%         \passvalueaftergroup{\moodle@answers@xml}%
%       \egroup
%       \csname writeshortanswerquestion\endcsname
%     }{}%
%   \egroup
% }%

\moodle@makefrontend{shortanswer}{\moodle@makelatextag@shortanswer}%

% LATEX PROCESSING

\def\moodle@makelatextag@shortanswer{%
  \csname ifmoodle@case sensitive\endcsname
    \moodle@makelatextagbox{Case-Sensitive}\relax
  \else
    \moodle@makelatextagbox{Case-Insensitive}\relax
  \fi
}

\def\moodle@shortanswer@latexprocessing{%
  \begin{itemize} \setlength\itemsep{0pt}\setlength\parskip{0pt}%
    \loopthroughitemswithcommand{\moodle@print@shortanswer@answer}%
  \end{itemize}%
}

   \def\moodle@print@shortanswer@answer#1{%
       \let\moodle@feedback=\@empty
       \moodle@print@shortanswer@answer@int#1\@rdelim
   }%
   \newcommand\moodle@print@shortanswer@answer@int[1][]{%
     \setkeys{moodle}{#1}%
     \moodle@print@shortanswer@answer@int@int%
   }%
   \def\moodle@print@shortanswer@answer@int@int#1\@rdelim{%
     \ifx\moodle@fraction\@hundred
       \item #1\quad$\checkmark$%
     \else
       \item #1\quad (\moodle@fraction\%)%
     \fi
     \ifx\moodle@feedback\@empty\relax\else
       \hfill \emph{$\rightarrow$ \moodle@feedback}
     \fi
   }%

% SAVING ANSWERS TO MEMORY
\def\saveshortansweranswer#1{%
  \bgroup
    \saveshortansweranswer@int#1\moodle@answer@rdelim
    \passvalueaftergroup{\moodle@answers@xml}%
  \egroup
}%
   \newcommand\saveshortansweranswer@int[1][]{%
     \setkeys{moodle}{fraction=100,#1}%                  %%%%%% DEFAULT VALUE IS 100%
     \saveshortansweranswer@int@int%
   }%
   \def\saveshortansweranswer@int@int#1\moodle@answer@rdelim{%
     \def\moodle@answertext{#1}%
     \trim@spaces@in\moodle@answertext
     \addto@xml[2]{\moodle@answers@xml}{<answer fraction="\moodle@fraction" format="plain_text">}%
     \addto@xml[4]{\moodle@answers@xml}{  <text>\moodle@answertext</text>}%
     \ifx\moodle@feedback\@empty\relax\else
       \trim@spaces@in\moodle@feedback
       \xa\converttohtmlmacro\xa\moodle@feedback@html\xa{\moodle@feedback}%
       \addto@xml[4]{\moodle@answers@xml}{  <feedback format="html"><text><![CDATA[<p>\moodle@feedback@html</p>]]></text></feedback>}%
     \fi
     \addto@xml[2]{\moodle@answers@xml}{</answer>}%
   }%

% WRITING QUESTION TO XML FILE
\gdef\writeshortanswerquestion{%
  \writetomoodle{<question type="shortanswer">}%
    \moodle@writecommondata%
    \moodle@writeusecase%
    \moodle@writeanswers%
  \writetomoodle{</question>}%
}%
%    \end{macrocode}
%
% \subsubsection{Numerical Question Front-End}
% 
%    \begin{macrocode}
\moodle@makefrontend{numerical}{\moodle@makelatextag@numerical}%

% LATEX PROCESSING

\def\moodle@makelatextag@numerical{}

\def\moodle@numerical@latexprocessing{%
      \begin{itemize} \setlength\itemsep{0pt}\setlength\parskip{0pt}%
        \loopthroughitemswithcommand{\moodle@print@numerical@answer}%
      \end{itemize}%
}

   \def\moodle@print@numerical@answer#1{%
       \let\moodle@feedback=\@empty
       \moodle@print@numerical@answer@int#1\@rdelim
   }%
   \newcommand\moodle@print@numerical@answer@int[1][]{%
     \setkeys{moodle}{#1}%
     \moodle@print@numerical@answer@int@int%
   }%
   \def\moodle@print@numerical@answer@int@int#1\@rdelim{%
     \ifdim0pt=\moodle@tolerance pt\relax
       \def\moodle@numericalprint@tolerance{}%
     \else
       \edef\moodle@numericalprint@tolerance{\noexpand\pm\moodle@tolerance}%
     \fi
     \ifx\moodle@fraction\@hundred
       \item $#1\moodle@numericalprint@tolerance\quad\checkmark$%
     \else
       \item $#1\moodle@numericalprint@tolerance$\quad (\moodle@fraction\%)%
     \fi
     \ifx\moodle@feedback\@empty\relax\else
       \hfill \emph{$\rightarrow$ \moodle@feedback}%
     \fi
   }%

% SAVING ANSWERS TO MEMORY
\def\savenumericalanswer#1{%
  \bgroup
    \savenumericalanswer@int#1\moodle@answer@rdelim
    \passvalueaftergroup{\moodle@answers@xml}%
  \egroup
}%
   \newcommand\savenumericalanswer@int[1][]{%
     \setkeys{moodle}{fraction=100,#1}%                  %%%%%% DEFAULT VALUE IS 100%
     \savenumericalanswer@int@int%
   }%
   \def\savenumericalanswer@int@int#1\moodle@answer@rdelim{%
     \def\moodle@answertext{#1}%
     \trim@spaces@in\moodle@answertext
     \addto@xml[2]{\moodle@answers@xml}{<answer fraction="\moodle@fraction" format="plain_text">}%
     \addto@xml[4]{\moodle@answers@xml}{  <text>\moodle@answertext</text>}%
     \addto@xml[4]{\moodle@answers@xml}{  <tolerance>\moodle@tolerance</tolerance>}%
     \ifx\moodle@feedback\@empty\relax\else
       \trim@spaces@in\moodle@feedback
       \xa\converttohtmlmacro\xa\moodle@feedback@html\xa{\moodle@feedback}%
       \addto@xml[4]{\moodle@answers@xml}{  <feedback format="html"><text><![CDATA[<p>\moodle@feedback@html</p>]]></text></feedback>}%
     \fi
     \addto@xml[2]{\moodle@answers@xml}{</answer>}%
   }%


% WRITING QUESTION TO XML FILE
\gdef\writenumericalquestion{%
  \writetomoodle{<question type="numerical">}%
    \moodle@writecommondata%
    \moodle@writeanswers%
  \writetomoodle{</question>}%
}%
%    \end{macrocode}
%
% \subsubsection{Multiple Choice Question Front-End}
% 
%    \begin{macrocode}
%Multiple choice has the structure
% \begin{multi}[options]{name}%
%   What is 5+7?
%   \item 13
%   \item* 12
%   \item 11
% \end{multi}%

\moodle@makefrontend{multi}{\moodle@makelatextag@multi}%

% LATEX PROCESSING

\def\moodle@makelatextag@multi{%
  \csname ifmoodle@multiple\endcsname
    \moodle@makelatextagbox{Multiple}\relax
  \else
    \moodle@makelatextagbox{Single}\relax
  \fi
  \csname ifmoodle@shuffle\endcsname
    \moodle@makelatextagbox{Shuffle}\relax
  \fi
}

\def\moodle@multi@latexprocessing{%
      \moodle@countcorrectanswers
      \begin{enumerate}\moodle@obeynumberingstyle
                       %\renewcommand{\theenumi}{\alph{enumi}}%
                       \setlength\itemsep{0pt}\setlength\parskip{0pt}%
        \loopthroughitemswithcommand{\moodle@print@multichoice@answer}%
      \end{enumerate}%
}
  \def\moodle@print@multichoice@answer#1{%
    \let\moodle@feedback=\@empty
    \moodle@print@multichoice@answer@int#1\@rdelim
  }%
  \newcommand\moodle@print@multichoice@answer@int[1][]{%
    \setkeys{moodle}{fraction=0,#1}%
    \moodle@print@multichoice@answer@int@int
  }%
  \def\moodle@print@multichoice@answer@int@int#1#2\@rdelim{%
    \def\test@i{#1}%
    \ifmoodle@single
      \ifx\test@i\@star
        \item #2$~\checkmark$%
      \else
        \item #1#2%
      \fi
    \else
      \ifx\test@i\@star
        \item #2%
        \setkeys{moodle}{fraction=\moodle@autopoints}%
      \else
        \item #1#2%
      \fi
      $~(\moodle@fraction\%)$
    \fi
    \ifx\moodle@feedback\@empty\relax\else
      \hfill \emph{$\rightarrow$ \moodle@feedback}%
    \fi
  }%

% COMMON UTILITY: COUNTING CORRECT ANSWERS
   \newcounter{moodle@numcorrectanswers}%
   \newlength{\moodle@pointspercorrect}%
   \def\moodle@countcorrectanswers{%
     \setcounter{moodle@numcorrectanswers}{0}%
     \global\setlength{\moodle@pointspercorrect}{100pt}%
     \loopthroughitemswithcommand{\moodle@countcorrectanswers@a}%
     \ifnum0=\c@moodle@numcorrectanswers\relax
       \PackageError{moodle}{No correct answers given for multiple choice question.}{Please mark at least one answer correct.}%
       \gdef\moodle@autopoints{0}%
     \else
       \global\divide\moodle@pointspercorrect by \c@moodle@numcorrectanswers\relax
       \edef\moodle@autopoints{\strip@pt\moodle@pointspercorrect}%
     \fi
   }
   \def\moodle@countcorrectanswers@a#1{%
     %The grouping is to keep key answer-specific key changes local.
     \bgroup
       \moodle@countcorrectanswers@b#1\moodle@answer@rdelim
     \egroup
   }%
   \newcommand\moodle@countcorrectanswers@b[1][]{%
     \setkeys{moodle}{fraction=0,#1}%
     \moodle@countcorrectanswers@c%
   }%
   \def\moodle@countcorrectanswers@c#1#2\moodle@answer@rdelim{%
     \def\test@i{#1}%
     \ifx\test@i\@star
       \stepcounter{moodle@numcorrectanswers}%
     \else
       \global\addtolength{\moodle@pointspercorrect}{-\moodle@fraction pt}%
     \fi
   }%
   \def\moodle@setautopoints#1pt{%
     \gdef\moodle@autopoints{#1}%
   }

% TODO: What if rounding errors make the percentages not add to 100%?  Will Moodle complain?
% TODO: Verify that negative scores are handled correctly.
% TODO: Put these macros in same order as other sections'.

% SAVING ANSWERS TO MEMORY
\def\savemultianswer#1{%
  \bgroup
    \savemultianswer@int#1\moodle@answer@rdelim
    \passvalueaftergroup{\moodle@answers@xml}%
  \egroup
}%
  \newcommand\savemultianswer@int[1][]{%
    \setkeys{moodle}{fraction=0,#1}%
    \savemultianswer@int@int%
  }%
  \def\savemultianswer@int@int#1#2\moodle@answer@rdelim{%
    \def\test@i{#1}%
    \ifx\test@i\@star
      \ifmoodle@single
        \setkeys{moodle}{fraction=100}%
      \else
        \setkeys{moodle}{fraction=\moodle@autopoints}%
      \fi
      \def\moodle@answertext{#2}%
    \else
      \def\moodle@answertext{#1#2}%
    \fi
    \trim@spaces@in\moodle@answertext
    \addto@xml[2]{\moodle@answers@xml}{<answer fraction="\moodle@fraction" format="html">}%
    \xa\converttohtmlmacro\xa\moodle@answertext@html\xa{\moodle@answertext}%
    \addto@xml[4]{\moodle@answers@xml}{  <text><![CDATA[<p>\moodle@answertext@html</p>]]></text>}%
    \ifx\moodle@feedback\@empty\relax\else
      \trim@spaces@in\moodle@feedback
      \xa\converttohtmlmacro\xa\moodle@feedback@html\xa{\moodle@feedback}%
      \addto@xml[4]{\moodle@answers@xml}{  <feedback format="html"><text><![CDATA[<p>\moodle@feedback@html</p>]]></text></feedback>}%
    \fi
    \addto@xml[2]{\moodle@answers@xml}{</answer>}%
  }%

% WRITING QUESTION TO XML FILE
\gdef\writemultiquestion{%
  \writetomoodle{<question type="multichoice">}%
    \moodle@writecommondata%
    \moodle@writesingle%
    \moodle@writeshuffle%
    \moodle@writeanswernumbering%
    \moodle@writeanswers%
  \writetomoodle{</question>}%
}%
%    \end{macrocode}
%
% \subsubsection{Matching Question Front-End}
% 
%    \begin{macrocode}
\let\answer=\hfill

\moodle@makefrontend{matching}{\moodle@makelatextag@mmatching}

% LATEX PROCESSING

\def\moodle@makelatextag@mmatching{%
  \csname ifmoodle@draganddrop\endcsname
    \moodle@makelatextagbox{Drag and drop}\relax
  \fi
  \csname ifmoodle@shuffle\endcsname
    \moodle@makelatextagbox{Shuffle}\relax
  \fi
}

\def\moodle@matching@latexprocessing{%
  \bgroup
    \let\answer=\hfill
    \begin{enumerate}\renewcommand{\theenumi}{\alph{enumi}}\setlength\itemsep{0pt}\setlength\parskip{0pt}%
      \loopthroughitemswithcommand{\moodle@print@matching@answer}%
    \end{enumerate}%
  \egroup
}
   \long\def\moodle@print@matching@answer#1{%
     \moodle@print@matching@answer@int#1\@rdelim
   }%
   \newcommand\moodle@print@matching@answer@int[1][]{%
     \moodle@print@matching@answer@int@int\relax
   }%
   \long\def\moodle@print@matching@answer@int@int#1\answer#2\@rdelim{%
       \item #1\hfill #2%
   }%


% SAVING ANSWERS TO MEMORY
\long\def\savematchinganswer#1{%
  \bgroup
    \savematchinganswer@int#1\moodle@answer@rdelim%
    \passvalueaftergroup{\moodle@answers@xml}%
  \egroup
}%
   \newcommand\savematchinganswer@int[1][]{%
     \setkeys{moodle}{#1}%
     \xa\savematchinganswer@int@int\space%
   }%
   \long\def\savematchinganswer@int@int#1\answer#2\moodle@answer@rdelim{%
     % Note that #1 may simply be \relax.
     \def\moodle@subquestiontext{#1}%
     \def\moodle@subanswertext{#2}%
     \trim@spaces@in\moodle@subquestiontext
     \xa\converttohtmlmacro\xa\moodle@subquestiontext@htmlized\xa{\moodle@subquestiontext}%
     \trim@spaces@in\moodle@subanswertext
     \ifmoodle@draganddrop
       \xa\converttohtmlmacro\xa\moodle@subanswertext@htmlized\xa{\moodle@subanswertext}%
     \fi
     \addto@xml[2]{\moodle@answers@xml}{<subquestion format="html">}%
     \ifx\moodle@subquestiontext\@empty
       \addto@xml[4]{\moodle@answers@xml}{  <text></text>}%
     \else
       \addto@xml[4]{\moodle@answers@xml}{  <text><![CDATA[<p>\moodle@subquestiontext@htmlized</p>]]></text>}%
     \fi
     \ifmoodle@draganddrop
%       \show\moodle@subanswertext@htmlized
       \addto@xml[4]{\moodle@answers@xml}{  <answer format="html"><text><![CDATA[<p>\moodle@subanswertext@htmlized</p>]]></text></answer>}%
     \else
%       \show\moodle@subanswertext
       \addto@xml[4]{\moodle@answers@xml}{  <answer><text>\moodle@subanswertext</text></answer>}%
     \fi
     \ifx\moodle@feedback\@empty\relax\else
       \trim@spaces@in\moodle@feedback
       \xa\converttohtmlmacro\xa\moodle@feedback@html\xa{\moodle@feedback}%
       \addto@xml[4]{\moodle@answers@xml}{  <feedback format="html"><text><![CDATA[<p>\moodle@feedback@html</p>]]></text></feedback>}%
     \fi
     \addto@xml[2]{\moodle@answers@xml}{</subquestion>}%
   }%

% WRITING QUESTION TO XML FILE
\gdef\writematchingquestion{%
  \ifmoodle@draganddrop
    \writetomoodle{<question type="ddmatch">}%
  \else
    \writetomoodle{<question type="matching">}%
  \fi
    \moodle@writecommondata%
    \moodle@writesingle%
    \moodle@writeshuffle%
    \moodle@writeanswernumbering%
    \moodle@writeanswers%
  \writetomoodle{</question>}%
%  \gdef\moodle@answ
}%
%    \end{macrocode}
%
% \subsection{Cloze Questions}
% Because cloze questions are so complicated, they get their own section of code.
% The cloze strategy is as follows.
% 
% All subquestions show up as part of the question body text.
% For each type of subquestion, we have a cloze-version environment
% that actually has 2 versions, depending on whether we are doing LaTeX or XML processing.
% So the main environment is quite typical: 
% \begin{enumerate}
%   \item Process the body as LaTeX.
%         During this run, a |\begin{multi}| etc.~will be processed for display onscreen.
%   \item Then save the body as the questiontext for XML.
%         During this run, a |\begin{multi}| etc.~will be parsed and turned into
%         cloze code as part of the XML questiontext.
% \end{enumerate}
%    \begin{macrocode}
% LATEX PROCESSING
% SAVING ANSWERS TO MEMORY

\NewEnviron{cloze}[2][]{%
  \bgroup
    \setkeys{moodle}{default grade=1}%
    \setkeys{moodle}{#1,questionname={#2}}%
    % A cloze question won't have any \item's in it, so we just use \BODY.
    \moodle@enableclozeenvironments
    %First, the LaTeX processing.
      \item \textbf{\moodle@questionname}\par
      \noindent
      \moodle@makelatextag@qtype{essay}
      \moodle@makelatextag@value{default grade}{points}
      \moodle@makelatextag@value{penalty}{penalty}\par
      \noindent
      \BODY
    %Now, writing information to memory.
    \@moodle@ifgeneratexml{%
      \xa\questiontext\xa{\BODY}% Save the question text as HTML.
      \writeclozequestion
    }{}%
  \egroup%
}

\def\moodle@enableclozeenvironments{%
  \let\multi=\clozemulti
  \let\endmulti=\endclozemulti
  \let\numerical=\clozenumerical
  \let\endnumerical=\endclozenumerical
  \let\shortanswer=\clozeshortanswer
  \let\endshortanswer=\endclozeshortanswer
}

% WRITING QUESTION TO XML FILE
\gdef\writeclozequestion{%
  \writetomoodle{<question type="cloze">}%
    \moodle@writecommondata%
  \writetomoodle{</question>}%
}%
%    \end{macrocode}
%
% \subsubsection{Cloze Multiple Choice Questions}
% 
%    \begin{macrocode}
% TODO: Guard against [single=false] option

\NewEnviron{clozemulti}[1][]{%
  \bgroup
    \setkeys{moodle}{default grade=1}%
    \setkeys{moodle}{#1}%
    \expandafter\gatheritems\xa{\BODY}%
    \let\moodle@questionheader=\gatheredheader
    \ifhtmlizer@active
      %HTML version
      \def\moodle@clozemulti@output{}%
      \xa\g@addto@macro\xa\moodle@clozemulti@output\xa{\moodle@questionheader}%
      \def\clozemulti@coding{}%
      \edef\clozemulti@coding{\otherlbrace\csname moodle@default grade\endcsname:}%
      \ifcase\moodle@multi@mode\relax
        % Case 0: dropdown box style
        \g@addto@macro{\clozemulti@coding}{MULTICHOICE}%
      \or
        % Case 1: vertical style
        \g@addto@macro{\clozemulti@coding}{MULTICHOICE_V}%
      \else
        % Case 2: horizontal radio buttons
        \g@addto@macro{\clozemulti@coding}{MULTICHOICE_H}%
      \fi
      \ifmoodle@shuffle
        \g@addto@macro{\clozemulti@coding}{S:}%
      \else
        \g@addto@macro{\clozemulti@coding}{:}%
      \fi
      \bgroup
        \setkeys{moodle}{feedback={}}%
        \loopthroughitemswithcommand{\saveclozemultichoiceanswer}%
      \egroup
      \xa\g@addto@macro\xa\clozemulti@coding\xa{\otherrbrace}%
      \xa\g@addto@macro\xa\moodle@clozemulti@output\xa{\clozemulti@coding}%
      \xa\gdef\xa\htmlize@afteraction@hook\xa{\moodle@clozemulti@output}%
      \def\endclozemulti@code{\htmlize@patchendenvironment}%
    \else
      %LaTeX version
      \moodle@questionheader% %Any introductory text just continues to be typeset.
      \par
      \noindent
      \moodle@makelatextag@qtype{multi}
      \moodle@makelatextag@value{default grade}{points}
      \moodle@makelatextag@value{penalty}{penalty}
      \moodle@makelatextag@multi
      \par
      \noindent
      \def\cloze@multichoice@table@text{}%
      \let\moodle@feedback=\@empty
      \loopthroughitemswithcommand{\moodle@print@clozemultichoice@answer}%
      \ifcase\moodle@multi@mode\relax
        %Case 0: dropdown box style
        \begin{tabular}[t]{|p{.45\linewidth}|p{.45\linewidth}|}
          \firsthline% (\firsthline is from the array package.)
%          answer & feedback \\\hline\hline
          \cloze@multichoice@table@text%
        \end{tabular}%
      \or
        %Case 1: vertical style
        \begin{itemize}\setlength\itemsep{0pt}\setlength\parskip{0pt}%
          \cloze@multichoice@table@text%
        \end{itemize}%
      \else
        %Case 2: horizontal radio buttons
        \par{\cloze@multichoice@table@text}\par%
      \fi
      \def\endclozemulti@code{\relax}%
    \fi
    \passvalueaftergroup\endclozemulti@code%
    \passvalueaftergroup\htmlize@afteraction@hook%
  \egroup%
}[\endclozemulti@code]%


\def\moodle@print@clozemultichoice@answer#1{%
  \let\moodle@feedback=\@empty
  \moodle@print@clozemultichoice@answer@int#1\@rdelim%
}%
\newcommand\moodle@print@clozemultichoice@answer@int[1][]{%
  \setkeys{moodle}{fraction=0,#1}%
  \moodle@print@clozemultichoice@answer@int@int%
}%
\def\moodle@print@clozemultichoice@answer@int@int#1#2\@rdelim{%
  % Case 0: "(answer) \\ \hline"
  % Case 1: "\item (answer)"
  % Case 2: "$\bullet~$(answer)\hfill"
  \ifcase\moodle@multi@mode\relax
    \relax% Case 0
  \or
    \g@addto@macro\cloze@multichoice@table@text{\item}% Case 1
  \else
    \g@addto@macro\cloze@multichoice@table@text{$\bullet~$}% Case 2
  \fi
  \def\test@i{#1}%
  \ifx\test@i\@star
    \setkeys{moodle}{fraction=100}%
    \g@addto@macro\cloze@multichoice@table@text{#2}%
  \else
    \g@addto@macro\cloze@multichoice@table@text{#1#2}%
  \fi
  \ifx\moodle@fraction\@hundred
    \g@addto@macro\cloze@multichoice@table@text{$~\checkmark$}%
  \fi
  \ifnum0<\moodle@fraction\relax
    \ifnum\moodle@fraction<100\relax
      \xdef\cloze@multichoice@table@text{\expandonce\cloze@multichoice@table@text$~(\moodle@fraction\%)$}%
    \fi
  \fi
  \ifcase\moodle@multi@mode\relax
    % Case 0
      \xdef\cloze@multichoice@table@text{\expandonce\cloze@multichoice@table@text &\expandonce\emph{\expandonce\moodle@feedback}}%
      \g@addto@macro{\cloze@multichoice@table@text}{\\\hline}
  \or % Case 1
    \ifx\moodle@feedback\@empty\relax\else
      \xdef\cloze@multichoice@table@text{\expandonce\cloze@multichoice@table@text \hfill \expandonce\emph{$\rightarrow$ \expandonce\moodle@feedback}}%
    \fi
  \else % otherwise
    \ifx\moodle@feedback\@empty\relax\else
      \xdef\cloze@multichoice@table@text{\expandonce\cloze@multichoice@table@text\,\expandonce\emph{$\rightarrow$ \expandonce\moodle@feedback}}%
    \fi
    \g@addto@macro{\cloze@multichoice@table@text}{\hfill}% Case 2
  \fi
}%

\def\saveclozemultichoiceanswer#1{%
  \bgroup
    \saveclozemultichoiceanswer@int#1\moodle@answer@rdelim
  \egroup
}%
\newcommand\saveclozemultichoiceanswer@int[1][]{%
  \setkeys{moodle}{fraction=0,#1}%
  \saveclozemultichoiceanswer@int@int%
}%
\def\saveclozemultichoiceanswer@int@int#1#2\moodle@answer@rdelim{%
  \def\test@i{#1}%
  \ifgatherbeginningofloop\else
    \xa\gdef\xa\clozemulti@coding\xa{\clozemulti@coding\clozetilde}% separator between answers
  \fi
  \ifx\test@i\@star
    \setkeys{moodle}{fraction=100}%
    \def\moodle@answertext{#2}%
  \else
    \def\moodle@answertext{#1#2}%
  \fi
  \trim@spaces@in\moodle@answertext
  \ifx\moodle@fraction\@hundred
    \g@addto@macro\clozemulti@coding{=}%
  \fi
  \ifnum0<\moodle@fraction\relax
    \ifnum\moodle@fraction<100\relax
      \xdef\clozemulti@coding{\expandonce\clozemulti@coding\otherpercent\moodle@fraction\otherpercent}%
    \fi
  \fi
  \xdef\clozemulti@coding{\expandonce\clozemulti@coding\expandonce\moodle@answertext}%
  \ifx\moodle@feedback\@empty\else
    \trim@spaces@in\moodle@feedback
%    \xa\converttohtmlmacro\xa\moodle@feedback@html\xa{\moodle@feedback}%
    \xdef\clozemulti@coding{\expandonce\clozemulti@coding\otherbackslash\otherhash\expandonce\moodle@feedback}%
  \fi
}%
%    \end{macrocode}
%
% \subsubsection{Cloze Numerical Questions}
% 
%    \begin{macrocode}
\NewEnviron{clozenumerical}[1][]{%
  \bgroup
    \expandafter\gatheritems\expandafter{\BODY}%
    \let\moodle@questionheader=\gatheredheader
    \setkeys{moodle}{default grade=1}%
    \setkeys{moodle}{#1}%
    \ifhtmlizer@active
      %HTML version
      \def\moodle@clozenumerical@output{}%
      \xa\g@addto@macro\xa\moodle@clozenumerical@output\xa{\moodle@questionheader}%
      \def\clozenumerical@coding{}%
      \edef\clozenumerical@coding{\otherlbrace\csname moodle@default grade\endcsname:NUMERICAL:}%
      \bgroup
        \setkeys{moodle}{feedback={}}%
        \loopthroughitemswithcommand{\saveclozenumericalanswer}%
      \egroup
      \xa\g@addto@macro\xa\clozenumerical@coding\xa{\otherrbrace}%
      \xa\g@addto@macro\xa\moodle@clozenumerical@output\xa{\clozenumerical@coding}%
      \xa\gdef\xa\htmlize@afteraction@hook\xa{\moodle@clozenumerical@output}%
      \def\endclozenumerical@code{\htmlize@patchendenvironment}%
    \else
      %LaTeX version
      \moodle@questionheader% %Any introductory text just continues to be typeset.
      \par
      \noindent
      \moodle@makelatextag@qtype{numerical}
      \moodle@makelatextag@value{default grade}{points}
      \moodle@makelatextag@value{penalty}{penalty}
      \moodle@makelatextag@numerical
      \par
      \noindent
      \def\cloze@numerical@table@text{}%
      \loopthroughitemswithcommand{\moodle@print@clozenumerical@answer}%
      \begin{tabular}[t]{|p{.45\linewidth}|p{.45\linewidth}|}
        \firsthline% (\firsthline is from the array package.)
%       answer & feedback \\\hline\hline
        \cloze@numerical@table@text%
      \end{tabular}%
      \def\endclozenumerical@code{\relax}%
    \fi
    \passvalueaftergroup\endclozenumerical@code%
    \passvalueaftergroup\htmlize@afteraction@hook%
  \egroup
}[\endclozenumerical@code]%

\def\moodle@print@clozenumerical@answer#1{%
  \bgroup
    \moodle@print@clozenumerical@answer@int#1\@rdelim
  \egroup
}%
\newcommand\moodle@print@clozenumerical@answer@int[1][]{%
  \setkeys{moodle}{#1}%
  \moodle@print@clozenumerical@answer@int@int%
}%
\def\moodle@print@clozenumerical@answer@int@int#1\@rdelim{%
  \ifx\moodle@fraction\@hundred
    \def\moodle@clozenumericalprint@fraction{$\checkmark$}%
  \else
    \edef\moodle@clozenumericalprint@fraction{(\moodle@fraction\%)}%
  \fi
  \ifdim0pt=\moodle@tolerance pt\relax
    \def\moodle@clozenumericalprint@tolerance{}%
  \else
    \edef\moodle@clozenumericalprint@tolerance{\noexpand\pm\moodle@tolerance}%
  \fi
  \xdef\moodle@clozenumericalprint@line{$#1\moodle@clozenumericalprint@tolerance$~\moodle@clozenumericalprint@fraction & \expandonce\emph{\expandonce\moodle@feedback}}%
  \xa\g@addto@macro\xa\cloze@numerical@table@text\xa{\moodle@clozenumericalprint@line \\\hline}%
}%



\def\saveclozenumericalanswer#1{%
  \bgroup
    \saveclozenumericalanswer@int#1\moodle@answer@rdelim
  \egroup
}%
\newcommand\saveclozenumericalanswer@int[1][]{%
  \setkeys{moodle}{fraction=100,#1}%                  %%%%%% DEFAULT VALUE IS 100%
  \saveclozenumericalanswer@int@int%
}%
\def\saveclozenumericalanswer@int@int#1\moodle@answer@rdelim{%
  \ifgatherbeginningofloop\else
    \xa\gdef\xa\clozenumerical@coding\xa{\clozenumerical@coding\clozetilde}% separator between answers
  \fi
  \def\moodle@answertext{#1}%
  \trim@spaces@in\moodle@answertext
  \ifx\moodle@fraction\@hundred
    \g@addto@macro\clozenumerical@coding{=}%
  \fi
  \ifnum0<\moodle@fraction\relax
    \ifnum\moodle@fraction<100\relax
      \xdef\clozenumerical@coding{\expandonce\clozenumerical@coding\otherpercent\moodle@fraction\otherpercent}%
    \fi
  \fi
  \xdef\clozenumerical@coding{\expandonce\clozenumerical@coding\moodle@answertext:\moodle@tolerance}%
  \ifx\moodle@feedback\@empty\else
    \trim@spaces@in\moodle@feedback
    %\xa\converttohtmlmacro\xa\moodle@feedback@html\xa{\moodle@feedback}%
    \xdef\clozenumerical@coding{\expandonce\clozenumerical@coding\otherbackslash\otherhash\expandonce\moodle@feedback}%
  \fi
}%
%    \end{macrocode}
%
% \subsubsection{Cloze Short Answer Questions}
% 
%    \begin{macrocode}
\NewEnviron{clozeshortanswer}[1][]{%
  \bgroup
    \expandafter\gatheritems\expandafter{\BODY}%
    \let\moodle@questionheader=\gatheredheader
    \setkeys{moodle}{default grade=1}%
    \setkeys{moodle}{#1}%
    %Because nesting conditionals built by \csname doesn't work well,
    %we'll test '\ifmoodle@case sensitive' now and save the result in \count0.
    \csname ifmoodle@case sensitive\endcsname
      \count0=1\relax
    \else
      \count0=0\relax
    \fi
    \ifhtmlizer@active
      %HTML version
      \def\moodle@clozeshortanswer@output{}%
      \xa\g@addto@macro\xa\moodle@clozeshortanswer@output\xa{\moodle@questionheader}%
      \def\clozeshortanswer@coding{}%
      \ifnum\count0=1\relax
        \edef\clozeshortanswer@coding{\otherlbrace\csname moodle@default grade\endcsname:SHORTANSWER_C:}%
      \else
        \edef\clozeshortanswer@coding{\otherlbrace\csname moodle@default grade\endcsname:SHORTANSWER:}%
      \fi
      \bgroup
        \setkeys{moodle}{feedback={}}%
        \loopthroughitemswithcommand{\saveclozeshortansweranswer}%
      \egroup
      \xa\g@addto@macro\xa\clozeshortanswer@coding\xa{\otherrbrace}%
      \xa\g@addto@macro\xa\moodle@clozeshortanswer@output\xa{\clozeshortanswer@coding}%
      \xa\gdef\xa\htmlize@afteraction@hook\xa{\moodle@clozeshortanswer@output}%
      \def\endclozeshortanswer@code{\htmlize@patchendenvironment}%
    \else
      %LaTeX version
      \moodle@questionheader% %Any introductory text just continues to be typeset.
      \par
      \noindent
      \moodle@makelatextag@qtype{shortanswer}
      \moodle@makelatextag@value{default grade}{points}
      \moodle@makelatextag@value{penalty}{penalty}
      \moodle@makelatextag@shortanswer
      \par
      \noindent
      \def\cloze@shortanswer@table@text{}%
      \loopthroughitemswithcommand{\moodle@print@clozeshortanswer@answer}%
      \begin{tabular}[t]{|p{.45\linewidth}|p{.45\linewidth}|}
        \firsthline% (\firsthline is from the array package.)
%       answer & feedback \\\hline\hline
        \cloze@shortanswer@table@text%
      \end{tabular}%
      \def\endclozeshortanswer@code{\relax}%
    \fi
    \passvalueaftergroup\endclozeshortanswer@code%
    \passvalueaftergroup\htmlize@afteraction@hook%
  \egroup
}[\endclozeshortanswer@code]%

\def\moodle@print@clozeshortanswer@answer#1{%
  \let\moodle@feedback=\@empty
  \bgroup
    \moodle@print@clozeshortanswer@answer@int#1\@rdelim
  \egroup
}%
\newcommand\moodle@print@clozeshortanswer@answer@int[1][]{%
  \setkeys{moodle}{#1}%
  \moodle@print@clozeshortanswer@answer@int@int%
}%
\def\moodle@print@clozeshortanswer@answer@int@int#1\@rdelim{%
  \ifx\moodle@fraction\@hundred
    \def\moodle@clozeshortanswerprint@fraction{$\checkmark$}%
  \else
    \edef\moodle@clozeshortanswerprint@fraction{(\moodle@fraction\%)}%
  \fi
  \xdef\moodle@clozeshortanswerprint@line{#1~\moodle@clozeshortanswerprint@fraction & \expandonce\emph{\expandonce\moodle@feedback}}%
  \xa\g@addto@macro\xa\cloze@shortanswer@table@text\xa{\moodle@clozeshortanswerprint@line \\\hline}%
}%



\def\saveclozeshortansweranswer#1{%
  \bgroup
    \saveclozeshortansweranswer@int#1\moodle@answer@rdelim
  \egroup
}%
\newcommand\saveclozeshortansweranswer@int[1][]{%
  \setkeys{moodle}{fraction=100,#1}%                  %%%%%% DEFAULT VALUE IS 100%
  \saveclozeshortansweranswer@int@int%
}%
\def\saveclozeshortansweranswer@int@int#1\moodle@answer@rdelim{%
  \ifgatherbeginningofloop\else
    \xa\gdef\xa\clozeshortanswer@coding\xa{\clozeshortanswer@coding\clozetilde}% separator between answers
  \fi
  \def\moodle@answertext{#1}%
  \trim@spaces@in\moodle@answertext
  \ifx\moodle@fraction\@hundred
    \g@addto@macro\clozeshortanswer@coding{=}%
  \fi
  \ifnum0<\moodle@fraction\relax
    \ifnum\moodle@fraction<100\relax
      \xdef\clozeshortanswer@coding{\expandonce\clozeshortanswer@coding\otherpercent\moodle@fraction\otherpercent}%
    \fi
  \fi
  \xdef\clozeshortanswer@coding{\expandonce\clozeshortanswer@coding\moodle@answertext}%
  \ifx\moodle@feedback\@empty\else
    %\trim@spaces@in\moodle@feedback
    %\xa\converttohtmlmacro\xa\moodle@feedback@html\xa{\moodle@feedback}%
    \xdef\clozeshortanswer@coding{\expandonce\clozeshortanswer@coding\otherbackslash\otherhash\expandonce\moodle@feedback}%
  \fi
}%
%    \end{macrocode}
%
% \section{Converting \LaTeX\ to HTML}
% A lot of work must now be done to convert the \LaTeX\ code 
% of a question or answer into HTML code with embedded \TeX\ for math.
% 
% \subsection{Catcode Setup}
% First, we create versions of the special characters with catcode 12, ``other.''
%    \begin{macrocode}
{\catcode`\#=12\gdef\otherhash{#}%
 \catcode`\~=12\gdef\othertilde{~}%
 \catcode`\&=12\gdef\otherampersand{&}%
 \catcode`\^=12\gdef\othercaret{^}%
 \catcode`\$=12\gdef\otherdollar{$}%
 \catcode`\%=12\gdef\otherpercent{%}}%
{\catcode`\ =12\gdef\otherspace{ }}%
{\tt\catcode`\|=0\catcode`\\=12\relax|gdef|otherbackslash{\}}%
{\catcode`\[=1\catcode`\]=2\catcode`\{=12\catcode`\}=12%
 \gdef\otherlbrace[{]\gdef\otherrbrace[}]]


\edef\@otherlbrace{\otherlbrace}%
\edef\@otherrbrace{\otherrbrace}%
\edef\@otherdollar{\otherdollar}%
\edef\@otherbackslash{\otherbackslash}%
\edef\@othertilde{\othertilde}%
%    \end{macrocode}
%
% Next, we define commands to change catcodes to a suitable verbatim mode
% for transcription.
% 
%    \begin{macrocode}
{ \catcode`\[=1\relax
  \catcode`\]=2\relax
  \catcode`\|=0\relax
  |gdef|verbcatcodesweirdest[
    |catcode`\{=12|relax
    |catcode`\}=12|relax
    |catcode`\\=12|relax
  ]%
}%
\def\verbcatcodes{%
  \catcode`\$=12\relax
  \catcode`\&=12\relax
  \catcode`\#=12\relax
  \catcode`\^=12\relax
  \catcode`\_=12\relax
  \catcode`\~=12\relax
  \makeatletter
  \catcode`\%=12\relax
  \catcode`\ =12\relax\catcode\newlinechar=12\verbcatcodesweirdest}%

\def\normalcatcodes{%
  \catcode`\\=0\relax
  \catcode`\{=1\relax
  \catcode`\}=2\relax
  \catcode`\$=3\relax
  \catcode`\&=4\relax
  \catcode\endlinechar=5\relax
  \catcode`\#=6\relax
  \catcode`\^=7\relax
  \catcode`\_=8\relax
  \catcode`\ =10\relax
  \makeatletter% We will be detokenizing and retokenizing internal control sequences, so we need this.
  \catcode`\~=13\relax
  \catcode`\%=14\relax}%

\def\retokenizingcatcodes{%
  %For rescanning previously scanned text, all true comments will already be gone,
  %but % signs may have been inserted by Cloze questions, so we want to treat them as 'other.'
  %
  % TODO: #'s are more worrisome.
  \normalcatcodes
  \catcode`\%=12\relax
}
%    \end{macrocode}
%
% \subsection{Detokenization and Retokenization}
% 
% We will be processing a \TeX\ token list.
% Based on its content, sometimes we will want it to be detokenized to 
% individual characters, but other times we want it retokenized so that
% \TeX's own parsing mechanism can gather up the parameters of macros.
% We use the e\TeX\ primitive command |\scantokens| to do this.
% 
% The following code (catcodes, groupings and all) defines a |\scantokens@to@macro| macro.
% That will assemble and disassemble strings of tokens
% using any changing schemes of catcodes we desire.
% 
% We define |^^A| to be |\gdef\stm@saved|, while |^^B| and |^^C| are substitutes for |{| and |}|, respectively.
% This permits us to define |\scantokens@to@macro| in a peculiar catcode regime.
%    \begin{macrocode}
\begingroup
  \catcode`\^^A=13\gdef^^A{\gdef\stm@saved}%
  \catcode`\^^B=1\catcode`\^^C=2\relax
  \long\gdef\scantokens@to@macro#1#2#3{%
    % #1 = control sequence to be defined
    % #2 = command to change catcodes, e.g. \verbcatcodes,
    %      and define any command sequences
    % #3 = text to be retokenized and saved into #1.
    \bgroup
      \def\texttorescan{#3}%
      \catcode`\^^A=13\catcode`\^^B=1\catcode`\^^C=2\relax
      \xa\def\xa\arg\xa{\xa^^A\xa^^B\texttorescan^^C}%
      #2%
      \catcode\endlinechar=9\relax% 
      %\scantokens always sees an end-of-line character at its end and converts it to a space.
      %The catcode change sets \scantokens to ignore end-of-line chars.
      %In practice, we're always calling \scantokens on previously scanned text anyway,
      %so we won't miss any real end-of-line chars, since they were already converted to spaces.
      \xa\scantokens\xa{\arg}%
    \egroup
    \xa\def\xa#1\xa{\stm@saved}%
  }%
\endgroup%

\long\def\ultradetokenize@to@macro#1#2{%
  \scantokens@to@macro#1{\verbcatcodes}{#2}%
}%
\def\retokenizenormal@to@macro#1#2{%
  \scantokens@to@macro#1{\retokenizingcatcodes}{#2}%
}%
%    \end{macrocode}
%
% \subsection{Level-Tracking}
% While parsing, we'll need to keep track of whether
% we are in math mode (and how many levels deep the math mode might be nested),
% and also how deeply nested in \TeX\ groups we are.
%    \begin{macrocode}
\newcount\moodle@mathmodedepth
\moodle@mathmodedepth=0\relax
\def\moodle@ifmathmode#1#2{%
  \ifnum\moodle@mathmodedepth>0\relax
    #1%
  \else
    #2%
  \fi
}

\newcount\grouplevel

\newif\ifhtmlizer@active
\htmlizer@activefalse
%    \end{macrocode}
%
% \subsection{Separation}
% 
% This code separates a string of tokens into two parts.
% Its parameters, |#1#2|, consist of tokenized text, 
% plus one terminal |\@htmlize@stop|.
% We are trying to break up the text into its first group and the remainder.
% This |\@htmlize@stop| is needed in case |#2| has the form ``|{...}|'',
% since we don't want \TeX\ to strip the braces off.
% Thus |\@htmlize@remainder| will definitely end in ``|\@htmlize@stop|''.
%    \begin{macrocode}
\long\def\htmlize@grabblock#1#2\htmlize@rdelim@ii{%
  \long\def\htmlize@blockinbraces{#1}%
  \long\def\htmlize@remainder{#2}%
}%
%    \end{macrocode}
%
% The next line defines the macro |\@htmlize@stop@detokenized| to contain
% the string of tokens |\@htmlize@stop|, all of category code 12 (other) or 10 (letter).
% We'll need this for comparison purposes later.
%    \begin{macrocode}
\ultradetokenize@to@macro\@htmlize@stop@detokenized{\@htmlize@stop}%
%    \end{macrocode}
%
% The next line creates the macro |\htmlize@remove@stopcode|,
% which removes the characters ``|\@htmlize@stop |'' from the end of a 
% detokenized sequence.
% Its syntax when called is simply |\htmlize@remove@stopcode |\meta{material},
% with no delimiters, since the ``|\@htmlize@stop |'' is itself the delimiter.
%    \begin{macrocode}
\xa\def\xa\htmlize@remove@stopcode\xa#\xa1\@htmlize@stop@detokenized{#1}%
%    \end{macrocode}
% 
% \subsection{Main Code: the HTMLizer}
% 
%    \begin{macrocode}
\newif\ifhtmlize@actioncs
\newif\ifhtmlize@expandcs
\newif\ifhtmlize@passcs

\long\def\@@begin@cs{\begin}%
\def\@@htmlize@stop{\@htmlize@stop}%

\long\def\converttohtmlmacro#1#2{%
  \grouplevel=0\relax
  \def\htmlize@output{}%
  \htmlizer@activetrue%
  \converttohtml@int{#2}%
  \htmlizer@activefalse%
  \let#1=\htmlize@output\relax
}

\long\def\converttohtml@int#1{%
  \advance\grouplevel by 1\relax
  \bgroup
    \ultradetokenize@to@macro\htmlize@texttoscan{#1}%
    \xa\htmlize@recursive@i\htmlize@texttoscan\@htmlize@stop\@htmlize@stop\@htmlize@stop\htmlize@rdelim@i
  \egroup
  \advance\grouplevel by -1\relax
}%

\def\@lt{<}%
\def\@gt{>}%

\long\def\htmlize@recursive@i#1#2#3\htmlize@rdelim@i{%
  % #1#2#3 is a sequence of tokens.  All should be categories 11 (letter) or 12 (other).
  % It terminates with the control sequences \@htmlize@stop\@htmlize@stop\@htmlize@stop.
  %\long\def\ds{(#1|#2|#3)}\show\ds
  \def\test@i{#1}%
  \def\test@ii{#2}%
  \ifx\test@i\@@htmlize@stop
    \let\htmlize@next@i=\relax
  \else
    \ifx\test@i\@otherlbrace%
      \xa\g@addto@macro\xa\htmlize@output\xa{\otherlbrace}%
      \bgroup
        \normalcatcodes
        %We need to rescan the input as TeX code,
        % so TeX can automatically pull off the first group in braces.
        % First, let's get rid of the terminal \@htmlize@stop codes.
        {\def\@htmlize@stop{}\xdef\htmlize@scrap{#1#2#3}}%
        \let\htmlize@text@to@rescan=\htmlize@scrap%
        % Next, we retokenize the code.
        \xa\retokenizenormal@to@macro\xa\htmlize@rescanned\xa{\htmlize@text@to@rescan}%
        % Now break it up into two pieces.
        \xa\htmlize@grabblock\htmlize@rescanned\@htmlize@stop\htmlize@rdelim@ii%
        % The first piece, \htmlize@blockinbraces, will be passed as a unit to \converttohtml@int.
        % The second part, \htmlize@remainder, will continue at this depth of grouping.
        % Therefore we'll detokenize \htmlize@remainder here.
        \xa\ultradetokenize@to@macro\xa\htmlize@remainder@detokenized\xa{\htmlize@remainder}%
        \edef\htmlize@remainder@detokenized{\xa\htmlize@remove@stopcode\htmlize@remainder@detokenized}%
        % 
        % Now build \htmlize@next@i.
        % When done, should look like
        %   \converttohtml@int{\htmlize@blockinbraces}%
        %   \g@addto@macro\htmlize@output{\otherrbrace}%
        %   \htmlize@recursive@i\htmlize@remainder@detokenized\@htmlize@stop\@htmlize@stop\@htmlize@stop\htmlize@rdelim@i%
        % but with all three arguments expanded.
        % Note that we are running
        \gdef\htmlize@scrap{\converttohtml@int}%
        \xa\g@addto@macro\xa\htmlize@scrap\xa{\xa{\htmlize@blockinbraces}}%
        \g@addto@macro\htmlize@scrap{\g@addto@macro\htmlize@output}%
        \xa\g@addto@macro\xa\htmlize@scrap\xa{\xa{\otherrbrace}}%
        \g@addto@macro\htmlize@scrap{\htmlize@recursive@i}%
        \xa\g@addto@macro\xa\htmlize@scrap\xa{\htmlize@remainder@detokenized\@htmlize@stop\@htmlize@stop\@htmlize@stop\htmlize@rdelim@i}%
        % Okay, that's done.  It's stored in a global macro.
        % Now we get it out of this group.
      \egroup
      \let\htmlize@next@i=\htmlize@scrap
    \else
      \ifx\test@i\@otherdollar%
        % Math shift character.
        \ifx\test@ii\@otherdollar
          % Double dollar sign, so we're entering display math mode.
          % We grab everything between $$...$$, sanitize it, and add it verbatim to
          % our output.
          \htmlize@displaymathshift@replace#1#2#3\htmlize@rdelim@iii%
        \else
          % Single dollar sign, so we're entering inline math mode.
          % We grab everything between $$...$$, sanitize it, and add it verbatim to
          % our output.
          \htmlize@inlinemathshift@replace#1#2#3\htmlize@rdelim@iii%
        \fi
        % Now we resume work.
        % The \htmlize@xxxxxxmathshift@replace macro stored the remaining text in \htmlize@remaining@text.
        % Note that since we never detokenized and retokenized #1#2#3, 
        % \htmlize@remaining@text still includes the terminating \@htmlize@stop\@htmlize@stop\@htmlize@stop.
        \def\htmlize@next@i{\xa\htmlize@recursive@i\htmlize@remaining@text\htmlize@rdelim@i}%
      \else
        \ifx\test@i\@otherbackslash%
          % Control sequence.  Oh boy.
          % There are three possible things to do:
          % 1. Retokenize everything, so we get a token list.
          %    Expand this control sequence, the first one in the list,
          %    to obtain a new token list.  Then resume processing that list.
          %    Examples: \def\emph#1{<EM>#1</EM>}, \def\rec#1{\frac{1}{#1}}, \def\inv{^{-1}}
          %              \& --> &amp; \# --> #; etc.
          %    Environments: \begin{center}...\end{center} --> <CENTER>...</CENTER>
          % 2. Retokenize everything, so we get a token list.
          %    Let this first command (with its parameters) ACT.
          %    This may involve work in TeX's stomach (e.g., with counters)
          %    or with external files (e.g., image processing).
          %    The command may directly add material to \htmlize@output,
          %    but it should not typeset anything and should vanish from the 
          %    input stream when it is done.
          %       When it's done, we somehow need to detokenize and resume 
          %    processing the remainder of the input stream.
          %       Only commands explicitly crafted (or modified) to work
          %    with moodle.sty can possibly do all this!
          %    Examples: (modified) \includegraphics
          %    Environments: \begin{clozemulti}, \begin{enumerate}
          % 3. Ignore that it's a command.  Pass it right on as a character
          %    sequence to \htmlize@output.
          %    Examples: \alpha, \frac
          %    Environments: \begin{array}
          %
          % #2 is only for items on a specific list.
          % #1 is anything that runs in TeX's mouth.  
          %    We could keep a list and give users a way to add to it.
          %    I could also try expanding macros, using \ifcsmacro from etoolbox.sty
          % 
          % The first step is to figure out what control sequence we're dealing with.
          % First, let's get rid of the terminal \@htmlize@stop codes.
          {\def\@htmlize@stop{}\xdef\htmlize@scrap{#1#2#3}}%
          \let\htmlize@text@to@rescan=\htmlize@scrap%
          % Next, we retokenize the code.
          \xa\retokenizenormal@to@macro\xa\htmlize@rescanned\xa{\htmlize@text@to@rescan}%
          % Now break it up into two pieces.
          \xa\htmlize@grabblock\htmlize@rescanned\@htmlize@stop\htmlize@rdelim@ii%
          \let\@htmlize@cs\htmlize@blockinbraces%
          \edef\htmlize@cs@string{\xa\string\@htmlize@cs}%
          % The first piece, \htmlize@blockinbraces, will contain the single token in \@htmlize@cs.
          % We'll need to keep the second part, \htmlize@remainder, since it probably
          % contains arguments to the cs in \@htmlize@cs.
          %\edef\ds{Encountered '\xa\string\@htmlize@cs'}\show\ds
          %
          % N.B. that \@htmlize@cs is a macro *containing* a single control sequence.
          % This is good for testing with \ifx.
          % \htmlize@cs@string contains the cs as a string, e.g., the characters "\emph".
          %
          \ifx\@htmlize@cs\@@begin@cs
            %This is a \begin.  Begin environment-handling routine.
            %
            % Grab the first {...} from \htmlize@remainder, which is the argument
            % to \begin.
            \xa\htmlize@grabblock\htmlize@remainder\@htmlize@stop\htmlize@rdelim@ii%
            \let\htmlize@envname=\htmlize@blockinbraces%
            %We do not need the rest, so we won't pay any attention to the new
            %content of \htmlize@remainder.
            %
            %Now environments are non-expandable,
            %so there are only two possibilities: action or pass.
            \xa\ifinlist\xa{\htmlize@envname}{\htmlize@env@actionlist}%
              {% Action environment!
                %\bgroup
                  %\def\ds{Encountered active environment \string\begin\{{\htmlize@envname}\}}\show\ds
                  \def\htmlize@next@i{\xa\htmlize@do@actionenv\htmlize@rescanned\@htmlize@stop\htmlize@actionsequence@rdelim}%
                %The \bgroup is to active the environments.
                %The matching \egroup is found in \htmlize@do@actionenv.
              }{%An environment to pass to the HTML
                %We just pass the backslash from "\begin" and move on. 
                \g@addto@macro\htmlize@output{#1}%
                \def\htmlize@next@i{\htmlize@recursive@i#2#3\htmlize@rdelim@i}%
              }%
          \else%
            %This is not an environment.  Begin macro-handling routine.
            \htmlize@actioncsfalse
            \htmlize@expandcsfalse
            \htmlize@passcsfalse
            \xa\ifinlist\xa{\htmlize@cs@string}{\htmlize@cs@actionlist}%
              {%Action sequence!
               \htmlize@actioncstrue}%
              {% Not action sequence!
               \xa\ifinlist\xa{\htmlize@cs@string}{\htmlize@cs@expandlist}%
                 {%CS to be expanded!
                  \htmlize@expandcstrue%
                 }%
                 {%CS to be transcribed to XML
                  \htmlize@passcstrue%
                 }%
              }%
            %Now exactly one of \ifhtmlize@actioncs, \ifhtmlize@expandcs, and \ifhtmlize@passcs is true.
            \ifhtmlize@actioncs
              % It's an action-sequence.
              %\edef\ds{Must let \xa\string\@htmlize@cs\ act!}\show\ds
              %\show\htmlize@rescanned
              \def\htmlize@next@i{\xa\htmlize@do@actioncs\htmlize@rescanned\@htmlize@stop\htmlize@actionsequence@rdelim}%
              %\show\htmlize@rescanned
              % Note that \htmlize@do@actioncs should patch the command to have it
              % restart the scanning in time.
            \else
              \ifhtmlize@expandcs
                % control sequence to be expanded
                %\edef\ds{Must expand \xa\string\@htmlize@cs}\show\ds
                \bgroup
                  \htmlize@redefine@expansionmacros
                  %The \expandafters first expand \htmlize@rescanned,
                  %and then expand its first token just once.
                  \xa\xa\xa\gdef\xa\xa\xa\htmlize@scrap\xa\xa\xa{\htmlize@rescanned}%
                \egroup
                \xa\ultradetokenize@to@macro\xa\htmlize@remaining@text\xa{\htmlize@scrap}%
                \def\htmlize@next@i{\xa\htmlize@recursive@i\htmlize@remaining@text\@htmlize@stop\@htmlize@stop\@htmlize@stop\htmlize@rdelim@i}%
              \else
                % control sequence to be transcribed to XML.
                %\edef\ds{Must pass on \xa\string\@htmlize@cs}\show\ds
                \g@addto@macro\htmlize@output{#1}%
                \def\htmlize@next@i{\htmlize@recursive@i#2#3\htmlize@rdelim@i}%
              \fi
            \fi
          \fi
        \else%
          \ifx\test@i\@othertilde%
            % The ~ becomes non-breaking space &nbsp;
            \g@addto@macro\htmlize@output{\otherampersand nbsp;}%
            \def\htmlize@next@i{\htmlize@recursive@i#2#3\htmlize@rdelim@i}%
          \else
            \ifx\test@i\@lsinglequote%
              \ifx\test@ii\@lsinglequote%
                % Double left quote
                \g@addto@macro\htmlize@output{\otherampersand ldquo;}%
                \def\htmlize@next@i{\htmlize@recursive@i#3\htmlize@rdelim@i}%
              \else
                \g@addto@macro\htmlize@output{\otherampersand lsquo;}%
                \def\htmlize@next@i{\htmlize@recursive@i#2#3\htmlize@rdelim@i}%
              \fi
            \else
              \ifx\test@i\@rsinglequote%
                \ifx\test@ii\@rsinglequote%
                  %Double right quote
                  \g@addto@macro\htmlize@output{\otherampersand rdquo;}%
                  \def\htmlize@next@i{\htmlize@recursive@i#3\htmlize@rdelim@i}%
                \else
                  \g@addto@macro\htmlize@output{\otherampersand rsquo;}%
                  \def\htmlize@next@i{\htmlize@recursive@i#2#3\htmlize@rdelim@i}%
                \fi
              \else
                \ifx\test@i\@doublequote
                  \g@addto@macro\htmlize@output{\otherampersand rdquo;}%
                  \def\htmlize@next@i{\htmlize@recursive@i#2#3\htmlize@rdelim@i}%
                \else
                  \ifx\test@i\@lt
                    \moodle@ifmathmode{\g@addto@macro\htmlize@output{\otherampersand lt;}}%
                                      {\g@addto@macro\htmlize@output{<}}%
                    \def\htmlize@next@i{\htmlize@recursive@i#2#3\htmlize@rdelim@i}%
                  \else
                    \ifx\test@i\@gt
                      \moodle@ifmathmode{\g@addto@macro\htmlize@output{\otherampersand gt;}}%
                                        {\g@addto@macro\htmlize@output{>}}%
                      \def\htmlize@next@i{\htmlize@recursive@i#2#3\htmlize@rdelim@i}%
                    \else                                      
                      % Default case: write first token to output, call self on remaining tokens.
                      \g@addto@macro\htmlize@output{#1}%
                      \def\htmlize@next@i{\htmlize@recursive@i#2#3\htmlize@rdelim@i}%
                    \fi
                  \fi
                \fi
              \fi
            \fi
          \fi
        \fi
      \fi
    \fi
  \fi
  \htmlize@next@i
}%

\def\@lsinglequote{`}%
\def\@rsinglequote{'}%
\def\@doublequote{"}%
%    \end{macrocode}
% 
% \subsection{Math Mode handling}
% 
% In the following, note that the |\|\meta{*}|mathrightdelim|'s gobble an argument.
% This is so ``|$a$ is...|" can turn into
%       ``\ldots |a\|\meta{*}|mathrightdelim{} is...|"
% and preserve a trailing space.
%    \begin{macrocode}
\edef\inlinemathleftdelim{\otherbackslash(}%
\def\inlinemathrightdelim#1{\advancemathmodecounter{-1}%
                            \g@addto@macro\htmlize@output{\otherbackslash)}}%
\edef\displaymathleftdelim{<CENTER>\otherbackslash[}%
\def\displaymathrightdelim#1{\advancemathmodecounter{-1}%
                             \g@addto@macro\htmlize@output{\otherbackslash]</CENTER>}}%
{\catcode`\$=12\relax%
  \gdef\htmlize@inlinemathshift@replace$#1$#2\htmlize@rdelim@iii{%
    %\def\ds{inline math shift has '#1' and '#2'}\show\ds
    \xa\g@addto@macro\xa\htmlize@output\xa{\inlinemathleftdelim}%
    \advancemathmodecounter{1}%
    \def\mathtext{#1}%
    \def\aftertext{#2}%
    \xdef\htmlize@remaining@text{\expandonce\mathtext%
                                 \otherbackslash inlinemathrightdelim{}%
                                 \expandonce\aftertext}%
    %\show\htmlize@remaining@text
  }%
  \gdef\htmlize@displaymathshift@replace$$#1$$#2\htmlize@rdelim@iii{%
    \xa\g@addto@macro\xa\htmlize@output\xa{\displaymathleftdelim}%
    \advancemathmodecounter{1}%
    \def\mathtext{#1}%
    \def\aftertext{#2}%
    \xdef\htmlize@remaining@text{\expandonce\mathtext%
                                 \otherbackslash displaymathrightdelim{}%
                                 \expandonce\aftertext}%
  }%
}
%    \end{macrocode}
% 
% \subsection{Engines for Control Sequences}
% 
% There are three kinds of control sequences that need special handling:
% \begin{enumerate}
%   \item Action environments
%   \item Action command sequences
%   \item Expansion macros
% \end{enumerate}
% 
% \subsubsection{Engine for running action environments}
%    \begin{macrocode}
\long\def\htmlize@do@actionenv#1#2\@htmlize@stop\htmlize@actionsequence@rdelim{%
  \bgroup %The corresponding \egroup is given in \htmlize@proceedwiththerest, 
          %to localize the changes to the environment definitions.
    \htmlize@activate@environments
    \gdef\htmlize@afteraction@hook{}%
    #1#2\@htmlize@stop\htmlize@actionsequence@rdelim%
}

\def\htmlize@patchendenvironment{\swaptotrueendenvironment{\xa\htmlize@proceedwiththerest\htmlize@afteraction@hook}}%

\def\swaptotrueendenvironment#1#2\if@ignore\@ignorefalse\ignorespaces\fi{#2\if@ignore\@ignorefalse\ignorespaces\fi#1}%


\long\def\htmlize@record@environment#1{%
  \listadd{\htmlize@env@actionlist}{#1}%
}
\long\def\html@newenvironment#1#2{%
  \listadd{\htmlize@env@actionlist}{#1}%
  \g@addto@macro\htmlize@activate@environments{%
    \xa\let\csname #1\endcsname\relax%
    \xa\let\csname end#1\endcsname\relax%
    \NewEnviron{#1}{%
      #2%
    }[\htmlize@patchendenvironment]%
  }%
}


\def\htmlize@activate@environments{}%
%    \end{macrocode}
% 
% \subsubsection{Engine for running action command sequences}
% 
% The following automatically adds the ``engine'' to do the command
% and then resume processing the \LaTeX\ into HTML.
% It uses the |xpatch| package, which says it works with anything
% defined using |\newcommand| etc. and |\newenvironment| etc.
%    \begin{macrocode}
\gdef\htmlize@afteraction@hook{}%

\long\def\htmlize@do@actioncs#1#2\htmlize@actionsequence@rdelim{%
  % #1#2 contains the current string to be rendered into HTML;
  %      N.B. it has been tokenized at this point, 
  %      so TeX can process it directly.
  % #1 = the command sequence we need to execute
  % #2 = the rest of the string
  %
  % First, we patch the desired command so that, when it is over,
  % it calls \htmlize@proceedwiththerest.
  % We do this within the group, so as not to permanently change the command.
  \bgroup
    % The matching \egroup is issued in \htmlize@proceedwiththerest,
    % so that the changes made by \htmlize@activate@css are localized to just the command itself.
    \gdef\htmlize@afteraction@hook{}%
    \htmlize@activate@css%
    \def\test@i{#1}%
    \ifx\test@i\@relax
      \def#1{\xa\htmlize@proceedwiththerest\htmlize@afteraction@hook}%
    \else
      \xapptocmd#1{\xa\htmlize@proceedwiththerest\htmlize@afteraction@hook}{}{\PackageError{Could not patch the command \string#1!}}%
    \fi
    % Now we call that patched command.
    #1#2\htmlize@actionsequence@rdelim%
  %The matching \egroup now is built into the command #1.
}
\long\def\htmlize@proceedwiththerest#1\htmlize@actionsequence@rdelim{%
    % The action cs has done its work.
    % Now we gather up the remaining tokens, detokenize them,
    % remove the \@htmlize@stop, and get back to work transcribing it.
  \egroup %This \egroup matches the \bgroup that was issued either in \htmlize@do@actioncs or in \htmlize@do@actionenv
  \ultradetokenize@to@macro\htmlize@remainder@detokenized{#1}%
  %This will contain an extra \@htmlize@stop, so we remove it.
  \xa\xa\xa\def\xa\xa\xa\htmlize@remainder@detokenized\xa\xa\xa{\xa\htmlize@remove@stopcode\htmlize@remainder@detokenized}%
  %Now we get back to work transcribing the remainder.
  \xa\htmlize@recursive@i\htmlize@remainder@detokenized\@htmlize@stop\@htmlize@stop\@htmlize@stop\htmlize@rdelim@i%  
}

\long\def\htmlize@record@action#1{%
  \xa\listadd\xa\htmlize@cs@actionlist\xa{\string#1}%
}

\def\htmlize@activate@css{}%
\def\html@action@def#1{%
  \htmlize@record@action{#1}%
  \xa\def\xa\htmlize@scrap\xa{\xa\let\xa#1\csname html@\string#1\endcsname}%
  \xa\g@addto@macro\xa\htmlize@activate@css\xa{\htmlize@scrap}%
  \xa\def\csname html@\string#1\endcsname% %And this \def\html@\oldcsname is follows by the remainder of the definition.
}
%    \end{macrocode}
% 
% \subsubsection{Engine for expansion control sequences}
% 
% Calling |\htmlize@redefine@expansionmacros| will redefine
% the macros for us.  It starts out empty.
%    \begin{macrocode}
\def\htmlize@redefine@expansionmacros{}%
%    \end{macrocode}
% If |\mymacro| needs no changes to be suited for expansion,
% you can simply call |\htmlize@record@expand{\mymacro}|
% or |\htmlregister{\mymacro}|
% to record that it should be expanded on its way to the HTML.
% Examples would be user-built macros such as |\inv|$\to$|^{-1}|
% or |\N|$\to$|\mathbb{N}|.
%    \begin{macrocode}
\long\def\htmlize@record@expand#1{%
  \xa\listadd\xa\htmlize@cs@expandlist\xa{\string#1}%
}
\let\htmlregister=\htmlize@record@expand
%    \end{macrocode}
% On the other hand, if an alternate version of the macro is 
% needed for HTML purposes, you can define its HTML version with
% |\html@def\mymacro...|
% Parameters are okay.
% An example would be
% |\html@def\emph#1{<EM>#1</EM>}|.
%    \begin{macrocode}
\long\def\html@def#1{%
  \htmlize@record@expand{#1}%
  \xa\def\xa\htmlize@scrap\xa{\xa\let\xa#1\csname html@\string#1\endcsname}%
  \xa\g@addto@macro\xa\htmlize@redefine@expansionmacros\xa{\htmlize@scrap}%
  \xa\def\csname html@\string#1\endcsname%
}
%    \end{macrocode}
% Note that when |\html@def| expands out, it ends with |\def\html@\oldcsname| 
% which abuts directly on the remainder of the definition.
% 
% \subsection{Specific Control Sequences for Action and Expansion}
% 
% Now that we have that machinery in place,
% we define specific environments, action control sequences, and macros to 
% expand to accomplish our purposes.
% 
% \subsubsection{Action Environments}
%    \begin{macrocode}
\htmlize@record@environment{clozemulti}
\htmlize@record@environment{multi}
\htmlize@record@environment{clozenumerical}
\htmlize@record@environment{numerical}
\htmlize@record@environment{clozeshortanswer}
\htmlize@record@environment{shortanswer}

\html@newenvironment{center}{\xdef\htmlize@afteraction@hook{<CENTER>\expandonce\BODY</CENTER>}}%
\html@newenvironment{enumerate}{%
    \xa\gatheritems\xa{\BODY}%
    \gdef\htmlize@afteraction@hook{<OL>}%
    \loopthroughitemswithcommand{\moodle@itemtoLI}%
    \g@addto@macro\htmlize@afteraction@hook{</OL>}%
}%
\html@newenvironment{itemize}{%
    \xa\gatheritems\xa{\BODY}%
    \gdef\htmlize@afteraction@hook{<UL>}%
    \loopthroughitemswithcommand{\moodle@itemtoLI}%
    \g@addto@macro\htmlize@afteraction@hook{</UL>}%
}%
\def\moodle@itemtoLI#1{%
  \g@addto@macro\htmlize@afteraction@hook{<LI>#1</LI>}%
}%

%    \end{macrocode}
% 
% \subsubsection{Action Control Sequences}
% 
%    \begin{macrocode}
\def\advancemathmodecounter#1{%
  \global\advance\moodle@mathmodedepth by #1\relax
}
\htmlize@record@action{\advancemathmodecounter}%




\htmlize@record@action{\relax}%

\html@action@def\#{\g@addto@macro\htmlize@output{\otherhash}}%
\html@action@def\&{\g@addto@macro\htmlize@output{\otherampersand}}%
\html@action@def\\{\moodle@ifmathmode{\g@addto@macro\htmlize@output{\otherbackslash\otherbackslash}}%
                                     {\g@addto@macro\htmlize@output{<BR/>}}}%
\html@action@def\{{%
    \moodle@ifmathmode{\g@addto@macro\htmlize@output{\otherbackslash\otherlbrace}}%
                      {\g@addto@macro\htmlize@output{\otherlbrace}}%
  }%
\html@action@def\}{%
    \moodle@ifmathmode{\g@addto@macro\htmlize@output{\otherbackslash\otherrbrace}}%
                      {\g@addto@macro\htmlize@output{\otherrbrace}}%
  }%
\html@action@def\ldots{%
    \moodle@ifmathmode{\g@addto@macro\htmlize@output{\string\ldots}}%
                      {\g@addto@macro\htmlize@output{...}}%
  }%
\html@action@def\dots{%
    \moodle@ifmathmode{\g@addto@macro\htmlize@output{\string\dots}}%
                      {\g@addto@macro\htmlize@output{...}}%
  }%
\html@action@def\ {%
    \moodle@ifmathmode{\g@addto@macro\htmlize@output{\otherbackslash\otherspace}}%
                      {\g@addto@macro\htmlize@output{\otherspace}}%
  }%
\html@action@def\${%
     \g@addto@macro\htmlize@output{\otherdollar}%
  }%

\html@action@def\clozetilde{%
    \xa\g@addto@macro\xa\htmlize@output\xa{\othertilde}%
  }%



%Diacritical marks over vowels
{\catcode`|=3\relax
 \gdef\htmlize@vowels{a|e|i|o|u|A|E|I|O|U|}}
\def\htmlize@define@diacritic#1#2{%
  \htmlize@record@action{#1}%
  \g@addto@macro\htmlize@activate@css{%
    \def#1##1{%
      \ifinlist{##1}{\htmlize@vowels}%
        {\g@addto@macro\htmlize@output{\otherampersand##1#2;}}%
        {\xa\g@addto@macro\htmlize@output\xa{\string#1##1}}%
    }%
  }%
}
\htmlize@define@diacritic{\^}{circ}%
\htmlize@define@diacritic{\'}{acute}%
\htmlize@define@diacritic{\"}{uml}%
\htmlize@define@diacritic{\`}{grave}%

%Hungarian long-umlaut diacritics
\def\@o{o}\def\@O{O}\def\@u{u}\def\@U{U}%
\html@action@def\H#1{%
  \bgroup
    \def\test@i{#1}%
    \ifx\test@i\@O
      \def\toadd{\otherampersand\otherhash337;}%
    \else
      \ifx\test@i\@o
        \def\toadd{\otherampersand\otherhash337;}%
      \else
        \ifx\test@i\U
          \def\toadd{\otherampersand\otherhash368;}%
        \else
          \ifx\test@i\u
            \def\toadd{\otherampersand\otherhash369;}%
          \else
            \def\toadd{\otherbackslash\otherlbrace#1\otherrbrace}%
          \fi
        \fi
      \fi
    \fi
    \xa\g@addto@macro\xa\htmlize@output\xa{\toadd}%
  \egroup
}%


%Tilde over a, n, o
{\catcode`|=3\relax
 \gdef\htmlize@tilde{A|N|O|a|n|o|}}
\html@action@def\~#1{%
    \ifinlist{#1}{\htmlize@tilde}%
      {\g@addto@macro\htmlize@output{\otherampersand#1tilde;}}%
      {\xa\g@addto@macro\htmlize@output\xa{\string\~#1}}%
  }%
%    \end{macrocode}
% 
% \subsubsection{Command sequences for Expansion}
% 
%    \begin{macrocode}
\html@def\underline#1{<SPAN STYLE="text-decoration: underline;">#1</SPAN>}
\html@def\emph#1{<EM>#1</EM>}%
\html@def\textbf#1{<B>#1</B>}%
\html@def\blank{____________}%
\html@def\par{</P><P>}%
\html@def\aa{\&aring;}%
\html@def\AA{\&Aring;}%
\html@def\ae{\&aelig;}%
\html@def\AE{\&AElig;}%
\html@def\S{\&sect;}%
\html@def\euro{\&euro;}%
\html@def\texteuro{\&euro;}%
\html@def\o{\&oslash;}%
\html@def\O{\&Oslash;}%
\html@def\ss{\&szlig;}%
\html@def\l{\&lstrok;}%
\html@def\L{\&Lstrok;}%

% AH CUSTOM MACROS TO EXPAND --- remove these before publishing!
%\htmlize@record@expand{\inv}%
%\htmlize@record@expand{\rec}%

\htmlize@record@action\inlinemathrightdelim
\htmlize@record@action\displaymathrightdelim


%    \end{macrocode}
% 
% \subsection{Graphics via {\tt\string\includegraphics}}
% 
% \subsubsection{External program command lines}
% We first set up commands for the external programs.
%    \begin{macrocode}
\def\htmlize@setexecutable#1{%
  % Defines macro #1 to be #2 in a verbatim mode suitable for filenames
  \def\htmlize@executable@macro{#1}%
  \bgroup\catcode`\|=0\catcode`\\=12\relax%
  \htmlize@setexecutable@int
}
\def\htmlize@setexecutable@int#1{%
  \egroup 
  \expandafter\def\htmlize@executable@macro{#1}%
}

\def\ghostscriptcommand{\htmlize@setexecutable\gs}
\def\opensslcommand{\htmlize@setexecutable\openssl}
\def\imagemagickcommand{\htmlize@setexecutable\htmlize@imagemagick@convert}
\def\optipngcommand{\htmlize@setexecutable\optipng}

\ifwindows%
  \ghostscriptcommand{gswin64c.exe}%
\else%
  \ghostscriptcommand{gs}%
\fi%

\opensslcommand{openssl}%
\imagemagickcommand{convert}%
\optipngcommand{optipng}
%    \end{macrocode}
% 
% \subsubsection{Graphics key-handling}
% Next, we get ready to handle keys like |height=4cm| or |width=3cm| or |ppi=72|.
%    \begin{macrocode}
\define@cmdkeys{moodle@includegraphics}[moodle@graphics@]{ppi}
\define@cmdkey{moodle}[moodle@graphics@]{ppi}{}% This is so the ppi key can be set at the document, quiz, or question level.
\define@cmdkeys{Gin}{ppi}% This is so the original \includegraphics will not object to a key of ppi.
\setkeys{moodle@includegraphics}{ppi=103}

\newdimen\moodle@graphics@temp@dimen
\newcount\moodle@graphics@height@pixels
\newcount\moodle@graphics@width@pixels
\def\moodle@graphics@dimentopixels#1#2{%
  \moodle@graphics@temp@dimen=#2\relax
  \moodle@graphics@temp@dimen=0.013837\moodle@graphics@temp@dimen
  \xa\moodle@graphics@temp@dimen\xa=\moodle@graphics@ppi\moodle@graphics@temp@dimen
  #1=\moodle@graphics@temp@dimen
  \divide #1 by 65536\relax
}
\define@key{moodle@includegraphics}{height}[]{%
  \moodle@graphics@dimentopixels{\moodle@graphics@height@pixels}{#1}%
}
\define@key{moodle@includegraphics}{width}[]{%
  \moodle@graphics@dimentopixels{\moodle@graphics@width@pixels}{#1}%
}
\setkeys{moodle@includegraphics}{height=0pt,width=0pt}
%    \end{macrocode}
% 
% \subsubsection{Graphics conversion to HTML}
%    \begin{macrocode}
\ifmoodle@draftmode
\else
\html@action@def\includegraphics{\moodle@includegraphics@int}%
\fi
  % Note that \htmlize@do@actioncs will 'patch' this by putting 
  % '\xa\htmlize@proceedwiththerest\htmlize@afteraction@hook'
  % at the end.  We want those 3 tokens to occur instead after
  % the graphics filename.
\def\moodle@includegraphics@int#1#2#3{\moodle@includegraphics@int@int}%
  % This gobbles up those three spurious tokens,
  % which we will re-insert after our work is done.
\newcommand\moodle@includegraphics@int@int[2][]{%
  \bgroup% The grouping is to localize the changes caused by \setkeys.
    \message{moodle.sty: Processing \string\includegraphics[#1]{#2} for HTML...^^J}
    \setkeys*{moodle@includegraphics}{#1}%
    % Height or width should be given in TeX dimensions like cm or pt or in,
    % and are converted to pixels for web use using the ppi key.  
    % TO DO: Can we modify \includegraphics to accept height or width in 
    %        pixels?
    % TO DO: What about \includegraphics[scale=0.7] ?
    %        Other keys: keepaspectratio=true|false, angle (rotation), clip & trim
    \ifnum\moodle@graphics@height@pixels=0\relax
      \ifnum\moodle@graphics@width@pixels=0\relax
        % No size specified.  Default to height of 200 pixels.
        \def\moodle@graphics@geometry{x200}%
        \def\moodle@graphics@htmlgeometry{}%
      \else
        % Width only specified.
        \edef\moodle@graphics@geometry{\number\moodle@graphics@width@pixels}%
        \edef\moodle@graphics@htmlgeometry{width=\number\moodle@graphics@width@pixels}%
      \fi
    \else
      \ifnum\moodle@graphics@width@pixels=0\relax
        % Height only specified.  The `x' is part of the syntax.
        \edef\moodle@graphics@geometry{x\number\moodle@graphics@height@pixels}%
        \edef\moodle@graphics@htmlgeometry{height=\number\moodle@graphics@height@pixels}%
      \else
        % Height and width both specified.  The `!' is part of the syntax.
        \edef\moodle@graphics@geometry{\number\moodle@graphics@width@pixels x\number\moodle@graphics@height@pixels!}%
        \edef\moodle@graphics@htmlgeometry{width=\number\moodle@graphics@width@pixels\otherspace height=\number\moodle@graphics@height@pixels}%
      \fi
    \fi
    %First, convert it to PNG
    \edef\cmdline{\htmlize@imagemagick@convert\otherspace #2 -resize \moodle@graphics@geometry\otherspace #2.png}%
    \message{moodle.sty:   Converting '#2' to PNG...^^J}%
    \expandafter\immediate\expandafter\write18\expandafter{\cmdline}%
    %Next, optimize
    \edef\cmdline{\optipng\otherspace -clobber -quiet \tikzexternalrealjob-tikztemp-\the\numconvertedpictures.png}
    \message{moodle.sty:   Optimizing '#2.png'...^^J}%
    \expandafter\immediate\expandafter\write18\expandafter{\cmdline}%
    %Next, convert the PNG to base64 encoding
    \def\cmdline{\openssl\otherspace enc -base64 -in #2.png -out #2.enc}%
    \message{moodle.sty:   Converting '#2.png' to base64...^^J}%
    \expandafter\immediate\expandafter\write18\expandafter{\cmdline}%
    %Now, save that base64 encoding in a TeX macro
    \def\moodle@newpic@baselxiv{}%
    \message{moodle.sty:   Reading base64 file '#2.enc'...^^J}%
    \openin\baseLXIVdatafile=#2.enc\relax
      \savebaselxivdata@recursive
    \closein\baseLXIVdatafile
    \ifwindows
      \immediate\write18{del #2.enc #2.png}%
    \else
      \immediate\write18{rm -f #2.enc #2.png}%
    \fi
    \xa\global\xa\let\csname picbaselxiv@graphics@#2\endcsname=\moodle@newpic@baselxiv%
    \edef\htmlize@imagetag{<IMG \moodle@graphics@htmlgeometry\otherspace SRC="data:image/png;base64,\csname picbaselxiv@graphics@#2\endcsname">}%
    \xa\g@addto@macro\xa\htmlize@output\xa{\htmlize@imagetag}%
    \message{moodle.sty:   <IMG> tag inserted.^^J}%
  \egroup
  % Now we re-insert the code to get the HTMLizing going again.
  \xa\htmlize@proceedwiththerest\htmlize@afteraction@hook
}
%    \end{macrocode}
% The following code accomplishes the reading of an |.enc| file into memory.
% It is also used by the \TikZ\ code below.
%    \begin{macrocode}
\newread\baseLXIVdatafile
\def\savebaselxivdata@recursive{%
  \ifeof\baseLXIVdatafile
    \let\baselxiv@next=\relax
  \else
    \read\baseLXIVdatafile to \datalinein
    %\message{<<\datalinein>>^^J}
    \ifx\datalinein\@moodle@par
      \let\baselxiv@next=\relax
    \else
      %We add tokens manually, rather than with \g@addto@macro, to save time.
      \xa\xa\xa\gdef\xa\xa\xa\moodle@newpic@baselxiv\xa\xa\xa{\xa\moodle@newpic@baselxiv\datalinein^^J}%
      \let\baselxiv@next=\savebaselxivdata@recursive
    \fi
  \fi
  \baselxiv@next
}
%    \end{macrocode}
% 
% \subsection{\TikZ\ Picture Handling}
% If the user is not using the \TikZ\ package, there is no need to waste time 
% loading it.  Without \TikZ\ loaded, however, many of the following commands
% are undefined.
% Our solution is to wait until |\AtBeginDocument| and then test whether 
% \TikZ\ is loaded.  If so, we make the appropriate definitions.
%    \begin{macrocode}
\newif\ifmoodle@tikzloaded
\moodle@tikzloadedfalse
\AtBeginDocument{
  \ifx\tikzpicture\@undefined
    \moodle@tikzloadedfalse
  \else
    \moodle@tikzloadedtrue
  \fi
  \ifmoodle@draftmode
    \long\def\tikzifexternalizing#1#2{#2}%
  \else
  \ifmoodle@tikzloaded
    \usetikzlibrary{external}
    \tikzexternalize%
    \tikzset{external/force remake}%

    \ifpdf
      \def\pdftopng{\edef\gscmdline{\gs\otherspace -dBATCH -dNOPAUSE -sDEVICE=pngalpha -sOutputFile=\tikzexternalrealjob-tikztemp-\the\numconvertedpictures.png -r150 \tikzexternalrealjob-tikztemp-\the\numconvertedpictures.pdf}%
                    \expandafter\immediate\expandafter\write18\expandafter{\gscmdline}}%
    \else
      \def\pstopng{\edef\gscmdline{\gs\otherspace -dBATCH -dNOPAUSE -sDEVICE=pngalpha -sOutputFile=\tikzexternalrealjob-tikztemp-\the\numconvertedpictures.png -r150 \tikzexternalrealjob-tikztemp-\the\numconvertedpictures.ps}%
                    \expandafter\immediate\expandafter\write18\expandafter{\gscmdline}}%
    \fi
    \def\pngoptim{\edef\cmdline{\optipng\otherspace -clobber -quiet \tikzexternalrealjob-tikztemp-\the\numconvertedpictures.png}%
      \expandafter\immediate\expandafter\write18\expandafter{\cmdline}}%
    \def\pngtobaselxiv{\edef\opensslcmdline{\openssl\otherspace enc -base64 -in \tikzexternalrealjob-tikztemp-\the\numconvertedpictures.png -out \tikzexternalrealjob-tikztemp-\the\numconvertedpictures.enc}%
                       \expandafter\immediate\expandafter\write18\expandafter{\opensslcmdline}}%

    \let\moodle@oldtikzpicture=\tikzpicture

    %The following code lets us run things *before* the normal \begin{tikzpicture}.
    \renewenvironment{tikzpicture}{%
      \global\advance\numconvertedpictures by 1\relax
      %\jobnamewithsuffixtomacro{\htmlize@picbasename}{-tikztemp-\the\numconvertedpictures}%
      %\xa\tikzsetnextfilename\xa{\htmlize@picbasename}%
      \tikzsetnextfilename{\tikzexternalrealjob-tikztemp-\the\numconvertedpictures}%
      \moodle@oldtikzpicture%
    }{}%
    % However, the tikz externalize library does *not* run \end{tikzpicture}.
    % In order to run commands after the tikz picture is done compiling, we need to 
    % use a hook into \tikzexternal@closeenvironments.

    \g@addto@macro{\tikzexternal@closeenvironments}{%
      \moodle@endtikzpicture@hook
    }
    \def\moodle@endtikzpicture@hook{%
      \@moodle@ifgeneratexml{%
        \ifpdf
          \message{moodle.sty: Converting picture '\tikzexternalrealjob-tikztemp-\the\numconvertedpictures.pdf' to PNG...^^J}%
          \pdftopng
        \else
          \pstopng
        \fi
        \message{moodle.sty:   Optimizing '\tikzexternalrealjob-tikztemp-\the\numconvertedpictures.png'...^^J}%
        \pngoptim
        \message{moodle.sty:   Converting '\tikzexternalrealjob-tikztemp-\the\numconvertedpictures.png' to base64...^^J}%
        \pngtobaselxiv
        \message{moodle.sty:   Reading base64 file '\tikzexternalrealjob-tikztemp-\the\numconvertedpictures.enc'...^^J}%
        \savebaselxivdata
        \message{moodle.sty:   base64 data saved.^^J}%
      }{}%
    }
    \tikzset{external/optimize=true}
    \tikzset{external/optimize command away={\VerbatimInput}{1}}
    %
    % The HTMLizer version of the tikzpicture environment,
    % which writes an <IMG> tag to the XML file.
    \htmlize@record@environment{tikzpicture}
    \g@addto@macro\htmlize@activate@environments{%
      \let\tikzpicture\relax\let\endtikzpicture\relax
      \NewEnviron{tikzpicture}[1][]{%
        \global\advance\numpicturesread by 1\relax
        \edef\htmlize@imagetag{<IMG SRC="data:image/png;base64,\csname picbaselxiv@\the\numpicturesread\endcsname">}%
        \xa\g@addto@macro\xa\htmlize@output\xa{\htmlize@imagetag}%
      }[\htmlize@patchendenvironment]%
    }%
  \else
    %TikZ not loaded.  Provide dummy definitions for commands.
    \long\def\tikzifexternalizing#1#2{#2}%
  \fi
  \fi
}

\newcount\numconvertedpictures
\numconvertedpictures=0\relax
\newcount\numpicturesread
\numpicturesread=0\relax

\def\savebaselxivdata{%
  \def\moodle@newpic@baselxiv{}%
  \openin\baseLXIVdatafile=\tikzexternalrealjob-tikztemp-\the\numconvertedpictures.enc\relax
    \savebaselxivdata@recursive
  \closein\baseLXIVdatafile
  \xa\global\xa\let\csname picbaselxiv@\the\numconvertedpictures\endcsname=\moodle@newpic@baselxiv%
}

\def\@moodle@par{\par}%
%    \end{macrocode}
% Finally, we clean up our mess by deleting the temporary PDF, PNG, and ENC 
% files we created.
%    \begin{macrocode}
\AtEndDocument{%
  \ifmoodle@tikzloaded
    \@moodle@ifgeneratexml{%
      \ifwindows
        \immediate\write18{del \tikzexternalrealjob-tikztemp-*.*}%
      \else
        \immediate\write18{rm -f \tikzexternalrealjob-tikztemp-*.*}%
      \fi
    }{}%
  \fi
}

% TO DO:
% * sizing options for TikZ pictures?
% * In cloze multi, how to handle HTML or TeX in answers?  In particular, what about ~?
%    \end{macrocode}
%
% \Finale
\endinput
