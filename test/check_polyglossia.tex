% !TeX encoding = UTF-8
% !TeX spellcheck = en_US
% !TEX TS-program = lualatex
\documentclass{article}
\usepackage[nostamp,draft]{moodle}
\usepackage{fontspec}
\usepackage{polyglossia} % polyglossia work only with XeTeX and LuaTeX
\forcsvlist{\listadd\FullSupportList}{english,french,german,italian,spanish}
\forcsvlist{\listadd\PartialSupportList}{catalan,croatian,czech,danish,dutch,
estonian,finnish,hungarian,icelandic,lithuanian,norsk,polish,portuguese,
romanian,swedish,turkish}
\newcounter{Lcount}
\def\do#1{\stepcounter{Lcount}\ifnum\the\value{Lcount}=1\setmainlanguage{#1}  
\else\setotherlanguage{#1}\fi}%
\dolistloop{\FullSupportList}
\dolistloop{\PartialSupportList}
\usepackage{translations} % overides "translator"
\DeclareLanguage{foobar}
%\AfterEndPreamble{
  \DeclareTranslation{foobar}{True}{Foo}%
  \DeclareTranslation{foobar}{False}{Bar}%
%}
\begin{document}
\section*{Introduction}
This document is intended to check the support of internationalization with 
\textsf{polyglossia}. This package requires the languages to be specified in 
the preamble.

Internationalization only matters for the PDF typesetting. It has no impact on 
the XML file generated.

\def\do#1{
\begin{#1}
\begin{quiz}{#1}
\begin{description}[tags={#1}]{Description}Text\end{description}
\begin{cloze}{Cloze}
\begin{multi}[shuffle]{Multichoice}?\item* A\item B\end{multi}
\begin{multi}[multiple,shuffle=false]{Multianswer}?\item* A\item B\end{multi}
\begin{numerical}{Numerical}?\item0\end{numerical}
\begin{shortanswer}[usecase]{Shortanswer}?\item0\end{shortanswer}
\begin{shortanswer}[usecase=false]{Shortanswer}?\item0\end{shortanswer}
\end{cloze}
\begin{essay}[template={Default}]{Essay}Text\item Info\end{essay}
\begin{essay}[response format=html+file]{Essay}Text\end{essay}
\begin{essay}[response format=text]{Essay}Text\end{essay}
\begin{essay}[response format=monospaced]{Essay}Text\end{essay}
\begin{essay}[response format=file]{Essay}Text\end{essay}
\begin{matching}[dd]{Matching}?\item A\answer1\item B\answer2\item 
C\answer3\end{matching}
\begin{multi}[shuffle]{Multichoice}?\item* A\item B\end{multi}
\begin{multi}[multiple,shuffle=false]{Multianswer}?\item* A\item B\end{multi}
\begin{multi}[allornothing]{All-or-nothing}?\item* A\item B\end{multi}
\begin{numerical}{Numerical}?\item0\end{numerical}
\begin{shortanswer}[usecase]{Shortanswer}?\item0\end{shortanswer}
\begin{shortanswer}[usecase=false]{Shortanswer}?\item0\end{shortanswer}
\begin{truefalse}{Truefalse}?\item*\end{truefalse}
\end{quiz}
\end{#1}
}
\dolistloop{\FullSupportList}% fully supported
\def\do#1{
\begin{#1}
\begin{truefalse}{#1}?\item*\end{truefalse}
\end{#1}
}
\begin{quiz}{True/False} % partly supported
\dolistloop{\PartialSupportList}
\end{quiz}

\end{document}