% !TeX encoding = UTF-8
% !TeX spellcheck = en_US
% !TEX TS-program = lualatex
\documentclass{article}
\usepackage[nostamp]{moodle}
\ifpdftex % FOR LATEX and PDFLATEX
	\usepackage[utf8]{inputenc} % necessary
	\usepackage[T1]{fontenc} % necessary
\else % assuming XELATEX or LUALATEX
	\usepackage{fontspec}
\fi
\usepackage{hyperref}
\begin{document}

\section*{Introduction}

This document is intended to check the possibility of generating 
pseudo-calculated questions with the help of Lua scripts.

Inspired by
\url{https://github.com/avohns/python-latex-moodle-quiz/blob/master/simple-examples-eng/example1_arithmetic.tex}

\begin{quiz}[tags={calculated}]{Example Quiz}
\directlua{
function clozenum_print(pair,op,result)
  tex.print("\\begin{numerical}$"..pair[1].." "..op.." "..pair[2].." 
  =$".."\\item ",result,"\\end{numerical}")
end
function cloze_print(pair,points)
  tex.print("\\begin{cloze}[points="..points.."]{Arithmetic Quiz 
  ("..pair[1]..", "..pair[2]..")}Solve the following tasks!\\\\")
  clozenum_print(pair,"+",pair[1]+pair[2])
  clozenum_print(pair,"-",pair[1]-pair[2])
  clozenum_print(pair,"*",pair[1]*pair[2])
  if pair[1]/pair[2]==math.floor(pair[1]/pair[2]) then
    clozenum_print(pair,":",math.floor(pair[1]/pair[2]))
  end
  tex.print("\\end{cloze}")
end
for x = 2,4 do
  for y = 2,4 do
    if x>y then
      if x/y==math.floor(x/y) then points=4 else points=3 end
      cloze_print({x,y},points)
    end
  end
end
}
\end{quiz}
\end{document}