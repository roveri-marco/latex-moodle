% !TeX encoding = UTF-8
% !TeX spellcheck = en_US
% !TEX TS-program = lualatex
\documentclass{article}
\usepackage[nostamp]{moodle}
\ifPDFTeX % FOR LATEX and PDFLATEX
	\usepackage[utf8]{inputenc} % necessary
	\usepackage[T1]{fontenc} % necessary
\else % assuming XELATEX or LUALATEX
	\usepackage{fontspec}
\fi
\newif\iffvextra
\fvextratrue % comment this line to use vanilla 'fancyvrb'
\iffvextra
  \usepackage{fvextra,xcolor}
\else
  \usepackage{fancyvrb} % must be loaded after fancybox 
\fi

\fvset{firstline=3,firstnumber=2,lastline=7,numbers=left,tabsize=4,numberblanklines=false}
\iffvextra
  \fvset{highlightlines={2-5,8},numbers=both}
\fi
\begin{document}

\section*{Introduction}

This document is intended to check the support of \textsf{fancyvrb} or  
\textsf{fvextra} packages.

The only command supported is \texttt{VerbatimInput[<options>]\{<filename>\}}.
The variant with a star is unsupported.

Outside of the moodle question bodies, \texttt{VerbatimOut} environments can be 
set to define portions of code.

\begin{quiz}{Fancy Verbatim}
% /!\ firstline and lastline set previously would apply to VerbatimOut
\begin{VerbatimOut}[firstline,lastline]{test.log}
  % !TeX encoding = UTF-8
  % !TeX spellcheck = en_US
  % !TEX TS-program = xelatex
  
  \documentclass{article}
  \usepackage[nostamp]{moodle}
  \newif\iffvextra
  \fvextratrue
  \ifxetex % FOR XELATEX
   \usepackage{fontspec}
  \else %% FOR PDFLATEX
  	\usepackage[utf8]{inputenc} % necessary
  	\usepackage[T1]{fontenc} % necessary
  \fi
\end{VerbatimOut}
\fvset{gobble=2} % remove two characters at the beginning of each line
\fvset{frame=none}% overide default display settings for PDF

% To avoid collision of line and item numbers in PDF, force `left-right' mode:
\RecustomVerbatimCommand{\VerbatimInput}{LVerbatimInput}{}
\begin{multi}{Multi}
Pick up the code decorated with line numbering on both sides.
\item[feedback={No. Here, line numbers lie on the left side.}] 
\VerbatimInput{test.log}
\item[feedback={No. Here, line numbers lie on the right side.}] 
\VerbatimInput[numbers=right,firstnumber=last,numberblanklines=true]{test.log}
\item[feedback={yes!}]* 
\VerbatimInput[numbers=left,firstline=9,firstnumber=4,lastline=23]{test.log}
\end{multi}

\end{quiz}

\end{document}