% !TeX encoding = UTF-8
% !TeX spellcheck = en_US
% !TEX TS-program = lualatex
\documentclass{article}
\usepackage[nostamp]{moodle}
\ifpdftex % FOR LATEX and PDFLATEX
	\usepackage[utf8]{inputenc} % necessary
	\usepackage[T1]{fontenc} % necessary
\else % assuming XELATEX or LUALATEX
	\usepackage{fontspec}
\fi
\usepackage{hyperref}
\usepackage{python}
\begin{document}

\section*{Introduction}

This document is intended to check the support of the tolerance key for 
numerical questions.

\begin{quiz}[tolerance=1]{Tolerance}
\begin{numerical}[tolerance=2]{Num}
Give a number
\item[tolerance=4] 1
\item 0
\end{numerical}
\begin{numerical}{Num}
Give a number
\item[tolerance=4] 1
\item 0
\end{numerical}
\begin{cloze}[tolerance=2]{ClozeNum}
\begin{numerical}[tolerance=3]
Give a number
\item[tolerance=4] 1
\item 0
\end{numerical}
\begin{numerical}
Give a number
\item[tolerance=4] 1
\item 0
\end{numerical}
\end{cloze}
\begin{cloze}{ClozeNum}
\begin{numerical}[tolerance=3]
Give a number
\item[tolerance=4] 1
\item 0
\end{numerical}
\begin{numerical}
Give a number
\item[tolerance=4] 1
\item 0
\end{numerical}
\end{cloze}
\end{quiz}
\end{document}